\subsection{nusdas\_cut\_raw: 領域限定の DATA 記録直接読取}
\APILabel{nusdas.cut.raw}

\Prototype
\begin{quote}
N\_SI4 {\bf nusdas\_cut\_raw}(const char {\it type1}[8], const char {\it type2}[4], const char {\it type3}[4], const N\_SI4 $\ast${\it basetime}, const char {\it member}[4], const N\_SI4 $\ast${\it validtime}, const char {\it plane}[6], const char {\it element}[6], void $\ast${\it udata}, const N\_SI4 $\ast${\it usize}, const N\_SI4 $\ast${\it ixstart}, const N\_SI4 $\ast${\it ixfinal}, const N\_SI4 $\ast${\it iystart}, const N\_SI4 $\ast${\it iyfinal});
\end{quote}

\begin{tabular}{l|rp{20em}}
\hline
\ArgName & \ArgType & \ArgRole \\
\hline
{\it type1} & const char [8] &  種別1  \\
{\it type2} & const char [4] &  種別2  \\
{\it type3} & const char [4] &  種別3  \\
{\it basetime} & const N\_SI4 $\ast$ &  基準時刻(通算分)  \\
{\it member} & const char [4] &  メンバー名  \\
{\it validtime} & const N\_SI4 $\ast$ &  対象時刻(通算分)  \\
{\it plane} & const char [6] &  面  \\
{\it element} & const char [6] &  要素名  \\
{\it udata} & void $\ast$ &  データ格納先配列  \\
{\it usize} & const N\_SI4 $\ast$ &  データ格納先配列のバイト数  \\
{\it ixstart} & const N\_SI4 $\ast$ &  $x$ 方向格子番号下限  \\
{\it ixfinal} & const N\_SI4 $\ast$ &  $x$ 方向格子番号上限  \\
{\it iystart} & const N\_SI4 $\ast$ &  $y$ 方向格子番号下限  \\
{\it iyfinal} & const N\_SI4 $\ast$ &  $y$ 方向格子番号上限  \\
\hline
\end{tabular}
\paragraph{\FuncDesc}
\APILink{nusdas.read2.raw}{nusdas\_read2\_raw} と類似だが、データレコードのうち格子点
({\it ixstart} , {\it iystart} )--({\it ixfinal} , {\it iyfinal} )
に対応する部分だけが {\it udata} に格納される。
ただしパッキングが 2UPP, 2UPJ, RLEN の場合、または欠損値が MASK の場合は全体が格納される。

\paragraph{\ResultCode}
\begin{quote}
\begin{description}
\item[{\bf 正}] 読み出して格納したバイト数
\item[{\bf 0}] 指定したデータは未記録(定義ファイルの elementmap によって書き込まれることは許容されているが、まだデータが書き込まれていない)
\item[{\bf -2}] 指定したデータは記録することが許容されていない(elementmap によって禁止されている場合と指定した面名、要素名が登録されていない場合の両方を含む)。
\item[{\bf -4}] 格納配列が不足
\end{description}\end{quote}

\paragraph{履歴}
この関数は NuSDaS 1.1 で導入された。
エラーコード -4 は NuSDaS 1.3 で新設されたもので、
それ以前はエラーチェックがなされていなかった。
