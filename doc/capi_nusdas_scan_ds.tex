\subsection{nusdas\_scan\_ds: データセットの一覧}
\APILabel{nusdas.scan.ds}

\Prototype
\begin{quote}
N\_SI4 {\bf nusdas\_scan\_ds}(char {\it type1}[8], char {\it type2}[4], char {\it type3}[4], N\_SI4 $\ast${\it nrd});
\end{quote}

\begin{tabular}{l|rp{20em}}
\hline
\ArgName & \ArgType & \ArgRole \\
\hline
{\it type1} & char [8] &  種別1  \\
{\it type2} & char [4] &  種別2  \\
{\it type3} & char [4] &  種別3  \\
{\it nrd} & N\_SI4 $\ast$ &  NRD番号  \\
\hline
\end{tabular}
\paragraph{\FuncDesc}
返却値が負になるまで呼出しを繰り返すと、ライブラリが認識している
データセットの一覧が得られる。

\paragraph{\ResultCode}
\begin{quote}
\begin{description}
\item[{\bf 0}] 引数の配列にデータセットの情報が格納された。
\item[{\bf -1}] もうこれ以上データセットは認識されていない。
\end{description}\end{quote}
\paragraph{履歴}
この関数は NuSDaS 1.3 で追加された。
pnusdas には非公開の nusdas\_list\_type という関数があり類似の機能を持つ。
