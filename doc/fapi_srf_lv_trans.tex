\subsection{SRF\_LV\_TRANS: レベル値を実数データ (代表値) に変換}
\APILabel{srf.lv.trans}

\Prototype
\begin{quote}
CALL {\bf SRF\_LV\_TRANS}({\it idat}, {\it fdat}, {\it dnum}, {\it ispec}, {\it result})
\end{quote}

\begin{tabular}{l|rllp{16em}}
\hline
\ArgName & \ArgType & \ArrayDim & I/O & \ArgRole \\
\hline
{\it idat} & INTEGER(4) & \AnySize & IN &  入力データ  \\
{\it fdat} & REAL(4) & \AnySize & OUT &  結果格納配列  \\
{\it dnum} & INTEGER(4) &  & IN &  データ要素数  \\
{\it ispec} & INTEGER(4) & \AnySize & IN &  ISPEC 配列  \\
{\it result} & INTEGER(4) &  & OUT & \ResultCode \\
\hline
\end{tabular}
\paragraph{\FuncDesc}
新 ISPEC 配列 {\it ispec} にしたがって
配列 {\it idat} のレベル値を代表値 {\it fdat} に変換する。

\paragraph{返却値}
不明値以外となったデータの要素数

\paragraph{注}
\begin{itemize}
\item 
不明値は -1 となる。
ただし、ISPEC のデータ種別 (先頭4バイト) が
SRR2, SRF2, SRRR, SRFR の場合に限り -9999.0 となる。
\item 
ISPEC のレベル表は通常 0.1mm 単位と解釈される。
ただし、ISPEC の先頭 3 バイトが `{\tt IER}' であるか、
あるいは ISPEC の先頭から 4 バイト目が `{\tt 1}' のときは
0.01mm 単位と解釈される。
\end{itemize}
\paragraph{履歴}
この関数は NAPS7 時代から存在した。
