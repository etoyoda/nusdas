\subsection{nusdas\_cut: 領域限定のデータ読取}
\APILabel{nusdas.cut}

\Prototype
\begin{quote}
N\_SI4 {\bf nusdas\_cut}(const char {\it type1}[8], const char {\it type2}[4], const char {\it type3}[4], const N\_SI4 $\ast${\it basetime}, const char {\it member}[4], const N\_SI4 $\ast${\it validtime}, const char {\it plane}[6], const char {\it element}[6], void $\ast${\it udata}, const char {\it utype}[2], const N\_SI4 $\ast${\it usize}, const N\_SI4 $\ast${\it ixstart}, const N\_SI4 $\ast${\it ixfinal}, const N\_SI4 $\ast${\it iystart}, const N\_SI4 $\ast${\it iyfinal});
\end{quote}

\begin{tabular}{l|rp{20em}}
\hline
\ArgName & \ArgType & \ArgRole \\
\hline
{\it type1} & const char [8] &  種別1  \\
{\it type2} & const char [4] &  種別2  \\
{\it type3} & const char [4] &  種別3  \\
{\it basetime} & const N\_SI4 $\ast$ &  基準時刻(通算分)  \\
{\it member} & const char [4] &  メンバー名  \\
{\it validtime} & const N\_SI4 $\ast$ &  対象時刻(通算分)  \\
{\it plane} & const char [6] &  面  \\
{\it element} & const char [6] &  要素名  \\
{\it udata} & void $\ast$ &  データ格納先配列  \\
{\it utype} & const char [2] &  データ格納先配列の型  \\
{\it usize} & const N\_SI4 $\ast$ &  データ格納先配列の要素数  \\
{\it ixstart} & const N\_SI4 $\ast$ &  $x$ 方向格子番号下限  \\
{\it ixfinal} & const N\_SI4 $\ast$ &  $x$ 方向格子番号上限  \\
{\it iystart} & const N\_SI4 $\ast$ &  $y$ 方向格子番号下限  \\
{\it iyfinal} & const N\_SI4 $\ast$ &  $y$ 方向格子番号上限  \\
\hline
\end{tabular}
\paragraph{\FuncDesc}
\APILink{nusdas.read}{nusdas\_read} $\ast$ と同様だが、データレコードのうち格子点
({\it ixstart} , {\it iystart} )--({\it ixfinal} , {\it iyfinal} )
だけが {\it udata} に格納される。

格子番号は 1 から始まるものとするため、
{\it ixstart} や {\it iystart} は正でなければならず、また
{\it ixfinal} や {\it iyfinal} はそれぞれ
{\it ixstart} や {\it iystart} 以上でなければならない。
この規則に反する指定を行った場合は、返却値-8のエラーとなる。
なお、{\it iyfinal}, {\it jyfinal} の上限が格子数を超えていることの
チェックはしていないので注意が必要。

utypeは\ref{tab:typename}のものを指定するが、これらの値はN\_を接頭辞につけた定数で参照できる。
例えば'R4'であれば定数N\_R4で参照できる。

\paragraph{\ResultCode}
\begin{quote}
\begin{description}
\item[{\bf 正}] 読み出して格納した格子数
\item[{\bf 0}] 指定したデータは未記録(定義ファイルの elementmap によって書き込まれることは許容されているが、まだデータが書き込まれていない)
\item[{\bf -2}] 指定したデータは記録することが許容されていない(elementmap によって禁止されている場合と指定した面名、要素名が登録されていない場合の両方を含む)。
\item[{\bf -4}] 格納配列が不足
\item[{\bf -5}] 格納配列の型とレコードの記録形式が不整合
\item[{\bf -8}] 領域指定パラメータが不正
\end{description}\end{quote}

\paragraph{履歴}
本関数は NuSDaS 1.1 で導入され、 NuSDaS 1.3 で初めてドキュメントされた。
\paragraph{互換性}
NuSDaS 1.1 では、ローカルのデータファイルに対しては、
{\it ixstart} $\le$ 0 の場合は {\it ixstart} = 1 に({\it jystart} も同様), 
{\it ixfinal} がX方向の格子数を超える場合には、{\it ixfinal} はX方向の格子数に
({\it jyfinal} も同様)に読み替えられていたが、 NuSDaS 1.3 からは返却値-8のエラー
とする。また、pandora データについては、{\it ixstart}, {\it ixfinal}, 
{\it jystart}, {\it jyfinal} が非負であることだけがチェックされていた。
NuSDaS 1.3 からはデータファイル、pandora とも上述の通りとなる。
