\subsection{NUSDAS\_INQ\_CNTL: データファイルの諸元問合せ }
\APILabel{nusdas.inq.cntl}

\Prototype
\begin{quote}
CALL {\bf NUSDAS\_INQ\_CNTL}({\it type1}, {\it type2}, {\it type3}, {\it basetime}, {\it member}, {\it validtime}, {\it param}, {\it data}, {\it datasize}, {\it result})
\end{quote}

\begin{tabular}{l|rllp{16em}}
\hline
\ArgName & \ArgType & \ArrayDim & I/O & \ArgRole \\
\hline
{\it type1} & CHARACTER(8) &  & IN &  種別1  \\
{\it type2} & CHARACTER(4) &  & IN &  種別2  \\
{\it type3} & CHARACTER(4) &  & IN &  種別3  \\
{\it basetime} & INTEGER(4) &  & IN &  基準時刻(通算分)  \\
{\it member} & CHARACTER(4) &  & IN &  メンバー名  \\
{\it validtime} & INTEGER(4) &  & IN &  対象時刻(通算分)  \\
{\it param} & INTEGER(4) &  & IN &  問合せ項目コード  \\
{\it data} & \AnyType & \AnySize & OUT &  問合せ結果配列  \\
{\it datasize} & INTEGER(4) &  & IN &  問合せ結果配列の要素数  \\
{\it result} & INTEGER(4) &  & OUT & \ResultCode \\
\hline
\end{tabular}
\paragraph{\FuncDesc}引数 {\it type1} から {\it validtime} で指定されるデータファイルに書かれた
CNTL 記録について、
引数 {\it param} で指定される問合せを行う。
\begin{quote}\begin{description}
\item[{\bf N\_MEMBER\_NUM}] 
メンバーの個数が4バイト整数型変数 {\it data} に書かれる。
\item[{\bf N\_MEMBER\_LIST}] 
データファイルに定義されたメンバー名が配列 {\it data} に書かれる。
配列 {\it data} は長さ 4 文字の文字型で
{\it N\_MEMBER\_NUM} 要素存在しなければならない。
\item[{\bf N\_VALIDTIME\_NUM}] 
validtimeの個数が4バイト整数型変数 {\it data} に書かれる。
\item[{\bf N\_VALIDTIME\_LIST}] 
データファイルに定義されたvalidtimeが配列 {\it data} に書かれる。
配列 {\it data} は長さ 4 byte整数型で
{\it N\_VALIDTIME\_NUM} 要素存在しなければならない。
\item[{\bf N\_VALIDTIME\_LIST2}] 
データファイルに定義されたvalidtime2が配列 {\it data} に書かれる。
配列 {\it data} は長さ 4 byte整数型で
{\it N\_VALIDTIME\_NUM} 要素存在しなければならない。
\item[{\bf N\_PLANE\_NUM}] 
面の個数が4バイト整数型変数 {\it data} に書かれる。
\item[{\bf N\_PLANE\_LIST}] 
データファイルに定義された面の名前が配列 {\it data} に書かれる。
配列 {\it data} は長さ 6 文字の文字型で
{\it N\_PLANE\_NUM} 要素存在しなければならない。
\item[{\bf N\_PLANE\_LIST2}] 
N\_PLANE\_LIST と全く同じ動作である。
\item[{\bf N\_ELEMENT\_NUM}] 
要素の個数が4バイト整数型変数 {\it data} に書かれる。
\item[{\bf N\_ELEMENT\_LIST}] 
データファイルに定義された要素の名前が配列 {\it data} に書かれる。
配列 {\it data} は長さ 6 文字の文字型で
{\it N\_ELEMENT\_NUM} 要素存在しなければならない。
\item[{\bf  N\_NUSD\_NBYTES }] 
NUSD レコードのサイズ(単位バイト)が4バイト整数型変数 {\it data} に書か
れる。(先頭・末尾に付加されるレコード長の大きさ(4$\ast$2バイト)を含む)
\item[{\bf  N\_NUSD\_CONTENT }] 
NUSD レコードの内容を配列 {\it data} に格納する。配列 {\it data} は
\newline N\_NUSD\_NBYTES バイト存在しなくてはならない。
(先頭・末尾に付加されるレコード長を含む)
\item[{\bf  N\_CNTL\_NBYTES }] 
CNTL レコードのサイズ(単位バイト)が4バイト整数型変数 {\it data} に書か
れる。(先頭・末尾に付加されるレコード長の大きさ(4$\ast$2バイト)を含む)
\item[{\bf  N\_CNTL\_CONTENT }] 
CNTL レコードの内容を配列 {\it data} に格納する。配列 {\it data} は
\newline N\_CNTL\_NBYTES バイト存在しなくてはならない。
(先頭・末尾に付加されるレコード長を含む)
\item[{\bf  N\_PROJECTION }] 
地図投影法の情報を4文字の文字型 {\it data} に格納する
(記号の意味は巻末の表参照)。
\item[{\bf  N\_GRID\_SIZE }] 
X方向、Y方向の格子数がこの順序で4バイト整数型の配列 {\it data} に
書かれる。配列 {\it data} は 2 要素存在しなくてはならない。
(この問い合わせは NuSDaS 1.3 で追加)
\item[{\bf  N\_GRID\_BASEPOINT }] 
基準点のx座標、y座標、緯度、経度がこの順序で4バイト単精度浮動小数点型の配
列 {\it data} に書かれる。配列 {\it data} は 4 要素存在しなくてはならない。
(この問い合わせは NuSDaS 1.3 で追加)
\item[{\bf  N\_GRID\_DISTANCE }] 
X方向、Y方向の格子間隔がこの順序で4バイト単精度浮動小数点型の配列
{\it data} に書かれる。配列 {\it data} は 2 要素存在しなくてはならない。
(この問い合わせは NuSDaS 1.3 で追加)
\item[{\bf  N\_STAND\_LATLON }] 
標準緯度、標準経度、第2標準緯度、第2標準経度がこの順序で
4バイト単精度浮動小数点型の配列 {\it data} に書かれる。
配列 {\it data} は 4 要素存在しなくてはならない。
(この問い合わせは NuSDaS 1.3 で追加)
\item[{\bf  N\_SPARE\_LATLON }] 
緯度1、経度1、緯度2、経度2がこの順序で
4バイト単精度浮動小数点型の配列 {\it data} に書かれる。
配列 {\it data} は 4 要素存在しなくてはならない。
(この問い合わせは NuSDaS 1.3 で追加)
\item[{\bf  N\_INDX\_SIZE }] 
INDX の個数が 4バイト整数型の変数 {\it data} に書かれる。
(この問い合わせは NuSDaS 1.3 で追加)
\item[{\bf  N\_ELEMENT\_MAP }] 
データの格納が許容されているか否かが1 or 0 によって、1バイト整数型
の配列 {\it data} に書かれる。配列 {\it data} は {\it N\_INDX\_SIZE} 要素存在
しなくてはならない。{\it dataはメンバー、validtime}, 面、要素をインデック
スにした配列で、それぞれの順序は {\it N\_MEMBER\_LIST}, 
{\it N\_VALIDTIME\_LIST}, {\it N\_PLANE\_LIST}, {\it N\_ELEMENT\_LIST}の問い合わせ
結果と一致する。
(この問い合わせは NuSDaS 1.3 で追加)
\item[{\bf  N\_DATA\_MAP }] 
データが書き込まれているか否かが1 or 0 によって、1バイト整数型
の配列 {\it data} に書かれる。配列 {\it data} は {\it N\_INDX\_SIZE} 要素存在
しなくてはならない。{\it dataはメンバー、validtime}, 面、要素をインデック
スにした配列で、それぞれの順序は {\it N\_MEMBER\_LIST}, 
{\it N\_VALIDTIME\_LIST}, {\it N\_PLANE\_LIST}, {\it N\_ELEMENT\_LIST}の問い合わせ
結果と一致する。
(この問い合わせは NuSDaS 1.3 で追加)
\end{description}\end{quote}

\paragraph{\ResultCode}
\begin{quote}
\begin{description}
\item[{\bf 正}] 格納要素数
\item[{\bf -1}] データの配列数が不足している。
\item[{\bf -2}] データの配列が確保されていない。
\item[{\bf -3}] 問い合わせ項目が不正
\end{description}\end{quote}
\paragraph{ 注意 }
NuSDaS1.1以前では、同じ構造のデータセットでも
N\_VALIDTIME\_NUM, N\_VALIDTIME\_LIST の問い合わせ結果が
1つの basetime に複数の validtime を格納するか否かによって異なっていた。
これは、validtime でファイルを分ける
(異なる validtime のファイルが異なる) 設定ならば
データファイルには 1 つの validtime だけが書かれていたからである。
しかし NuSDaS 1.3 からは定義ファイルに指定されたすべての validtime が
各データファイルの validtime に格納されているので、問い合わせ結果は
格納形態を問わず一定である。
