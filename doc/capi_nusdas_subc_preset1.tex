\subsection{nusdas\_subc\_preset1: SUBC ETA/SIGM のデフォルト値設定 }
\APILabel{nusdas.subc.preset1}

\Prototype
\begin{quote}
N\_SI4 {\bf nusdas\_subc\_preset1}(const char {\it type1}[8], const char {\it type2}[4], const char {\it type3}[4], const char {\it group}[4], const N\_SI4 $\ast${\it n\_levels}, float {\it a}[\,], float {\it b}[\,], float $\ast${\it c});
\end{quote}

\begin{tabular}{l|rp{20em}}
\hline
\ArgName & \ArgType & \ArgRole \\
\hline
{\it type1} & const char [8] &  種別1  \\
{\it type2} & const char [4] &  種別2  \\
{\it type3} & const char [4] &  種別3  \\
{\it group} & const char [4] &  群名  \\
{\it n\_levels} & const N\_SI4 $\ast$ &  鉛直層数  \\
{\it a} & float [\,] &  係数 a  \\
{\it b} & float [\,] &  係数 b  \\
{\it c} & float $\ast$ &  係数 c  \\
\hline
\end{tabular}
\paragraph{\FuncDesc}ファイルが新たに生成される際にETA, SIGMに書き込む値を設定する。
SIGM や引数については nusdas\_subc\_eta を参照。
引数の「群名」には、"ETA " または "SIGM" を指定する。
\paragraph{\ResultCode}
\begin{quote}
\begin{description}
\item[{\bf 0}] 正常終了
\item[{\bf -1}] 定義ファイルに指定した群名が登録されていない
\item[{\bf -2}] メモリの確保に失敗した
\item[{\bf -3}] レコードのサイズが不正
\end{description}\end{quote}

\paragraph{ 互換性 }
NuSDaS1.1 では、一つのNuSDaSデータセットに設定できる補助管理部の数は最大
10 に制限されており、それを超えると-2が返された。一方、 NuSDaS 1.3 からは
メモリが確保できる限り数に制限はなく、-2 をメモリ確保失敗のエラーコードに
読み替えている。
