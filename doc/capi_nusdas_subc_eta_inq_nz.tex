\subsection{nusdas\_subc\_eta\_inq\_nz: SUBC 記録の鉛直層数問合せ}
\APILabel{nusdas.subc.eta.inq.nz}

\Prototype
\begin{quote}
N\_SI4 {\bf nusdas\_subc\_eta\_inq\_nz}(const char {\it type1}[8], const char {\it type2}[4], const char {\it type3}[4], const N\_SI4 $\ast${\it basetime}, const char {\it member}[4], const N\_SI4 $\ast${\it validtime}, const char {\it group}[4], N\_SI4 $\ast${\it n\_levels});
\end{quote}

\begin{tabular}{l|rp{20em}}
\hline
\ArgName & \ArgType & \ArgRole \\
\hline
{\it type1} & const char [8] &  種別1  \\
{\it type2} & const char [4] &  種別2  \\
{\it type3} & const char [4] &  種別3  \\
{\it basetime} & const N\_SI4 $\ast$ &  基準時刻(通算分)  \\
{\it member} & const char [4] &  メンバー名  \\
{\it validtime} & const N\_SI4 $\ast$ &  対象時刻(通算分)  \\
{\it group} & const char [4] &  群名  \\
{\it n\_levels} & N\_SI4 $\ast$ &  鉛直層数  \\
\hline
\end{tabular}
\paragraph{\FuncDesc}SUBC レコードの ETA, SIGM, ZHYB に記録された鉛直層数を問い合わせる。
群名には "ETA ", "SIGM", "ZHYB" のいずれかを指定する。
\paragraph{\ResultCode}
\begin{quote}
\begin{description}
\item[{\bf 正}] 正常終了
\end{description}\end{quote}
\paragraph{ 履歴 }
この関数は NuSDaS1.2 で導入された。
