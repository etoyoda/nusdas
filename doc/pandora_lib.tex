\section{pandora client のためのTCP/IP通信ライブラリ: pandora\_lib}
\label{pandora_lib}
pandora サーバーとのTCP/IP通信を支援することを念頭に構成されたライブラリ
である。pandora サーバーとの通信は HTTP プロトコルで行われるが、このライ
ブラリは対pandora サーバーと意識しているものの、汎用のHTTPプロトコルによ
る通信のためのライブラリとしても利用できる。

C言語で記述されており、C言語での利用が前提になっている。

\subsection{ソースファイルの構成}
{\tt pandora\_lib.c} およびヘッダファイル {\tt pandora\_lib.h} 
から構成されている。

pandora\_lib の関数を使用するソースファイルには、
{\tt pandora\_lib.h}をincludeする。

\subsection{関数リファレンス}
\input pandora_lib_doc.tex

\subsection{使用例}
\subsubsection{pandora server からNuSDaSデータを取得するクライアントの例}
pandora server({\tt localhost:8080}) から引数で指定した日時の
全球速報解析 \\*
(NuSDaS TYPE名: {\tt \_GSMLLPP.EASV.STD1})の地表面データを取得する
サンプルである。

使用例:
\begin{verbatim}
$ ./panlib_samlple1 2017/08/01/00:00
\end{verbatim}

\begin{verbatim}
#include <stdio.h>
#include <stdlib.h>
#include <string.h>
#include "pandora_lib.h"

#define TYPE1 "_GSMLLPP"
#define TYPE2 "EASV"
#define TYPE3 "STD1"
#define MEMBER "none"
#define LAYER "SURF"
#define SERVER "localhost:8080"
#define ROOT "/NAPS/db2/eternal/Ea/Anl/anl_p.nus"

int main(int argc, char* argv[])
{
    pandora_data *pdr;
    char elem[3][7]={"T","U","V"};
    int i;
    int code, len;
    FILE *fp;
    char filename[256];
    char nus_path[256];
    int year, month, day, hour, min;
    int data_num;
    char *data_num_str;
    
    if(argc <2){
        fprintf(stderr,"usage: %s yyyy/mm/dd/hh:mm\n",argv[0]);
        return -1;
    }
    if(sscanf(argv[1],"%d/%d/%d/%d:%d",&year, &month, &day, &hour, &min)!=5){
        fprintf(stderr,"Time Format Error\n");
        return -2;
    }
    
    pdr = pdr_new();
    /* オブジェクトを作成 */
    
    pdr_set_server(pdr,SERVER);
    /* アクセスするサーバー名をセット*/
    
    pdr_set_root(pdr, ROOT);
    /* 振り分け先をセット */
    
    pdr_set_resource_type(pdr,"data.txt");
    /* 資源の種類と形式をセット */
    
    for(i=0;i<3;i++){
        sprintf(nus_path,
            "%s.%s.%s/%04d-%02d-%02dt%02d%02d/%s/%04d-%02d-%02dt%02d%02d//%s//%s",
            TYPE1, TYPE2, TYPE3,
            year, month, day, hour, min, MEMBER,
            year, month, day, hour, min, LAYER, elem[i]
        );
        /* アクセスパスを作成する validtime2, plane2の値は不要だがスラッシュは必要 */
        pdr_set_path(pdr, nus_path);
        if(pdr_process(pdr) < 0){ 
            /* リクエストの送信,レスポンスの受信*/
            code = pdr_get_status_code(pdr); /*エラーコードの取得*/
            fprintf(stderr, "%d: Request failed. code=%d\n",i, code);
        }
        len = pdr_get_data_len(pdr); /* レスポンスの長さの取得 */
        if((data_num_str = pdr_header_find(pdr,"X-Data-Num"))!=NULL){
            data_num = atoi(data_num_str);
        }
        else{
            data_num = 0;
        }
        fprintf(stderr,"%d: Data received: %d byte\n",i, len);
        fprintf(stderr,"Data-Num: %d\n", data_num);
        sprintf(filename, "data-%04d%02d%02d%02d%02d-%d.txt", 
                year, month, day, hour, min, i);
        fp = fopen(filename,"w");
        fwrite(pdr_get_data(pdr),len,1,fp);
        /* pdr_get_data(pdr)から始まるメモリー上のデータを長さlen
           (つまり全部)出力 */
        
        fclose(fp);
        
    }
    pdr_delete(pdr);
    
    return 0;
}
\end{verbatim}

\subsubsection{汎用HTTPクライアントの例}
引数に与えたURLの内容をHTTPによって取得する。

\begin{verbatim}
panlib_sample2 http://<server>:<port>/<path>
\end{verbatim}

\begin{verbatim}
/* -------------------------------------------------------------------------*/
#include <stdio.h>
#include <stdlib.h>
#include <string.h>

#include "pandora_lib.h"
#define DEFAULT_TIMEOUT 60

/*-------------------------------------------------------------------------*/
void usage(char *argv[]){
    fprintf(stderr,"usage: %s [option] http://<server>:<port>/<path>\n",argv[0]);
    fprintf(stderr, "        option: -h   print all headers to stderr\n");
}
/*------------------------------------------------------------------------*/
int main(int argc, char *argv[])
{
    char *server_path, *server, *path, *host_header;
    pandora_data *pdr;
    char *p;
    int code;
    int i, len;
    int print_header_flag;
    int timeout;
    int rt;

    if(argc < 2){
        usage(argv);
        return -1;
    }

    print_header_flag = 0;
    server_path = NULL;
    timeout = DEFAULT_TIMEOUT;
    for(i=1;argv[i]!=NULL;i++){
        if(argv[i][0]=='-'){
            switch(argv[i][1]){
            case 'h':
                print_header_flag = 1;
                break;
            case 't':
                timeout = atoi(argv[++i]);
                if(timeout == 0){
                    fprintf(stderr,"Invalid Timeout Parameter:%s\n",
                            argv[i]);
                }
                break;
            case 'j':
                host_header = argv[++i];
                break;
            default:
                break;
            }
            continue;
        }
        
        if(server_path == NULL){
            server_path = argv[i];
        }
    }

    if(server_path == NULL){
        usage(argv);
        return -2;
    }

    /* serverとpathの分解 */
    if((p=strstr(server_path, "http://"))==NULL){
        usage(argv);
        return -3;
    }
    server = p + strlen("http://");
    for(p = server; *p!='\0'; p++){
        if(*p=='/'){
            *p = '\0';
            path = p+1;
            break;
        }
    }

    /*
      pandora_lib では,root,, path, resource_typeを設定するのが原則で
      あるが,今回のようにパスが完全にわかっている場合には,以下のよう
      にpdr_set_resource_typeに完全なパスをセットすることも可能である.
      (rootとpathは設定しなくてよい)

      なお,このように3つにわかれているのでは,
      1) pathに nusdims_to_path の出力をそのままセットできる.
         (関数pdr_set_path_nus はNuSDaSの要素をpandora pathに変換して,
         pdr_set_pathを実行する)
      2) pathだけが違う(たとえば時刻がちがうなど)とか,同じ資源のdataとmeta
        をとりたい場合など,変更のある部分だけを変更できる

      という利点があると考えている.
     */

    pdr = pdr_new();
    if(pdr == NULL){
        fprintf(stderr, "malloc error:%s,%d\n",__FILE__,__LINE__);
        return -10;
    }
    
    if((rt = pdr_set_timeout(pdr, 60))<0){
        fprintf(stderr, "pdr_set_timeout error:%d\n", rt);
        return rt;
    }
    else if((rt = pdr_set_server(pdr,server))<0){
        fprintf(stderr, "pdr_set_server error:%d\n", rt);
        return rt;
    }
    else if((rt = pdr_set_resource_type(pdr, path))<0){
        fprintf(stderr, "pdr_set_resource error:%d\n", rt);
        return rt;
    }

    if(host_header != NULL){
        if((rt = pdr_set_host_header(pdr, host_header)) < 0){
            fprintf(stderr, "pdr_set_host_header error:%d\n", rt);
            return rt;
        }
    }

    if(pdr_process(pdr)<0){
        code = pdr_get_status_code(pdr);
        fprintf(stderr, "Error! status code=%d\n", code);

    }
    fprintf(stderr,"Data Length: %d\n", pdr_get_data_len(pdr));
    if(print_header_flag == 1){
        pdr_print_all_headers(pdr,stderr);
    }
    fwrite(pdr_get_data(pdr), pdr_get_data_len(pdr),1,stdout);    

    pdr_delete(pdr);
    return 0;
}
\end{verbatim}
