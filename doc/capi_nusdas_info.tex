\subsection{nusdas\_info: INFO 記録へのアクセス }
\APILabel{nusdas.info}

\Prototype
\begin{quote}
N\_SI4 {\bf nusdas\_info}(const char {\it type1}[8], const char {\it type2}[4], const char {\it type3}[4], const N\_SI4 $\ast${\it basetime}, const char {\it member}[4], const N\_SI4 $\ast${\it validtime}, const char {\it group}[4], char {\it info}[\,], const N\_SI4 $\ast${\it bytesize}, const char {\it getput}[3]);
\end{quote}

\begin{tabular}{l|rp{20em}}
\hline
\ArgName & \ArgType & \ArgRole \\
\hline
{\it type1} & const char [8] &  種別1  \\
{\it type2} & const char [4] &  種別2  \\
{\it type3} & const char [4] &  種別3  \\
{\it basetime} & const N\_SI4 $\ast$ &  基準時刻(通算分)  \\
{\it member} & const char [4] &  メンバー名  \\
{\it validtime} & const N\_SI4 $\ast$ &  対象時刻(通算分)  \\
{\it group} & const char [4] &  群名  \\
{\it info} & char [\,] &  INFO 記録内容  \\
{\it bytesize} & const N\_SI4 $\ast$ &  INFO 記録のバイト数  \\
{\it getput} & const char [3] &  入出力指示 ({\it "GET}" または {\it "PUT}")  \\
\hline
\end{tabular}
\paragraph{\FuncDesc}\paragraph{\ResultCode}
\begin{quote}
\begin{description}
\item[{\bf 非負}] 書き出したINFOのバイト数
\item[{\bf -3}] バッファが不足している
\item[{\bf -5}] 入出力指示が不正
\end{description}\end{quote}

\paragraph{ 注意 }
NuSDaS1.1では、バッファが不足している場合でもバッファの大きさの分だけを
書き込み、そのサイズを返していたが、 NuSDaS 1.3 からはこのような場合は-3が返る。
また、INFO のサイズは NuSDaS 1.3 で新設された nusdas\_inq\_subcinfo で
問い合わせ項目を N\_INFO\_NUM にすれば得ることができる。
