\subsection{srf\_rd\_rdic: レーダーサイト情報の問い合わせ}
\APILabel{srf.rd.rdic}

\Prototype
\begin{quote}
int {\bf srf\_rd\_rdic}(int {\it stnum}, int {\it iseq}, float $\ast${\it lat}, float $\ast${\it lon}, float $\ast${\it hh}, N\_SI4 $\ast${\it offx}, N\_SI4 $\ast${\it offy}, N\_SI4 $\ast${\it type1}, N\_SI4 $\ast${\it type2});
\end{quote}

\begin{tabular}{l|rp{20em}}
\hline
\ArgName & \ArgType & \ArgRole \\
\hline
{\it stnum} & int &  地点番号  \\
{\it iseq} & int &  日時(通算時)  \\
{\it lat} & float $\ast$ &  緯度  \\
{\it lon} & float $\ast$ &  経度  \\
{\it hh} & float $\ast$ &  高度  \\
{\it offx} & N\_SI4 $\ast$ &  中心のオフセット  \\
{\it offy} & N\_SI4 $\ast$ &  中心のオフセット  \\
{\it type1} & N\_SI4 $\ast$ &  デジタル化タイプ  \\
{\it type2} & N\_SI4 $\ast$ &  デジタル化タイプ  \\
\hline
\end{tabular}
\paragraph{\FuncDesc}
ファイル名 RADAR\_DIC のレーダー地点辞書から
日時 {\it iseq} (通算時であって通算分でないことに注意)
における地点番号 {\it stnum} のレーダーサイトの情報を読出す。

\paragraph{\ResultCode}
\begin{quote}
\begin{description}
\item[{\bf 1}] 正常終了
\item[{\bf 0}] 指定されたレーダーサイトがみつからなかった
\item[{\bf -1}] レーダー地点辞書が開けなかった
\end{description}\end{quote}

\paragraph{注}
\begin{itemize}
\item レーダー地点辞書は NAPS8 では
/grpK/nwp/Open/Const/Vsrf/Dcd/rdrdic.txt に置かれている。
\item NAPS8 初期版 libsrf.a には経度のかわりに誤って緯度を返すバグがある。
\end{itemize}
\paragraph{履歴}
この関数は NAPS7 時代からルーチン環境には存在したが、
pnusdas から NuSDaS 1.1 に至る CVS 版ソースには含まれていなかった。
NuSDaS 1.3 で両者が統合された。
