\subsection{NUSDAS\_CUT: 領域限定のデータ読取}
\APILabel{nusdas.cut}

\Prototype
\begin{quote}
CALL {\bf NUSDAS\_CUT}({\it type1}, {\it type2}, {\it type3}, {\it basetime}, {\it member}, {\it validtime}, {\it plane}, {\it element}, {\it udata}, {\it utype}, {\it usize}, {\it ixstart}, {\it ixfinal}, {\it iystart}, {\it iyfinal}, {\it result})
\end{quote}

\begin{tabular}{l|rllp{16em}}
\hline
\ArgName & \ArgType & \ArrayDim & I/O & \ArgRole \\
\hline
{\it type1} & CHARACTER(8) &  & IN &  種別1  \\
{\it type2} & CHARACTER(4) &  & IN &  種別2  \\
{\it type3} & CHARACTER(4) &  & IN &  種別3  \\
{\it basetime} & INTEGER(4) &  & IN &  基準時刻(通算分)  \\
{\it member} & CHARACTER(4) &  & IN &  メンバー名  \\
{\it validtime} & INTEGER(4) &  & IN &  対象時刻(通算分)  \\
{\it plane} & CHARACTER(6) &  & IN &  面  \\
{\it element} & CHARACTER(6) &  & IN &  要素名  \\
{\it udata} & \AnyType & \AnySize & OUT &  データ格納先配列  \\
{\it utype} & CHARACTER(2) &  & IN &  データ格納先配列の型  \\
{\it usize} & INTEGER(4) &  & IN &  データ格納先配列の要素数  \\
{\it ixstart} & INTEGER(4) &  & IN &  $x$ 方向格子番号下限  \\
{\it ixfinal} & INTEGER(4) &  & IN &  $x$ 方向格子番号上限  \\
{\it iystart} & INTEGER(4) &  & IN &  $y$ 方向格子番号下限  \\
{\it iyfinal} & INTEGER(4) &  & IN &  $y$ 方向格子番号上限  \\
{\it result} & INTEGER(4) &  & OUT & \ResultCode \\
\hline
\end{tabular}
\paragraph{\FuncDesc}
\APILink{nusdas.read}{nusdas\_read} $\ast$ と同様だが、データレコードのうち格子点
({\it ixstart} , {\it iystart} )--({\it ixfinal} , {\it iyfinal} )
だけが {\it udata} に格納される。

格子番号は 1 から始まるものとするため、
{\it ixstart} や {\it iystart} は正でなければならず、また
{\it ixfinal} や {\it iyfinal} はそれぞれ
{\it ixstart} や {\it iystart} 以上でなければならない。
この規則に反する指定を行った場合は、返却値-8のエラーとなる。
なお、{\it iyfinal}, {\it jyfinal} の上限が格子数を超えていることの
チェックはしていないので注意が必要。

utypeは\ref{tab:typename}のものを指定するが、これらの値はN\_を接頭辞につけた定数で参照できる。
例えば'R4'であれば定数N\_R4で参照できる。

\paragraph{\ResultCode}
\begin{quote}
\begin{description}
\item[{\bf 正}] 読み出して格納した格子数
\item[{\bf 0}] 指定したデータは未記録(定義ファイルの elementmap によって書き込まれることは許容されているが、まだデータが書き込まれていない)
\item[{\bf -2}] 指定したデータは記録することが許容されていない(elementmap によって禁止されている場合と指定した面名、要素名が登録されていない場合の両方を含む)。
\item[{\bf -4}] 格納配列が不足
\item[{\bf -5}] 格納配列の型とレコードの記録形式が不整合
\item[{\bf -8}] 領域指定パラメータが不正
\end{description}\end{quote}

\paragraph{履歴}
本関数は NuSDaS 1.1 で導入され、 NuSDaS 1.3 で初めてドキュメントされた。
\paragraph{互換性}
NuSDaS 1.1 では、ローカルのデータファイルに対しては、
{\it ixstart} $\le$ 0 の場合は {\it ixstart} = 1 に({\it jystart} も同様), 
{\it ixfinal} がX方向の格子数を超える場合には、{\it ixfinal} はX方向の格子数に
({\it jyfinal} も同様)に読み替えられていたが、 NuSDaS 1.3 からは返却値-8のエラー
とする。また、pandora データについては、{\it ixstart}, {\it ixfinal}, 
{\it jystart}, {\it jyfinal} が非負であることだけがチェックされていた。
NuSDaS 1.3 からはデータファイル、pandora とも上述の通りとなる。
