\subsection{n\_decode\_rlen\_nbit\_I1: RLE データを展開する}
\APILabel{n.decode.rlen.nbit.I1}

\Prototype
\begin{quote}
N\_SI4 {\bf n\_decode\_rlen\_nbit\_I1}(unsigned char {\it udata}[\,], const unsigned char {\it compressed\_data}[\,], N\_SI4 {\it compressed\_nbytes}, N\_SI4 {\it udata\_nelems}, N\_SI4 {\it maxvalue}, N\_SI4 {\it nbit});
\end{quote}

\begin{tabular}{l|rp{20em}}
\hline
\ArgName & \ArgType & \ArgRole \\
\hline
{\it udata} & unsigned char [\,] &  結果格納配列  \\
{\it compressed\_data} & const unsigned char [\,] &  圧縮データ  \\
{\it compressed\_nbytes} & N\_SI4 &  圧縮データのバイト数  \\
{\it udata\_nelems} & N\_SI4 &  圧縮データの要素数  \\
{\it maxvalue} & N\_SI4 &  データの最大値  \\
{\it nbit} & N\_SI4 &  圧縮データのビット数  \\
\hline
\end{tabular}
\paragraph{\FuncDesc}\paragraph{履歴}
この関数は NuSDaS 1.0 から存在するが、ドキュメントされていなかった。
NuSDaS 1.3 から Fortran API を伴う
サービスサブルーチンとして採録された。
