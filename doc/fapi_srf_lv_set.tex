\subsection{SRF\_LV\_SET: 実数からレベル値への変換}
\APILabel{srf.lv.set}

\Prototype
\begin{quote}
CALL {\bf SRF\_LV\_SET}({\it idat}, {\it fdat}, {\it dnum}, {\it ispec}, {\it param}, {\it result})
\end{quote}

\begin{tabular}{l|rllp{16em}}
\hline
\ArgName & \ArgType & \ArrayDim & I/O & \ArgRole \\
\hline
{\it idat} & INTEGER(4) & \AnySize & OUT &  レベル値格納配列  \\
{\it fdat} & REAL(4) & \AnySize & IN &  変換元データ配列  \\
{\it dnum} & INTEGER(4) &  & IN &  データ配列要素数  \\
{\it ispec} & INTEGER(4) & \AnySize & I/O &  新 ISPC  \\
{\it param} & CHARACTER($\ast$) & \AnySize & IN &  変換テーブル名  \\
{\it result} & INTEGER(4) &  & OUT & \ResultCode \\
\hline
\end{tabular}
\paragraph{\FuncDesc}
配列 {\it fdat} の実数データをレベル値 {\it idat} に変換し、
ISPC 配列 {\it ispec} にレベル代表値をセットする。
変換テーブルとしてファイル ./SRF\_LV\_TABLE/param.ltb を読む。
ここで {\it param} は変換テーブル名 (最長 4 字) である。

\paragraph{\ResultCode}
\begin{quote}
\begin{description}
\item[{\bf -1}] 変換テーブルを開くことができない
\item[{\bf -2}] 変換テーブルに 256 以上のレベルが指定されている
\item[{\bf 正}] 変換に成功した。返却値はレベル数
\end{description}\end{quote}

\paragraph{注}
\begin{itemize}
\item 
NAPS7 では変換テーブルとして
her ier prr pmf srr srf pr2
を用意していた。
NAPS8 では \newline 
/grpK/nwp/Open/Const/Vsrf/Comm/lvtbl.txd 以下に
her ie2 ier kor pft pm2 pmf pr2 prr sr1 sr2 sr3 srf srj srr yar yrr
が置かれている。
ルーチンジョブではこのディレクトリへシンボリックリンク SRF\_LV\_TABLE
を張って利用する。
\item 
変換テーブル名が ie2, kor, pft のとき,
ISPEC には変換テーブルに書かれた代表値の 1/10 が書かれる。
それ以外の場合は変換テーブルの代表値がそのまま書かれる。
\item 
変換テーブル名が sr2 または srj のときは実数データが -900.0 より小さい
ものが欠損値とみなされる。そうでなければ、負値が欠損値とみなされる
(NAPS7 のマニュアルでは欠損値は -1 を指定することとされている)。
\item 
変換テーブル名が srj のときは、実数データが変換テーブルの下限値に
正確に一致しないと最も上の階級 (具体的には 42 で 21.0を意味する)
に割り当てられる。この挙動はバグかもしれない。
\item 
変換テーブルに 191 行以上書かれているとき、
最初の 190 行だけが用いられ、レベル値は 0..190 となるが、
返却値には実際のレベル数 (変換テーブルの行数 + 1) が返される。
これは ispec の配列をオーバーフローしないためである。
\end{itemize}
\paragraph{履歴}
この関数は NAPS7 時代から存在した。
Fortran ラッパーが文字列の長さを伝えないバグは NuSDaS 1.3 で解決した。
