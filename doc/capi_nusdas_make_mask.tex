\subsection{nusdas\_make\_mask: マスクビット配列の作成}
\APILabel{nusdas.make.mask}

\Prototype
\begin{quote}
N\_SI4 {\bf nusdas\_make\_mask}(const void $\ast${\it udata}, const char {\it utype}[2], const N\_SI4 $\ast${\it usize}, void $\ast${\it output}, const N\_SI4 $\ast${\it mb\_bytes});
\end{quote}

\begin{tabular}{l|rp{20em}}
\hline
\ArgName & \ArgType & \ArgRole \\
\hline
{\it udata} & const void $\ast$ &  格子データ  \\
{\it utype} & const char [2] &  格子データの型  \\
{\it usize} & const N\_SI4 $\ast$ &  格子データの要素数  \\
{\it output} & void $\ast$ &  マスクビット配列  \\
{\it mb\_bytes} & const N\_SI4 $\ast$ &  マスクビット配列のバイト数  \\
\hline
\end{tabular}
\paragraph{\FuncDesc}
配列 {\it udata} の内容をチェックしてマスクビット列を作成し
{\it output} に書き込む。
引数 {\it utype} と欠損値は配列の型に応じて次のように指定する。
\begin{quote}\begin{description}
\item[{\bf 1バイト整数型}] 
引数 {\it utype} に N\_I1 を指定する。
配列中の欠損扱いしたい要素に N\_MV\_UI1 を設定しておく。
\item[{\bf 2バイト整数型}] 
引数 {\it utype} に N\_I2 を指定する。
配列中の欠損扱いしたい要素に N\_MV\_SI2 を設定しておく。
\item[{\bf 4バイト整数型}] 
引数 {\it utype} に N\_I4 を指定する。
配列中の欠損扱いしたい要素に N\_MV\_SI4 を設定しておく。
\item[{\bf 4バイト実数型}] 
引数 {\it utype} に N\_R4 を指定する。
配列中の欠損扱いしたい要素に N\_MV\_R4 を設定しておく。
\item[{\bf 8バイト実数型}] 
引数 {\it utype} に N\_R8 を指定する。
配列中の欠損扱いしたい要素に N\_MV\_R8 を設定しておく。
\end{description}\end{quote}

\paragraph{\ResultCode}
\begin{quote}
\begin{description}
\item[{\bf 0}] 正常終了
\item[{\bf -1}] 配列長 {\it mb\_bytes} が不足している
\item[{\bf -5}] 未知の型名 {\it utype} が与えられた
\end{description}\end{quote}

\paragraph{サイズ要件}
{\it mb\_bytes} は少なくとも ({\it usize} + 7) / 8 バイト以上必要である。

\paragraph{履歴}
\APILink{nusdas.make.mask}{nusdas\_make\_mask} は NuSDaS 1.0 から存在する。
