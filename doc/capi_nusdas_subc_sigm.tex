\subsection{nusdas\_subc\_sigm: SUBC SIGM へのアクセス}
\APILabel{nusdas.subc.sigm}

\Prototype
\begin{quote}
N\_SI4 {\bf nusdas\_subc\_sigm}(const char {\it type1}[8], const char {\it type2}[4], const char {\it type3}[4], const N\_SI4 $\ast${\it basetime}, const char {\it member}[4], const N\_SI4 $\ast${\it validtime}, N\_SI4 $\ast${\it n\_levels}, float {\it a}[\,], float {\it b}[\,], float $\ast${\it c}, const char {\it getput}[3]);
\end{quote}

\begin{tabular}{l|rp{20em}}
\hline
\ArgName & \ArgType & \ArgRole \\
\hline
{\it type1} & const char [8] &  種別1  \\
{\it type2} & const char [4] &  種別2  \\
{\it type3} & const char [4] &  種別3  \\
{\it basetime} & const N\_SI4 $\ast$ &  基準時刻(通算分)  \\
{\it member} & const char [4] &  メンバー名  \\
{\it validtime} & const N\_SI4 $\ast$ &  対象時刻(通算分)  \\
{\it n\_levels} & N\_SI4 $\ast$ &  鉛直層数  \\
{\it a} & float [\,] &  係数 a  \\
{\it b} & float [\,] &  係数 b  \\
{\it c} & float $\ast$ &  係数 c  \\
{\it getput} & const char [3] &  入出力指示 ({\it "GET}" または {\it "PUT}")  \\
\hline
\end{tabular}
\paragraph{\FuncDesc}鉛直座標に ETA 座標系を用いるときに、鉛直座標を定めるパラメータへの
アクセスを提供する。 
関数の仕様は、nusdas\_subc\_eta と同じである。
