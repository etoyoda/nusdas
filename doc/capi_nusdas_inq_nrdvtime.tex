\subsection{nusdas\_inq\_nrdvtime: データセットの対象時刻リスト取得}
\APILabel{nusdas.inq.nrdvtime}

\Prototype
\begin{quote}
N\_SI4 {\bf nusdas\_inq\_nrdvtime}(const char {\it type1}[8], const char {\it type2}[4], const char {\it type3}[4], N\_SI4 $\ast${\it vtlist}, const N\_SI4 $\ast${\it vtlistsize}, const N\_SI4 $\ast${\it basetime}, N\_SI4 {\it pflag});
\end{quote}

\begin{tabular}{l|rp{20em}}
\hline
\ArgName & \ArgType & \ArgRole \\
\hline
{\it type1} & const char [8] &  種別1  \\
{\it type2} & const char [4] &  種別2  \\
{\it type3} & const char [4] &  種別3  \\
{\it vtlist} & N\_SI4 $\ast$ &  対象時刻が書かれる配列  \\
{\it vtlistsize} & const N\_SI4 $\ast$ &  配列の要素数  \\
{\it basetime} & const N\_SI4 $\ast$ &  基準時刻(通算分)  \\
{\it pflag} & N\_SI4 &  動作詳細印字フラグ  \\
\hline
\end{tabular}
\paragraph{\FuncDesc}
種別1〜種別3で指示されるデータセットに基準時刻 {\it basetime} のもとで
存在する対象時刻を配列 {\it vtlist} に書き込む。
引数 {\it pflag} に非零値を設定すると動作過程の情報を警告メッセージとして
印字するようになる。
\paragraph{\ResultCode}
\begin{quote}
\begin{description}
\item[{\bf 非負}] 対象時刻の個数
\end{description}\end{quote}
\paragraph{履歴}
本関数は NuSDaS 1.0 から存在したがドキュメントされていなかった。
\paragraph{注意}
\begin{itemize}
\item 配列長 {\it vtlistsize} より多くの対象時刻が存在する場合は、
配列長を越えて書き込むことはない。リターンコードと配列長を比較して、
リターンコードが大きかったらその数だけ配列を確保し直して
本関数を呼び直すことにより、すべてのリストを得ることができる。
\item 対象時刻の探索はファイルの有無または CNTL レコードによる。
リスト中の対象時刻についてデータレコードが書かれていない場合もありうる。
\item 基準時刻 {\it basetime} に -1 を指定すると、
基準時刻を問わない検索になる。
\item 検索にあたってメンバー名は問わない。
\item 
NuSDaS 1.1 までは見付かったデータセットがネットワークでなければ、
それについてだけ探索が行われた。
NuSDaS 1.3 からは、
指定した種別にマッチするすべてのデータセットについて探索が行われる。
\item 
種別に対応するデータセットが見つからない場合
(たとえば種別名を間違えた場合)、
返却値はゼロとなる。
データセットが存在して空の場合と異なり、
このとき ``Can not find NUSDAS root directory for selected type1-3''
``type1$<$...$>$ type2$<$...$>$ type3$<$...$>$ NRD=...''
というメッセージが標準エラー出力に表示される。
NRD= の後の数値が -1 でなければ、
NRD 番号を指定したために存在しているデータセットが見つからなくなっている
可能性がある。
\end{itemize}
