\subsection{nusdas\_inq\_nrdbtime: データセットの基準時刻リスト取得}
\APILabel{nusdas.inq.nrdbtime}

\Prototype
\begin{quote}
N\_SI4 {\bf nusdas\_inq\_nrdbtime}(const char {\it type1}[8], const char {\it type2}[4], const char {\it type3}[4], N\_SI4 $\ast${\it btlist}, const N\_SI4 $\ast${\it btlistsize}, N\_SI4 {\it pflag});
\end{quote}

\begin{tabular}{l|rp{20em}}
\hline
\ArgName & \ArgType & \ArgRole \\
\hline
{\it type1} & const char [8] &  種別1  \\
{\it type2} & const char [4] &  種別2  \\
{\it type3} & const char [4] &  種別3  \\
{\it btlist} & N\_SI4 $\ast$ &  基準時刻が格納される配列  \\
{\it btlistsize} & const N\_SI4 $\ast$ &  配列の要素数  \\
{\it pflag} & N\_SI4 &  動作過程印字フラグ  \\
\hline
\end{tabular}
\paragraph{\FuncDesc}
種別1〜種別3で指示されるデータセットに存在する基準時刻を
配列 {\it btlist} に書き込む。
引数 {\it pflag} に非零値を設定すると動作過程の情報を警告メッセージとして
印字するようになる。
\paragraph{\ResultCode}
\begin{quote}
\begin{description}
\item[{\bf 非負}] 基準時刻の個数
\item[{\bf -1}] ファイル I/O エラー
\item[{\bf -2}] ファイルに管理部が存在しない
\item[{\bf -3}] ファイルのレコード長が不正
\item[{\bf -4}] ファイルあるいはディレクトリのオープンに失敗
\end{description}\end{quote}
\paragraph{履歴}
本関数は NuSDaS 1.0 から存在した。
\paragraph{注意}
\begin{itemize}
\item 
配列長 {\it btlistsize} より多くの基準時刻が存在する場合は、
配列長を越えて書き込むことはない。リターンコードと配列長を比較して、
リターンコードが大きかったらその数だけ配列を確保し直して
本関数を呼び直すことにより、すべてのリストを得ることができる。
\item 
NuSDaS 1.1 までは見付かったデータセットがネットワークでなければ、
それについてだけ探索が行われた。
NuSDaS 1.3 からは、
指定した種別にマッチするすべてのデータセットについて探索が行われる。
\item 
種別に対応するデータセットが見つからない場合
(たとえば種別名を間違えた場合)、
返却値はゼロとなる。
データセットが存在して空の場合と異なり、
このとき ``Can not find NUSDAS root directory for selected type1-3''
``type1$<$...$>$ type2$<$...$>$ type3$<$...$>$ NRD=...''
というメッセージが標準エラー出力に表示される。
NRD= の後の数値が -1 でなければ、
NRD 番号を指定したために存在しているデータセットが見つからなくなっている
可能性がある。
\end{itemize}
