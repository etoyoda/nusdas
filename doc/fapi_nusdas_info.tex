\subsection{NUSDAS\_INFO: INFO 記録へのアクセス }
\APILabel{nusdas.info}

\Prototype
\begin{quote}
CALL {\bf NUSDAS\_INFO}({\it type1}, {\it type2}, {\it type3}, {\it basetime}, {\it member}, {\it validtime}, {\it group}, {\it info}, {\it bytesize}, {\it getput}, {\it result})
\end{quote}

\begin{tabular}{l|rllp{16em}}
\hline
\ArgName & \ArgType & \ArrayDim & I/O & \ArgRole \\
\hline
{\it type1} & CHARACTER(8) &  & IN &  種別1  \\
{\it type2} & CHARACTER(4) &  & IN &  種別2  \\
{\it type3} & CHARACTER(4) &  & IN &  種別3  \\
{\it basetime} & INTEGER(4) &  & IN &  基準時刻(通算分)  \\
{\it member} & CHARACTER(4) &  & IN &  メンバー名  \\
{\it validtime} & INTEGER(4) &  & IN &  対象時刻(通算分)  \\
{\it group} & CHARACTER(4) &  & IN &  群名  \\
{\it info} & CHARACTER & \AnySize & I/O &  INFO 記録内容  \\
{\it bytesize} & INTEGER(4) &  & IN &  INFO 記録のバイト数  \\
{\it getput} & CHARACTER(3) &  & IN &  入出力指示 ({\it "GET}" または {\it "PUT}")  \\
{\it result} & INTEGER(4) &  & OUT & \ResultCode \\
\hline
\end{tabular}
\paragraph{\FuncDesc}\paragraph{\ResultCode}
\begin{quote}
\begin{description}
\item[{\bf 非負}] 書き出したINFOのバイト数
\item[{\bf -3}] バッファが不足している
\item[{\bf -5}] 入出力指示が不正
\end{description}\end{quote}

\paragraph{ 注意 }
NuSDaS1.1では、バッファが不足している場合でもバッファの大きさの分だけを
書き込み、そのサイズを返していたが、 NuSDaS 1.3 からはこのような場合は-3が返る。
また、INFO のサイズは NuSDaS 1.3 で新設された nusdas\_inq\_subcinfo で
問い合わせ項目を N\_INFO\_NUM にすれば得ることができる。
