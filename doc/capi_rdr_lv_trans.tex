\subsection{rdr\_lv\_trans: レベル値から代表値への変換}
\APILabel{rdr.lv.trans}

\Prototype
\begin{quote}
int {\bf rdr\_lv\_trans}(N\_SI4 {\it idat}[\,], float {\it fdat}[\,], int {\it dnum}, const char $\ast${\it param});
\end{quote}

\begin{tabular}{l|rp{20em}}
\hline
\ArgName & \ArgType & \ArgRole \\
\hline
{\it idat} & N\_SI4 [\,] &  入力データ  \\
{\it fdat} & float [\,] &  結果格納配列  \\
{\it dnum} & int &  データ要素数  \\
{\it param} & const char $\ast$ &  テーブル名  \\
\hline
\end{tabular}
\paragraph{\FuncDesc}
配列 {\it idat} のレベル値を代表値 {\it fdat} に変換する。
変換テーブルとしてファイル ./SRF\_LV\_TABLE/param.ltb を読む。
ここで {\it param} は変換テーブル名 (最長 4 字) である。

\paragraph{\ResultCode}
\begin{quote}
\begin{description}
\item[{\bf -1}] 変換テーブルを開くことができない
\item[{\bf -2}] 変換テーブルに 256 以上のレベルが指定されている
\item[{\bf 非負}] 変換に成功。返却値は不明値以外となったデータの要素数
\end{description}\end{quote}

\paragraph{注}
\begin{itemize}
\item 不明値は -1 となる。
\item 
NAPS8 では変換テーブルとして
/grpK/nwp/Open/Const/Vsrf/Comm/lvtbl.txd 以下に
her ie2 ier kor pft pi10 pm2 pmf pr2 prr rr60
sr1 sr2 sr3 srf srj srr yar yrr
が置かれている。
ルーチンジョブではこのディレクトリへシンボリックリンク SRF\_LV\_TABLE
を張って利用する。
\item 
上記変換テーブルのうち、
pi10 と rr60 は1行にレベル値と代表値の2列が書かれており、
その他はレベル値、最小値、代表値の3列が書かれているが、
本サブルーチンはどちらにも対応している。
\end{itemize}
\paragraph{履歴}
この関数は NAPS7 時代には存在しなかったようである。
レーダー情報作成装置に関連して開発されたと考えられているが、
NuSDaS 1.3 以前にはきちんとメンテナンスされていなかった。
