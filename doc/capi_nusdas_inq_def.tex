\subsection{nusdas\_inq\_def: データセットの諸元問合せ }
\APILabel{nusdas.inq.def}

\Prototype
\begin{quote}
N\_SI4 {\bf nusdas\_inq\_def}(const char {\it type1}[8], const char {\it type2}[4], const char {\it type3}[4], const N\_SI4 {\it param}, void $\ast${\it data}, const N\_SI4 $\ast${\it datasize});
\end{quote}

\begin{tabular}{l|rp{20em}}
\hline
\ArgName & \ArgType & \ArgRole \\
\hline
{\it type1} & const char [8] &  種別1  \\
{\it type2} & const char [4] &  種別2  \\
{\it type3} & const char [4] &  種別3  \\
{\it param} & const N\_SI4 &  問合せ項目コード  \\
{\it data} & void $\ast$ &  結果格納配列  \\
{\it datasize} & const N\_SI4 $\ast$ &  結果格納配列の要素数  \\
\hline
\end{tabular}
\paragraph{\FuncDesc}引数 {\it type1} から {\it type3}で指定されるデータセットの定義ファイル
に書かれた内容について、引数 {\it param} で指定される問合せを行う。
\begin{quote}\begin{description}
\item[{\bf N\_MEMBER\_NUM}] 
定義ファイルに書かれたメンバーの個数が4バイト整数型変数 {\it data} に書かれる。
\item[{\bf N\_MEMBER\_LIST}] 
定義ファイルに書かれたメンバー名が配列 {\it data} に書かれる。
配列 {\it data} は長さ 4 文字の文字型で
{\it N\_MEMBER\_NUM} 要素存在しなければならない。
\item[{\bf N\_VALIDTIME\_NUM}] 
定義ファイルに書かれたvalidtimeの個数が4バイト整数型変数 
{\it data} に書かれる。
\item[{\bf N\_VALIDTIME\_LIST}] 
定義ファイルに書かれたvalidtimeが配列 {\it data} に書かれる。
配列 {\it data} は長さ 4 byte整数型で
{\it N\_VALIDTIME\_NUM} 要素存在しなければならない。
\item[{\bf N\_VALIDTIME\_LIST2}] 
定義ファイルに書かれた validtime2 が配列 {\it data} に書かれる。
\newline 配列 {\it data} は長さ 4 byte整数型で
{\it N\_VALIDTIME\_NUM} 要素存在しなければならない。
\item[{\bf N\_VALIDTIME\_UNIT}] 
定義ファイルに書かれた validtime の単位が4文字の文字型変数 {\it data} 
に書かれる。
\item[{\bf N\_PLANE\_NUM}] 
定義ファイルに書かれた面の個数が4バイト整数型変数 {\it data} に書かれる。
\item[{\bf N\_PLANE\_LIST}] 
定義ファイルに書かれた面の名前が配列 {\it data} に書かれる。
配列 {\it data} は長さ 6 文字の文字型で
{\it N\_PLANE\_NUM} 要素存在しなければならない。
\item[{\bf N\_PLANE\_LIST2}] 
定義ファイルに書かれた面2の名前が配列 {\it data} に書かれる。
配列 {\it data} は長さ 6 文字の文字型で
{\it N\_PLANE\_NUM} 要素存在しなければならない。
\item[{\bf N\_ELEMENT\_NUM}] 
定義ファイルに書かれた要素の個数が4バイト整数型変数 {\it data} に書かれる。
\item[{\bf N\_ELEMENT\_LIST}] 
定義ファイルに書かれた要素の名前が配列 {\it data} に書かれる。
配列 {\it data} は長さ 6 文字の文字型で
{\it N\_ELEMENT\_NUM} 要素存在しなければならない。
\item[{\bf  N\_PROJECTION }] 
定義ファイルに書かれた地図投影法の情報を4文字の文字型 {\it data} に格納する
(記号の意味は巻末の表参照)。
\item[{\bf  N\_GRID\_SIZE }] 
定義ファイルに書かれたX方向、Y方向の格子数がこの順序で4バイト整数型の
配列 {\it data} に書かれる。配列 {\it data} は 2 要素存在しなくてはならない。
\item[{\bf  N\_GRID\_BASEPOINT }] 
定義ファイルに書かれた基準点のx座標、y座標、緯度、経度が
この順序で4バイト単精度浮動小数点型の配列 {\it data} に書かれる。
配列 {\it data} は 4 要素存在しなくてはならない。
\item[{\bf  N\_GRID\_DISTANCE }] 
定義ファイルに書かれたX方向、Y方向の格子間隔がこの順序で
4バイト単精度浮動小数点型の配列{\it data} に書かれる。
配列 {\it data} は 2 要素存在しなくてはならない。
\item[{\bf  N\_STAND\_LATLON }] 
定義ファイルに書かれた標準緯度、標準経度、第2標準緯度、第2標準経度が
この順序で4バイト単精度浮動小数点型の配列 {\it data} に書かれる。
配列 {\it data} は 4 要素存在しなくてはならない。
\item[{\bf  N\_SPARE\_LATLON }] 
定義ファイルに書かれた緯度1、経度1、緯度2、経度2がこの順序で
4バイト単精度浮動小数点型の配列 {\it data} に書かれる。
配列 {\it data} は 4 要素存在しなくてはならない。
\item[{\bf  N\_INDX\_SIZE }] 
定義ファイルから算出されるINDX の個数が 4バイト整数型の変数 {\it data} 
に書かれる。(この問い合わせはNuSDaS1.3で追加)
\item[{\bf  N\_ELEMENT\_MAP }] 
定義ファイルでデータの格納が許容されているか否かが1 or 0 によって、
1バイト整数型の配列 {\it data} に書かれる。
配列 {\it data} は {\it N\_INDX\_SIZE} 要素存在
しなくてはならない。{\it dataはメンバー、validtime}, 面、要素をインデック
スにした配列で、それぞれの順序は {\it N\_MEMBER\_LIST}, 
{\it N\_VALIDTIME\_LIST}, {\it N\_PLANE\_LIST}, {\it N\_ELEMENT\_LIST}の問い合わせ
結果と一致する。
\item[{\bf N\_SUBC\_NUM}] 
定義ファイルに書かれた SUBC 記録の個数が4バイト整数型変数 {\it buf} に書かれる。
\item[{\bf N\_SUBC\_LIST}] 
定義ファイルに書かれた SUBC 記録の群名が配列 {\it buf} に書かれる。
配列 {\it buf} は長さ 4 文字の文字型で
{\it N\_SUBC\_NUM} 要素存在しなければならない。
\item[{\bf N\_INFO\_NUM}] 
定義ファイルに書かれた INFO 記録の個数が4バイト整数型変数 {\it buf} に書かれる。
\item[{\bf N\_INFO\_LIST}] 
定義ファイルに書かれた INFO 記録の群名が配列 {\it buf} に書かれる。
配列 {\it buf} は長さ 4 文字の文字型で
{\it N\_INFO\_NUM} 要素存在しなければならない。
\end{description}\end{quote}
\paragraph{\ResultCode}
\begin{quote}
\begin{description}
\item[{\bf 正}] 格納要素数
\item[{\bf -1}] 格納配列が不足
\item[{\bf -2}] 格納配列が確保されていない
\item[{\bf -3}] 問い合わせが不正
\end{description}\end{quote}
\paragraph{ 履歴 }
この関数は NuSDaS1.0 より実装されていたが、NuSDaS1.3 で N\_INDX\_SIZE の問い合わせ
機能が追加されている。
