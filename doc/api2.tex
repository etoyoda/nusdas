\Chapter{範囲指定型 API (廃止予定)}
\label{api2}

\section{概要}

NuSDaS 内部では、対象時刻および面は点ではなく
範囲として表現できるようになっています
(GRIB から変換ができるように配慮したのでしょう)。
対象時刻は整数 2 つで期間の始点と終点を表わし、
面は 6 文字の名前 2 つで上端と下端の高度を表わすことができます。
これらはそれぞれ「対象時刻1」「対象時刻2」「面1」「面2」と呼ばれます。

対象時刻2や面2が指定されない「範囲でないデータ」は、
対象時刻2が値 $1$%
\footnote{通算分 1 は欠損をあらわします。
1801 年 1 月 1 日 00:01Z と区別つきませんが、おそらく問題にはならないでしょう}
で面2と面1が同じものとして表現されています。

対象時刻または面名に範囲を用いるデータセットを読み書きするために、
本章で説明する API が用意されています。
これらは関数名の末尾に `2' がつくので区別されます。
しかし、実際には積算・平均の時間範囲は範囲指定ではなく、
SUBC TDIF レコード {}[\TabRef{table.fmt.subc.tdif}] によって表現されます。

対象時刻2や面2を用いた範囲指定は Pandora では利用できなくなりました。
本章で説明される関数・サブルーチンも、
将来のライブラリ再設計時には削除される予定ですので、
開発環境のアプリケーションに用例があった場合は
速やかに `2なし' API に移行するように努めてください。

\section{Fortran API}
\label{fapi2}

\renewcommand{\APILabel}[1]{\label{fapi:#1}}

\subsection{NUSDAS\_CUT2: 領域限定のデータ読取 }
\APILabel{nusdas.cut2}

\Prototype
\begin{quote}
CALL {\bf NUSDAS\_CUT2}({\it type1}, {\it type2}, {\it type3}, {\it basetime}, {\it member}, {\it validtime1}, {\it validtime2}, {\it plane1}, {\it plane2}, {\it element}, {\it udata}, {\it utype}, {\it usize}, {\it ixstart}, {\it ixfinal}, {\it iystart}, {\it iyfinal}, {\it result})
\end{quote}

\begin{tabular}{l|rllp{16em}}
\hline
\ArgName & \ArgType & \ArrayDim & I/O & \ArgRole \\
\hline
{\it type1} & CHARACTER(8) &  & IN &  種別1  \\
{\it type2} & CHARACTER(4) &  & IN &  種別2  \\
{\it type3} & CHARACTER(4) &  & IN &  種別3  \\
{\it basetime} & INTEGER(4) &  & IN &  基準時刻(通算分)  \\
{\it member} & CHARACTER(4) &  & IN &  メンバー名  \\
{\it validtime1} & INTEGER(4) &  & IN &  対象時刻1  \\
{\it validtime2} & INTEGER(4) &  & IN &  対象時刻2  \\
{\it plane1} & CHARACTER(6) &  & IN &  面1  \\
{\it plane2} & CHARACTER(6) &  & IN &  面2  \\
{\it element} & CHARACTER(6) &  & IN &  要素名  \\
{\it udata} & \AnyType & \AnySize & OUT &  データ格納配列  \\
{\it utype} & CHARACTER(2) &  & IN &  データ格納配列の型  \\
{\it usize} & INTEGER(4) &  & IN &  データ格納配列の要素数  \\
{\it ixstart} & INTEGER(4) &  & IN &  $x$ 方向格子番号下限  \\
{\it ixfinal} & INTEGER(4) &  & IN &  $x$ 方向格子番号上限  \\
{\it iystart} & INTEGER(4) &  & IN &  $y$ 方向格子番号下限  \\
{\it iyfinal} & INTEGER(4) &  & IN &  $y$ 方向格子番号上限  \\
{\it result} & INTEGER(4) &  & OUT & \ResultCode \\
\hline
\end{tabular}

\subsection{NUSDAS\_CUT2\_RAW: 領域限定の DATA 記録直接読取 }
\APILabel{nusdas.cut2.raw}

\Prototype
\begin{quote}
CALL {\bf NUSDAS\_CUT2\_RAW}({\it type1}, {\it type2}, {\it type3}, {\it basetime}, {\it member}, {\it validtime1}, {\it validtime2}, {\it plane1}, {\it plane2}, {\it element}, {\it udata}, {\it usize}, {\it ixstart}, {\it ixfinal}, {\it iystart}, {\it iyfinal}, {\it result})
\end{quote}

\begin{tabular}{l|rllp{16em}}
\hline
\ArgName & \ArgType & \ArrayDim & I/O & \ArgRole \\
\hline
{\it type1} & CHARACTER(8) &  & IN &  種別1  \\
{\it type2} & CHARACTER(4) &  & IN &  種別2  \\
{\it type3} & CHARACTER(4) &  & IN &  種別3  \\
{\it basetime} & INTEGER(4) &  & IN &  基準時刻(通算分)  \\
{\it member} & CHARACTER(4) &  & IN &  メンバー名  \\
{\it validtime1} & INTEGER(4) &  & IN &  対象時刻1(通算分)  \\
{\it validtime2} & INTEGER(4) &  & IN &  対象時刻2(通算分)  \\
{\it plane1} & CHARACTER(6) &  & IN &  面1  \\
{\it plane2} & CHARACTER(6) &  & IN &  面2  \\
{\it element} & CHARACTER(6) &  & IN &  要素名  \\
{\it udata} & \AnyType & \AnySize & OUT &  データ格納先配列  \\
{\it usize} & INTEGER(4) &  & IN &  データ格納先配列のバイト数  \\
{\it ixstart} & INTEGER(4) &  & IN &  $x$ 方向格子番号下限  \\
{\it ixfinal} & INTEGER(4) &  & IN &  $x$ 方向格子番号上限  \\
{\it iystart} & INTEGER(4) &  & IN &  $y$ 方向格子番号下限  \\
{\it iyfinal} & INTEGER(4) &  & IN &  $y$ 方向格子番号上限  \\
{\it result} & INTEGER(4) &  & OUT & \ResultCode \\
\hline
\end{tabular}
\paragraph{\FuncDesc}nusdas\_cut\_raw の範囲指定版。nusdas\_cut を参照。

\subsection{NUSDAS\_GRID2: 格子情報へのアクセス }
\APILabel{nusdas.grid2}

\Prototype
\begin{quote}
CALL {\bf NUSDAS\_GRID2}({\it type1}, {\it type2}, {\it type3}, {\it basetime}, {\it member}, {\it validtime1}, {\it validtime2}, {\it proj}, {\it gridsize}, {\it gridinfo}, {\it value}, {\it getput}, {\it result})
\end{quote}

\begin{tabular}{l|rllp{16em}}
\hline
\ArgName & \ArgType & \ArrayDim & I/O & \ArgRole \\
\hline
{\it type1} & CHARACTER(8) &  & IN &  種別1  \\
{\it type2} & CHARACTER(4) &  & IN &  種別2  \\
{\it type3} & CHARACTER(4) &  & IN &  種別3  \\
{\it basetime} & INTEGER(4) &  & IN &  基準時刻(通算分)  \\
{\it member} & CHARACTER(4) &  & IN &  メンバー名  \\
{\it validtime1} & INTEGER(4) &  & IN &  対象時刻1(通算分)  \\
{\it validtime2} & INTEGER(4) &  & IN &  対象時刻2(通算分)  \\
{\it proj} & CHARACTER(4) &  & I/O &  投影法3字略号  \\
{\it gridsize} & INTEGER(4) & 2 & I/O &  格子数  \\
{\it gridinfo} & REAL(4) & 14 & I/O &  投影法緒元  \\
{\it value} & CHARACTER(4) &  & I/O &  格子点値が周囲の場を代表する方法  \\
{\it getput} & CHARACTER(3) &  & IN &  入出力指示 ({\it "GET}" または {\it "PUT}")  \\
{\it result} & INTEGER(4) &  & OUT & \ResultCode \\
\hline
\end{tabular}

\subsection{NUSDAS\_INFO2: INFO 記録へのアクセス }
\APILabel{nusdas.info2}

\Prototype
\begin{quote}
CALL {\bf NUSDAS\_INFO2}({\it type1}, {\it type2}, {\it type3}, {\it basetime}, {\it member}, {\it validtime1}, {\it validtime2}, {\it group}, {\it info}, {\it bytesize}, {\it getput}, {\it result})
\end{quote}

\begin{tabular}{l|rllp{16em}}
\hline
\ArgName & \ArgType & \ArrayDim & I/O & \ArgRole \\
\hline
{\it type1} & CHARACTER(8) &  & IN &  種別1  \\
{\it type2} & CHARACTER(4) &  & IN &  種別2  \\
{\it type3} & CHARACTER(4) &  & IN &  種別3  \\
{\it basetime} & INTEGER(4) &  & IN &  基準時刻(通算分)  \\
{\it member} & CHARACTER(4) &  & IN &  メンバー名  \\
{\it validtime1} & INTEGER(4) &  & IN &  対象時刻1(通算分)  \\
{\it validtime2} & INTEGER(4) &  & IN &  対象時刻2(通算分)  \\
{\it group} & CHARACTER(4) &  & IN &  群名  \\
{\it info} & CHARACTER & \AnySize & I/O &  INFO 記録内容  \\
{\it bytesize} & INTEGER(4) &  & IN &  INFO 記録のバイト数  \\
{\it getput} & CHARACTER(3) &  & IN &  入出力指示 ({\it "GET}" または {\it "PUT}")  \\
{\it result} & INTEGER(4) &  & OUT & \ResultCode \\
\hline
\end{tabular}

\subsection{NUSDAS\_INQ\_CNTL2: データファイルの諸元問合せ }
\APILabel{nusdas.inq.cntl2}

\Prototype
\begin{quote}
CALL {\bf NUSDAS\_INQ\_CNTL2}({\it type1}, {\it type2}, {\it type3}, {\it basetime}, {\it member}, {\it validtime1}, {\it validtime2}, {\it param}, {\it data}, {\it datasize}, {\it result})
\end{quote}

\begin{tabular}{l|rllp{16em}}
\hline
\ArgName & \ArgType & \ArrayDim & I/O & \ArgRole \\
\hline
{\it type1} & CHARACTER(8) &  & IN &  種別1  \\
{\it type2} & CHARACTER(4) &  & IN &  種別2  \\
{\it type3} & CHARACTER(4) &  & IN &  種別3  \\
{\it basetime} & INTEGER(4) &  & IN &  基準時刻(通算分)  \\
{\it member} & CHARACTER(4) &  & IN &  メンバー名  \\
{\it validtime1} & INTEGER(4) &  & IN &  対象時刻1(通算分)  \\
{\it validtime2} & INTEGER(4) &  & IN &  対象時刻2(通算分)  \\
{\it param} & INTEGER(4) &  & IN &  問合せ項目コード  \\
{\it data} & \AnyType & \AnySize & OUT &  問合せ結果配列  \\
{\it datasize} & INTEGER(4) &  & IN &  問合せ結果配列の要素数  \\
{\it result} & INTEGER(4) &  & OUT & \ResultCode \\
\hline
\end{tabular}

\subsection{NUSDAS\_INQ\_DATA2: データ記録の諸元問合せ }
\APILabel{nusdas.inq.data2}

\Prototype
\begin{quote}
CALL {\bf NUSDAS\_INQ\_DATA2}({\it type1}, {\it type2}, {\it type3}, {\it basetime}, {\it member}, {\it validtime1}, {\it validtime2}, {\it plane1}, {\it plane2}, {\it element}, {\it item}, {\it data}, {\it nelems}, {\it result})
\end{quote}

\begin{tabular}{l|rllp{16em}}
\hline
\ArgName & \ArgType & \ArrayDim & I/O & \ArgRole \\
\hline
{\it type1} & CHARACTER(8) &  & IN &  種別1  \\
{\it type2} & CHARACTER(4) &  & IN &  種別2  \\
{\it type3} & CHARACTER(4) &  & IN &  種別3  \\
{\it basetime} & INTEGER(4) &  & IN &  基準時刻(通算分)  \\
{\it member} & CHARACTER(4) &  & IN &  メンバー名  \\
{\it validtime1} & INTEGER(4) &  & IN &  対象時刻1(通算分)  \\
{\it validtime2} & INTEGER(4) &  & IN &  対象時刻2(通算分)  \\
{\it plane1} & CHARACTER(6) &  & IN &  面1  \\
{\it plane2} & CHARACTER(6) &  & IN &  面2  \\
{\it element} & CHARACTER(6) &  & IN &  要素名  \\
{\it item} & INTEGER(4) &  & IN &  問合せ項目コード  \\
{\it data} & \AnyType & \AnySize & OUT &  結果格納配列  \\
{\it nelems} & INTEGER(4) &  & IN &  結果格納配列要素数  \\
{\it result} & INTEGER(4) &  & OUT & \ResultCode \\
\hline
\end{tabular}

\subsection{NUSDAS\_ONEFILE\_CLOSE2: ひとつのファイルを閉じる}
\APILabel{nusdas.onefile.close2}

\Prototype
\begin{quote}
CALL {\bf NUSDAS\_ONEFILE\_CLOSE2}({\it type1}, {\it type2}, {\it type3}, {\it basetime}, {\it member}, {\it validtime1}, {\it validtime2}, {\it result})
\end{quote}

\begin{tabular}{l|rllp{16em}}
\hline
\ArgName & \ArgType & \ArrayDim & I/O & \ArgRole \\
\hline
{\it type1} & CHARACTER(8) &  & IN &  種別1  \\
{\it type2} & CHARACTER(4) &  & IN &  種別2  \\
{\it type3} & CHARACTER(4) &  & IN &  種別3  \\
{\it basetime} & INTEGER(4) &  & IN &  基準時刻(通算分)  \\
{\it member} & CHARACTER(4) &  & IN &  メンバー名  \\
{\it validtime1} & INTEGER(4) &  & IN &  対象時刻1(通算分)  \\
{\it validtime2} & INTEGER(4) &  & IN &  対象時刻2(通算分)  \\
{\it result} & INTEGER(4) &  & OUT & \ResultCode \\
\hline
\end{tabular}

\subsection{NUSDAS\_READ2: データ記録の読取}
\APILabel{nusdas.read2}

\Prototype
\begin{quote}
CALL {\bf NUSDAS\_READ2}({\it utype1}, {\it utype2}, {\it utype3}, {\it basetime}, {\it member}, {\it validtime1}, {\it validtime2}, {\it plane1}, {\it plane2}, {\it element}, {\it data}, {\it fmt}, {\it size}, {\it result})
\end{quote}

\begin{tabular}{l|rllp{16em}}
\hline
\ArgName & \ArgType & \ArrayDim & I/O & \ArgRole \\
\hline
{\it utype1} & CHARACTER(8) &  & IN &  種別1  \\
{\it utype2} & CHARACTER(4) &  & IN &  種別2  \\
{\it utype3} & CHARACTER(4) &  & IN &  種別3  \\
{\it basetime} & INTEGER(4) &  & IN &  基準時刻(通算分)  \\
{\it member} & CHARACTER(4) &  & IN &  メンバー  \\
{\it validtime1} & INTEGER(4) &  & IN &  対象時刻1(通算分)  \\
{\it validtime2} & INTEGER(4) &  & IN &  対象時刻2(通算分)  \\
{\it plane1} & CHARACTER(6) &  & IN &  面の名前1  \\
{\it plane2} & CHARACTER(6) &  & IN &  面の名前2  \\
{\it element} & CHARACTER(6) &  & IN &  要素名  \\
{\it data} & \AnyType & \AnySize & OUT &  結果格納配列  \\
{\it fmt} & CHARACTER(2) &  & IN &  結果格納配列の型  \\
{\it size} & INTEGER(4) &  & IN &  結果格納配列の要素数  \\
{\it result} & INTEGER(4) &  & OUT & \ResultCode \\
\hline
\end{tabular}

\subsection{NUSDAS\_SUBC\_DELT2: SUBC DELT へのアクセス }
\APILabel{nusdas.subc.delt2}

\Prototype
\begin{quote}
CALL {\bf NUSDAS\_SUBC\_DELT2}({\it type1}, {\it type2}, {\it type3}, {\it basetime}, {\it member}, {\it validtime1}, {\it validtime2}, {\it delt}, {\it getput}, {\it result})
\end{quote}

\begin{tabular}{l|rllp{16em}}
\hline
\ArgName & \ArgType & \ArrayDim & I/O & \ArgRole \\
\hline
{\it type1} & CHARACTER(8) &  & IN &  種別1  \\
{\it type2} & CHARACTER(4) &  & IN &  種別2  \\
{\it type3} & CHARACTER(4) &  & IN &  種別3  \\
{\it basetime} & INTEGER(4) &  & IN &  基準時刻(通算分)  \\
{\it member} & CHARACTER(4) &  & IN &  メンバー名  \\
{\it validtime1} & INTEGER(4) &  & IN &  対象時刻1(通算分)  \\
{\it validtime2} & INTEGER(4) &  & IN &  対象時刻2(通算分)  \\
{\it delt} & REAL(4) &  & I/O &  DELT 数値へのポインタ  \\
{\it getput} & CHARACTER(3) &  & IN &  入出力指示 ({\it "GET}" または {\it "PUT}")  \\
{\it result} & INTEGER(4) &  & OUT & \ResultCode \\
\hline
\end{tabular}

\subsection{NUSDAS\_SUBC\_ETA2: SUBC ETA へのアクセス }
\APILabel{nusdas.subc.eta2}

\Prototype
\begin{quote}
CALL {\bf NUSDAS\_SUBC\_ETA2}({\it type1}, {\it type2}, {\it type3}, {\it basetime}, {\it member}, {\it validtime1}, {\it validtime2}, {\it n\_levels}, {\it a}, {\it b}, {\it c}, {\it getput}, {\it result})
\end{quote}

\begin{tabular}{l|rllp{16em}}
\hline
\ArgName & \ArgType & \ArrayDim & I/O & \ArgRole \\
\hline
{\it type1} & CHARACTER(8) &  & IN &  種別1  \\
{\it type2} & CHARACTER(4) &  & IN &  種別2  \\
{\it type3} & CHARACTER(4) &  & IN &  種別3  \\
{\it basetime} & INTEGER(4) &  & IN &  基準時刻(通算分)  \\
{\it member} & CHARACTER(4) &  & IN &  メンバー名  \\
{\it validtime1} & INTEGER(4) &  & IN &  対象時刻1(通算分)  \\
{\it validtime2} & INTEGER(4) &  & IN &  対象時刻2(通算分)  \\
{\it n\_levels} & INTEGER(4) &  & I/O &  鉛直層数  \\
{\it a} & REAL(4) & \AnySize & I/O &  係数 a  \\
{\it b} & REAL(4) & \AnySize & I/O &  係数 b  \\
{\it c} & REAL(4) &  & I/O &  係数 c  \\
{\it getput} & CHARACTER(3) &  & IN &  入出力指示 ({\it "GET}" または {\it "PUT}")  \\
{\it result} & INTEGER(4) &  & OUT & \ResultCode \\
\hline
\end{tabular}

\subsection{NUSDAS\_SUBC\_ETA\_INQ\_NZ2: SUBC 記録の鉛直層数問合せ }
\APILabel{nusdas.subc.eta.inq.nz2}

\Prototype
\begin{quote}
CALL {\bf NUSDAS\_SUBC\_ETA\_INQ\_NZ2}({\it type1}, {\it type2}, {\it type3}, {\it basetime}, {\it member}, {\it validtime1}, {\it validtime2}, {\it group}, {\it n\_levels}, {\it result})
\end{quote}

\begin{tabular}{l|rllp{16em}}
\hline
\ArgName & \ArgType & \ArrayDim & I/O & \ArgRole \\
\hline
{\it type1} & CHARACTER(8) &  & IN &  種別1  \\
{\it type2} & CHARACTER(4) &  & IN &  種別2  \\
{\it type3} & CHARACTER(4) &  & IN &  種別3  \\
{\it basetime} & INTEGER(4) &  & IN &  基準時刻(通算分)  \\
{\it member} & CHARACTER(4) &  & IN &  メンバー名  \\
{\it validtime1} & INTEGER(4) &  & IN &  対象時刻1(通算分)  \\
{\it validtime2} & INTEGER(4) &  & IN &  対象時刻2(通算分)  \\
{\it group} & CHARACTER(4) &  & IN &  群名  \\
{\it n\_levels} & INTEGER(4) &  & OUT &  鉛直層数  \\
{\it result} & INTEGER(4) &  & OUT & \ResultCode \\
\hline
\end{tabular}

\subsection{NUSDAS\_SUBC\_RGAU2: SUBC RGAU へのアクセス }
\APILabel{nusdas.subc.rgau2}

\Prototype
\begin{quote}
CALL {\bf NUSDAS\_SUBC\_RGAU2}({\it type1}, {\it type2}, {\it type3}, {\it basetime}, {\it member}, {\it validtime1}, {\it validtime2}, {\it j}, {\it j\_start}, {\it j\_n}, {\it i}, {\it i\_start}, {\it i\_n}, {\it lat}, {\it getput}, {\it result})
\end{quote}

\begin{tabular}{l|rllp{16em}}
\hline
\ArgName & \ArgType & \ArrayDim & I/O & \ArgRole \\
\hline
{\it type1} & CHARACTER(8) &  & IN &  種別1  \\
{\it type2} & CHARACTER(4) &  & IN &  種別2  \\
{\it type3} & CHARACTER(4) &  & IN &  種別3  \\
{\it basetime} & INTEGER(4) &  & IN &  基準時刻(通算分)  \\
{\it member} & CHARACTER(4) &  & IN &  メンバー名  \\
{\it validtime1} & INTEGER(4) &  & IN &  対象時刻1(通算分)  \\
{\it validtime2} & INTEGER(4) &  & IN &  対象時刻2(通算分)  \\
{\it j} & INTEGER(4) &  & I/O &  全球の南北分割数  \\
{\it j\_start} & INTEGER(4) &  & I/O &  データの最北格子の番号(1始まり)  \\
{\it j\_n} & INTEGER(4) &  & I/O &  データの南北格子数  \\
{\it i} & INTEGER(4) & \AnySize & I/O &  全球の東西格子数  \\
{\it i\_start} & INTEGER(4) & \AnySize & I/O &  データの最西格子の番号(1始まり)  \\
{\it i\_n} & INTEGER(4) & \AnySize & I/O &  データの東西格子数  \\
{\it lat} & REAL(4) & \AnySize & I/O &  緯度  \\
{\it getput} & CHARACTER(3) &  & IN &  入出力指示 ({\it "GET}" または {\it "PUT}")  \\
{\it result} & INTEGER(4) &  & OUT & \ResultCode \\
\hline
\end{tabular}

\subsection{NUSDAS\_SUBC\_RGAU\_INQ\_JN2: SUBC RGAU 記録の大きさを問合せ }
\APILabel{nusdas.subc.rgau.inq.jn2}

\Prototype
\begin{quote}
CALL {\bf NUSDAS\_SUBC\_RGAU\_INQ\_JN2}({\it type1}, {\it type2}, {\it type3}, {\it basetime}, {\it member}, {\it validtime1}, {\it validtime2}, {\it j\_n}, {\it result})
\end{quote}

\begin{tabular}{l|rllp{16em}}
\hline
\ArgName & \ArgType & \ArrayDim & I/O & \ArgRole \\
\hline
{\it type1} & CHARACTER(8) &  & IN &  種別1  \\
{\it type2} & CHARACTER(4) &  & IN &  種別2  \\
{\it type3} & CHARACTER(4) &  & IN &  種別3  \\
{\it basetime} & INTEGER(4) &  & IN &  基準時刻(通算分)  \\
{\it member} & CHARACTER(4) &  & IN &  メンバー名  \\
{\it validtime1} & INTEGER(4) &  & IN &  対象時刻1(通算分)  \\
{\it validtime2} & INTEGER(4) &  & IN &  対象時刻2(通算分)  \\
{\it j\_n} & INTEGER(4) &  & OUT &  南北格子数  \\
{\it result} & INTEGER(4) &  & OUT & \ResultCode \\
\hline
\end{tabular}

\subsection{NUSDAS\_SUBC\_SIGM2: SUBC SIGM へのアクセス }
\APILabel{nusdas.subc.sigm2}

\Prototype
\begin{quote}
CALL {\bf NUSDAS\_SUBC\_SIGM2}({\it type1}, {\it type2}, {\it type3}, {\it basetime}, {\it member}, {\it validtime1}, {\it validtime2}, {\it n\_levels}, {\it a}, {\it b}, {\it c}, {\it getput}, {\it result})
\end{quote}

\begin{tabular}{l|rllp{16em}}
\hline
\ArgName & \ArgType & \ArrayDim & I/O & \ArgRole \\
\hline
{\it type1} & CHARACTER(8) &  & IN &  種別1  \\
{\it type2} & CHARACTER(4) &  & IN &  種別2  \\
{\it type3} & CHARACTER(4) &  & IN &  種別3  \\
{\it basetime} & INTEGER(4) &  & IN &  基準時刻(通算分)  \\
{\it member} & CHARACTER(4) &  & IN &  メンバー名  \\
{\it validtime1} & INTEGER(4) &  & IN &  対象時刻1(通算分)  \\
{\it validtime2} & INTEGER(4) &  & IN &  対象時刻2(通算分)  \\
{\it n\_levels} & INTEGER(4) &  & I/O &  鉛直層数  \\
{\it a} & REAL(4) & \AnySize & I/O &  係数 a  \\
{\it b} & REAL(4) & \AnySize & I/O &  係数 b  \\
{\it c} & REAL(4) &  & I/O &  係数 c  \\
{\it getput} & CHARACTER(3) &  & IN &  入出力指示 ({\it "GET}" または {\it "PUT}")  \\
{\it result} & INTEGER(4) &  & OUT & \ResultCode \\
\hline
\end{tabular}

\subsection{NUSDAS\_SUBC\_SRF2: 降短系 SUBC へのアクセス }
\APILabel{nusdas.subc.srf2}

\Prototype
\begin{quote}
CALL {\bf NUSDAS\_SUBC\_SRF2}({\it type1}, {\it type2}, {\it type3}, {\it basetime}, {\it member}, {\it validtime1}, {\it validtime2}, {\it plane1}, {\it plane2}, {\it element}, {\it group}, {\it data}, {\it getput}, {\it result})
\end{quote}

\begin{tabular}{l|rllp{16em}}
\hline
\ArgName & \ArgType & \ArrayDim & I/O & \ArgRole \\
\hline
{\it type1} & CHARACTER(8) &  & IN &  種別1  \\
{\it type2} & CHARACTER(4) &  & IN &  種別2  \\
{\it type3} & CHARACTER(4) &  & IN &  種別3  \\
{\it basetime} & INTEGER(4) &  & IN &  基準時刻(通算分)  \\
{\it member} & CHARACTER(4) &  & IN &  メンバー名  \\
{\it validtime1} & INTEGER(4) &  & IN &  対象時刻1(通算分)  \\
{\it validtime2} & INTEGER(4) &  & IN &  対象時刻2(通算分)  \\
{\it plane1} & CHARACTER(6) &  & IN &  面1  \\
{\it plane2} & CHARACTER(6) &  & IN &  面2  \\
{\it element} & CHARACTER(6) &  & IN &  要素名  \\
{\it group} & CHARACTER(4) &  & IN &  群名  \\
{\it data} & INTEGER(4) &  & I/O &  データ配列  \\
{\it getput} & CHARACTER(3) &  & IN &  入出力指示 ({\it "GET}" または {\it "PUT}")  \\
{\it result} & INTEGER(4) &  & OUT & \ResultCode \\
\hline
\end{tabular}

\subsection{NUSDAS\_SUBC\_TDIF2: SUBC TDIF へのアクセス }
\APILabel{nusdas.subc.tdif2}

\Prototype
\begin{quote}
CALL {\bf NUSDAS\_SUBC\_TDIF2}({\it type1}, {\it type2}, {\it type3}, {\it basetime}, {\it member}, {\it validtime1}, {\it validtime2}, {\it diff\_time}, {\it total\_sec}, {\it getput}, {\it result})
\end{quote}

\begin{tabular}{l|rllp{16em}}
\hline
\ArgName & \ArgType & \ArrayDim & I/O & \ArgRole \\
\hline
{\it type1} & CHARACTER(8) &  & IN &  種別1  \\
{\it type2} & CHARACTER(4) &  & IN &  種別2  \\
{\it type3} & CHARACTER(4) &  & IN &  種別3  \\
{\it basetime} & INTEGER(4) &  & IN &  基準時刻(通算分)  \\
{\it member} & CHARACTER(4) &  & IN &  メンバー名  \\
{\it validtime1} & INTEGER(4) &  & IN &  対象時刻1(通算分)  \\
{\it validtime2} & INTEGER(4) &  & IN &  対象時刻2(通算分)  \\
{\it diff\_time} & INTEGER(4) &  & I/O &  対象時刻からのずれ(秒)  \\
{\it total\_sec} & INTEGER(4) &  & I/O &  総予報時間(秒)  \\
{\it getput} & CHARACTER(3) &  & IN &  入出力指示 ({\it "GET}" または {\it "PUT}")  \\
{\it result} & INTEGER(4) &  & OUT & \ResultCode \\
\hline
\end{tabular}

\subsection{NUSDAS\_SUBC\_ZHYB2: SUBC ZHYB へのアクセス }
\APILabel{nusdas.subc.zhyb2}

\Prototype
\begin{quote}
CALL {\bf NUSDAS\_SUBC\_ZHYB2}({\it type1}, {\it type2}, {\it type3}, {\it basetime}, {\it member}, {\it validtime1}, {\it validtime2}, {\it nz}, {\it ptrf}, {\it presrf}, {\it zrp}, {\it zrw}, {\it vctrans\_p}, {\it vctrans\_w}, {\it dvtrans\_p}, {\it dvtrans\_w}, {\it getput}, {\it result})
\end{quote}

\begin{tabular}{l|rllp{16em}}
\hline
\ArgName & \ArgType & \ArrayDim & I/O & \ArgRole \\
\hline
{\it type1} & CHARACTER(8) &  & IN &  種別1  \\
{\it type2} & CHARACTER(4) &  & IN &  種別2  \\
{\it type3} & CHARACTER(4) &  & IN &  種別3  \\
{\it basetime} & INTEGER(4) &  & IN &  基準時刻(通算分)  \\
{\it member} & CHARACTER(4) &  & IN &  メンバー名  \\
{\it validtime1} & INTEGER(4) &  & IN &  対象時刻1(通算分)  \\
{\it validtime2} & INTEGER(4) &  & IN &  対象時刻2(通算分)  \\
{\it nz} & INTEGER(4) &  & I/O &  鉛直層数  \\
{\it ptrf} & REAL(4) &  & I/O &  温位の参照値  \\
{\it presrf} & REAL(4) &  & I/O &  気圧の参照値  \\
{\it zrp} & REAL(4) & \AnySize & I/O &  モデル面高度 (フルレベル)  \\
{\it zrw} & REAL(4) & \AnySize & I/O &  モデル面高度 (ハーフレベル)  \\
{\it vctrans\_p} & REAL(4) & \AnySize & I/O &  座標変換関数 (フルレベル)  \\
{\it vctrans\_w} & REAL(4) & \AnySize & I/O &  座標変換関数 (ハーフレベル)  \\
{\it dvtrans\_p} & REAL(4) & \AnySize & I/O &  座標変換関数の鉛直微分 (フルレベル)  \\
{\it dvtrans\_w} & REAL(4) & \AnySize & I/O &  座標変換関数の鉛直微分 (ハーフレベル)  \\
{\it getput} & CHARACTER(3) &  & IN &  入出力指示 ({\it "GET}" または {\it "PUT}")  \\
{\it result} & INTEGER(4) &  & OUT & \ResultCode \\
\hline
\end{tabular}

\subsection{NUSDAS\_WRITE2: データ記録の書出}
\APILabel{nusdas.write2}

\Prototype
\begin{quote}
CALL {\bf NUSDAS\_WRITE2}({\it utype1}, {\it utype2}, {\it utype3}, {\it basetime}, {\it member}, {\it validtime1}, {\it validtime2}, {\it plane1}, {\it plane2}, {\it element}, {\it data}, {\it fmt}, {\it nelems}, {\it result})
\end{quote}

\begin{tabular}{l|rllp{16em}}
\hline
\ArgName & \ArgType & \ArrayDim & I/O & \ArgRole \\
\hline
{\it utype1} & CHARACTER(8) &  & IN &  種別1  \\
{\it utype2} & CHARACTER(4) &  & IN &  種別2  \\
{\it utype3} & CHARACTER(4) &  & IN &  種別3  \\
{\it basetime} & INTEGER(4) &  & IN &  基準時刻(通算分)  \\
{\it member} & CHARACTER(4) &  & IN &  メンバー名  \\
{\it validtime1} & INTEGER(4) &  & IN &  対象時刻1(通算分)  \\
{\it validtime2} & INTEGER(4) &  & IN &  対象時刻2(通算分)  \\
{\it plane1} & CHARACTER(6) &  & IN &  面の名前1  \\
{\it plane2} & CHARACTER(6) &  & IN &  面の名前2  \\
{\it element} & CHARACTER(6) &  & IN &  要素名  \\
{\it data} & \AnyType & \AnySize & IN &  データを与える配列  \\
{\it fmt} & CHARACTER(2) &  & IN &  data の型  \\
{\it nelems} & INTEGER(4) &  & IN &  data の要素数  \\
{\it result} & INTEGER(4) &  & OUT & \ResultCode \\
\hline
\end{tabular}


\section{C API}
\label{capi2}

\renewcommand{\APILabel}[1]{\label{capi:#1}}

\subsection{nusdas\_cut2: 領域限定のデータ読取 }
\APILabel{nusdas.cut2}

\Prototype
\begin{quote}
N\_SI4 {\bf nusdas\_cut2}(const char {\it type1}[8], const char {\it type2}[4], const char {\it type3}[4], const N\_SI4 $\ast${\it basetime}, const char {\it member}[4], const N\_SI4 $\ast${\it validtime1}, const N\_SI4 $\ast${\it validtime2}, const char {\it plane1}[6], const char {\it plane2}[6], const char {\it element}[6], void $\ast${\it udata}, const char {\it utype}[2], const N\_SI4 $\ast${\it usize}, const N\_SI4 $\ast${\it ixstart}, const N\_SI4 $\ast${\it ixfinal}, const N\_SI4 $\ast${\it iystart}, const N\_SI4 $\ast${\it iyfinal});
\end{quote}

\begin{tabular}{l|rp{20em}}
\hline
\ArgName & \ArgType & \ArgRole \\
\hline
{\it type1} & const char [8] &  種別1  \\
{\it type2} & const char [4] &  種別2  \\
{\it type3} & const char [4] &  種別3  \\
{\it basetime} & const N\_SI4 $\ast$ &  基準時刻(通算分)  \\
{\it member} & const char [4] &  メンバー名  \\
{\it validtime1} & const N\_SI4 $\ast$ &  対象時刻1  \\
{\it validtime2} & const N\_SI4 $\ast$ &  対象時刻2  \\
{\it plane1} & const char [6] &  面1  \\
{\it plane2} & const char [6] &  面2  \\
{\it element} & const char [6] &  要素名  \\
{\it udata} & void $\ast$ &  データ格納配列  \\
{\it utype} & const char [2] &  データ格納配列の型  \\
{\it usize} & const N\_SI4 $\ast$ &  データ格納配列の要素数  \\
{\it ixstart} & const N\_SI4 $\ast$ &  $x$ 方向格子番号下限  \\
{\it ixfinal} & const N\_SI4 $\ast$ &  $x$ 方向格子番号上限  \\
{\it iystart} & const N\_SI4 $\ast$ &  $y$ 方向格子番号下限  \\
{\it iyfinal} & const N\_SI4 $\ast$ &  $y$ 方向格子番号上限  \\
\hline
\end{tabular}

\subsection{nusdas\_cut2\_raw: 領域限定の DATA 記録直接読取 }
\APILabel{nusdas.cut2.raw}

\Prototype
\begin{quote}
N\_SI4 {\bf nusdas\_cut2\_raw}(const char {\it type1}[8], const char {\it type2}[4], const char {\it type3}[4], const N\_SI4 $\ast${\it basetime}, const char {\it member}[4], const N\_SI4 $\ast${\it validtime1}, const N\_SI4 $\ast${\it validtime2}, const char {\it plane1}[6], const char {\it plane2}[6], const char {\it element}[6], void $\ast${\it udata}, const N\_SI4 $\ast${\it usize}, const N\_SI4 $\ast${\it ixstart}, const N\_SI4 $\ast${\it ixfinal}, const N\_SI4 $\ast${\it iystart}, const N\_SI4 $\ast${\it iyfinal});
\end{quote}

\begin{tabular}{l|rp{20em}}
\hline
\ArgName & \ArgType & \ArgRole \\
\hline
{\it type1} & const char [8] &  種別1  \\
{\it type2} & const char [4] &  種別2  \\
{\it type3} & const char [4] &  種別3  \\
{\it basetime} & const N\_SI4 $\ast$ &  基準時刻(通算分)  \\
{\it member} & const char [4] &  メンバー名  \\
{\it validtime1} & const N\_SI4 $\ast$ &  対象時刻1(通算分)  \\
{\it validtime2} & const N\_SI4 $\ast$ &  対象時刻2(通算分)  \\
{\it plane1} & const char [6] &  面1  \\
{\it plane2} & const char [6] &  面2  \\
{\it element} & const char [6] &  要素名  \\
{\it udata} & void $\ast$ &  データ格納先配列  \\
{\it usize} & const N\_SI4 $\ast$ &  データ格納先配列のバイト数  \\
{\it ixstart} & const N\_SI4 $\ast$ &  $x$ 方向格子番号下限  \\
{\it ixfinal} & const N\_SI4 $\ast$ &  $x$ 方向格子番号上限  \\
{\it iystart} & const N\_SI4 $\ast$ &  $y$ 方向格子番号下限  \\
{\it iyfinal} & const N\_SI4 $\ast$ &  $y$ 方向格子番号上限  \\
\hline
\end{tabular}
\paragraph{\FuncDesc}nusdas\_cut\_raw の範囲指定版。nusdas\_cut を参照。

\subsection{nusdas\_grid2: 格子情報へのアクセス }
\APILabel{nusdas.grid2}

\Prototype
\begin{quote}
N\_SI4 {\bf nusdas\_grid2}(const char {\it type1}[8], const char {\it type2}[4], const char {\it type3}[4], const N\_SI4 $\ast${\it basetime}, const char {\it member}[4], const N\_SI4 $\ast${\it validtime1}, const N\_SI4 $\ast${\it validtime2}, char {\it proj}[4], N\_SI4 {\it gridsize}[2], float {\it gridinfo}[14], char {\it value}[4], const char {\it getput}[3]);
\end{quote}

\begin{tabular}{l|rp{20em}}
\hline
\ArgName & \ArgType & \ArgRole \\
\hline
{\it type1} & const char [8] &  種別1  \\
{\it type2} & const char [4] &  種別2  \\
{\it type3} & const char [4] &  種別3  \\
{\it basetime} & const N\_SI4 $\ast$ &  基準時刻(通算分)  \\
{\it member} & const char [4] &  メンバー名  \\
{\it validtime1} & const N\_SI4 $\ast$ &  対象時刻1(通算分)  \\
{\it validtime2} & const N\_SI4 $\ast$ &  対象時刻2(通算分)  \\
{\it proj} & char [4] &  投影法3字略号  \\
{\it gridsize} & N\_SI4 [2] &  格子数  \\
{\it gridinfo} & float [14] &  投影法緒元  \\
{\it value} & char [4] &  格子点値が周囲の場を代表する方法  \\
{\it getput} & const char [3] &  入出力指示 ({\it "GET}" または {\it "PUT}")  \\
\hline
\end{tabular}

\subsection{nusdas\_info2: INFO 記録へのアクセス }
\APILabel{nusdas.info2}

\Prototype
\begin{quote}
N\_SI4 {\bf nusdas\_info2}(const char {\it type1}[8], const char {\it type2}[4], const char {\it type3}[4], const N\_SI4 $\ast${\it basetime}, const char {\it member}[4], const N\_SI4 $\ast${\it validtime1}, const N\_SI4 $\ast${\it validtime2}, const char {\it group}[4], char {\it info}[\,], const N\_SI4 $\ast${\it bytesize}, const char {\it getput}[3]);
\end{quote}

\begin{tabular}{l|rp{20em}}
\hline
\ArgName & \ArgType & \ArgRole \\
\hline
{\it type1} & const char [8] &  種別1  \\
{\it type2} & const char [4] &  種別2  \\
{\it type3} & const char [4] &  種別3  \\
{\it basetime} & const N\_SI4 $\ast$ &  基準時刻(通算分)  \\
{\it member} & const char [4] &  メンバー名  \\
{\it validtime1} & const N\_SI4 $\ast$ &  対象時刻1(通算分)  \\
{\it validtime2} & const N\_SI4 $\ast$ &  対象時刻2(通算分)  \\
{\it group} & const char [4] &  群名  \\
{\it info} & char [\,] &  INFO 記録内容  \\
{\it bytesize} & const N\_SI4 $\ast$ &  INFO 記録のバイト数  \\
{\it getput} & const char [3] &  入出力指示 ({\it "GET}" または {\it "PUT}")  \\
\hline
\end{tabular}

\subsection{nusdas\_inq\_cntl2: データファイルの諸元問合せ }
\APILabel{nusdas.inq.cntl2}

\Prototype
\begin{quote}
N\_SI4 {\bf nusdas\_inq\_cntl2}(const char {\it type1}[8], const char {\it type2}[4], const char {\it type3}[4], const N\_SI4 $\ast${\it basetime}, const char {\it member}[4], const N\_SI4 $\ast${\it validtime1}, const N\_SI4 $\ast${\it validtime2}, N\_SI4 {\it param}, void $\ast${\it data}, const N\_SI4 $\ast${\it datasize});
\end{quote}

\begin{tabular}{l|rp{20em}}
\hline
\ArgName & \ArgType & \ArgRole \\
\hline
{\it type1} & const char [8] &  種別1  \\
{\it type2} & const char [4] &  種別2  \\
{\it type3} & const char [4] &  種別3  \\
{\it basetime} & const N\_SI4 $\ast$ &  基準時刻(通算分)  \\
{\it member} & const char [4] &  メンバー名  \\
{\it validtime1} & const N\_SI4 $\ast$ &  対象時刻1(通算分)  \\
{\it validtime2} & const N\_SI4 $\ast$ &  対象時刻2(通算分)  \\
{\it param} & N\_SI4 &  問合せ項目コード  \\
{\it data} & void $\ast$ &  問合せ結果配列  \\
{\it datasize} & const N\_SI4 $\ast$ &  問合せ結果配列の要素数  \\
\hline
\end{tabular}

\subsection{nusdas\_inq\_data2: データ記録の諸元問合せ }
\APILabel{nusdas.inq.data2}

\Prototype
\begin{quote}
N\_SI4 {\bf nusdas\_inq\_data2}(const char {\it type1}[8], const char {\it type2}[4], const char {\it type3}[4], const N\_SI4 $\ast${\it basetime}, const char {\it member}[4], const N\_SI4 $\ast${\it validtime1}, const N\_SI4 $\ast${\it validtime2}, const char {\it plane1}[6], const char {\it plane2}[6], const char {\it element}[6], N\_SI4 {\it item}, void $\ast${\it data}, const N\_SI4 $\ast${\it nelems});
\end{quote}

\begin{tabular}{l|rp{20em}}
\hline
\ArgName & \ArgType & \ArgRole \\
\hline
{\it type1} & const char [8] &  種別1  \\
{\it type2} & const char [4] &  種別2  \\
{\it type3} & const char [4] &  種別3  \\
{\it basetime} & const N\_SI4 $\ast$ &  基準時刻(通算分)  \\
{\it member} & const char [4] &  メンバー名  \\
{\it validtime1} & const N\_SI4 $\ast$ &  対象時刻1(通算分)  \\
{\it validtime2} & const N\_SI4 $\ast$ &  対象時刻2(通算分)  \\
{\it plane1} & const char [6] &  面1  \\
{\it plane2} & const char [6] &  面2  \\
{\it element} & const char [6] &  要素名  \\
{\it item} & N\_SI4 &  問合せ項目コード  \\
{\it data} & void $\ast$ &  結果格納配列  \\
{\it nelems} & const N\_SI4 $\ast$ &  結果格納配列要素数  \\
\hline
\end{tabular}

\subsection{nusdas\_onefile\_close2: ひとつのファイルを閉じる}
\APILabel{nusdas.onefile.close2}

\Prototype
\begin{quote}
N\_SI4 {\bf nusdas\_onefile\_close2}(const char {\it type1}[8], const char {\it type2}[4], const char {\it type3}[4], const N\_SI4 $\ast${\it basetime}, const char {\it member}[4], const N\_SI4 $\ast${\it validtime1}, const N\_SI4 $\ast${\it validtime2});
\end{quote}

\begin{tabular}{l|rp{20em}}
\hline
\ArgName & \ArgType & \ArgRole \\
\hline
{\it type1} & const char [8] &  種別1  \\
{\it type2} & const char [4] &  種別2  \\
{\it type3} & const char [4] &  種別3  \\
{\it basetime} & const N\_SI4 $\ast$ &  基準時刻(通算分)  \\
{\it member} & const char [4] &  メンバー名  \\
{\it validtime1} & const N\_SI4 $\ast$ &  対象時刻1(通算分)  \\
{\it validtime2} & const N\_SI4 $\ast$ &  対象時刻2(通算分)  \\
\hline
\end{tabular}

\subsection{nusdas\_read2: データ記録の読取}
\APILabel{nusdas.read2}

\Prototype
\begin{quote}
N\_SI4 {\bf nusdas\_read2}(const char {\it utype1}[8], const char {\it utype2}[4], const char {\it utype3}[4], const N\_SI4 $\ast${\it basetime}, const char {\it member}[4], const N\_SI4 $\ast${\it validtime1}, const N\_SI4 $\ast${\it validtime2}, const char {\it plane1}[6], const char {\it plane2}[6], const char {\it element}[6], void $\ast${\it data}, const char {\it fmt}[2], const N\_SI4 $\ast${\it size});
\end{quote}

\begin{tabular}{l|rp{20em}}
\hline
\ArgName & \ArgType & \ArgRole \\
\hline
{\it utype1} & const char [8] &  種別1  \\
{\it utype2} & const char [4] &  種別2  \\
{\it utype3} & const char [4] &  種別3  \\
{\it basetime} & const N\_SI4 $\ast$ &  基準時刻(通算分)  \\
{\it member} & const char [4] &  メンバー  \\
{\it validtime1} & const N\_SI4 $\ast$ &  対象時刻1(通算分)  \\
{\it validtime2} & const N\_SI4 $\ast$ &  対象時刻2(通算分)  \\
{\it plane1} & const char [6] &  面の名前1  \\
{\it plane2} & const char [6] &  面の名前2  \\
{\it element} & const char [6] &  要素名  \\
{\it data} & void $\ast$ &  結果格納配列  \\
{\it fmt} & const char [2] &  結果格納配列の型  \\
{\it size} & const N\_SI4 $\ast$ &  結果格納配列の要素数  \\
\hline
\end{tabular}

\subsection{nusdas\_subc\_delt2: SUBC DELT へのアクセス }
\APILabel{nusdas.subc.delt2}

\Prototype
\begin{quote}
N\_SI4 {\bf nusdas\_subc\_delt2}(const char {\it type1}[8], const char {\it type2}[4], const char {\it type3}[4], const N\_SI4 $\ast${\it basetime}, const char {\it member}[4], const N\_SI4 $\ast${\it validtime1}, const N\_SI4 $\ast${\it validtime2}, float $\ast${\it delt}, const char {\it getput}[3]);
\end{quote}

\begin{tabular}{l|rp{20em}}
\hline
\ArgName & \ArgType & \ArgRole \\
\hline
{\it type1} & const char [8] &  種別1  \\
{\it type2} & const char [4] &  種別2  \\
{\it type3} & const char [4] &  種別3  \\
{\it basetime} & const N\_SI4 $\ast$ &  基準時刻(通算分)  \\
{\it member} & const char [4] &  メンバー名  \\
{\it validtime1} & const N\_SI4 $\ast$ &  対象時刻1(通算分)  \\
{\it validtime2} & const N\_SI4 $\ast$ &  対象時刻2(通算分)  \\
{\it delt} & float $\ast$ &  DELT 数値へのポインタ  \\
{\it getput} & const char [3] &  入出力指示 ({\it "GET}" または {\it "PUT}")  \\
\hline
\end{tabular}

\subsection{nusdas\_subc\_eta2: SUBC ETA へのアクセス }
\APILabel{nusdas.subc.eta2}

\Prototype
\begin{quote}
N\_SI4 {\bf nusdas\_subc\_eta2}(const char {\it type1}[8], const char {\it type2}[4], const char {\it type3}[4], const N\_SI4 $\ast${\it basetime}, const char {\it member}[4], const N\_SI4 $\ast${\it validtime1}, const N\_SI4 $\ast${\it validtime2}, N\_SI4 $\ast${\it n\_levels}, float {\it a}[\,], float {\it b}[\,], float $\ast${\it c}, const char {\it getput}[3]);
\end{quote}

\begin{tabular}{l|rp{20em}}
\hline
\ArgName & \ArgType & \ArgRole \\
\hline
{\it type1} & const char [8] &  種別1  \\
{\it type2} & const char [4] &  種別2  \\
{\it type3} & const char [4] &  種別3  \\
{\it basetime} & const N\_SI4 $\ast$ &  基準時刻(通算分)  \\
{\it member} & const char [4] &  メンバー名  \\
{\it validtime1} & const N\_SI4 $\ast$ &  対象時刻1(通算分)  \\
{\it validtime2} & const N\_SI4 $\ast$ &  対象時刻2(通算分)  \\
{\it n\_levels} & N\_SI4 $\ast$ &  鉛直層数  \\
{\it a} & float [\,] &  係数 a  \\
{\it b} & float [\,] &  係数 b  \\
{\it c} & float $\ast$ &  係数 c  \\
{\it getput} & const char [3] &  入出力指示 ({\it "GET}" または {\it "PUT}")  \\
\hline
\end{tabular}

\subsection{nusdas\_subc\_eta\_inq\_nz2: SUBC 記録の鉛直層数問合せ }
\APILabel{nusdas.subc.eta.inq.nz2}

\Prototype
\begin{quote}
N\_SI4 {\bf nusdas\_subc\_eta\_inq\_nz2}(const char {\it type1}[8], const char {\it type2}[4], const char {\it type3}[4], const N\_SI4 $\ast${\it basetime}, const char {\it member}[4], const N\_SI4 $\ast${\it validtime1}, const N\_SI4 $\ast${\it validtime2}, const char {\it group}[4], N\_SI4 $\ast${\it n\_levels});
\end{quote}

\begin{tabular}{l|rp{20em}}
\hline
\ArgName & \ArgType & \ArgRole \\
\hline
{\it type1} & const char [8] &  種別1  \\
{\it type2} & const char [4] &  種別2  \\
{\it type3} & const char [4] &  種別3  \\
{\it basetime} & const N\_SI4 $\ast$ &  基準時刻(通算分)  \\
{\it member} & const char [4] &  メンバー名  \\
{\it validtime1} & const N\_SI4 $\ast$ &  対象時刻1(通算分)  \\
{\it validtime2} & const N\_SI4 $\ast$ &  対象時刻2(通算分)  \\
{\it group} & const char [4] &  群名  \\
{\it n\_levels} & N\_SI4 $\ast$ &  鉛直層数  \\
\hline
\end{tabular}

\subsection{nusdas\_subc\_rgau2: SUBC RGAU へのアクセス }
\APILabel{nusdas.subc.rgau2}

\Prototype
\begin{quote}
N\_SI4 {\bf nusdas\_subc\_rgau2}(const char {\it type1}[8], const char {\it type2}[4], const char {\it type3}[4], const N\_SI4 $\ast${\it basetime}, const char {\it member}[4], const N\_SI4 $\ast${\it validtime1}, const N\_SI4 $\ast${\it validtime2}, N\_SI4 $\ast${\it j}, N\_SI4 $\ast${\it j\_start}, N\_SI4 $\ast${\it j\_n}, N\_SI4 {\it i}[\,], N\_SI4 {\it i\_start}[\,], N\_SI4 {\it i\_n}[\,], float {\it lat}[\,], const char {\it getput}[3]);
\end{quote}

\begin{tabular}{l|rp{20em}}
\hline
\ArgName & \ArgType & \ArgRole \\
\hline
{\it type1} & const char [8] &  種別1  \\
{\it type2} & const char [4] &  種別2  \\
{\it type3} & const char [4] &  種別3  \\
{\it basetime} & const N\_SI4 $\ast$ &  基準時刻(通算分)  \\
{\it member} & const char [4] &  メンバー名  \\
{\it validtime1} & const N\_SI4 $\ast$ &  対象時刻1(通算分)  \\
{\it validtime2} & const N\_SI4 $\ast$ &  対象時刻2(通算分)  \\
{\it j} & N\_SI4 $\ast$ &  全球の南北分割数  \\
{\it j\_start} & N\_SI4 $\ast$ &  データの最北格子の番号(1始まり)  \\
{\it j\_n} & N\_SI4 $\ast$ &  データの南北格子数  \\
{\it i} & N\_SI4 [\,] &  全球の東西格子数  \\
{\it i\_start} & N\_SI4 [\,] &  データの最西格子の番号(1始まり)  \\
{\it i\_n} & N\_SI4 [\,] &  データの東西格子数  \\
{\it lat} & float [\,] &  緯度  \\
{\it getput} & const char [3] &  入出力指示 ({\it "GET}" または {\it "PUT}")  \\
\hline
\end{tabular}

\subsection{nusdas\_subc\_rgau\_inq\_jn2: SUBC RGAU 記録の大きさを問合せ }
\APILabel{nusdas.subc.rgau.inq.jn2}

\Prototype
\begin{quote}
N\_SI4 {\bf nusdas\_subc\_rgau\_inq\_jn2}(const char {\it type1}[8], const char {\it type2}[4], const char {\it type3}[4], const N\_SI4 $\ast${\it basetime}, const char {\it member}[4], const N\_SI4 $\ast${\it validtime1}, const N\_SI4 $\ast${\it validtime2}, N\_SI4 $\ast${\it j\_n});
\end{quote}

\begin{tabular}{l|rp{20em}}
\hline
\ArgName & \ArgType & \ArgRole \\
\hline
{\it type1} & const char [8] &  種別1  \\
{\it type2} & const char [4] &  種別2  \\
{\it type3} & const char [4] &  種別3  \\
{\it basetime} & const N\_SI4 $\ast$ &  基準時刻(通算分)  \\
{\it member} & const char [4] &  メンバー名  \\
{\it validtime1} & const N\_SI4 $\ast$ &  対象時刻1(通算分)  \\
{\it validtime2} & const N\_SI4 $\ast$ &  対象時刻2(通算分)  \\
{\it j\_n} & N\_SI4 $\ast$ &  南北格子数  \\
\hline
\end{tabular}

\subsection{nusdas\_subc\_sigm2: SUBC SIGM へのアクセス }
\APILabel{nusdas.subc.sigm2}

\Prototype
\begin{quote}
N\_SI4 {\bf nusdas\_subc\_sigm2}(const char {\it type1}[8], const char {\it type2}[4], const char {\it type3}[4], const N\_SI4 $\ast${\it basetime}, const char {\it member}[4], const N\_SI4 $\ast${\it validtime1}, const N\_SI4 $\ast${\it validtime2}, N\_SI4 $\ast${\it n\_levels}, float {\it a}[\,], float {\it b}[\,], float $\ast${\it c}, const char {\it getput}[3]);
\end{quote}

\begin{tabular}{l|rp{20em}}
\hline
\ArgName & \ArgType & \ArgRole \\
\hline
{\it type1} & const char [8] &  種別1  \\
{\it type2} & const char [4] &  種別2  \\
{\it type3} & const char [4] &  種別3  \\
{\it basetime} & const N\_SI4 $\ast$ &  基準時刻(通算分)  \\
{\it member} & const char [4] &  メンバー名  \\
{\it validtime1} & const N\_SI4 $\ast$ &  対象時刻1(通算分)  \\
{\it validtime2} & const N\_SI4 $\ast$ &  対象時刻2(通算分)  \\
{\it n\_levels} & N\_SI4 $\ast$ &  鉛直層数  \\
{\it a} & float [\,] &  係数 a  \\
{\it b} & float [\,] &  係数 b  \\
{\it c} & float $\ast$ &  係数 c  \\
{\it getput} & const char [3] &  入出力指示 ({\it "GET}" または {\it "PUT}")  \\
\hline
\end{tabular}

\subsection{nusdas\_subc\_srf2: 降短系 SUBC へのアクセス }
\APILabel{nusdas.subc.srf2}

\Prototype
\begin{quote}
N\_SI4 {\bf nusdas\_subc\_srf2}(const char {\it type1}[8], const char {\it type2}[4], const char {\it type3}[4], const N\_SI4 $\ast${\it basetime}, const char {\it member}[4], const N\_SI4 $\ast${\it validtime1}, const N\_SI4 $\ast${\it validtime2}, const char {\it plane1}[6], const char {\it plane2}[6], const char {\it element}[6], const char {\it group}[4], N\_SI4 $\ast${\it data}, const char {\it getput}[3]);
\end{quote}

\begin{tabular}{l|rp{20em}}
\hline
\ArgName & \ArgType & \ArgRole \\
\hline
{\it type1} & const char [8] &  種別1  \\
{\it type2} & const char [4] &  種別2  \\
{\it type3} & const char [4] &  種別3  \\
{\it basetime} & const N\_SI4 $\ast$ &  基準時刻(通算分)  \\
{\it member} & const char [4] &  メンバー名  \\
{\it validtime1} & const N\_SI4 $\ast$ &  対象時刻1(通算分)  \\
{\it validtime2} & const N\_SI4 $\ast$ &  対象時刻2(通算分)  \\
{\it plane1} & const char [6] &  面1  \\
{\it plane2} & const char [6] &  面2  \\
{\it element} & const char [6] &  要素名  \\
{\it group} & const char [4] &  群名  \\
{\it data} & N\_SI4 $\ast$ &  データ配列  \\
{\it getput} & const char [3] &  入出力指示 ({\it "GET}" または {\it "PUT}")  \\
\hline
\end{tabular}

\subsection{nusdas\_subc\_srf\_ship2: SUBC LOCA へのアクセス }
\APILabel{nusdas.subc.srf.ship2}

\Prototype
\begin{quote}
N\_SI4 {\bf nusdas\_subc\_srf\_ship2}(const char {\it type1}[8], const char {\it type2}[4], const char {\it type3}[4], const N\_SI4 $\ast${\it basetime}, const char {\it member}[4], const N\_SI4 $\ast${\it validtime1}, const N\_SI4 $\ast${\it validtime2}, N\_SI4 $\ast${\it lat}, N\_SI4 $\ast${\it lon}, const char {\it getput}[3]);
\end{quote}

\begin{tabular}{l|rp{20em}}
\hline
\ArgName & \ArgType & \ArgRole \\
\hline
{\it type1} & const char [8] &  種別1  \\
{\it type2} & const char [4] &  種別2  \\
{\it type3} & const char [4] &  種別3  \\
{\it basetime} & const N\_SI4 $\ast$ &  基準時刻(通算分)  \\
{\it member} & const char [4] &  メンバー名  \\
{\it validtime1} & const N\_SI4 $\ast$ &  対象時刻1(通算分)  \\
{\it validtime2} & const N\_SI4 $\ast$ &  対象時刻2(通算分)  \\
{\it lat} & N\_SI4 $\ast$ &  緯度  \\
{\it lon} & N\_SI4 $\ast$ &  経度  \\
{\it getput} & const char [3] &  入出力指示 ({\it "GET}" または {\it "PUT}")  \\
\hline
\end{tabular}

\subsection{nusdas\_subc\_tdif2: SUBC TDIF へのアクセス }
\APILabel{nusdas.subc.tdif2}

\Prototype
\begin{quote}
N\_SI4 {\bf nusdas\_subc\_tdif2}(const char {\it type1}[8], const char {\it type2}[4], const char {\it type3}[4], const N\_SI4 $\ast${\it basetime}, const char {\it member}[4], const N\_SI4 $\ast${\it validtime1}, const N\_SI4 $\ast${\it validtime2}, N\_SI4 $\ast${\it diff\_time}, N\_SI4 $\ast${\it total\_sec}, const char {\it getput}[3]);
\end{quote}

\begin{tabular}{l|rp{20em}}
\hline
\ArgName & \ArgType & \ArgRole \\
\hline
{\it type1} & const char [8] &  種別1  \\
{\it type2} & const char [4] &  種別2  \\
{\it type3} & const char [4] &  種別3  \\
{\it basetime} & const N\_SI4 $\ast$ &  基準時刻(通算分)  \\
{\it member} & const char [4] &  メンバー名  \\
{\it validtime1} & const N\_SI4 $\ast$ &  対象時刻1(通算分)  \\
{\it validtime2} & const N\_SI4 $\ast$ &  対象時刻2(通算分)  \\
{\it diff\_time} & N\_SI4 $\ast$ &  対象時刻からのずれ(秒)  \\
{\it total\_sec} & N\_SI4 $\ast$ &  総予報時間(秒)  \\
{\it getput} & const char [3] &  入出力指示 ({\it "GET}" または {\it "PUT}")  \\
\hline
\end{tabular}

\subsection{nusdas\_subc\_zhyb2: SUBC ZHYB へのアクセス }
\APILabel{nusdas.subc.zhyb2}

\Prototype
\begin{quote}
N\_SI4 {\bf nusdas\_subc\_zhyb2}(const char {\it type1}[8], const char {\it type2}[4], const char {\it type3}[4], const N\_SI4 $\ast${\it basetime}, const char {\it member}[4], const N\_SI4 $\ast${\it validtime1}, const N\_SI4 $\ast${\it validtime2}, N\_SI4 $\ast${\it nz}, float $\ast${\it ptrf}, float $\ast${\it presrf}, float {\it zrp}[\,], float {\it zrw}[\,], float {\it vctrans\_p}[\,], float {\it vctrans\_w}[\,], float {\it dvtrans\_p}[\,], float {\it dvtrans\_w}[\,], const char {\it getput}[3]);
\end{quote}

\begin{tabular}{l|rp{20em}}
\hline
\ArgName & \ArgType & \ArgRole \\
\hline
{\it type1} & const char [8] &  種別1  \\
{\it type2} & const char [4] &  種別2  \\
{\it type3} & const char [4] &  種別3  \\
{\it basetime} & const N\_SI4 $\ast$ &  基準時刻(通算分)  \\
{\it member} & const char [4] &  メンバー名  \\
{\it validtime1} & const N\_SI4 $\ast$ &  対象時刻1(通算分)  \\
{\it validtime2} & const N\_SI4 $\ast$ &  対象時刻2(通算分)  \\
{\it nz} & N\_SI4 $\ast$ &  鉛直層数  \\
{\it ptrf} & float $\ast$ &  温位の参照値  \\
{\it presrf} & float $\ast$ &  気圧の参照値  \\
{\it zrp} & float [\,] &  モデル面高度 (フルレベル)  \\
{\it zrw} & float [\,] &  モデル面高度 (ハーフレベル)  \\
{\it vctrans\_p} & float [\,] &  座標変換関数 (フルレベル)  \\
{\it vctrans\_w} & float [\,] &  座標変換関数 (ハーフレベル)  \\
{\it dvtrans\_p} & float [\,] &  座標変換関数の鉛直微分 (フルレベル)  \\
{\it dvtrans\_w} & float [\,] &  座標変換関数の鉛直微分 (ハーフレベル)  \\
{\it getput} & const char [3] &  入出力指示 ({\it "GET}" または {\it "PUT}")  \\
\hline
\end{tabular}

\subsection{nusdas\_write2: データ記録の書出}
\APILabel{nusdas.write2}

\Prototype
\begin{quote}
N\_SI4 {\bf nusdas\_write2}(const char {\it utype1}[8], const char {\it utype2}[4], const char {\it utype3}[4], const N\_SI4 $\ast${\it basetime}, const char {\it member}[4], const N\_SI4 $\ast${\it validtime1}, const N\_SI4 $\ast${\it validtime2}, const char {\it plane1}[6], const char {\it plane2}[6], const char {\it element}[6], const void $\ast${\it data}, const char {\it fmt}[2], const N\_SI4 $\ast${\it nelems});
\end{quote}

\begin{tabular}{l|rp{20em}}
\hline
\ArgName & \ArgType & \ArgRole \\
\hline
{\it utype1} & const char [8] &  種別1  \\
{\it utype2} & const char [4] &  種別2  \\
{\it utype3} & const char [4] &  種別3  \\
{\it basetime} & const N\_SI4 $\ast$ &  基準時刻(通算分)  \\
{\it member} & const char [4] &  メンバー名  \\
{\it validtime1} & const N\_SI4 $\ast$ &  対象時刻1(通算分)  \\
{\it validtime2} & const N\_SI4 $\ast$ &  対象時刻2(通算分)  \\
{\it plane1} & const char [6] &  面の名前1  \\
{\it plane2} & const char [6] &  面の名前2  \\
{\it element} & const char [6] &  要素名  \\
{\it data} & const void $\ast$ &  データを与える配列  \\
{\it fmt} & const char [2] &  data の型  \\
{\it nelems} & const N\_SI4 $\ast$ &  data の要素数  \\
\hline
\end{tabular}

