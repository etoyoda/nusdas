\Chapter{Fortran リファレンスマニュアル}
\label{chap:fapi}

\section{凡例}

\newcommand{\Prototype}{\paragraph{書式}}
\newcommand{\ArgName}{引数名}
\newcommand{\ArgType}{引数の型}
\newcommand{\ArrayDim}{配列長}
\newcommand{\ArgRole}{役割}
\newcommand{\AnyType}{任意}
\newcommand{\AnySize}{可変}
\newcommand{\ResultCode}{終了コード}
\newcommand{\FuncDesc}{説明}
\newcommand{\Bug}{バグ}
\newcommand{\APILabel}[1]{\label{fapi:#1}}
\newcommand{\APILink}[2]{{#2(p. \pageref{fapi:#1})}}

\begin{itemize}
 \item
  引数 {\it fmt} または {\it utype} (配列の型) に与えるべき定数名は、
  \TabRef{tab:typename}参照。
\end{itemize}

\clearpage
\section{最低限知るべきサブルーチン}

\subsection{NUSDAS\_READ: データ記録の読取}
\APILabel{nusdas.read}

\Prototype
\begin{quote}
CALL {\bf NUSDAS\_READ}({\it utype1}, {\it utype2}, {\it utype3}, {\it basetime}, {\it member}, {\it validtime}, {\it plane}, {\it element}, {\it data}, {\it fmt}, {\it size}, {\it result})
\end{quote}

\begin{tabular}{l|rllp{16em}}
\hline
\ArgName & \ArgType & \ArrayDim & I/O & \ArgRole \\
\hline
{\it utype1} & CHARACTER(8) &  & IN &  種別1  \\
{\it utype2} & CHARACTER(4) &  & IN &  種別2  \\
{\it utype3} & CHARACTER(4) &  & IN &  種別3  \\
{\it basetime} & INTEGER(4) &  & IN &  基準時刻(通算分)  \\
{\it member} & CHARACTER(4) &  & IN &  メンバー  \\
{\it validtime} & INTEGER(4) &  & IN &  対象時刻(通算分)  \\
{\it plane} & CHARACTER(6) &  & IN &  面の名前  \\
{\it element} & CHARACTER(6) &  & IN &  要素名  \\
{\it data} & \AnyType & \AnySize & OUT &  結果格納配列  \\
{\it fmt} & CHARACTER(2) &  & IN &  結果格納配列の型  \\
{\it size} & INTEGER(4) &  & IN &  結果格納配列の要素数  \\
{\it result} & INTEGER(4) &  & OUT & \ResultCode \\
\hline
\end{tabular}
\paragraph{\FuncDesc}引数で指定したTYPE, 基準時刻、メンバー、対象時刻、面、要素のデータを
読み出す。 

fmtは\ref{tab:typename}のものを指定するが、これらの値はN\_を接頭辞につけた定数で参照できる。
例えば'R4'であれば定数N\_R4で参照できる。

\paragraph{\ResultCode}
\begin{quote}
\begin{description}
\item[{\bf 正}] 読み出して格納した格子数
\item[{\bf 0}] 指定したデータは未記録(定義ファイルの elementmap によって書き込まれることは許容されているが、まだデータが書き込まれていない)
\item[{\bf -2}] 指定したデータは記録することが許容されていない(elementmap によって禁止されている場合と指定した面名、要素名が登録されていない場合の両方を含む)。
\item[{\bf -4}] 格納配列が不足
\item[{\bf -5}] 格納配列の型とレコードの記録形式が不整合
\end{description}\end{quote}

\paragraph{ 注意 }
nusdas\_read では、返却値 0 はエラーであることに注意が必要。
nusdas\_read のエラーチェックは返却値が求めている格子数と一致していること
を確認するのが望ましい。
\paragraph{ 互換性 }
NuSDaS1.1 では「ランレングス圧縮で、データが指定最大値を超えている」
(返却値-6)が定義されていたが、はデータの最初だけを
見ているだけで意味がないと思われるので、 NuSDaS 1.3 からはこのエラーは
返さない。また、「ユーザーオープンファイルの管理部又はアドレス部が不正
である」(返却値-7)は、共通部分の-54〜-57に対応するので、このエラーは返さない


\subsection{NUSDAS\_WRITE: データ記録の書出}
\APILabel{nusdas.write}

\Prototype
\begin{quote}
CALL {\bf NUSDAS\_WRITE}({\it utype1}, {\it utype2}, {\it utype3}, {\it basetime}, {\it member}, {\it validtime}, {\it plane}, {\it element}, {\it data}, {\it fmt}, {\it nelems}, {\it result})
\end{quote}

\begin{tabular}{l|rllp{16em}}
\hline
\ArgName & \ArgType & \ArrayDim & I/O & \ArgRole \\
\hline
{\it utype1} & CHARACTER(8) &  & IN &  種別1  \\
{\it utype2} & CHARACTER(4) &  & IN &  種別2  \\
{\it utype3} & CHARACTER(4) &  & IN &  種別3  \\
{\it basetime} & INTEGER(4) &  & IN &  基準時刻(通算分)  \\
{\it member} & CHARACTER(4) &  & IN &  メンバー名  \\
{\it validtime} & INTEGER(4) &  & IN &  対象時刻(通算分)  \\
{\it plane} & CHARACTER(6) &  & IN &  面の名前  \\
{\it element} & CHARACTER(6) &  & IN &  要素名  \\
{\it data} & \AnyType & \AnySize & IN &  データ配列  \\
{\it fmt} & CHARACTER(2) &  & IN &  データ配列の型  \\
{\it nelems} & INTEGER(4) &  & IN &  データ配列の要素数  \\
{\it result} & INTEGER(4) &  & OUT & \ResultCode \\
\hline
\end{tabular}
\paragraph{\FuncDesc}
データレコードを指定された場所に書き出す。

fmtは\ref{tab:typename}のものを指定するが、これらの値はN\_を接頭辞につけた定数で参照できる。
例えば'R4'であれば定数N\_R4で参照できる。

\paragraph{\ResultCode}
\begin{quote}
\begin{description}
\item[{\bf 正}] 実際に書き出された要素数
\item[{\bf -2}] メンバー名、面名、要素名が間違っている
\item[{\bf -2}] このレコードは ELEMENTMAP によって書き出しが禁止されている
\item[{\bf -3}] 与えられたデータ要素数 {\it nelems} が必要より小さい
\item[{\bf -4}] 指定データセットにはデータ配列の型 {\it fmt} は書き出せない
\item[{\bf -5}] データレコード長が固定レコード長を超える
\item[{\bf -6}] データセットの欠損値指定方式と RLEN 圧縮は併用できない
\item[{\bf -7}] マスクビットの設定がされていない
\item[{\bf -8}] エンコード過程でのエラー (数値が過大または RLEN 圧縮エラー)
\end{description}\end{quote}

\paragraph{注意}
\begin{itemize}
\item データセットの指定と異なる大きさのレコードを書き出すにはあらかじめ
\APILink{nusdas.parameter.change}{nusdas\_parameter\_change} を使って設定を変えておく。
\item 格子数 (データセットの指定または \APILink{nusdas.parameter.change}{nusdas\_parameter\_change} 設定)
より大きい要素数 {\it nelems} を指定するとエラーにはならず、
余った要素が書き出されない結果となるので注意されたい。
\end{itemize}

\paragraph{履歴}
この関数は NuSDaS 1.0 から存在した。


\input{fapi_nusdas_iocntl}

\subsection{NUSDAS\_ALLFILE\_CLOSE: 全てのデータファイルを閉じる}
\APILabel{nusdas.allfile.close}

\Prototype
\begin{quote}
CALL {\bf NUSDAS\_ALLFILE\_CLOSE}({\it param}, {\it result})
\end{quote}

\begin{tabular}{l|rllp{16em}}
\hline
\ArgName & \ArgType & \ArrayDim & I/O & \ArgRole \\
\hline
{\it param} & INTEGER(4) &  & IN &  閉じるファイルの種類  \\
{\it result} & INTEGER(4) &  & OUT & \ResultCode \\
\hline
\end{tabular}
\paragraph{\FuncDesc}
今までに NuSDaS インターフェイスで開いた全てのファイルを閉じる.
引数 param は次のいずれかを用いる:
\begin{quote}\begin{description}
\item[{\bf N\_FOPEN\_READ}] 読み込み用に開いたファイルだけを閉じる
\item[{\bf N\_FOPEN\_WRITE}] 書き込み可で開いたファイルだけを閉じる
\item[{\bf N\_FOPEN\_ALL}] すべてのファイルを閉じる
\end{description}\end{quote}

\paragraph{\ResultCode}
\begin{quote}
\begin{description}
\item[{\bf 正}] 正常に閉じられたファイルの数
\item[{\bf 0}] 閉じるべきファイルがなかった
\item[{\bf 負}] 閉じる際にエラーが起こったファイルの数
\end{description}\end{quote}

\paragraph{履歴}
この関数は NuSDaS 1.0 から存在した.


\clearpage
\section{データ読書サブルーチン}

\subsection{NUSDAS\_CUT: 領域限定のデータ読取}
\APILabel{nusdas.cut}

\Prototype
\begin{quote}
CALL {\bf NUSDAS\_CUT}({\it type1}, {\it type2}, {\it type3}, {\it basetime}, {\it member}, {\it validtime}, {\it plane}, {\it element}, {\it udata}, {\it utype}, {\it usize}, {\it ixstart}, {\it ixfinal}, {\it iystart}, {\it iyfinal}, {\it result})
\end{quote}

\begin{tabular}{l|rllp{16em}}
\hline
\ArgName & \ArgType & \ArrayDim & I/O & \ArgRole \\
\hline
{\it type1} & CHARACTER(8) &  & IN &  種別1  \\
{\it type2} & CHARACTER(4) &  & IN &  種別2  \\
{\it type3} & CHARACTER(4) &  & IN &  種別3  \\
{\it basetime} & INTEGER(4) &  & IN &  基準時刻(通算分)  \\
{\it member} & CHARACTER(4) &  & IN &  メンバー名  \\
{\it validtime} & INTEGER(4) &  & IN &  対象時刻(通算分)  \\
{\it plane} & CHARACTER(6) &  & IN &  面  \\
{\it element} & CHARACTER(6) &  & IN &  要素名  \\
{\it udata} & \AnyType & \AnySize & OUT &  データ格納先配列  \\
{\it utype} & CHARACTER(2) &  & IN &  データ格納先配列の型  \\
{\it usize} & INTEGER(4) &  & IN &  データ格納先配列の要素数  \\
{\it ixstart} & INTEGER(4) &  & IN &  $x$ 方向格子番号下限  \\
{\it ixfinal} & INTEGER(4) &  & IN &  $x$ 方向格子番号上限  \\
{\it iystart} & INTEGER(4) &  & IN &  $y$ 方向格子番号下限  \\
{\it iyfinal} & INTEGER(4) &  & IN &  $y$ 方向格子番号上限  \\
{\it result} & INTEGER(4) &  & OUT & \ResultCode \\
\hline
\end{tabular}
\paragraph{\FuncDesc}
\APILink{nusdas.read}{nusdas\_read} $\ast$ と同様だが、データレコードのうち格子点
({\it ixstart} , {\it iystart} )--({\it ixfinal} , {\it iyfinal} )
だけが {\it udata} に格納される。

格子番号は 1 から始まるものとするため、
{\it ixstart} や {\it iystart} は正でなければならず、また
{\it ixfinal} や {\it iyfinal} はそれぞれ
{\it ixstart} や {\it iystart} 以上でなければならない。
この規則に反する指定を行った場合は、返却値-8のエラーとなる。
なお、{\it iyfinal}, {\it jyfinal} の上限が格子数を超えていることの
チェックはしていないので注意が必要。

utypeは\ref{tab:typename}のものを指定するが、これらの値はN\_を接頭辞につけた定数で参照できる。
例えば'R4'であれば定数N\_R4で参照できる。

\paragraph{\ResultCode}
\begin{quote}
\begin{description}
\item[{\bf 正}] 読み出して格納した格子数
\item[{\bf 0}] 指定したデータは未記録(定義ファイルの elementmap によって書き込まれることは許容されているが、まだデータが書き込まれていない)
\item[{\bf -2}] 指定したデータは記録することが許容されていない(elementmap によって禁止されている場合と指定した面名、要素名が登録されていない場合の両方を含む)。
\item[{\bf -4}] 格納配列が不足
\item[{\bf -5}] 格納配列の型とレコードの記録形式が不整合
\item[{\bf -8}] 領域指定パラメータが不正
\end{description}\end{quote}

\paragraph{履歴}
本関数は NuSDaS 1.1 で導入され、 NuSDaS 1.3 で初めてドキュメントされた。
\paragraph{互換性}
NuSDaS 1.1 では、ローカルのデータファイルに対しては、
{\it ixstart} $\le$ 0 の場合は {\it ixstart} = 1 に({\it jystart} も同様), 
{\it ixfinal} がX方向の格子数を超える場合には、{\it ixfinal} はX方向の格子数に
({\it jyfinal} も同様)に読み替えられていたが、 NuSDaS 1.3 からは返却値-8のエラー
とする。また、pandora データについては、{\it ixstart}, {\it ixfinal}, 
{\it jystart}, {\it jyfinal} が非負であることだけがチェックされていた。
NuSDaS 1.3 からはデータファイル、pandora とも上述の通りとなる。

\subsection{NUSDAS\_CUT\_RAW: 領域限定の DATA 記録直接読取}
\APILabel{nusdas.cut.raw}

\Prototype
\begin{quote}
CALL {\bf NUSDAS\_CUT\_RAW}({\it type1}, {\it type2}, {\it type3}, {\it basetime}, {\it member}, {\it validtime}, {\it plane}, {\it element}, {\it udata}, {\it usize}, {\it ixstart}, {\it ixfinal}, {\it iystart}, {\it iyfinal}, {\it result})
\end{quote}

\begin{tabular}{l|rllp{16em}}
\hline
\ArgName & \ArgType & \ArrayDim & I/O & \ArgRole \\
\hline
{\it type1} & CHARACTER(8) &  & IN &  種別1  \\
{\it type2} & CHARACTER(4) &  & IN &  種別2  \\
{\it type3} & CHARACTER(4) &  & IN &  種別3  \\
{\it basetime} & INTEGER(4) &  & IN &  基準時刻(通算分)  \\
{\it member} & CHARACTER(4) &  & IN &  メンバー名  \\
{\it validtime} & INTEGER(4) &  & IN &  対象時刻(通算分)  \\
{\it plane} & CHARACTER(6) &  & IN &  面  \\
{\it element} & CHARACTER(6) &  & IN &  要素名  \\
{\it udata} & \AnyType & \AnySize & OUT &  データ格納先配列  \\
{\it usize} & INTEGER(4) &  & IN &  データ格納先配列のバイト数  \\
{\it ixstart} & INTEGER(4) &  & IN &  $x$ 方向格子番号下限  \\
{\it ixfinal} & INTEGER(4) &  & IN &  $x$ 方向格子番号上限  \\
{\it iystart} & INTEGER(4) &  & IN &  $y$ 方向格子番号下限  \\
{\it iyfinal} & INTEGER(4) &  & IN &  $y$ 方向格子番号上限  \\
{\it result} & INTEGER(4) &  & OUT & \ResultCode \\
\hline
\end{tabular}
\paragraph{\FuncDesc}
\APILink{nusdas.read2.raw}{nusdas\_read2\_raw} と類似だが、データレコードのうち格子点
({\it ixstart} , {\it iystart} )--({\it ixfinal} , {\it iyfinal} )
に対応する部分だけが {\it udata} に格納される。
ただしパッキングが 2UPP, 2UPJ, RLEN の場合、または欠損値が MASK の場合は全体が格納される。

\paragraph{\ResultCode}
\begin{quote}
\begin{description}
\item[{\bf 正}] 読み出して格納したバイト数
\item[{\bf 0}] 指定したデータは未記録(定義ファイルの elementmap によって書き込まれることは許容されているが、まだデータが書き込まれていない)
\item[{\bf -2}] 指定したデータは記録することが許容されていない(elementmap によって禁止されている場合と指定した面名、要素名が登録されていない場合の両方を含む)。
\item[{\bf -4}] 格納配列が不足
\end{description}\end{quote}

\paragraph{履歴}
この関数は NuSDaS 1.1 で導入された。
エラーコード -4 は NuSDaS 1.3 で新設されたもので、
それ以前はエラーチェックがなされていなかった。

\subsection{NUSDAS\_READ2\_RAW: DATA記録内容の直接読取}
\APILabel{nusdas.read2.raw}

\Prototype
\begin{quote}
CALL {\bf NUSDAS\_READ2\_RAW}({\it type1}, {\it type2}, {\it type3}, {\it basetime}, {\it member}, {\it validtime1}, {\it validtime2}, {\it plane1}, {\it plane2}, {\it element}, {\it buf}, {\it buf\_nbytes}, {\it result})
\end{quote}

\begin{tabular}{l|rllp{16em}}
\hline
\ArgName & \ArgType & \ArrayDim & I/O & \ArgRole \\
\hline
{\it type1} & CHARACTER(8) &  & IN &  種別1  \\
{\it type2} & CHARACTER(4) &  & IN &  種別2  \\
{\it type3} & CHARACTER(4) &  & IN &  種別3  \\
{\it basetime} & INTEGER(4) &  & IN &  基準時刻(通算分)  \\
{\it member} & CHARACTER(4) &  & IN &  メンバー名  \\
{\it validtime1} & INTEGER(4) &  & IN &  対象時刻1  \\
{\it validtime2} & INTEGER(4) &  & IN &  対象時刻2  \\
{\it plane1} & CHARACTER(6) &  & IN &  面1  \\
{\it plane2} & CHARACTER(6) &  & IN &  面2  \\
{\it element} & CHARACTER(6) &  & IN &  要素名  \\
{\it buf} & \AnyType & \AnySize & OUT &  データ格納配列  \\
{\it buf\_nbytes} & INTEGER(4) &  & IN &  データ格納配列のバイト数  \\
{\it result} & INTEGER(4) &  & OUT & \ResultCode \\
\hline
\end{tabular}
\paragraph{\FuncDesc}引数で指定したTYPE, 基準時刻、メンバー、対象時刻、面、要素のデータを
ファイルに格納されたままの形式で読み出す。
データは、DATA レコードのフォーマット表\ref{table.fmt.data}の項番10〜13までのデータが
格納される。
\paragraph{\ResultCode}
\begin{quote}
\begin{description}
\item[{\bf 正}] 読み出して格納したバイト数。
\item[{\bf 0}] 指定したデータは未記録(定義ファイルの elementmap によって書き込まれることは許容されているが、まだデータが書き込まれていない)
\item[{\bf -2}] 指定したデータは記録することが許容されていない(elementmap によって禁止されている場合と指定した面名、要素名が登録されていない場合の両方を含む)。
\item[{\bf -4}] 格納配列が不足
\end{description}\end{quote}
\paragraph{ 履歴 }
この関数は NuSDaS1.1 で導入された。

\subsection{NUSDAS\_READ\_3D: 高次元読み込み}
\APILabel{nusdas.read.3d}

\Prototype
\begin{quote}
CALL {\bf NUSDAS\_READ\_3D}({\it type1}, {\it type2}, {\it type3}, {\it basetime}, {\it member}, {\it validtime}, {\it plane}, {\it element}, {\it nrecs}, {\it udata}, {\it utype}, {\it usize}, {\it result})
\end{quote}

\begin{tabular}{l|rllp{16em}}
\hline
\ArgName & \ArgType & \ArrayDim & I/O & \ArgRole \\
\hline
{\it type1} & CHARACTER(8) &  & IN &  種別1  \\
{\it type2} & CHARACTER(4) &  & IN &  種別2  \\
{\it type3} & CHARACTER(4) &  & IN &  種別3  \\
{\it basetime} & INTEGER(4) &  & IN &  基準時刻  \\
{\it member} & CHARACTER(4) & \AnySize & IN &  メンバ名の配列  \\
{\it validtime} & INTEGER(4) & \AnySize & IN &  対象時刻の配列  \\
{\it plane} & CHARACTER(6) & \AnySize & IN &  面名の配列  \\
{\it element} & CHARACTER(6) & \AnySize & IN &  要素名の配列  \\
{\it nrecs} & INTEGER(4) &  & IN &  レコード数  \\
{\it udata} & \AnyType & \AnySize & OUT &  結果格納配列  \\
{\it utype} & CHARACTER(2) &  & IN &  結果格納配列の型  \\
{\it usize} & INTEGER(4) &  & IN &  レコードあたり要素数  \\
{\it result} & INTEGER(4) &  & OUT & \ResultCode \\
\hline
\end{tabular}

\paragraph{\FuncDesc}
{\it member}, {\it validtime}, {\it plane}, {\it element} には {\it nrecs} 個の大きさを持つ配列を指定する。
これらの配列の各要素を指定して \APILink{nusdas.read}{nusdas\_read} を順次呼びだし {\it udata} に格納する。
{\it udata} の要素数は {\it nrecs} * {\it usize} 個以上でなければならない。
nusdas\_read の返却値(終了コード)が {\it usize} と一致しなかった場合は読み込みを終了する。

utypeは\ref{tab:typename}のものを指定するが、これらの値はN\_を接頭辞につけた定数で参照できる。
例えば'R4'であれば定数N\_R4で参照できる。

\paragraph{\ResultCode}
\begin{quote}
\begin{description}
\item[{\bf 正}] 全てのデータを読むことができた場合は {\it nrecs} * {\it usize} を返す。
\item[{\bf 負}] 最後に呼び出した \APILink{nusdas.read}{nusdas\_read} の終了コードを返す。
\end{description}\end{quote}

\paragraph{注意}
読み込みを途中で終了した場合、どこまで処理したかを知る方法は無い。

\subsection{NUSDAS\_WRITE\_3D: 高次元書き出し}
\APILabel{nusdas.write.3d}

\Prototype
\begin{quote}
CALL {\bf NUSDAS\_WRITE\_3D}({\it type1}, {\it type2}, {\it type3}, {\it basetime}, {\it member}, {\it validtime}, {\it plane}, {\it element}, {\it nrecs}, {\it udata}, {\it utype}, {\it usize}, {\it result})
\end{quote}

\begin{tabular}{l|rllp{16em}}
\hline
\ArgName & \ArgType & \ArrayDim & I/O & \ArgRole \\
\hline
{\it type1} & CHARACTER(8) &  & IN &  種別1  \\
{\it type2} & CHARACTER(4) &  & IN &  種別2  \\
{\it type3} & CHARACTER(4) &  & IN &  種別3  \\
{\it basetime} & INTEGER(4) &  & IN &  基準時刻  \\
{\it member} & CHARACTER(4) & \AnySize & IN &  メンバ名の配列  \\
{\it validtime} & INTEGER(4) & \AnySize & IN &  対象時刻の配列  \\
{\it plane} & CHARACTER(6) & \AnySize & IN &  面名の配列  \\
{\it element} & CHARACTER(6) & \AnySize & IN &  要素名の配列  \\
{\it nrecs} & INTEGER(4) &  & IN &  レコード数  \\
{\it udata} & \AnyType & \AnySize & IN &  データ配列  \\
{\it utype} & CHARACTER(2) &  & IN &  データ配列の型  \\
{\it usize} & INTEGER(4) &  & IN &  レコードあたり要素数  \\
{\it result} & INTEGER(4) &  & OUT & \ResultCode \\
\hline
\end{tabular}

\paragraph{\FuncDesc}
{\it member}, {\it validtime}, {\it plane}, {\it element} には {\it nrecs} 個の大きさを持つ配列を指定する。
これらの配列の各要素を指定して \APILink{nusdas.write}{nusdas\_write} を順次呼びだす。
データ配列の要素数は {\it nrecs} * {\it usize} 個でなければならない。
nusdas\_write の返却値(終了コード)が {\it usize} と一致しなかった場合は書き出しを終了する。

utypeは\ref{tab:typename}のものを指定するが、これらの値はN\_を接頭辞につけた定数で参照できる。
例えば'R4'であれば定数N\_R4で参照できる。

\paragraph{\ResultCode}
\begin{quote}
\begin{description}
\item[{\bf 正}] 全てのデータを書き出すことができた場合は {\it nrecs} * {\it usize} を返す。
\item[{\bf 負}] 最後に呼び出した \APILink{nusdas.write}{nusdas\_write} の終了コードを返す。
\end{description}\end{quote}

\paragraph{注意}
読み込みを途中で終了した場合、どこまで処理したかを知る方法は無い。


\clearpage
\section{動作制御用サブルーチン}

\subsection{NUSDAS\_ESF\_FLUSH: NAPS7型ESファイルの出力完了}
\APILabel{nusdas.esf.flush}

\Prototype
\begin{quote}
CALL {\bf NUSDAS\_ESF\_FLUSH}({\it type1}, {\it type2}, {\it type3}, {\it basetime}, {\it member}, {\it validtime}, {\it result})
\end{quote}

\begin{tabular}{l|rllp{16em}}
\hline
\ArgName & \ArgType & \ArrayDim & I/O & \ArgRole \\
\hline
{\it type1} & CHARACTER(8) &  & IN &  種別1  \\
{\it type2} & CHARACTER(4) &  & IN &  種別2  \\
{\it type3} & CHARACTER(4) &  & IN &  種別3  \\
{\it basetime} & INTEGER(4) &  & IN &  基準時刻  \\
{\it member} & CHARACTER(4) &  & IN &  メンバー名  \\
{\it validtime} & INTEGER(4) &  & IN &  対象時刻  \\
{\it result} & INTEGER(4) &  & OUT & \ResultCode \\
\hline
\end{tabular}
\paragraph{\FuncDesc}
\paragraph{履歴} \APILink{nusdas.esf.flush}{nusdas\_esf\_flush} は NuSDaS 1.0 から存在する。
\paragraph{\Bug} NuSDaS 1.3 からは ES をサポートしていないため、
この関数はダミーである。

\subsection{NUSDAS\_MAKE\_MASK: マスクビット配列の作成}
\APILabel{nusdas.make.mask}

\Prototype
\begin{quote}
CALL {\bf NUSDAS\_MAKE\_MASK}({\it udata}, {\it utype}, {\it usize}, {\it output}, {\it mb\_bytes}, {\it result})
\end{quote}

\begin{tabular}{l|rllp{16em}}
\hline
\ArgName & \ArgType & \ArrayDim & I/O & \ArgRole \\
\hline
{\it udata} & \AnyType & \AnySize & IN &  格子データ  \\
{\it utype} & CHARACTER(2) &  & IN &  格子データの型  \\
{\it usize} & INTEGER(4) &  & IN &  格子データの要素数  \\
{\it output} & \AnyType & \AnySize & OUT &  マスクビット配列  \\
{\it mb\_bytes} & INTEGER(4) &  & IN &  マスクビット配列のバイト数  \\
{\it result} & INTEGER(4) &  & OUT & \ResultCode \\
\hline
\end{tabular}
\paragraph{\FuncDesc}
配列 {\it udata} の内容をチェックしてマスクビット列を作成し
{\it output} に書き込む。
引数 {\it utype} と欠損値は配列の型に応じて次のように指定する。
\begin{quote}\begin{description}
\item[{\bf 1バイト整数型}] 
引数 {\it utype} に N\_I1 を指定する。
配列中の欠損扱いしたい要素に N\_MV\_UI1 を設定しておく。
\item[{\bf 2バイト整数型}] 
引数 {\it utype} に N\_I2 を指定する。
配列中の欠損扱いしたい要素に N\_MV\_SI2 を設定しておく。
\item[{\bf 4バイト整数型}] 
引数 {\it utype} に N\_I4 を指定する。
配列中の欠損扱いしたい要素に N\_MV\_SI4 を設定しておく。
\item[{\bf 4バイト実数型}] 
引数 {\it utype} に N\_R4 を指定する。
配列中の欠損扱いしたい要素に N\_MV\_R4 を設定しておく。
\item[{\bf 8バイト実数型}] 
引数 {\it utype} に N\_R8 を指定する。
配列中の欠損扱いしたい要素に N\_MV\_R8 を設定しておく。
\end{description}\end{quote}

\paragraph{\ResultCode}
\begin{quote}
\begin{description}
\item[{\bf 0}] 正常終了
\item[{\bf -1}] 配列長 {\it mb\_bytes} が不足している
\item[{\bf -5}] 未知の型名 {\it utype} が与えられた
\end{description}\end{quote}

\paragraph{サイズ要件}
{\it mb\_bytes} は少なくとも ({\it usize} + 7) / 8 バイト以上必要である。

\paragraph{履歴}
\APILink{nusdas.make.mask}{nusdas\_make\_mask} は NuSDaS 1.0 から存在する。

\subsection{NUSDAS\_SET\_MASK: 改善型マスクビット設定関数}
\APILabel{nusdas.set.mask}

\Prototype
\begin{quote}
CALL {\bf NUSDAS\_SET\_MASK}({\it type1}, {\it type2}, {\it type3}, {\it udata}, {\it utype}, {\it usize}, {\it result})
\end{quote}

\begin{tabular}{l|rllp{16em}}
\hline
\ArgName & \ArgType & \ArrayDim & I/O & \ArgRole \\
\hline
{\it type1} & CHARACTER(8) &  & IN &  種別1  \\
{\it type2} & CHARACTER(4) &  & IN &  種別2  \\
{\it type3} & CHARACTER(4) &  & IN &  種別3  \\
{\it udata} & \AnyType & \AnySize & IN &  データ配列  \\
{\it utype} & CHARACTER(2) &  & IN &  データ配列の型  \\
{\it usize} & INTEGER(4) &  & IN &  配列の要素数  \\
{\it result} & INTEGER(4) &  & OUT & \ResultCode \\
\hline
\end{tabular}
\paragraph{\FuncDesc}
配列 {\it udata} の内容に従って \APILink{nusdas.make.mask}{nusdas\_make\_mask} と同様に
マスクビット列を作成し
指定した種別のデータセットに対して設定する。

\paragraph{\ResultCode}
\begin{quote}
\begin{description}
\item[{\bf 0}] 正常終了
\item[{\bf -5}] 未知の型名 {\it utype} が与えられた
\end{description}\end{quote}

\paragraph{注意}
本関数によるマスクビットの設定は \APILink{nusdas.parameter.change}{nusdas\_parameter\_change} に
優先するが、他のデータセットには効果をもたない。

\paragraph{履歴}
本関数は NuSDaS 1.3 で新設された。

\subsection{NUSDAS\_ONEFILE\_CLOSE: 指定データファイルを閉じる}
\APILabel{nusdas.onefile.close}

\Prototype
\begin{quote}
CALL {\bf NUSDAS\_ONEFILE\_CLOSE}({\it type1}, {\it type2}, {\it type3}, {\it basetime}, {\it member}, {\it validtime}, {\it result})
\end{quote}

\begin{tabular}{l|rllp{16em}}
\hline
\ArgName & \ArgType & \ArrayDim & I/O & \ArgRole \\
\hline
{\it type1} & CHARACTER(8) &  & IN &  種別1  \\
{\it type2} & CHARACTER(4) &  & IN &  種別2  \\
{\it type3} & CHARACTER(4) &  & IN &  種別3  \\
{\it basetime} & INTEGER(4) &  & IN &  基準時刻(通算分)  \\
{\it member} & CHARACTER(4) &  & IN &  メンバー名  \\
{\it validtime} & INTEGER(4) &  & IN &  対象時刻  \\
{\it result} & INTEGER(4) &  & OUT & \ResultCode \\
\hline
\end{tabular}

\paragraph{\ResultCode}
\begin{quote}
\begin{description}
\item[{\bf 0}] 正常終了
\item[{\bf 1}] ファイルクローズ前の書き込み時に定義ファイルを読み込めなかった
\item[{\bf -1}] ファイルクローズ前の書き込み時にIOエラーが発生
\end{description}\end{quote}

\paragraph{\FuncDesc}\paragraph{履歴}
この関数は NuSDaS 1.0 から存在した.

\input{fapi_nusdas_parameter_change}
 \paragraph{Fortran 版の仕様変更}
 NuSDaS 1.2 以前では C API と同様、
 引数 {\it value} に nusdas\_fort.h で定義される変数 NULL を与えると
 パラメタを既定値に戻すことができたが、
 この機能は可搬性の問題から廃止された。
 上述の既定値を明示的に設定するか、
 nusdas\_parameter\_reset() を利用されたい。
\subsection{NUSDAS\_INQ\_PARAMETER: オプション取得}
\APILabel{nusdas.inq.parameter}

\Prototype
\begin{quote}
CALL {\bf NUSDAS\_INQ\_PARAMETER}({\it param}, {\it value}, {\it result})
\end{quote}

\begin{tabular}{l|rllp{16em}}
\hline
\ArgName & \ArgType & \ArrayDim & I/O & \ArgRole \\
\hline
{\it param} & INTEGER(4) &  & IN &  設定項目コード  \\
{\it value} & \AnyType & \AnySize & OUT &  設定値  \\
{\it result} & INTEGER(4) &  & OUT & \ResultCode \\
\hline
\end{tabular}
\paragraph{\FuncDesc}
\APILink{nusdas.parameter.change}{nusdas\_parameter\_change} の項目 {\it param} で設定される
パラメータの値を {\it value} の指す領域 (型は以下を参照) に書き込む。
\begin{quote}\begin{description}
\item[{\bf N\_PC\_MISSING\_UI1}] 1バイト整数の欠損値
\item[{\bf N\_PC\_MISSING\_SI2}] 2バイト整数の欠損値
\item[{\bf N\_PC\_MISSING\_SI4}] 4バイト整数の欠損値
\item[{\bf N\_PC\_MISSING\_R4}] 4バイト実数の欠損値
\item[{\bf N\_PC\_MISSING\_R8}] 8バイト実数の欠損値
\item[{\bf N\_PC\_SIZEX}] 4バイト整数に x 方向強制格子サイズを与える
\item[{\bf N\_PC\_SIZEY}] 4バイト整数に y 方向強制格子サイズを与える
\item[{\bf N\_PC\_MASK\_BIT}] 
マスクビット配列を返す。
この問合せは設定値が \APILink{nusdas.make.mask}{nusdas\_make\_mask} で作られた場合にしか機能しない。
\item[{\bf N\_PC\_PACKING}] 
4バイトの文字列に強制パック方式名を与える。
設定されていない場合は 4 文字のスペースが書き込まれる。
\item[{\bf N\_PC\_ID\_SET}] 
NRD 番号制約がかかっている場合その値、かかっていない場合 -1 を与える。
\item[{\bf N\_PC\_WBUFFER}] 
4バイト整数に書き込みバッファサイズ (実行時オプション FWBF) を与える。
\item[{\bf N\_PC\_RBUFFER}] 
4バイト整数に読み取りバッファサイズ (実行時オプション FRBF) を与える。
\end{description}\end{quote}

\paragraph{\ResultCode}
\begin{quote}
\begin{description}
\item[{\bf 0}] 正常終了
\item[{\bf -1}] サポートされていないパラメタである
\item[{\bf -2}] マスクビット配列は設定されていない
\item[{\bf -3}] マスクビット配列は設定されているが長さがわからない
\end{description}\end{quote}

\paragraph{履歴}
NuSDaS 1.3 で導入された。


\subsection{NUSDAS\_PARAMETER\_RESET: オプションを既定値に戻す}
\APILabel{nusdas.parameter.reset}

\Prototype
\begin{quote}
CALL {\bf NUSDAS\_PARAMETER\_RESET}({\it param}, {\it result})
\end{quote}

\begin{tabular}{l|rllp{16em}}
\hline
\ArgName & \ArgType & \ArrayDim & I/O & \ArgRole \\
\hline
{\it param} & INTEGER(4) &  & IN &  設定項目コード  \\
{\it result} & INTEGER(4) &  & OUT & \ResultCode \\
\hline
\end{tabular}
\paragraph{\FuncDesc}
\APILink{nusdas.parameter.change}{nusdas\_parameter\_change}
で設定されたパラメタを既定値に戻します。

\paragraph{履歴}
この関数 は NuSDaS 1.3 で導入されました。
それ以前のバージョンでは \APILink{nusdas.parameter.change}{nusdas\_parameter\_change} に既定値
または定数 NULL を与える方法が使われていました。


\clearpage
\section{問合せサブルーチン}

\subsection{NUSDAS\_GRID: 格子情報へのアクセス}
\APILabel{nusdas.grid}

\Prototype
\begin{quote}
CALL {\bf NUSDAS\_GRID}({\it type1}, {\it type2}, {\it type3}, {\it basetime}, {\it member}, {\it validtime}, {\it proj}, {\it gridsize}, {\it gridinfo}, {\it value}, {\it getput}, {\it result})
\end{quote}

\begin{tabular}{l|rllp{16em}}
\hline
\ArgName & \ArgType & \ArrayDim & I/O & \ArgRole \\
\hline
{\it type1} & CHARACTER(8) &  & IN &  種別1  \\
{\it type2} & CHARACTER(4) &  & IN &  種別2  \\
{\it type3} & CHARACTER(4) &  & IN &  種別3  \\
{\it basetime} & INTEGER(4) &  & IN &  基準時刻(通算分)  \\
{\it member} & CHARACTER(4) &  & IN &  メンバー名  \\
{\it validtime} & INTEGER(4) &  & IN &  対象時刻(通算分)  \\
{\it proj} & CHARACTER(4) &  & I/O &  投影法3字略号  \\
{\it gridsize} & INTEGER(4) & 2 & I/O &  格子数  \\
{\it gridinfo} & REAL(4) & 14 & I/O &  投影法緒元  \\
{\it value} & CHARACTER(4) &  & I/O &  格子点値が周囲の場を代表する方法  \\
{\it getput} & CHARACTER(3) &  & IN &  入出力指示 ({\it "GET}" または {\it "PUT}")  \\
{\it result} & INTEGER(4) &  & OUT & \ResultCode \\
\hline
\end{tabular}
\paragraph{\FuncDesc}このAPIは、CNTLレコードに格納された格子情報(つまり定義ファイルに書かれた
格子情報)を返す。nusdas\_parameter\_change を使って、定義ファイルに書いた
格子数から変更した場合には正しい情報が得られない。このような場合は 
nusdas\_inq\_data を使う。

gridinfo には4バイト単精度浮動小数点型の配列で14要素存在するものを指定する。

これはCNTLレコードの項番 15 〜 21に対応する。
順に基準点X座標、基準点Y座標、基準点緯度、基準点経度、
X方向格子間隔、Y方向格子間隔、標準緯度、標準経度、第2標準緯度、第2標準経度、
緯度1、経度1、緯度2、経度2となる。

value の値については\TabRef{tab:value}を参照。

\paragraph{\ResultCode}
\begin{quote}
\begin{description}
\item[{\bf 0}] 正常
\item[{\bf -5}] 入出力指示が不正
\end{description}\end{quote}
\paragraph{ 履歴 }
この関数は NuSDaS 1.0 から実装されていた。

\subsection{NUSDAS\_INFO: INFO 記録へのアクセス }
\APILabel{nusdas.info}

\Prototype
\begin{quote}
CALL {\bf NUSDAS\_INFO}({\it type1}, {\it type2}, {\it type3}, {\it basetime}, {\it member}, {\it validtime}, {\it group}, {\it info}, {\it bytesize}, {\it getput}, {\it result})
\end{quote}

\begin{tabular}{l|rllp{16em}}
\hline
\ArgName & \ArgType & \ArrayDim & I/O & \ArgRole \\
\hline
{\it type1} & CHARACTER(8) &  & IN &  種別1  \\
{\it type2} & CHARACTER(4) &  & IN &  種別2  \\
{\it type3} & CHARACTER(4) &  & IN &  種別3  \\
{\it basetime} & INTEGER(4) &  & IN &  基準時刻(通算分)  \\
{\it member} & CHARACTER(4) &  & IN &  メンバー名  \\
{\it validtime} & INTEGER(4) &  & IN &  対象時刻(通算分)  \\
{\it group} & CHARACTER(4) &  & IN &  群名  \\
{\it info} & CHARACTER & \AnySize & I/O &  INFO 記録内容  \\
{\it bytesize} & INTEGER(4) &  & IN &  INFO 記録のバイト数  \\
{\it getput} & CHARACTER(3) &  & IN &  入出力指示 ({\it "GET}" または {\it "PUT}")  \\
{\it result} & INTEGER(4) &  & OUT & \ResultCode \\
\hline
\end{tabular}
\paragraph{\FuncDesc}\paragraph{\ResultCode}
\begin{quote}
\begin{description}
\item[{\bf 非負}] 書き出したINFOのバイト数
\item[{\bf -3}] バッファが不足している
\item[{\bf -5}] 入出力指示が不正
\end{description}\end{quote}

\paragraph{ 注意 }
NuSDaS1.1では、バッファが不足している場合でもバッファの大きさの分だけを
書き込み、そのサイズを返していたが、 NuSDaS 1.3 からはこのような場合は-3が返る。
また、INFO のサイズは NuSDaS 1.3 で新設された nusdas\_inq\_subcinfo で
問い合わせ項目を N\_INFO\_NUM にすれば得ることができる。

\subsection{NUSDAS\_INQ\_CNTL: データファイルの諸元問合せ }
\APILabel{nusdas.inq.cntl}

\Prototype
\begin{quote}
CALL {\bf NUSDAS\_INQ\_CNTL}({\it type1}, {\it type2}, {\it type3}, {\it basetime}, {\it member}, {\it validtime}, {\it param}, {\it data}, {\it datasize}, {\it result})
\end{quote}

\begin{tabular}{l|rllp{16em}}
\hline
\ArgName & \ArgType & \ArrayDim & I/O & \ArgRole \\
\hline
{\it type1} & CHARACTER(8) &  & IN &  種別1  \\
{\it type2} & CHARACTER(4) &  & IN &  種別2  \\
{\it type3} & CHARACTER(4) &  & IN &  種別3  \\
{\it basetime} & INTEGER(4) &  & IN &  基準時刻(通算分)  \\
{\it member} & CHARACTER(4) &  & IN &  メンバー名  \\
{\it validtime} & INTEGER(4) &  & IN &  対象時刻(通算分)  \\
{\it param} & INTEGER(4) &  & IN &  問合せ項目コード  \\
{\it data} & \AnyType & \AnySize & OUT &  問合せ結果配列  \\
{\it datasize} & INTEGER(4) &  & IN &  問合せ結果配列の要素数  \\
{\it result} & INTEGER(4) &  & OUT & \ResultCode \\
\hline
\end{tabular}
\paragraph{\FuncDesc}引数 {\it type1} から {\it validtime} で指定されるデータファイルに書かれた
CNTL 記録について、
引数 {\it param} で指定される問合せを行う。
\begin{quote}\begin{description}
\item[{\bf N\_MEMBER\_NUM}] 
メンバーの個数が4バイト整数型変数 {\it data} に書かれる。
\item[{\bf N\_MEMBER\_LIST}] 
データファイルに定義されたメンバー名が配列 {\it data} に書かれる。
配列 {\it data} は長さ 4 文字の文字型で
{\it N\_MEMBER\_NUM} 要素存在しなければならない。
\item[{\bf N\_VALIDTIME\_NUM}] 
validtimeの個数が4バイト整数型変数 {\it data} に書かれる。
\item[{\bf N\_VALIDTIME\_LIST}] 
データファイルに定義されたvalidtimeが配列 {\it data} に書かれる。
配列 {\it data} は長さ 4 byte整数型で
{\it N\_VALIDTIME\_NUM} 要素存在しなければならない。
\item[{\bf N\_VALIDTIME\_LIST2}] 
データファイルに定義されたvalidtime2が配列 {\it data} に書かれる。
配列 {\it data} は長さ 4 byte整数型で
{\it N\_VALIDTIME\_NUM} 要素存在しなければならない。
\item[{\bf N\_PLANE\_NUM}] 
面の個数が4バイト整数型変数 {\it data} に書かれる。
\item[{\bf N\_PLANE\_LIST}] 
データファイルに定義された面の名前が配列 {\it data} に書かれる。
配列 {\it data} は長さ 6 文字の文字型で
{\it N\_PLANE\_NUM} 要素存在しなければならない。
\item[{\bf N\_PLANE\_LIST2}] 
N\_PLANE\_LIST と全く同じ動作である。
\item[{\bf N\_ELEMENT\_NUM}] 
要素の個数が4バイト整数型変数 {\it data} に書かれる。
\item[{\bf N\_ELEMENT\_LIST}] 
データファイルに定義された要素の名前が配列 {\it data} に書かれる。
配列 {\it data} は長さ 6 文字の文字型で
{\it N\_ELEMENT\_NUM} 要素存在しなければならない。
\item[{\bf  N\_NUSD\_NBYTES }] 
NUSD レコードのサイズ(単位バイト)が4バイト整数型変数 {\it data} に書か
れる。(先頭・末尾に付加されるレコード長の大きさ(4$\ast$2バイト)を含む)
\item[{\bf  N\_NUSD\_CONTENT }] 
NUSD レコードの内容を配列 {\it data} に格納する。配列 {\it data} は
\newline N\_NUSD\_NBYTES バイト存在しなくてはならない。
(先頭・末尾に付加されるレコード長を含む)
\item[{\bf  N\_CNTL\_NBYTES }] 
CNTL レコードのサイズ(単位バイト)が4バイト整数型変数 {\it data} に書か
れる。(先頭・末尾に付加されるレコード長の大きさ(4$\ast$2バイト)を含む)
\item[{\bf  N\_CNTL\_CONTENT }] 
CNTL レコードの内容を配列 {\it data} に格納する。配列 {\it data} は
\newline N\_CNTL\_NBYTES バイト存在しなくてはならない。
(先頭・末尾に付加されるレコード長を含む)
\item[{\bf  N\_PROJECTION }] 
地図投影法の情報を4文字の文字型 {\it data} に格納する
(記号の意味は巻末の表参照)。
\item[{\bf  N\_GRID\_SIZE }] 
X方向、Y方向の格子数がこの順序で4バイト整数型の配列 {\it data} に
書かれる。配列 {\it data} は 2 要素存在しなくてはならない。
(この問い合わせは NuSDaS 1.3 で追加)
\item[{\bf  N\_GRID\_BASEPOINT }] 
基準点のx座標、y座標、緯度、経度がこの順序で4バイト単精度浮動小数点型の配
列 {\it data} に書かれる。配列 {\it data} は 4 要素存在しなくてはならない。
(この問い合わせは NuSDaS 1.3 で追加)
\item[{\bf  N\_GRID\_DISTANCE }] 
X方向、Y方向の格子間隔がこの順序で4バイト単精度浮動小数点型の配列
{\it data} に書かれる。配列 {\it data} は 2 要素存在しなくてはならない。
(この問い合わせは NuSDaS 1.3 で追加)
\item[{\bf  N\_STAND\_LATLON }] 
標準緯度、標準経度、第2標準緯度、第2標準経度がこの順序で
4バイト単精度浮動小数点型の配列 {\it data} に書かれる。
配列 {\it data} は 4 要素存在しなくてはならない。
(この問い合わせは NuSDaS 1.3 で追加)
\item[{\bf  N\_SPARE\_LATLON }] 
緯度1、経度1、緯度2、経度2がこの順序で
4バイト単精度浮動小数点型の配列 {\it data} に書かれる。
配列 {\it data} は 4 要素存在しなくてはならない。
(この問い合わせは NuSDaS 1.3 で追加)
\item[{\bf  N\_INDX\_SIZE }] 
INDX の個数が 4バイト整数型の変数 {\it data} に書かれる。
(この問い合わせは NuSDaS 1.3 で追加)
\item[{\bf  N\_ELEMENT\_MAP }] 
データの格納が許容されているか否かが1 or 0 によって、1バイト整数型
の配列 {\it data} に書かれる。配列 {\it data} は {\it N\_INDX\_SIZE} 要素存在
しなくてはならない。{\it dataはメンバー、validtime}, 面、要素をインデック
スにした配列で、それぞれの順序は {\it N\_MEMBER\_LIST}, 
{\it N\_VALIDTIME\_LIST}, {\it N\_PLANE\_LIST}, {\it N\_ELEMENT\_LIST}の問い合わせ
結果と一致する。
(この問い合わせは NuSDaS 1.3 で追加)
\item[{\bf  N\_DATA\_MAP }] 
データが書き込まれているか否かが1 or 0 によって、1バイト整数型
の配列 {\it data} に書かれる。配列 {\it data} は {\it N\_INDX\_SIZE} 要素存在
しなくてはならない。{\it dataはメンバー、validtime}, 面、要素をインデック
スにした配列で、それぞれの順序は {\it N\_MEMBER\_LIST}, 
{\it N\_VALIDTIME\_LIST}, {\it N\_PLANE\_LIST}, {\it N\_ELEMENT\_LIST}の問い合わせ
結果と一致する。
(この問い合わせは NuSDaS 1.3 で追加)
\end{description}\end{quote}

\paragraph{\ResultCode}
\begin{quote}
\begin{description}
\item[{\bf 正}] 格納要素数
\item[{\bf -1}] データの配列数が不足している。
\item[{\bf -2}] データの配列が確保されていない。
\item[{\bf -3}] 問い合わせ項目が不正
\end{description}\end{quote}
\paragraph{ 注意 }
NuSDaS1.1以前では、同じ構造のデータセットでも
N\_VALIDTIME\_NUM, N\_VALIDTIME\_LIST の問い合わせ結果が
1つの basetime に複数の validtime を格納するか否かによって異なっていた。
これは、validtime でファイルを分ける
(異なる validtime のファイルが異なる) 設定ならば
データファイルには 1 つの validtime だけが書かれていたからである。
しかし NuSDaS 1.3 からは定義ファイルに指定されたすべての validtime が
各データファイルの validtime に格納されているので、問い合わせ結果は
格納形態を問わず一定である。

\subsection{NUSDAS\_INQ\_DATA: データ記録の諸元問合せ}
\APILabel{nusdas.inq.data}

\Prototype
\begin{quote}
CALL {\bf NUSDAS\_INQ\_DATA}({\it type1}, {\it type2}, {\it type3}, {\it basetime}, {\it member}, {\it validtime}, {\it plane}, {\it element}, {\it param}, {\it data}, {\it nelems}, {\it result})
\end{quote}

\begin{tabular}{l|rllp{16em}}
\hline
\ArgName & \ArgType & \ArrayDim & I/O & \ArgRole \\
\hline
{\it type1} & CHARACTER(8) &  & IN &  種別1  \\
{\it type2} & CHARACTER(4) &  & IN &  種別2  \\
{\it type3} & CHARACTER(4) &  & IN &  種別3  \\
{\it basetime} & INTEGER(4) &  & IN &  基準時刻(通算分)  \\
{\it member} & CHARACTER(4) &  & IN &  メンバー名  \\
{\it validtime} & INTEGER(4) &  & IN &  対象時刻(通算分)  \\
{\it plane} & CHARACTER(6) &  & IN &  面  \\
{\it element} & CHARACTER(6) &  & IN &  要素名  \\
{\it param} & INTEGER(4) &  & IN &  問合せ項目コード  \\
{\it data} & \AnyType & \AnySize & OUT &  結果格納配列  \\
{\it nelems} & INTEGER(4) &  & IN &  結果格納配列の要素数  \\
{\it result} & INTEGER(4) &  & OUT & \ResultCode \\
\hline
\end{tabular}
\paragraph{\FuncDesc}
引数 {\it type1} から {\it element} までで指定されるデータ記録について
引数 {\it query} で指定される問合せを行う。

\begin{quote}\begin{description}
\item[{\bf N\_DATA\_QUADRUPLET}] 
16 バイトのメモリ領域を引数に取り、 N\_GRID\_SIZE から
\newline N\_MISSING\_VALUE までの情報が返される。
\item[{\bf N\_GRID\_SIZE}] 
引数 {\it data} に4バイト整数の長さ2の配列を取り、
そこに X, Y 方向の格子数が書かれる。
\item[{\bf N\_PC\_PACKING}] 
引数 {\it data} に4バイトの文字列を取り、
そこにパック方式名称が書かれる。
文字列はヌル終端されないことに注意。
\item[{\bf N\_MISSING\_MODE}] 
引数 {\it data} に4バイトの文字列を取り、
そこに欠損値表現方式名が書かれる。
文字列はヌル終端されないことに注意。
\item[{\bf N\_MISSING\_VALUE}] 
引数には上述 N\_PC\_PACKING 項目によって決まる型の変数を取り、
そこにデータ記録上の欠損値が書かれる。
この値は \APILink{nusdas.read}{nusdas\_read} で得られる配列で用いられる
欠損値とは異なることに注意。
\item[{\bf N\_DATA\_EXIST}] 
引数 {\it data} に4バイト整数型変数をとり、
そこにデータの存在有無を示す値が書かれる。
0はデータの不在、1は存在を示す。
\item[{\bf N\_DATA\_NBYTES}] 
引数 {\it data} に4バイト整数型変数をとり、
そこにデータ記録のバイト数が書かれる。
\item[{\bf N\_DATA\_CONTENT}] 
引数 {\it data} が指すバイト列にデータ記録がそのまま書かれる。
データ記録は、DATA レコードのフォーマット表\ref{table.fmt.data}の
項番10〜13までのデータが格納される。
\item[{\bf N\_RECORD\_TIME}] 
引数 {\it data} に4バイト整数型変数をとり、
そこにデータ記録の作成時刻が書かれる。
この問合せはデータ記録の更新確認用に用意されており、
結果は大小比較だけに用いるべきもので日時等を算出すべきではない。
この値は time システムコールの返す値の下位 32 ビットであり、
2038 年問題の対策のためいずれ機種依存の意味を持つように
なるものと思われる。
\end{description}\end{quote}
\paragraph{\ResultCode}
\begin{quote}
\begin{description}
\item[{\bf 正}] 格納要素数
\item[{\bf -1}] データの配列数が不足している
\item[{\bf -2}] データの配列が確保されていない
\item[{\bf -3}] 問い合わせ項目が不正 
\end{description}\end{quote}
\paragraph{ 履歴 }
この関数は pnusdas では実装はされていたが、ドキュメント化されていなかった。

\input{fapi_nusdas_inq_def}
\subsection{NUSDAS\_INQ\_NRDBTIME: データセットの基準時刻リスト取得}
\APILabel{nusdas.inq.nrdbtime}

\Prototype
\begin{quote}
CALL {\bf NUSDAS\_INQ\_NRDBTIME}({\it type1}, {\it type2}, {\it type3}, {\it btlist}, {\it btlistsize}, {\it pflag}, {\it result})
\end{quote}

\begin{tabular}{l|rllp{16em}}
\hline
\ArgName & \ArgType & \ArrayDim & I/O & \ArgRole \\
\hline
{\it type1} & CHARACTER(8) &  & IN &  種別1  \\
{\it type2} & CHARACTER(4) &  & IN &  種別2  \\
{\it type3} & CHARACTER(4) &  & IN &  種別3  \\
{\it btlist} & INTEGER(4) & 可変 & OUT &  基準時刻が格納される配列  \\
{\it btlistsize} & INTEGER(4) &  & IN &  配列の要素数  \\
{\it pflag} & INTEGER(4) &  & IN &  動作過程印字フラグ  \\
{\it result} & INTEGER(4) &  & OUT & \ResultCode \\
\hline
\end{tabular}
\paragraph{\FuncDesc}
種別1〜種別3で指示されるデータセットに存在する基準時刻を
配列 {\it btlist} に書き込む。
引数 {\it pflag} に非零値を設定すると動作過程の情報を警告メッセージとして
印字するようになる。
\paragraph{\ResultCode}
\begin{quote}
\begin{description}
\item[{\bf 非負}] 基準時刻の個数
\item[{\bf -1}] ファイル I/O エラー
\item[{\bf -2}] ファイルに管理部が存在しない
\item[{\bf -3}] ファイルのレコード長が不正
\item[{\bf -4}] ファイルあるいはディレクトリのオープンに失敗
\end{description}\end{quote}
\paragraph{履歴}
本関数は NuSDaS 1.0 から存在した。
\paragraph{注意}
\begin{itemize}
\item 
配列長 {\it btlistsize} より多くの基準時刻が存在する場合は、
配列長を越えて書き込むことはない。リターンコードと配列長を比較して、
リターンコードが大きかったらその数だけ配列を確保し直して
本関数を呼び直すことにより、すべてのリストを得ることができる。
\item 
NuSDaS 1.1 までは見付かったデータセットがネットワークでなければ、
それについてだけ探索が行われた。
NuSDaS 1.3 からは、
指定した種別にマッチするすべてのデータセットについて探索が行われる。
\item 
種別に対応するデータセットが見つからない場合
(たとえば種別名を間違えた場合)、
返却値はゼロとなる。
データセットが存在して空の場合と異なり、
このとき ``Can not find NUSDAS root directory for selected type1-3''
``type1$<$...$>$ type2$<$...$>$ type3$<$...$>$ NRD=...''
というメッセージが標準エラー出力に表示される。
NRD= の後の数値が -1 でなければ、
NRD 番号を指定したために存在しているデータセットが見つからなくなっている
可能性がある。
\end{itemize}

\subsection{NUSDAS\_INQ\_NRDVTIME: データセットの対象時刻リスト取得}
\APILabel{nusdas.inq.nrdvtime}

\Prototype
\begin{quote}
CALL {\bf NUSDAS\_INQ\_NRDVTIME}({\it type1}, {\it type2}, {\it type3}, {\it vtlist}, {\it vtlistsize}, {\it basetime}, {\it pflag}, {\it result})
\end{quote}

\begin{tabular}{l|rllp{16em}}
\hline
\ArgName & \ArgType & \ArrayDim & I/O & \ArgRole \\
\hline
{\it type1} & CHARACTER(8) &  & IN &  種別1  \\
{\it type2} & CHARACTER(4) &  & IN &  種別2  \\
{\it type3} & CHARACTER(4) &  & IN &  種別3  \\
{\it vtlist} & INTEGER(4) & 可変 & OUT &  対象時刻が書かれる配列  \\
{\it vtlistsize} & INTEGER(4) &  & IN &  配列の要素数  \\
{\it basetime} & INTEGER(4) &  & IN &  基準時刻(通算分)  \\
{\it pflag} & INTEGER(4) &  & IN &  動作詳細印字フラグ  \\
{\it result} & INTEGER(4) &  & OUT & \ResultCode \\
\hline
\end{tabular}
\paragraph{\FuncDesc}
種別1〜種別3で指示されるデータセットに基準時刻 {\it basetime} のもとで
存在する対象時刻を配列 {\it vtlist} に書き込む。
引数 {\it pflag} に非零値を設定すると動作過程の情報を警告メッセージとして
印字するようになる。
\paragraph{\ResultCode}
\begin{quote}
\begin{description}
\item[{\bf 非負}] 対象時刻の個数
\end{description}\end{quote}
\paragraph{履歴}
本関数は NuSDaS 1.0 から存在したがドキュメントされていなかった。
\paragraph{注意}
\begin{itemize}
\item 配列長 {\it vtlistsize} より多くの対象時刻が存在する場合は、
配列長を越えて書き込むことはない。リターンコードと配列長を比較して、
リターンコードが大きかったらその数だけ配列を確保し直して
本関数を呼び直すことにより、すべてのリストを得ることができる。
\item 対象時刻の探索はファイルの有無または CNTL レコードによる。
リスト中の対象時刻についてデータレコードが書かれていない場合もありうる。
\item 基準時刻 {\it basetime} に -1 を指定すると、
基準時刻を問わない検索になる。
\item 検索にあたってメンバー名は問わない。
\item 
NuSDaS 1.1 までは見付かったデータセットがネットワークでなければ、
それについてだけ探索が行われた。
NuSDaS 1.3 からは、
指定した種別にマッチするすべてのデータセットについて探索が行われる。
\item 
種別に対応するデータセットが見つからない場合
(たとえば種別名を間違えた場合)、
返却値はゼロとなる。
データセットが存在して空の場合と異なり、
このとき ``Can not find NUSDAS root directory for selected type1-3''
``type1$<$...$>$ type2$<$...$>$ type3$<$...$>$ NRD=...''
というメッセージが標準エラー出力に表示される。
NRD= の後の数値が -1 でなければ、
NRD 番号を指定したために存在しているデータセットが見つからなくなっている
可能性がある。
\end{itemize}

\subsection{NUSDAS\_INQ\_SUBCINFO: SUBC/INFO の問合せ}
\APILabel{nusdas.inq.subcinfo}

\Prototype
\begin{quote}
CALL {\bf NUSDAS\_INQ\_SUBCINFO}({\it type1}, {\it type2}, {\it type3}, {\it basetime}, {\it member}, {\it validtime}, {\it query}, {\it group}, {\it buf}, {\it bufnelems}, {\it result})
\end{quote}

\begin{tabular}{l|rllp{16em}}
\hline
\ArgName & \ArgType & \ArrayDim & I/O & \ArgRole \\
\hline
{\it type1} & CHARACTER(8) &  & IN &  種別1  \\
{\it type2} & CHARACTER(4) &  & IN &  種別2  \\
{\it type3} & CHARACTER(4) &  & IN &  種別3  \\
{\it basetime} & INTEGER(4) &  & IN &  基準時刻  \\
{\it member} & CHARACTER(4) &  & IN &  メンバー  \\
{\it validtime} & INTEGER(4) &  & IN &  対象時刻  \\
{\it query} & INTEGER(4) &  & IN &  問合せ項目  \\
{\it group} & CHARACTER(4) &  & IN &  群名  \\
{\it buf} & \AnyType & \AnySize & OUT &  結果格納配列  \\
{\it bufnelems} & INTEGER(4) &  & IN &  結果格納配列の要素数  \\
{\it result} & INTEGER(4) &  & OUT & \ResultCode \\
\hline
\end{tabular}
\paragraph{\FuncDesc}
引数 {\it type1} から {\it validtime} で指定されるデータファイルに書かれた
SUBC または INFO 記録について、
引数 {\it query} で指定される問合せを行う。
\begin{quote}\begin{description}
\item[{\bf N\_SUBC\_NUM}] 
SUBC 記録の個数が4バイト整数型変数 {\it buf} に書かれる。
引数 {\it group} は無視される。
\item[{\bf N\_SUBC\_LIST}] 
データファイルに定義された SUBC 記録の群名が配列 {\it buf} に書かれる。
配列 {\it buf} は長さ 4 文字の文字型で
{\it N\_SUBC\_NUM} 要素存在しなければならない。
引数 {\it group} は無視される。
\item[{\bf N\_SUBC\_NBYTES}] 
群名 {\it group} の SUBC 記録のバイト数が4バイト整数型変数 {\it buf} に書かれる。
\item[{\bf N\_SUBC\_CONTENT}] 
群名 {\it group} の SUBC 記録が配列 {\it buf} に書かれる。
上述のバイト数だけの長さを確保しておかねばならない。
\item[{\bf N\_INFO\_NUM}] 
INFO 記録の個数が4バイト整数型変数 {\it buf} に書かれる。
引数 {\it group} は無視される。
\item[{\bf N\_INFO\_LIST}] 
データファイルに定義された INFO 記録の群名が配列 {\it buf} に書かれる。
配列 {\it buf} は長さ 4 文字の文字型で
{\it N\_INFO\_NUM} 要素存在しなければならない。
引数 {\it group} は無視される。
\item[{\bf N\_INFO\_NBYTES}] 
群名 {\it group} の INFO 記録のバイト数が4バイト整数型変数 {\it buf} に書かれる。
\end{description}\end{quote}

\paragraph{\ResultCode}
\begin{quote}
\begin{description}
\item[{\bf 正}] 格納要素数
\end{description}\end{quote}
\paragraph{履歴}
この関数は NuSDaS 1.3 で新設された。


\paragraph{注意}
「レコード内容」として取得されるのは表 \ref{table.fmt.subc} 項番 6 と同じで
あり、その長さはレコード有効長から 16 を引いたものに等しい。

\subsection{NUSDAS\_SCAN\_DS: データセットの一覧}
\APILabel{nusdas.scan.ds}

\Prototype
\begin{quote}
CALL {\bf NUSDAS\_SCAN\_DS}({\it type1}, {\it type2}, {\it type3}, {\it nrd}, {\it result})
\end{quote}

\begin{tabular}{l|rllp{16em}}
\hline
\ArgName & \ArgType & \ArrayDim & I/O & \ArgRole \\
\hline
{\it type1} & CHARACTER(8) &  & OUT &  種別1  \\
{\it type2} & CHARACTER(4) &  & OUT &  種別2  \\
{\it type3} & CHARACTER(4) &  & OUT &  種別3  \\
{\it nrd} & INTEGER(4) &  & OUT &  NRD番号 \\
{\it result} & INTEGER(4) &  & OUT & \ResultCode \\
\hline
\end{tabular}
\paragraph{\FuncDesc}
返却値が負になるまで呼出しを繰り返すと、ライブラリが認識している
データセットの一覧が得られる。

\paragraph{\ResultCode}
\begin{quote}
\begin{description}
\item[{\bf 0}] 引数の配列にデータセットの情報が格納された。
\item[{\bf -1}] もうこれ以上データセットは認識されていない。
\end{description}\end{quote}
\paragraph{履歴}
この関数は NuSDaS 1.3 で追加された。
pnusdas には非公開の nusdas\_list\_type という関数があり類似の機能を持つ。


\clearpage
\section{メタデータ用サブルーチン}

\subsection{NUSDAS\_SUBC\_DELT: SUBC DELT へのアクセス }
\APILabel{nusdas.subc.delt}

\Prototype
\begin{quote}
CALL {\bf NUSDAS\_SUBC\_DELT}({\it type1}, {\it type2}, {\it type3}, {\it basetime}, {\it member}, {\it validtime}, {\it delt}, {\it getput}, {\it result})
\end{quote}

\begin{tabular}{l|rllp{16em}}
\hline
\ArgName & \ArgType & \ArrayDim & I/O & \ArgRole \\
\hline
{\it type1} & CHARACTER(8) &  & IN &  種別1  \\
{\it type2} & CHARACTER(4) &  & IN &  種別2  \\
{\it type3} & CHARACTER(4) &  & IN &  種別3  \\
{\it basetime} & INTEGER(4) &  & IN &  基準時刻(通算分)  \\
{\it member} & CHARACTER(4) &  & IN &  メンバー名  \\
{\it validtime} & INTEGER(4) &  & IN &  対象時刻(通算分)  \\
{\it delt} & REAL(4) &  & I/O &  DELT 数値へのポインタ  \\
{\it getput} & CHARACTER(3) &  & IN &  入出力指示 ({\it "GET}" または {\it "PUT}")  \\
{\it result} & INTEGER(4) &  & OUT & \ResultCode \\
\hline
\end{tabular}
\paragraph{\FuncDesc}モデルの時間積分間隔を補助管理情報に記録しておくものである。
\paragraph{\ResultCode}
\begin{quote}
\begin{description}
\item[{\bf 0}] 正常終了
\item[{\bf -2}] レコードが存在しない、または書き込まれていない。
\item[{\bf -3}] レコードサイズが不正
\item[{\bf -5}] 入出力指示が不正
\end{description}\end{quote}
\paragraph{ 履歴 }
この関数は NuSDaS1.2で導入された。

\subsection{NUSDAS\_SUBC\_DELT\_PRESET1: SUBC DELT のデフォルト設定}
\APILabel{nusdas.subc.delt.preset1}

\Prototype
\begin{quote}
CALL {\bf NUSDAS\_SUBC\_DELT\_PRESET1}({\it type1}, {\it type2}, {\it type3}, {\it delt}, {\it result})
\end{quote}

\begin{tabular}{l|rllp{16em}}
\hline
\ArgName & \ArgType & \ArrayDim & I/O & \ArgRole \\
\hline
{\it type1} & CHARACTER(8) &  & IN &  種別1  \\
{\it type2} & CHARACTER(4) &  & IN &  種別2  \\
{\it type3} & CHARACTER(4) &  & IN &  種別3  \\
{\it delt} & REAL(4) &  & IN &  DELT 数値へのポインタ  \\
{\it result} & INTEGER(4) &  & OUT & \ResultCode \\
\hline
\end{tabular}
\paragraph{\FuncDesc}ファイルが新たに生成される際にDELTレコードに書き込む値を設定する。
DELT レコードや引数についてはnusdas\_subc\_delt を参照。
\paragraph{\ResultCode}
\begin{quote}
\begin{description}
\item[{\bf 0}] 正常終了
\item[{\bf -1}] 定義ファイルに "DELT" が登録されていない
\item[{\bf -2}] メモリの確保に失敗した
\end{description}\end{quote}
\paragraph{ 互換性 }
NuSDaS1.1 では、一つのNuSDaSデータセットに設定できる補助管理部の数は最大
10 に制限されており、それを超えると-2が返された。一方、 NuSDaS 1.3 からは
メモリが確保できる限り数に制限はなく、-2 をメモリ確保失敗のエラーコードに
読み替えている。

\subsection{NUSDAS\_SUBC\_ETA: SUBC ETA へのアクセス}
\APILabel{nusdas.subc.eta}

\Prototype
\begin{quote}
CALL {\bf NUSDAS\_SUBC\_ETA}({\it type1}, {\it type2}, {\it type3}, {\it basetime}, {\it member}, {\it validtime}, {\it n\_levels}, {\it a}, {\it b}, {\it c}, {\it getput}, {\it result})
\end{quote}

\begin{tabular}{l|rllp{16em}}
\hline
\ArgName & \ArgType & \ArrayDim & I/O & \ArgRole \\
\hline
{\it type1} & CHARACTER(8) &  & IN &  種別1  \\
{\it type2} & CHARACTER(4) &  & IN &  種別2  \\
{\it type3} & CHARACTER(4) &  & IN &  種別3  \\
{\it basetime} & INTEGER(4) &  & IN &  基準時刻(通算分)  \\
{\it member} & CHARACTER(4) &  & IN &  メンバー名  \\
{\it validtime} & INTEGER(4) &  & IN &  対象時刻(通算分)  \\
{\it n\_levels} & INTEGER(4) &  & I/O &  鉛直層数  \\
{\it a} & REAL(4) & \AnySize & I/O &  係数 a  \\
{\it b} & REAL(4) & \AnySize & I/O &  係数 b  \\
{\it c} & REAL(4) &  & I/O &  係数 c  \\
{\it getput} & CHARACTER(3) &  & IN &  入出力指示 ({\it "GET}" または {\it "PUT}")  \\
{\it result} & INTEGER(4) &  & OUT & \ResultCode \\
\hline
\end{tabular}
\paragraph{\FuncDesc}鉛直座標に ETA 座標系を用いるときに、鉛直座標を定めるパラメータへの
アクセスを提供する。 
パラメータは4バイト単精度浮動小数点型の配列{\it a}, {\it b}, {\it c} で構成され、
{\it a}, {\it b}, は鉛直層数 {\it n\_levels} に対して、{\it n\_levels}+1 要素の配列、
{\it c} は1要素の配列(変数)を確保する必要がある。
{\it n\_levels} は nusdas\_subc\_inq\_nz で問い合わせることができる。
入出力指示が {\it GET} の場合においても、{\it n\_levels} は書込み対象変数として扱われる。
特に parameter 宣言された変数を {\it n\_levels} に指定してはならない。
\paragraph{\ResultCode}
\begin{quote}
\begin{description}
\item[{\bf 0}] 正常終了
\item[{\bf -2}] レコードが存在しない、またはレコードの書き込みがされていない。
\item[{\bf -3}] レコードサイズが不正
\item[{\bf -4}] ユーザーの鉛直層数がファイルの中の鉛直層数より小さい
\item[{\bf -5}] 入出力指示が不正。
\end{description}\end{quote}
\paragraph{ 履歴 }
この関数は NuSDaS1.0 から存在した。
NuSDaS1.1までは、レコードが書き込まれたかの情報を持ち合わせていなかった
ために無記録のレコードをファイルから読んで正常終了していた。 NuSDaS 1.3 からは
ファイルの初期化時にレコードを初期化し、未記録を判定できるようにした。
その場合のエラーは-2としている。
\paragraph{ 注意 }
SUBC ETA に使われている鉛直層数 {\it n\_levels} は実際のモデルの鉛直層数と
異なっている場合があるので、配列確保の際にはnusdas\_subc\_inq\_nzで問い
合わせた結果を用いること。

\subsection{NUSDAS\_SUBC\_ETA\_INQ\_NZ: SUBC 記録の鉛直層数問合せ}
\APILabel{nusdas.subc.eta.inq.nz}

\Prototype
\begin{quote}
CALL {\bf NUSDAS\_SUBC\_ETA\_INQ\_NZ}({\it type1}, {\it type2}, {\it type3}, {\it basetime}, {\it member}, {\it validtime}, {\it group}, {\it n\_levels}, {\it result})
\end{quote}

\begin{tabular}{l|rllp{16em}}
\hline
\ArgName & \ArgType & \ArrayDim & I/O & \ArgRole \\
\hline
{\it type1} & CHARACTER(8) &  & IN &  種別1  \\
{\it type2} & CHARACTER(4) &  & IN &  種別2  \\
{\it type3} & CHARACTER(4) &  & IN &  種別3  \\
{\it basetime} & INTEGER(4) &  & IN &  基準時刻(通算分)  \\
{\it member} & CHARACTER(4) &  & IN &  メンバー名  \\
{\it validtime} & INTEGER(4) &  & IN &  対象時刻(通算分)  \\
{\it group} & CHARACTER(4) &  & IN &  群名  \\
{\it n\_levels} & INTEGER(4) &  & OUT &  鉛直層数  \\
{\it result} & INTEGER(4) &  & OUT & \ResultCode \\
\hline
\end{tabular}
\paragraph{\FuncDesc}SUBC レコードの ETA, SIGM, ZHYB に記録された鉛直層数を問い合わせる。
群名には "ETA ", "SIGM", "ZHYB" のいずれかを指定する。
\paragraph{\ResultCode}
\begin{quote}
\begin{description}
\item[{\bf 正}] 正常終了
\end{description}\end{quote}
\paragraph{ 履歴 }
この関数は NuSDaS1.2 で導入された。

\subsection{NUSDAS\_SUBC\_PRESET1: SUBC ETA/SIGM のデフォルト値設定 }
\APILabel{nusdas.subc.preset1}

\Prototype
\begin{quote}
CALL {\bf NUSDAS\_SUBC\_PRESET1}({\it type1}, {\it type2}, {\it type3}, {\it group}, {\it n\_levels}, {\it a}, {\it b}, {\it c}, {\it result})
\end{quote}

\begin{tabular}{l|rllp{16em}}
\hline
\ArgName & \ArgType & \ArrayDim & I/O & \ArgRole \\
\hline
{\it type1} & CHARACTER(8) &  & IN &  種別1  \\
{\it type2} & CHARACTER(4) &  & IN &  種別2  \\
{\it type3} & CHARACTER(4) &  & IN &  種別3  \\
{\it group} & CHARACTER(4) &  & IN &  群名  \\
{\it n\_levels} & INTEGER(4) &  & IN &  鉛直層数  \\
{\it a} & REAL(4) & \AnySize & I/O &  係数 a  \\
{\it b} & REAL(4) & \AnySize & I/O &  係数 b  \\
{\it c} & REAL(4) &  & I/O &  係数 c  \\
{\it result} & INTEGER(4) &  & OUT & \ResultCode \\
\hline
\end{tabular}
\paragraph{\FuncDesc}ファイルが新たに生成される際にETA, SIGMに書き込む値を設定する。
SIGM や引数については nusdas\_subc\_eta を参照。
引数の「群名」には、"ETA " または "SIGM" を指定する。
\paragraph{\ResultCode}
\begin{quote}
\begin{description}
\item[{\bf 0}] 正常終了
\item[{\bf -1}] 定義ファイルに指定した群名が登録されていない
\item[{\bf -2}] メモリの確保に失敗した
\item[{\bf -3}] レコードのサイズが不正
\end{description}\end{quote}

\paragraph{ 互換性 }
NuSDaS1.1 では、一つのNuSDaSデータセットに設定できる補助管理部の数は最大
10 に制限されており、それを超えると-2が返された。一方、 NuSDaS 1.3 からは
メモリが確保できる限り数に制限はなく、-2 をメモリ確保失敗のエラーコードに
読み替えている。

\subsection{NUSDAS\_SUBC\_RGAU: SUBC RGAU へのアクセス }
\APILabel{nusdas.subc.rgau}

\Prototype
\begin{quote}
CALL {\bf NUSDAS\_SUBC\_RGAU}({\it type1}, {\it type2}, {\it type3}, {\it basetime}, {\it member}, {\it validtime}, {\it j}, {\it j\_start}, {\it j\_n}, {\it i}, {\it i\_start}, {\it i\_n}, {\it lat}, {\it getput}, {\it result})
\end{quote}

\begin{tabular}{l|rllp{16em}}
\hline
\ArgName & \ArgType & \ArrayDim & I/O & \ArgRole \\
\hline
{\it type1} & CHARACTER(8) &  & IN &  種別1  \\
{\it type2} & CHARACTER(4) &  & IN &  種別2  \\
{\it type3} & CHARACTER(4) &  & IN &  種別3  \\
{\it basetime} & INTEGER(4) &  & IN &  基準時刻(通算分)  \\
{\it member} & CHARACTER(4) &  & IN &  メンバー名  \\
{\it validtime} & INTEGER(4) &  & IN &  対象時刻(通算分)  \\
{\it j} & INTEGER(4) &  & I/O &  全球の南北分割数  \\
{\it j\_start} & INTEGER(4) &  & I/O &  データの最北格子の番号(1始まり)  \\
{\it j\_n} & INTEGER(4) &  & I/O &  データの南北格子数  \\
{\it i} & INTEGER(4) & \AnySize & I/O &  全球の東西格子数  \\
{\it i\_start} & INTEGER(4) & \AnySize & I/O &  データの最西格子の番号(1始まり)  \\
{\it i\_n} & INTEGER(4) & \AnySize & I/O &  データの東西格子数  \\
{\it lat} & REAL(4) & \AnySize & I/O &  緯度  \\
{\it getput} & CHARACTER(3) &  & IN &  入出力指示 ({\it "GET}" または {\it "PUT}")  \\
{\it result} & INTEGER(4) &  & OUT & \ResultCode \\
\hline
\end{tabular}
\paragraph{\FuncDesc}Reduced Gauss 格子を使う場合の補助管理情報へのアクセスを提供する。
入出力指示が {\it GET} の場合においても、j\_n の値はセットする。
特に parameter 宣言された変数を j\_n に指定してはならない。この j\_n の値は
nusdas\_subc\_rgau\_inq\_jn を使って問い合わせできる。
i, i\_start, i\_n, lat は j\_n 要素をもった配列を用意する。
\paragraph{\ResultCode}
\begin{quote}
\begin{description}
\item[{\bf 0}] 正常終了
\item[{\bf -2}] レコードが存在しない、または書き込まれていない。
\item[{\bf -3}] サイズの情報が引数と定義ファイルで不一致
\item[{\bf -4}] 指定した入力値(j\_n, j\_start, j\_n, i, i\_start, i\_n)が不正(PUTのときのみ)
\item[{\bf -5}] 入出力指示が不正
\item[{\bf -6}] 指定した入力値(j\_n)が不正(GETのときのみ)
\end{description}\end{quote}
\paragraph{ 注意 }
Reduced Gauss 格子を使う場合は1次元でデータを格納するので、定義ファイルの
size(格子数)には (実際の格子数) 1 と指定する。また、SUBC のサイズは 
16 $\ast$ j\_n + 12 を計算した値を定義ファイルに書く。
\paragraph{ 履歴 }
この関数はNuSDaS1.2で実装された

\subsection{NUSDAS\_SUBC\_RGAU\_INQ\_JN: SUBC RGAU 記録の大きさを問合せ}
\APILabel{nusdas.subc.rgau.inq.jn}

\Prototype
\begin{quote}
CALL {\bf NUSDAS\_SUBC\_RGAU\_INQ\_JN}({\it type1}, {\it type2}, {\it type3}, {\it basetime}, {\it member}, {\it validtime}, {\it j\_n}, {\it result})
\end{quote}

\begin{tabular}{l|rllp{16em}}
\hline
\ArgName & \ArgType & \ArrayDim & I/O & \ArgRole \\
\hline
{\it type1} & CHARACTER(8) &  & IN &  種別1  \\
{\it type2} & CHARACTER(4) &  & IN &  種別2  \\
{\it type3} & CHARACTER(4) &  & IN &  種別3  \\
{\it basetime} & INTEGER(4) &  & IN &  基準時刻(通算分)  \\
{\it member} & CHARACTER(4) &  & IN &  メンバー名  \\
{\it validtime} & INTEGER(4) &  & IN &  対象時刻(通算分)  \\
{\it j\_n} & INTEGER(4) &  & OUT &  南北格子数  \\
{\it result} & INTEGER(4) &  & OUT & \ResultCode \\
\hline
\end{tabular}
\paragraph{\FuncDesc}RGAU に記録されている j\_n (南北格子数) を問い合わせる。
\paragraph{\ResultCode}
\begin{quote}
\begin{description}
\item[{\bf 正}] 正常終了
\item[{\bf -2}] 要求されたレコードが存在しない、または書き込まれていない。
\item[{\bf -3}] レコードのサイズが不正
\end{description}\end{quote}
\paragraph{ 履歴 }
この関数は NuSDaS1.2で導入された。

\subsection{NUSDAS\_SUBC\_RGAU\_PRESET1: SUBC RGAU のデフォルト値を設定}
\APILabel{nusdas.subc.rgau.preset1}

\Prototype
\begin{quote}
CALL {\bf NUSDAS\_SUBC\_RGAU\_PRESET1}({\it type1}, {\it type2}, {\it type3}, {\it j}, {\it j\_start}, {\it j\_n}, {\it i}, {\it i\_start}, {\it i\_n}, {\it lat}, {\it result})
\end{quote}

\begin{tabular}{l|rllp{16em}}
\hline
\ArgName & \ArgType & \ArrayDim & I/O & \ArgRole \\
\hline
{\it type1} & CHARACTER(8) &  & IN &  種別1  \\
{\it type2} & CHARACTER(4) &  & IN &  種別2  \\
{\it type3} & CHARACTER(4) &  & IN &  種別3  \\
{\it j} & INTEGER(4) &  & IN &  全球の南北分割数  \\
{\it j\_start} & INTEGER(4) &  & IN &  データの最北格子の番号(1始まり)  \\
{\it j\_n} & INTEGER(4) &  & IN &  データの南北格子数  \\
{\it i} & INTEGER(4) & \AnySize & IN &  全球の東西格子数  \\
{\it i\_start} & INTEGER(4) & \AnySize & IN &  データの最西格子の番号(1始まり)  \\
{\it i\_n} & INTEGER(4) & \AnySize & IN &  データの東西格子数  \\
{\it lat} & REAL(4) & \AnySize & IN &  緯度  \\
{\it result} & INTEGER(4) &  & OUT & \ResultCode \\
\hline
\end{tabular}
\paragraph{\FuncDesc}ファイルが新たに生成される際にRGAUレコードに書き込む値を設定する。
RGAU レコードや引数については nusdas\_subc\_rgau を参照。
\paragraph{\ResultCode}
\begin{quote}
\begin{description}
\item[{\bf 0}] 正常終了
\item[{\bf -1}] 定義ファイルに "RGAU" が登録されていない
\item[{\bf -2}] メモリの確保に失敗した
\end{description}\end{quote}
\paragraph{ 互換性 }
NuSDaS1.1 では、一つのNuSDaSデータセットに設定できる補助管理部の数は最大
10 に制限されており、それを超えると-2が返された。一方、 NuSDaS 1.3 からは
メモリが確保できる限り数に制限はなく、-2 をメモリ確保失敗のエラーコードに
読み替えている。

\subsection{NUSDAS\_SUBC\_SIGM: SUBC SIGM へのアクセス}
\APILabel{nusdas.subc.sigm}

\Prototype
\begin{quote}
CALL {\bf NUSDAS\_SUBC\_SIGM}({\it type1}, {\it type2}, {\it type3}, {\it basetime}, {\it member}, {\it validtime}, {\it n\_levels}, {\it a}, {\it b}, {\it c}, {\it getput}, {\it result})
\end{quote}

\begin{tabular}{l|rllp{16em}}
\hline
\ArgName & \ArgType & \ArrayDim & I/O & \ArgRole \\
\hline
{\it type1} & CHARACTER(8) &  & IN &  種別1  \\
{\it type2} & CHARACTER(4) &  & IN &  種別2  \\
{\it type3} & CHARACTER(4) &  & IN &  種別3  \\
{\it basetime} & INTEGER(4) &  & IN &  基準時刻(通算分)  \\
{\it member} & CHARACTER(4) &  & IN &  メンバー名  \\
{\it validtime} & INTEGER(4) &  & IN &  対象時刻(通算分)  \\
{\it n\_levels} & INTEGER(4) &  & I/O &  鉛直層数  \\
{\it a} & REAL(4) & \AnySize & I/O &  係数 a  \\
{\it b} & REAL(4) & \AnySize & I/O &  係数 b  \\
{\it c} & REAL(4) &  & I/O &  係数 c  \\
{\it getput} & CHARACTER(3) &  & IN &  入出力指示 ({\it "GET}" または {\it "PUT}")  \\
{\it result} & INTEGER(4) &  & OUT & \ResultCode \\
\hline
\end{tabular}
\paragraph{\FuncDesc}鉛直座標に ETA 座標系を用いるときに、鉛直座標を定めるパラメータへの
アクセスを提供する。 
関数の仕様は、nusdas\_subc\_eta と同じである。

\subsection{NUSDAS\_SUBC\_SRF: 降短系 SUBC へのアクセス}
\APILabel{nusdas.subc.srf}

\Prototype
\begin{quote}
CALL {\bf NUSDAS\_SUBC\_SRF}({\it type1}, {\it type2}, {\it type3}, {\it basetime}, {\it member}, {\it validtime}, {\it plane}, {\it element}, {\it group}, {\it data}, {\it getput}, {\it result})
\end{quote}

\begin{tabular}{l|rllp{16em}}
\hline
\ArgName & \ArgType & \ArrayDim & I/O & \ArgRole \\
\hline
{\it type1} & CHARACTER(8) &  & IN &  種別1  \\
{\it type2} & CHARACTER(4) &  & IN &  種別2  \\
{\it type3} & CHARACTER(4) &  & IN &  種別3  \\
{\it basetime} & INTEGER(4) &  & IN &  基準時刻(通算分)  \\
{\it member} & CHARACTER(4) &  & IN &  メンバー名  \\
{\it validtime} & INTEGER(4) &  & IN &  対象時刻(通算分)  \\
{\it plane} & CHARACTER(6) &  & IN &  面  \\
{\it element} & CHARACTER(6) &  & IN &  要素名  \\
{\it group} & CHARACTER(4) &  & IN &  群名  \\
{\it data} & INTEGER(4) &  & I/O &  データ配列  \\
{\it getput} & CHARACTER(3) &  & IN &  入出力指示 ({\it "GET}" または {\it "PUT}")  \\
{\it result} & INTEGER(4) &  & OUT & \ResultCode \\
\hline
\end{tabular}
\paragraph{\FuncDesc}降水短時間予報系のデータの補助管理部へのアクセスを提供する。
群名には次のもののいずれかを指定する。
\begin{quote}\begin{description}
\item[{\bf ISPC}] 
レーダーや雨量計の運用情報、レベル値変換テーブルが格納される。
data には 128要素の4バイト整数型配列を用意する。内部のフォーマットは
4バイト整数型であることは関係ないが、バイトオーダーの変換はされるので
注意が必要。
\item[{\bf THUN}] 
詳細未詳。
data には 4バイト整数型変数を用意する。
\item[{\bf RADR}] 
レーダー観測に関する情報。data には 4バイト整数型変数を用意する。
\item[{\bf RADS}] 
レーダー観測に関する情報。data には 6要素の4バイト整数型配列を用意する。
\item[{\bf DPRD}] 
ドップラーレーダー観測に関する情報。
data には 8要素の4バイト整数型配列を用意する。
\end{description}\end{quote}
\paragraph{\ResultCode}
\begin{quote}
\begin{description}
\item[{\bf 0}] 正常終了
\item[{\bf -2}] 要求されたレコードが存在しない、または書かれていない。
\item[{\bf -3}] レコードサイズが不正
\item[{\bf -4}] 群名が不正
\item[{\bf -5}] 入出力指示が不正
\end{description}\end{quote}
\paragraph{ 履歴 }
この関数は NuSDaS1.0 から存在した。

\subsection{NUSDAS\_SUBC\_TDIF: SUBC TDIF へのアクセス}
\APILabel{nusdas.subc.tdif}

\Prototype
\begin{quote}
CALL {\bf NUSDAS\_SUBC\_TDIF}({\it type1}, {\it type2}, {\it type3}, {\it basetime}, {\it member}, {\it validtime}, {\it diff\_time}, {\it total\_sec}, {\it getput}, {\it result})
\end{quote}

\begin{tabular}{l|rllp{16em}}
\hline
\ArgName & \ArgType & \ArrayDim & I/O & \ArgRole \\
\hline
{\it type1} & CHARACTER(8) &  & IN &  種別1  \\
{\it type2} & CHARACTER(4) &  & IN &  種別2  \\
{\it type3} & CHARACTER(4) &  & IN &  種別3  \\
{\it basetime} & INTEGER(4) &  & IN &  基準時刻(通算分)  \\
{\it member} & CHARACTER(4) &  & IN &  メンバー名  \\
{\it validtime} & INTEGER(4) &  & IN &  対象時刻(通算分)  \\
{\it diff\_time} & INTEGER(4) &  & I/O &  対象時刻からのずれ(秒)  \\
{\it total\_sec} & INTEGER(4) &  & I/O &  総予報時間(秒)  \\
{\it getput} & CHARACTER(3) &  & IN &  入出力指示 ({\it "GET}" または {\it "PUT}")  \\
{\it result} & INTEGER(4) &  & OUT & \ResultCode \\
\hline
\end{tabular}
\paragraph{\FuncDesc}格納した値の時刻の対象時間とのずれ、積算時間を格納する補助管理部 TDIF 
へのアクセスを提供する。
\paragraph{\ResultCode}
\begin{quote}
\begin{description}
\item[{\bf 0}] 正常終了
\item[{\bf -2}] 要求されたレコードが存在しない、または書き込まれていない。
\item[{\bf -3}] レコードサイズが不正
\item[{\bf -5}] 入出力指示が不正
\end{description}\end{quote}

\paragraph{補足}
\begin{itemize}
\item  diff\_time = 時間範囲始点 - 対象時刻 [秒単位]
\item  total\_sec = 時間範囲終点 - 時間範囲始点 [秒単位]
\end{itemize}

\paragraph{履歴}
この関数は NuSDaS1.0 から存在した。

\subsection{NUSDAS\_SUBC\_ZHYB: SUBC ZHYB へのアクセス }
\APILabel{nusdas.subc.zhyb}

\Prototype
\begin{quote}
CALL {\bf NUSDAS\_SUBC\_ZHYB}({\it type1}, {\it type2}, {\it type3}, {\it basetime}, {\it member}, {\it validtime}, {\it nz}, {\it ptrf}, {\it presrf}, {\it zrp}, {\it zrw}, {\it vctrans\_p}, {\it vctrans\_w}, {\it dvtrans\_p}, {\it dvtrans\_w}, {\it getput}, {\it result})
\end{quote}

\begin{tabular}{l|rllp{16em}}
\hline
\ArgName & \ArgType & \ArrayDim & I/O & \ArgRole \\
\hline
{\it type1} & CHARACTER(8) &  & IN &  種別1  \\
{\it type2} & CHARACTER(4) &  & IN &  種別2  \\
{\it type3} & CHARACTER(4) &  & IN &  種別3  \\
{\it basetime} & INTEGER(4) &  & IN &  基準時刻(通算分)  \\
{\it member} & CHARACTER(4) &  & IN &  メンバー名  \\
{\it validtime} & INTEGER(4) &  & IN &  対象時刻(通算分)  \\
{\it nz} & INTEGER(4) &  & I/O &  鉛直層数  \\
{\it ptrf} & REAL(4) &  & I/O &  温位の参照値  \\
{\it presrf} & REAL(4) &  & I/O &  気圧の参照値  \\
{\it zrp} & REAL(4) & \AnySize & I/O &  モデル面高度 (フルレベル)  \\
{\it zrw} & REAL(4) & \AnySize & I/O &  モデル面高度 (ハーフレベル)  \\
{\it vctrans\_p} & REAL(4) & \AnySize & I/O &  座標変換関数 (フルレベル)  \\
{\it vctrans\_w} & REAL(4) & \AnySize & I/O &  座標変換関数 (ハーフレベル)  \\
{\it dvtrans\_p} & REAL(4) & \AnySize & I/O &  座標変換関数の鉛直微分 (フルレベル)  \\
{\it dvtrans\_w} & REAL(4) & \AnySize & I/O &  座標変換関数の鉛直微分 (ハーフレベル)  \\
{\it getput} & CHARACTER(3) &  & IN &  入出力指示 ({\it "GET}" または {\it "PUT}")  \\
{\it result} & INTEGER(4) &  & OUT & \ResultCode \\
\hline
\end{tabular}
\paragraph{\FuncDesc}鉛直座標に鉛直ハイブリッド座標をを使う場合の補助管理情報ZHYB
へのアクセスを提供する。
入出力指示が {\it GET} の場合においても、nz の値をセットする。
特に parameter 宣言された変数を nz に指定してはならない。この nz の値は
nusdas\_subc\_eta\_inq\_nz を使って問い合わせできる。
zrp, zrw, vctrans\_p, vctrans\_w, dvtrans\_p, dvtrans\_w は 
nz 要素をもった配列を用意する。
\paragraph{\ResultCode}
\begin{quote}
\begin{description}
\item[{\bf 0}] 正常終了
\item[{\bf -2}] レコードが存在しない、または書き込まれていない。
\item[{\bf -3}] サイズの情報が引数と定義ファイルで不一致
\item[{\bf -4}] 指定した入力値(ptrf, presrf)が不正(PUTのときのみ)
\item[{\bf -5}] 入出力指示が不正
\item[{\bf -6}] 指定した入力値(nz)が不正(GETのときのみ)
\end{description}\end{quote}
\paragraph{ 注意 }
SUBC のサイズは 24 $\ast$ nz + 12 を計算した値を定義ファイルに書く。
\paragraph{ 履歴 }
この関数はNuSDaS1.2で実装された

\subsection{NUSDAS\_SUBC\_ZHYB\_PRESET1: SUBC ZHYB のデフォルト値を設定}
\APILabel{nusdas.subc.zhyb.preset1}

\Prototype
\begin{quote}
CALL {\bf NUSDAS\_SUBC\_ZHYB\_PRESET1}({\it type1}, {\it type2}, {\it type3}, {\it nz}, {\it ptrf}, {\it presrf}, {\it zrp}, {\it zrw}, {\it vctrans\_p}, {\it vctrans\_w}, {\it dvtrans\_p}, {\it dvtrans\_w}, {\it result})
\end{quote}

\begin{tabular}{l|rllp{16em}}
\hline
\ArgName & \ArgType & \ArrayDim & I/O & \ArgRole \\
\hline
{\it type1} & CHARACTER(8) &  & IN &  種別1  \\
{\it type2} & CHARACTER(4) &  & IN &  種別2  \\
{\it type3} & CHARACTER(4) &  & IN &  種別3  \\
{\it nz} & INTEGER(4) &  & IN &  鉛直層数  \\
{\it ptrf} & REAL(4) &  & IN &  温位の参照値  \\
{\it presrf} & REAL(4) &  & IN &  気圧の参照値  \\
{\it zrp} & REAL(4) & \AnySize & IN &  モデル面高度 (フルレベル)  \\
{\it zrw} & REAL(4) & \AnySize & IN &  モデル面高度 (ハーフレベル)  \\
{\it vctrans\_p} & REAL(4) & \AnySize & IN &  座標変換関数 (フルレベル)  \\
{\it vctrans\_w} & REAL(4) & \AnySize & IN &  座標変換関数 (ハーフレベル)  \\
{\it dvtrans\_p} & REAL(4) & \AnySize & IN &  座標変換関数の鉛直微分 (フルレベル)  \\
{\it dvtrans\_w} & REAL(4) & \AnySize & IN &  座標変換関数の鉛直微分 (ハーフレベル)  \\
{\it result} & INTEGER(4) &  & OUT & \ResultCode \\
\hline
\end{tabular}
\paragraph{\FuncDesc}ファイルが新たに生成される際にZHYBレコードに書き込む値を設定する。
ZHYB レコードや引数についてはnusdas\_subc\_zhyb を参照。
\paragraph{\ResultCode}
\begin{quote}
\begin{description}
\item[{\bf 0}] 正常終了
\item[{\bf -1}] 定義ファイルに "ZHYB" が登録されていない
\item[{\bf -2}] メモリの確保に失敗した
\end{description}\end{quote}
\paragraph{ 互換性 }
NuSDaS1.1 では、一つのNuSDaSデータセットに設定できる補助管理部の数は最大
10 に制限されており、それを超えると-2が返された。一方、 NuSDaS 1.3 からは
メモリが確保できる限り数に制限はなく、-2 をメモリ確保失敗のエラーコードに
読み替えている。


\clearpage
\section{サービスサブルーチン}
\label{fapi:service}

\input{fapi_endian_swab2}
\input{fapi_endian_swab4}
\subsection{ENDIAN\_SWAB8: 8バイト整数のバイトオーダー変換}
\APILabel{endian.swab8}

\Prototype
\begin{quote}
CALL {\bf ENDIAN\_SWAB8}({\it ary}, {\it count})
\end{quote}

\begin{tabular}{l|rllp{16em}}
\hline
\ArgName & \ArgType & \ArrayDim & I/O & \ArgRole \\
\hline
{\it ary} & \AnyType & \AnySize & I/O &  配列  \\
{\it count} & INTEGER(4) &  & IN &  配列の要素数  \\
\hline
\end{tabular}
\paragraph{\FuncDesc}
リトルエンディアン機では、8バイト整数または実数の配列 {\it ary} の
バイトオーダーを逆順にする。
ビッグエンディアンのデータを読んだ後整数として解釈する前、
または整数として値を格納した後ビッグエンディアンで書き出す前に呼ぶ。

ビッグエンディアン機ではなにもしない。

\subsection{ENDIAN\_SWAB\_FMT: 任意構造のバイトオーダー変換}
\APILabel{endian.swab.fmt}

\Prototype
\begin{quote}
CALL {\bf ENDIAN\_SWAB\_FMT}({\it ptr}, {\it fmt})
\end{quote}

\begin{tabular}{l|rllp{16em}}
\hline
\ArgName & \ArgType & \ArrayDim & I/O & \ArgRole \\
\hline
{\it ptr} & \AnyType & \AnySize & I/O &  変換対象  \\
{\it fmt} & CHARACTER($\ast$) & \AnySize & IN &  書式  \\
\hline
\end{tabular}
\paragraph{\FuncDesc}
リトルエンディアン機では、
さまざまな長さのデータが混在するメモリ領域 {\it ptr} の
バイトオーダーを逆順にする。
ビッグエンディアンのデータを読んだ後整数として解釈する前、
または整数として値を格納した後ビッグエンディアンで書き出す前に呼ぶ。

ビッグエンディアン機ではなにもしない。

メモリのレイアウトは文字列 {\it fmt} で指定される。
文字列は以下に示す型を表わす文字の羅列である。
\begin{quote}\begin{description}
\item[{\bf D, d, L, l}] 8バイト
\item[{\bf F, f, I, i}] 4バイト
\item[{\bf H, h}] 2バイト
\item[{\bf B, b, N, n}] 1バイト (なにもしない)
\end{description}\end{quote}
文字の前に数字をつけると繰り返し数をあらわす。
たとえば ``{\tt 4c8i}'' は最初の 4 バイトが無変換、
次に 4 バイト単位で 8 個変換を行うことを示す。

\paragraph{注意}
\begin{itemize}
\item 数字は strtoul(3) で解釈しているので十進だけではなく八進や十六進
も使える。
たとえば ``{\tt 0xFFi}'' は 4 バイト単位で 255 個変換することを示し、
``{\tt 0100h}'' は 2 バイト単位で 64 個変換することを示す。
\end{itemize}

\paragraph{履歴}
本関数は pnusdas から存在し、NuSDaS 1.3 で Fortran ラッパーを伴う
サービスサブルーチンとしてドキュメントされた。

\subsection{NUSDAS\_GUNZIP: gzip 圧縮データを展開}
\APILabel{nusdas.gunzip}

\Prototype
\begin{quote}
CALL {\bf NUSDAS\_GUNZIP}({\it in\_data}, {\it in\_nbytes}, {\it out\_buf}, {\it out\_nbytes}, {\it result})
\end{quote}

\begin{tabular}{l|rllp{16em}}
\hline
\ArgName & \ArgType & \ArrayDim & I/O & \ArgRole \\
\hline
{\it in\_data} & \AnyType & \AnySize & IN &  圧縮データ  \\
{\it in\_nbytes} & INTEGER(4) &  & IN &  圧縮データのバイト数  \\
{\it out\_buf} & \AnyType & \AnySize & OUT &  展開結果を格納する領域  \\
{\it out\_nbytes} & INTEGER(4) &  & IN &  結果領域のバイト数  \\
{\it result} & INTEGER(4) &  & OUT & \ResultCode \\
\hline
\end{tabular}
\paragraph{\FuncDesc}
入力データ {\it in\_data} を gzip 展開して {\it out\_buf} に格納する。
\paragraph{\ResultCode}
\begin{quote}
\begin{description}
\item[{\bf -98}] NuSDaS が ZLib を使うように設定されていない。
\item[{\bf -99}] 入力は gzip 圧縮形式ではない。
\item[{\bf -5}] 展開結果の長さが圧縮データと不整合。
\item[{\bf -4}] 結果領域の長さ {\it out\_nbytes} が不足している。
\item[{\bf -3}] 展開結果の CRC32 が圧縮データと不整合。
\item[{\bf -2}] ZLib の inflateInit 関数がエラーを起こした。
\item[{\bf -1}] ZLib の inflate 関数がエラーを起こした。
\item[{\bf 他}] 展開データのバイト数
\end{description}\end{quote}
\paragraph{履歴}
本関数は NuSDaS 1.3 で新設された。

\subsection{NUSDAS\_GUNZIP\_NBYTES: gzip 圧縮データの展開後の長さを得る}
\APILabel{nusdas.gunzip.nbytes}

\Prototype
\begin{quote}
CALL {\bf NUSDAS\_GUNZIP\_NBYTES}({\it in\_data}, {\it in\_nbytes}, {\it result})
\end{quote}

\begin{tabular}{l|rllp{16em}}
\hline
\ArgName & \ArgType & \ArrayDim & I/O & \ArgRole \\
\hline
{\it in\_data} & \AnyType & \AnySize & IN &  圧縮データ  \\
{\it in\_nbytes} & INTEGER(4) &  & IN &  圧縮データのバイト数  \\
{\it result} & INTEGER(4) &  & OUT & \ResultCode \\
\hline
\end{tabular}
\paragraph{\FuncDesc}
入力データ {\it in\_data} を gzip 展開するときに必要となる
結果格納領域のバイト数を返す。
\paragraph{\ResultCode}
\begin{quote}
\begin{description}
\item[{\bf -98}] NuSDaS が ZLib を使うように設定されていない。
\item[{\bf 正}] 展開後の長さ
\end{description}\end{quote}
\paragraph{履歴}
本関数は NuSDaS 1.3 で新設された。

\subsection{NUSDAS\_GZIP: gzip 圧縮}
\APILabel{nusdas.gzip}

\Prototype
\begin{quote}
CALL {\bf NUSDAS\_GZIP}({\it in\_data}, {\it in\_nbytes}, {\it out\_buf}, {\it out\_nbytes}, {\it result})
\end{quote}

\begin{tabular}{l|rllp{16em}}
\hline
\ArgName & \ArgType & \ArrayDim & I/O & \ArgRole \\
\hline
{\it in\_data} & \AnyType & \AnySize & IN &  入力データ  \\
{\it in\_nbytes} & INTEGER(4) &  & IN &  入力データのバイト数  \\
{\it out\_buf} & \AnyType & \AnySize & OUT &  圧縮結果を格納する領域  \\
{\it out\_nbytes} & INTEGER(4) &  & IN &  結果領域のバイト数  \\
{\it result} & INTEGER(4) &  & OUT & \ResultCode \\
\hline
\end{tabular}
\paragraph{\FuncDesc}
入力データ {\it in\_data} を gzip 圧縮して {\it out\_buf} に格納する。
\paragraph{\ResultCode}
\begin{quote}
\begin{description}
\item[{\bf -98}] NuSDaS が ZLib を使うように設定されていない。
\item[{\bf -9}] ZLib の deflateEnd 関数がエラーを起こした。
\item[{\bf -4}] 結果領域の長さ {\it out\_nbytes} が不足している。
\item[{\bf -2}] ZLib の deflateInit2 関数がエラーを起こした。
\item[{\bf -1}] ZLib の deflate 関数がエラーを起こした。
\item[{\bf 他}] 圧縮データの長さ
\end{description}\end{quote}
\paragraph{履歴}
本関数は NuSDaS 1.3 で新設された。

\subsection{NUSDAS\_UNPACK: 生DATAレコードの解読}
\APILabel{nusdas.unpack}

\Prototype
\begin{quote}
CALL {\bf NUSDAS\_UNPACK}({\it pdata}, {\it udata}, {\it utype}, {\it usize}, {\it result})
\end{quote}

\begin{tabular}{l|rllp{16em}}
\hline
\ArgName & \ArgType & \ArrayDim & I/O & \ArgRole \\
\hline
{\it pdata} & \AnyType & \AnySize & IN &  パックされたバイト列  \\
{\it udata} & \AnyType & \AnySize & I/O &  展開先配列  \\
{\it utype} & CHARACTER(2) &  & IN &  展開する型  \\
{\it usize} & INTEGER(4) &  & IN &  展開先配列の要素数  \\
{\it result} & INTEGER(4) &  & OUT & \ResultCode \\
\hline
\end{tabular}
\paragraph{\FuncDesc}
\APILink{nusdas.inq.data}{nusdas\_inq\_data} の問い合わせ N\_DATA\_CONTENT で得られるバイト列を
解読して数値配列を得る。パッキング型2UPJでは利用できない(2UPPは利用可)。

\paragraph{\ResultCode}
\begin{quote}
\begin{description}
\item[{\bf 正}] 正常終了、値は要素数
\item[{\bf -4}] 展開先の大きさ {\it usize} がデータレコードの要素数より少ない
\item[{\bf -5}] パッキング型・欠損値型・展開型の組合せが不適
\item[{\bf -6}] 利用できないパッキング型が与えられた
\end{description}\end{quote}

\paragraph{履歴}
本関数は NuSDaS 1.3 で追加された。
エラーコード -6 は NuSDaS 1.4 で新設されたもので、 
それ以前はエラーチェックがなされていなかった。

\subsection{NUSDAS\_UNCPSD: 2UPPを2UPCに展開する}
\APILabel{nusdas.uncpsd}

\Prototype
\begin{quote}
CALL {\bf NUSDAS\_UNCPSD}({\it pdata}, {\it cdata}, {\it ctype}, {\it result})
\end{quote}

\begin{tabular}{l|rllp{16em}}
\hline
\ArgName & \ArgType & \ArrayDim & I/O & \ArgRole \\
\hline
{\it pdata} & \AnyType & \AnySize & IN &  入力する2UPPデータ  \\
{\it cdata} & \AnyType & \AnySize & I/O &  展開先配列  \\
{\it csize} & INTEGER(4) &  & IN &  展開先配列のバイト数  \\
{\it result} & INTEGER(4) &  & OUT & \ResultCode \\
\hline
\end{tabular}
\paragraph{\FuncDesc}
\APILink{nusdas.inq.data}{nusdas\_inq\_data} の問い合わせ N\_DATA\_CONTENT で得られる 2UPP のバイト列を
展開して 2UPC のバイト列として得ます。結果として返るバイト列は 2UPC 形式のデータに対して
\APILink{nusdas.inq.data}{nusdas\_inq\_data} の問い合わせ N\_DATA\_CONTENT で得られるデータと同じ形式です。

\paragraph{\ResultCode}
\begin{quote}
\begin{description}
\item[{\bf 正}] 正常終了、値は展開後のバイト数
\item[{\bf -4}] 展開先配列の大きさ {\it csize} が必要バイト数より少ない
\item[{\bf -5}] 入力データが2UPPではない
\item[{\bf -6}] 2UPPから2UPCへの展開時にエラーが発生
\end{description}\end{quote}

\paragraph{履歴}
本関数は NuSDaS 1.4 で追加された。

\subsection{NUSDAS\_UNCPSD\_NBYTES: 2UPC展開後の長さを得る}
\APILabel{nusdas.unpack}

\Prototype
\begin{quote}
CALL {\bf NUSDAS\_UNCPSD\_NBYTES}({\it pdata}, {\it result})
\end{quote}

\begin{tabular}{l|rllp{16em}}
\hline
\ArgName & \ArgType & \ArrayDim & I/O & \ArgRole \\
\hline
{\it pdata} & \AnyType & \AnySize & IN &  入力する2UPPデータ  \\
{\it result} & INTEGER(4) &  & OUT & \ResultCode \\
\hline
\end{tabular}
\paragraph{\FuncDesc}
入力データ {\it pdata} を \APILink{nusdas.uncpsd}{nusdas\_uncpsd} で展開した後の
展開後バイト数を得る。

\paragraph{\ResultCode}
\begin{quote}
\begin{description}
\item[{\bf 正}] 正常終了、値は展開後のバイト数
\item[{\bf -5}] 入力データが2UPPではない
\end{description}\end{quote}

\paragraph{履歴}
本関数は NuSDaS 1.4 で追加された。

\subsection{N\_DECODE\_RLEN\_NBIT\_I1: RLE データを展開する}
\APILabel{n.decode.rlen.nbit.i1}

\Prototype
\begin{quote}
CALL {\bf N\_DECODE\_RLEN\_NBIT\_I1}({\it udata}, {\it compressed\_data}, {\it compressed\_nbytes}, {\it udata\_nelems}, {\it maxvalue}, {\it nbit}, {\it result})
\end{quote}

\begin{tabular}{l|rllp{16em}}
\hline
\ArgName & \ArgType & \ArrayDim & I/O & \ArgRole \\
\hline
{\it udata} & CHARACTER & \AnySize & OUT &  結果格納配列  \\
{\it compressed\_data} & CHARACTER & \AnySize & IN &  圧縮データ  \\
{\it compressed\_nbytes} & INTEGER(4) &  & IN &  圧縮データのバイト数  \\
{\it udata\_nelems} & INTEGER(4) &  & IN &  圧縮データの要素数  \\
{\it maxvalue} & INTEGER(4) &  & IN &  データの最大値  \\
{\it nbit} & INTEGER(4) &  & IN &  圧縮データのビット数  \\
{\it result} & INTEGER(4) &  & OUT & \ResultCode \\
\hline
\end{tabular}
\paragraph{\FuncDesc}\paragraph{履歴}
この関数は NuSDaS 1.0 から存在するが、ドキュメントされていなかった。
NuSDaS 1.3 から Fortran API を伴う
サービスサブルーチンとして採録された。

\subsection{N\_ENCODE\_RLEN\_8BIT: 4バイト整数を RLE 圧縮する}
\APILabel{n.encode.rlen.8bit}

\Prototype
\begin{quote}
CALL {\bf N\_ENCODE\_RLEN\_8BIT}({\it udata}, {\it compressed\_data}, {\it udata\_nelems}, {\it max\_compress\_nbytes}, {\it maxvalue}, {\it result})
\end{quote}

\begin{tabular}{l|rllp{16em}}
\hline
\ArgName & \ArgType & \ArrayDim & I/O & \ArgRole \\
\hline
{\it udata} & INTEGER(4) & \AnySize & IN &  元データ配列  \\
{\it compressed\_data} & CHARACTER & \AnySize & OUT &  結果格納配列  \\
{\it udata\_nelems} & INTEGER(4) &  & IN &  元データの要素数  \\
{\it max\_compress\_nbytes} & INTEGER(4) &  & IN &  結果配列のバイト数  \\
{\it maxvalue} & INTEGER(4) &  & OUT &  元データの最大値  \\
{\it result} & INTEGER(4) &  & OUT & \ResultCode \\
\hline
\end{tabular}
\paragraph{\FuncDesc}\paragraph{履歴}
この関数は NuSDaS 1.0 から存在するが、ドキュメントされていなかった。
NuSDaS 1.3 から Fortran API を伴う
サービスサブルーチンとして採録された。

\input{fapi_n_encode_rlen_8bit_i1.tex}

\clearpage
\section{降水短時間ライブラリ}
\label{fapi:libsrf}

\subsection{概要}

降水短時間ルーチンに関連するアメダスデータ・レーダーデータに
特有な処理のためのサブルーチンをまとめたものが
降水短時間ライブラリ libsrf.a として提供される。
ライブラリ全体の関数プロトタイプを与えるヘッダは存在しない%
\footnote{このため IBM 系環境では Fortran ラッパーが提供できない。}が、
\APILink{srf.amd.rdic}{SRF\_AMD\_RDIC} 及び
\APILink{srf.search.amdstn}{SRF\_SEARCH\_AMDSTN} を呼ぶ時は
SRF\_AMD\_SINFO 型や定数の定義を参照するため
\verb|"srf_amedas_fort.h"| をインクルードする必要がある。

\subsection{RDR\_LV\_TRANS: レベル値から代表値への変換}
\APILabel{rdr.lv.trans}

\Prototype
\begin{quote}
CALL {\bf RDR\_LV\_TRANS}({\it idat}, {\it fdat}, {\it dnum}, {\it param}, {\it result})
\end{quote}

\begin{tabular}{l|rllp{16em}}
\hline
\ArgName & \ArgType & \ArrayDim & I/O & \ArgRole \\
\hline
{\it idat} & INTEGER(4) & \AnySize & I/O &  入力データ  \\
{\it fdat} & REAL(4) & \AnySize & OUT &  結果格納配列  \\
{\it dnum} & INTEGER(4) &  & IN &  データ要素数  \\
{\it param} & CHARACTER($\ast$) & \AnySize & IN &  テーブル名  \\
{\it result} & INTEGER(4) &  & OUT & \ResultCode \\
\hline
\end{tabular}
\paragraph{\FuncDesc}
配列 {\it idat} のレベル値を代表値 {\it fdat} に変換する。
変換テーブルとしてファイル ./SRF\_LV\_TABLE/param.ltb を読む。
ここで {\it param} は変換テーブル名 (最長 4 字) である。

\paragraph{\ResultCode}
\begin{quote}
\begin{description}
\item[{\bf -1}] 変換テーブルを開くことができない
\item[{\bf -2}] 変換テーブルに 256 以上のレベルが指定されている
\item[{\bf 非負}] 変換に成功。返却値は不明値以外となったデータの要素数
\end{description}\end{quote}

\paragraph{注}
\begin{itemize}
\item 不明値は -1 となる。
\item 
NAPS8 では変換テーブルとして
/grpK/nwp/Open/Const/Vsrf/Comm/lvtbl.txd 以下に
her ie2 ier kor pft pi10 pm2 pmf pr2 prr rr60
sr1 sr2 sr3 srf srj srr yar yrr
が置かれている。
ルーチンジョブではこのディレクトリへシンボリックリンク SRF\_LV\_TABLE
を張って利用する。
\item 
上記変換テーブルのうち、
pi10 と rr60 は1行にレベル値と代表値の2列が書かれており、
その他はレベル値、最小値、代表値の3列が書かれているが、
本サブルーチンはどちらにも対応している。
\end{itemize}
\paragraph{履歴}
この関数は NAPS7 時代には存在しなかったようである。
レーダー情報作成装置に関連して開発されたと考えられているが、
NuSDaS 1.3 以前にはきちんとメンテナンスされていなかった。

\subsection{SRF\_AMD\_AQC: AQCのパックを展開}
\APILabel{srf.amd.aqc}

\Prototype
\begin{quote}
CALL {\bf SRF\_AMD\_AQC}({\it aqc\_in}, {\it num}, {\it aqc\_out}, {\it param})
\end{quote}

\begin{tabular}{l|rllp{16em}}
\hline
\ArgName & \ArgType & \ArrayDim & I/O & \ArgRole \\
\hline
{\it aqc\_in} & INTEGER(2) & \AnySize & IN &  AQC 配列  \\
{\it num} & INTEGER(4) &  & IN &  配列要素数  \\
{\it aqc\_out} & INTEGER(2) & \AnySize & I/O &  結果配列  \\
{\it param} & CHARACTER($\ast$) & \AnySize & IN &  要素名  \\
\hline
\end{tabular}
\paragraph{\FuncDesc}
アメダス デコード データセットに含まれる AQC から
要素名 {\it param} で指定される各ビットフィールドを取り出す。
\begin{quote}\begin{description}
\item[{\bf UNYOU△}] 入電・休止・運用情報 (-1:休止, 0:入電無し, 正:入電回数)
\item[{\bf RRfr0△}] 降水量の情報
(0:入電無し, 1:ハードエラー・欠測・休止, 2:AQC該当値, 3:正常値; 以下同じ)
\item[{\bf SSfr0△}] 日照時間の情報
\item[{\bf T△△△△△}] 気温の情報
\item[{\bf WindD△}] 風向の情報
\item[{\bf WindS△}] 風速の情報
\item[{\bf SnowD△}] 積雪深の情報
\end{description}\end{quote}

\paragraph{注意}
要素名が不正な場合は警告後なにもせず終了する。
(NuSDaS 1.3 より前は不定動作)
\paragraph{履歴}
この関数は NAPS7 時代から存在した。

\input{fapi_srf_amd_rdic.tex}
\subsection{SRF\_AMD\_SLCT: アメダスデータを指定の地点番号順に並べる}
\APILabel{srf.amd.slct}

\Prototype
\begin{quote}
CALL {\bf SRF\_AMD\_SLCT}({\it st\_r}, {\it n\_st\_r}, {\it d\_r}, {\it t\_r}, {\it st\_n}, {\it n\_st\_n}, {\it d\_n}, {\it t\_n}, {\it sort\_f}, {\it result})
\end{quote}

\begin{tabular}{l|rllp{16em}}
\hline
\ArgName & \ArgType & \ArrayDim & I/O & \ArgRole \\
\hline
{\it st\_r} & INTEGER(4) &  & IN &  結果の順を指示する地点番号表  \\
{\it n\_st\_r} & INTEGER(4) &  & IN &  結果配列長  \\
{\it d\_r} & \AnyType & \AnySize & OUT &  結果配列  \\
{\it t\_r} & CHARACTER($\ast$) & \AnySize & IN &  結果配列の型  \\
{\it st\_n} & INTEGER(4) &  & I/O &  元データ地点番号配列  \\
{\it n\_st\_n} & INTEGER(4) &  & IN &  元データ配列長  \\
{\it d\_n} & \AnyType & \AnySize & I/O &  元データ配列  \\
{\it t\_n} & CHARACTER($\ast$) & \AnySize & IN &  元データ配列の型  \\
{\it sort\_f} & INTEGER(4) &  & IN &  未ソートフラグ  \\
{\it result} & INTEGER(4) &  & OUT & \ResultCode \\
\hline
\end{tabular}
\paragraph{\FuncDesc}
長さ {\it n\_st\_n} の地点番号配列 {\it st\_n} と
対応する順に並んだ {\it t\_n} 型の配列 {\it d\_n} から、
別の地点番号配列 {\it st\_r} (要素数 {\it n\_st\_r} 個) の順に並んだ
{\it t\_r} 型の配列 {\it d\_r} (要素数 {\it n\_st\_r} 個) を作る。

配列 {\it st\_n} と {\it d\_n} があらかじめソートされている場合 {\it sort\_f} に
N\_OFF (nusdas.h で定義される) を指定する。
そうでない場合 {\it sort\_f} に N\_ON を指定するとソートされる。

\paragraph{\ResultCode}
\begin{quote}
\begin{description}
\item[{\bf 0}] 配列 {\it st\_r} の全地点が見付かった
\item[{\bf 正}] みつからなかった地点数
\end{description}\end{quote}

\begin{itemize}
\item 型は nusdas.h で定義される N\_R4, N\_I4, N\_I2 のいずれかで指定する。
\item 配列 {\it st\_r} に含まれる地点番号が {\it st\_n} で見付からない場合は
nusdas.h で定義される欠損値 N\_MV\_R4, N\_MV\_SI4, N\_MV\_SI2 が入る。
\end{itemize}
\paragraph{履歴}
この関数は NAPS7 時代から存在した。

\subsection{SRF\_LV\_SET: 実数からレベル値への変換}
\APILabel{srf.lv.set}

\Prototype
\begin{quote}
CALL {\bf SRF\_LV\_SET}({\it idat}, {\it fdat}, {\it dnum}, {\it ispec}, {\it param}, {\it result})
\end{quote}

\begin{tabular}{l|rllp{16em}}
\hline
\ArgName & \ArgType & \ArrayDim & I/O & \ArgRole \\
\hline
{\it idat} & INTEGER(4) & \AnySize & OUT &  レベル値格納配列  \\
{\it fdat} & REAL(4) & \AnySize & IN &  変換元データ配列  \\
{\it dnum} & INTEGER(4) &  & IN &  データ配列要素数  \\
{\it ispec} & INTEGER(4) & \AnySize & I/O &  新 ISPC  \\
{\it param} & CHARACTER($\ast$) & \AnySize & IN &  変換テーブル名  \\
{\it result} & INTEGER(4) &  & OUT & \ResultCode \\
\hline
\end{tabular}
\paragraph{\FuncDesc}
配列 {\it fdat} の実数データをレベル値 {\it idat} に変換し、
ISPC 配列 {\it ispec} にレベル代表値をセットする。
変換テーブルとしてファイル ./SRF\_LV\_TABLE/param.ltb を読む。
ここで {\it param} は変換テーブル名 (最長 4 字) である。

\paragraph{\ResultCode}
\begin{quote}
\begin{description}
\item[{\bf -1}] 変換テーブルを開くことができない
\item[{\bf -2}] 変換テーブルに 256 以上のレベルが指定されている
\item[{\bf 正}] 変換に成功した。返却値はレベル数
\end{description}\end{quote}

\paragraph{注}
\begin{itemize}
\item 
NAPS7 では変換テーブルとして
her ier prr pmf srr srf pr2
を用意していた。
NAPS8 では \newline 
/grpK/nwp/Open/Const/Vsrf/Comm/lvtbl.txd 以下に
her ie2 ier kor pft pm2 pmf pr2 prr sr1 sr2 sr3 srf srj srr yar yrr
が置かれている。
ルーチンジョブではこのディレクトリへシンボリックリンク SRF\_LV\_TABLE
を張って利用する。
\item 
変換テーブル名が ie2, kor, pft のとき,
ISPEC には変換テーブルに書かれた代表値の 1/10 が書かれる。
それ以外の場合は変換テーブルの代表値がそのまま書かれる。
\item 
変換テーブル名が sr2 または srj のときは実数データが -900.0 より小さい
ものが欠損値とみなされる。そうでなければ、負値が欠損値とみなされる
(NAPS7 のマニュアルでは欠損値は -1 を指定することとされている)。
\item 
変換テーブル名が srj のときは、実数データが変換テーブルの下限値に
正確に一致しないと最も上の階級 (具体的には 42 で 21.0を意味する)
に割り当てられる。この挙動はバグかもしれない。
\item 
変換テーブルに 191 行以上書かれているとき、
最初の 190 行だけが用いられ、レベル値は 0..190 となるが、
返却値には実際のレベル数 (変換テーブルの行数 + 1) が返される。
これは ispec の配列をオーバーフローしないためである。
\end{itemize}
\paragraph{履歴}
この関数は NAPS7 時代から存在した。
Fortran ラッパーが文字列の長さを伝えないバグは NuSDaS 1.3 で解決した。

\subsection{SRF\_LV\_TRANS: レベル値を実数データ (代表値) に変換}
\APILabel{srf.lv.trans}

\Prototype
\begin{quote}
CALL {\bf SRF\_LV\_TRANS}({\it idat}, {\it fdat}, {\it dnum}, {\it ispec}, {\it result})
\end{quote}

\begin{tabular}{l|rllp{16em}}
\hline
\ArgName & \ArgType & \ArrayDim & I/O & \ArgRole \\
\hline
{\it idat} & INTEGER(4) & \AnySize & IN &  入力データ  \\
{\it fdat} & REAL(4) & \AnySize & OUT &  結果格納配列  \\
{\it dnum} & INTEGER(4) &  & IN &  データ要素数  \\
{\it ispec} & INTEGER(4) & \AnySize & IN &  ISPEC 配列  \\
{\it result} & INTEGER(4) &  & OUT & \ResultCode \\
\hline
\end{tabular}
\paragraph{\FuncDesc}
新 ISPEC 配列 {\it ispec} にしたがって
配列 {\it idat} のレベル値を代表値 {\it fdat} に変換する。

\paragraph{返却値}
不明値以外となったデータの要素数

\paragraph{注}
\begin{itemize}
\item 
不明値は -1 となる。
ただし、ISPEC のデータ種別 (先頭4バイト) が
SRR2, SRF2, SRRR, SRFR の場合に限り -9999.0 となる。
\item 
ISPEC のレベル表は通常 0.1mm 単位と解釈される。
ただし、ISPEC の先頭 3 バイトが `{\tt IER}' であるか、
あるいは ISPEC の先頭から 4 バイト目が `{\tt 1}' のときは
0.01mm 単位と解釈される。
\end{itemize}
\paragraph{履歴}
この関数は NAPS7 時代から存在した。

\subsection{SRF\_RD\_RDIC: レーダーサイト情報の問い合わせ}
\APILabel{srf.rd.rdic}

\Prototype
\begin{quote}
CALL {\bf SRF\_RD\_RDIC}({\it stnum}, {\it iseq}, {\it lat}, {\it lon}, {\it hh}, {\it offx}, {\it offy}, {\it type1}, {\it type2}, {\it result})
\end{quote}

\begin{tabular}{l|rllp{16em}}
\hline
\ArgName & \ArgType & \ArrayDim & I/O & \ArgRole \\
\hline
{\it stnum} & INTEGER(4) &  & IN &  地点番号  \\
{\it iseq} & INTEGER(4) &  & IN &  日時(通算時)  \\
{\it lat} & REAL(4) &  & OUT &  緯度  \\
{\it lon} & REAL(4) &  & OUT &  経度  \\
{\it hh} & REAL(4) &  & OUT &  高度  \\
{\it offx} & INTEGER(4) &  & OUT &  中心のオフセット  \\
{\it offy} & INTEGER(4) &  & OUT &  中心のオフセット  \\
{\it type1} & INTEGER(4) &  & OUT &  デジタル化タイプ  \\
{\it type2} & INTEGER(4) &  & OUT &  デジタル化タイプ  \\
{\it result} & INTEGER(4) &  & OUT & \ResultCode \\
\hline
\end{tabular}
\paragraph{\FuncDesc}
ファイル名 RADAR\_DIC のレーダー地点辞書から
日時 {\it iseq} (通算時であって通算分でないことに注意)
における地点番号 {\it stnum} のレーダーサイトの情報を読出す。

\paragraph{\ResultCode}
\begin{quote}
\begin{description}
\item[{\bf 1}] 正常終了
\item[{\bf 0}] 指定されたレーダーサイトがみつからなかった
\item[{\bf -1}] レーダー地点辞書が開けなかった
\end{description}\end{quote}

\paragraph{注}
\begin{itemize}
\item レーダー地点辞書は NAPS8 では
/grpK/nwp/Open/Const/Vsrf/Dcd/rdrdic.txt に置かれている。
\item NAPS8 初期版 libsrf.a には経度のかわりに誤って緯度を返すバグがある。
\end{itemize}
\paragraph{履歴}
この関数は NAPS7 時代からルーチン環境には存在したが、
pnusdas から NuSDaS 1.1 に至る CVS 版ソースには含まれていなかった。
NuSDaS 1.3 で両者が統合された。

\subsection{SRF\_SEARCH\_AMDSTN: 地点番号の辞書内通番を探す}
\APILabel{srf.search.amdstn}

\Prototype
\begin{quote}
CALL {\bf SRF\_SEARCH\_AMDSTN}({\it amd}, {\it ac}, {\it stn}, {\it amd\_type}, {\it result})
\end{quote}

\begin{tabular}{l|rllp{16em}}
\hline
\ArgName & \ArgType & \ArrayDim & I/O & \ArgRole \\
\hline
{\it amd} & type(SRF\_AMD\_SINFO) & \AnySize & IN &  地点辞書配列  \\
{\it ac} & INTEGER(4) &  & IN &  地点辞書配列の長さ  \\
{\it stn} & INTEGER(4) &  & IN &  地点番号  \\
{\it amd\_type} & INTEGER(4) &  & IN &  地点種別  \\
{\it result} & INTEGER(4) &  & OUT & \ResultCode \\
\hline
\end{tabular}
\paragraph{\FuncDesc}
SRF\_AMD\_SINFO 構造型の配列 {\it amd} (地点数 {\it ac} 個) から
地点番号 {\it stn} の地点情報を収めた添字 (1始まり) を返す。

\paragraph{\ResultCode}
\begin{quote}
\begin{description}
\item[{\bf 正}] 地点の辞書内格納順位 (1始まり)
\item[{\bf -1}] 地点がみつからない
\end{description}\end{quote}

\paragraph{注意}
\begin{itemize}
\item 地点種別 {\it amd\_type} は無視される。
\item 配列が地点番号順にソートされていることを前提に二分探索を使っている。
\end{itemize}
\paragraph{履歴}
この関数は NAPS7 時代から存在したようであるが
ドキュメントされていなかった。
NuSDaS 1.3 リリースに際してドキュメントされるようになった。

