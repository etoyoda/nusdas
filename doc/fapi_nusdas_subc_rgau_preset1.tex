\subsection{NUSDAS\_SUBC\_RGAU\_PRESET1: SUBC RGAU のデフォルト値を設定}
\APILabel{nusdas.subc.rgau.preset1}

\Prototype
\begin{quote}
CALL {\bf NUSDAS\_SUBC\_RGAU\_PRESET1}({\it type1}, {\it type2}, {\it type3}, {\it j}, {\it j\_start}, {\it j\_n}, {\it i}, {\it i\_start}, {\it i\_n}, {\it lat}, {\it result})
\end{quote}

\begin{tabular}{l|rllp{16em}}
\hline
\ArgName & \ArgType & \ArrayDim & I/O & \ArgRole \\
\hline
{\it type1} & CHARACTER(8) &  & IN &  種別1  \\
{\it type2} & CHARACTER(4) &  & IN &  種別2  \\
{\it type3} & CHARACTER(4) &  & IN &  種別3  \\
{\it j} & INTEGER(4) &  & IN &  全球の南北分割数  \\
{\it j\_start} & INTEGER(4) &  & IN &  データの最北格子の番号(1始まり)  \\
{\it j\_n} & INTEGER(4) &  & IN &  データの南北格子数  \\
{\it i} & INTEGER(4) & \AnySize & IN &  全球の東西格子数  \\
{\it i\_start} & INTEGER(4) & \AnySize & IN &  データの最西格子の番号(1始まり)  \\
{\it i\_n} & INTEGER(4) & \AnySize & IN &  データの東西格子数  \\
{\it lat} & REAL(4) & \AnySize & IN &  緯度  \\
{\it result} & INTEGER(4) &  & OUT & \ResultCode \\
\hline
\end{tabular}
\paragraph{\FuncDesc}ファイルが新たに生成される際にRGAUレコードに書き込む値を設定する。
RGAU レコードや引数については nusdas\_subc\_rgau を参照。
\paragraph{\ResultCode}
\begin{quote}
\begin{description}
\item[{\bf 0}] 正常終了
\item[{\bf -1}] 定義ファイルに "RGAU" が登録されていない
\item[{\bf -2}] メモリの確保に失敗した
\end{description}\end{quote}
\paragraph{ 互換性 }
NuSDaS1.1 では、一つのNuSDaSデータセットに設定できる補助管理部の数は最大
10 に制限されており、それを超えると-2が返された。一方、 NuSDaS 1.3 からは
メモリが確保できる限り数に制限はなく、-2 をメモリ確保失敗のエラーコードに
読み替えている。
