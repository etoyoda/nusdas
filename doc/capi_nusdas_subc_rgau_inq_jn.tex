\subsection{nusdas\_subc\_rgau\_inq\_jn: SUBC RGAU 記録の大きさを問合せ}
\APILabel{nusdas.subc.rgau.inq.jn}

\Prototype
\begin{quote}
N\_SI4 {\bf nusdas\_subc\_rgau\_inq\_jn}(const char {\it type1}[8], const char {\it type2}[4], const char {\it type3}[4], const N\_SI4 $\ast${\it basetime}, const char {\it member}[4], const N\_SI4 $\ast${\it validtime}, N\_SI4 $\ast${\it j\_n});
\end{quote}

\begin{tabular}{l|rp{20em}}
\hline
\ArgName & \ArgType & \ArgRole \\
\hline
{\it type1} & const char [8] &  種別1  \\
{\it type2} & const char [4] &  種別2  \\
{\it type3} & const char [4] &  種別3  \\
{\it basetime} & const N\_SI4 $\ast$ &  基準時刻(通算分)  \\
{\it member} & const char [4] &  メンバー名  \\
{\it validtime} & const N\_SI4 $\ast$ &  対象時刻(通算分)  \\
{\it j\_n} & N\_SI4 $\ast$ &  南北格子数  \\
\hline
\end{tabular}
\paragraph{\FuncDesc}RGAU に記録されている j\_n (南北格子数) を問い合わせる。
\paragraph{\ResultCode}
\begin{quote}
\begin{description}
\item[{\bf 正}] 正常終了
\item[{\bf -2}] 要求されたレコードが存在しない、または書き込まれていない。
\item[{\bf -3}] レコードのサイズが不正
\end{description}\end{quote}
\paragraph{ 履歴 }
この関数は NuSDaS1.2で導入された。
