\subsection{NUSDAS\_SUBC\_ETA\_INQ\_NZ: SUBC 記録の鉛直層数問合せ}
\APILabel{nusdas.subc.eta.inq.nz}

\Prototype
\begin{quote}
CALL {\bf NUSDAS\_SUBC\_ETA\_INQ\_NZ}({\it type1}, {\it type2}, {\it type3}, {\it basetime}, {\it member}, {\it validtime}, {\it group}, {\it n\_levels}, {\it result})
\end{quote}

\begin{tabular}{l|rllp{16em}}
\hline
\ArgName & \ArgType & \ArrayDim & I/O & \ArgRole \\
\hline
{\it type1} & CHARACTER(8) &  & IN &  種別1  \\
{\it type2} & CHARACTER(4) &  & IN &  種別2  \\
{\it type3} & CHARACTER(4) &  & IN &  種別3  \\
{\it basetime} & INTEGER(4) &  & IN &  基準時刻(通算分)  \\
{\it member} & CHARACTER(4) &  & IN &  メンバー名  \\
{\it validtime} & INTEGER(4) &  & IN &  対象時刻(通算分)  \\
{\it group} & CHARACTER(4) &  & IN &  群名  \\
{\it n\_levels} & INTEGER(4) &  & OUT &  鉛直層数  \\
{\it result} & INTEGER(4) &  & OUT & \ResultCode \\
\hline
\end{tabular}
\paragraph{\FuncDesc}SUBC レコードの ETA, SIGM, ZHYB に記録された鉛直層数を問い合わせる。
群名には "ETA ", "SIGM", "ZHYB" のいずれかを指定する。
\paragraph{\ResultCode}
\begin{quote}
\begin{description}
\item[{\bf 正}] 正常終了
\end{description}\end{quote}
\paragraph{ 履歴 }
この関数は NuSDaS1.2 で導入された。
