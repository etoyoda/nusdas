\subsection{NUSDAS\_SET\_MASK: 改善型マスクビット設定関数}
\APILabel{nusdas.set.mask}

\Prototype
\begin{quote}
CALL {\bf NUSDAS\_SET\_MASK}({\it type1}, {\it type2}, {\it type3}, {\it udata}, {\it utype}, {\it usize}, {\it result})
\end{quote}

\begin{tabular}{l|rllp{16em}}
\hline
\ArgName & \ArgType & \ArrayDim & I/O & \ArgRole \\
\hline
{\it type1} & CHARACTER(8) &  & IN &  種別1  \\
{\it type2} & CHARACTER(4) &  & IN &  種別2  \\
{\it type3} & CHARACTER(4) &  & IN &  種別3  \\
{\it udata} & \AnyType & \AnySize & IN &  データ配列  \\
{\it utype} & CHARACTER(2) &  & IN &  データ配列の型  \\
{\it usize} & INTEGER(4) &  & IN &  配列の要素数  \\
{\it result} & INTEGER(4) &  & OUT & \ResultCode \\
\hline
\end{tabular}
\paragraph{\FuncDesc}
配列 {\it udata} の内容に従って \APILink{nusdas.make.mask}{nusdas\_make\_mask} と同様に
マスクビット列を作成し
指定した種別のデータセットに対して設定する。

\paragraph{\ResultCode}
\begin{quote}
\begin{description}
\item[{\bf 0}] 正常終了
\item[{\bf -5}] 未知の型名 {\it utype} が与えられた
\end{description}\end{quote}

\paragraph{注意}
本関数によるマスクビットの設定は \APILink{nusdas.parameter.change}{nusdas\_parameter\_change} に
優先するが、他のデータセットには効果をもたない。

\paragraph{履歴}
本関数は NuSDaS 1.3 で新設された。
