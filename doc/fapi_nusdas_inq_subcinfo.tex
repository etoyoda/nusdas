\subsection{NUSDAS\_INQ\_SUBCINFO: SUBC/INFO の問合せ}
\APILabel{nusdas.inq.subcinfo}

\Prototype
\begin{quote}
CALL {\bf NUSDAS\_INQ\_SUBCINFO}({\it type1}, {\it type2}, {\it type3}, {\it basetime}, {\it member}, {\it validtime}, {\it query}, {\it group}, {\it buf}, {\it bufnelems}, {\it result})
\end{quote}

\begin{tabular}{l|rllp{16em}}
\hline
\ArgName & \ArgType & \ArrayDim & I/O & \ArgRole \\
\hline
{\it type1} & CHARACTER(8) &  & IN &  種別1  \\
{\it type2} & CHARACTER(4) &  & IN &  種別2  \\
{\it type3} & CHARACTER(4) &  & IN &  種別3  \\
{\it basetime} & INTEGER(4) &  & IN &  基準時刻  \\
{\it member} & CHARACTER(4) &  & IN &  メンバー  \\
{\it validtime} & INTEGER(4) &  & IN &  対象時刻  \\
{\it query} & INTEGER(4) &  & IN &  問合せ項目  \\
{\it group} & CHARACTER(4) &  & IN &  群名  \\
{\it buf} & \AnyType & \AnySize & OUT &  結果格納配列  \\
{\it bufnelems} & INTEGER(4) &  & IN &  結果格納配列の要素数  \\
{\it result} & INTEGER(4) &  & OUT & \ResultCode \\
\hline
\end{tabular}
\paragraph{\FuncDesc}
引数 {\it type1} から {\it validtime} で指定されるデータファイルに書かれた
SUBC または INFO 記録について、
引数 {\it query} で指定される問合せを行う。
\begin{quote}\begin{description}
\item[{\bf N\_SUBC\_NUM}] 
SUBC 記録の個数が4バイト整数型変数 {\it buf} に書かれる。
引数 {\it group} は無視される。
\item[{\bf N\_SUBC\_LIST}] 
データファイルに定義された SUBC 記録の群名が配列 {\it buf} に書かれる。
配列 {\it buf} は長さ 4 文字の文字型で
{\it N\_SUBC\_NUM} 要素存在しなければならない。
引数 {\it group} は無視される。
\item[{\bf N\_SUBC\_NBYTES}] 
群名 {\it group} の SUBC 記録のバイト数が4バイト整数型変数 {\it buf} に書かれる。
\item[{\bf N\_SUBC\_CONTENT}] 
群名 {\it group} の SUBC 記録が配列 {\it buf} に書かれる。
上述のバイト数だけの長さを確保しておかねばならない。
\item[{\bf N\_INFO\_NUM}] 
INFO 記録の個数が4バイト整数型変数 {\it buf} に書かれる。
引数 {\it group} は無視される。
\item[{\bf N\_INFO\_LIST}] 
データファイルに定義された INFO 記録の群名が配列 {\it buf} に書かれる。
配列 {\it buf} は長さ 4 文字の文字型で
{\it N\_INFO\_NUM} 要素存在しなければならない。
引数 {\it group} は無視される。
\item[{\bf N\_INFO\_NBYTES}] 
群名 {\it group} の INFO 記録のバイト数が4バイト整数型変数 {\it buf} に書かれる。
\end{description}\end{quote}

\paragraph{\ResultCode}
\begin{quote}
\begin{description}
\item[{\bf 正}] 格納要素数
\end{description}\end{quote}
\paragraph{履歴}
この関数は NuSDaS 1.3 で新設された。
