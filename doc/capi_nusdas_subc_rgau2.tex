\subsection{nusdas\_subc\_rgau2: SUBC RGAU へのアクセス }
\APILabel{nusdas.subc.rgau2}

\Prototype
\begin{quote}
N\_SI4 {\bf nusdas\_subc\_rgau2}(const char {\it type1}[8], const char {\it type2}[4], const char {\it type3}[4], const N\_SI4 $\ast${\it basetime}, const char {\it member}[4], const N\_SI4 $\ast${\it validtime1}, const N\_SI4 $\ast${\it validtime2}, N\_SI4 $\ast${\it j}, N\_SI4 $\ast${\it j\_start}, N\_SI4 $\ast${\it j\_n}, N\_SI4 {\it i}[\,], N\_SI4 {\it i\_start}[\,], N\_SI4 {\it i\_n}[\,], float {\it lat}[\,], const char {\it getput}[3]);
\end{quote}

\begin{tabular}{l|rp{20em}}
\hline
\ArgName & \ArgType & \ArgRole \\
\hline
{\it type1} & const char [8] &  種別1  \\
{\it type2} & const char [4] &  種別2  \\
{\it type3} & const char [4] &  種別3  \\
{\it basetime} & const N\_SI4 $\ast$ &  基準時刻(通算分)  \\
{\it member} & const char [4] &  メンバー名  \\
{\it validtime1} & const N\_SI4 $\ast$ &  対象時刻1(通算分)  \\
{\it validtime2} & const N\_SI4 $\ast$ &  対象時刻2(通算分)  \\
{\it j} & N\_SI4 $\ast$ &  全球の南北分割数  \\
{\it j\_start} & N\_SI4 $\ast$ &  データの最北格子の番号(1始まり)  \\
{\it j\_n} & N\_SI4 $\ast$ &  データの南北格子数  \\
{\it i} & N\_SI4 [\,] &  全球の東西格子数  \\
{\it i\_start} & N\_SI4 [\,] &  データの最西格子の番号(1始まり)  \\
{\it i\_n} & N\_SI4 [\,] &  データの東西格子数  \\
{\it lat} & float [\,] &  緯度  \\
{\it getput} & const char [3] &  入出力指示 ({\it "GET}" または {\it "PUT}")  \\
\hline
\end{tabular}
