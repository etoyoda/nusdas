\subsection{NUSDAS\_SUBC\_ETA: SUBC ETA へのアクセス}
\APILabel{nusdas.subc.eta}

\Prototype
\begin{quote}
CALL {\bf NUSDAS\_SUBC\_ETA}({\it type1}, {\it type2}, {\it type3}, {\it basetime}, {\it member}, {\it validtime}, {\it n\_levels}, {\it a}, {\it b}, {\it c}, {\it getput}, {\it result})
\end{quote}

\begin{tabular}{l|rllp{16em}}
\hline
\ArgName & \ArgType & \ArrayDim & I/O & \ArgRole \\
\hline
{\it type1} & CHARACTER(8) &  & IN &  種別1  \\
{\it type2} & CHARACTER(4) &  & IN &  種別2  \\
{\it type3} & CHARACTER(4) &  & IN &  種別3  \\
{\it basetime} & INTEGER(4) &  & IN &  基準時刻(通算分)  \\
{\it member} & CHARACTER(4) &  & IN &  メンバー名  \\
{\it validtime} & INTEGER(4) &  & IN &  対象時刻(通算分)  \\
{\it n\_levels} & INTEGER(4) &  & I/O &  鉛直層数  \\
{\it a} & REAL(4) & \AnySize & I/O &  係数 a  \\
{\it b} & REAL(4) & \AnySize & I/O &  係数 b  \\
{\it c} & REAL(4) &  & I/O &  係数 c  \\
{\it getput} & CHARACTER(3) &  & IN &  入出力指示 ({\it "GET}" または {\it "PUT}")  \\
{\it result} & INTEGER(4) &  & OUT & \ResultCode \\
\hline
\end{tabular}
\paragraph{\FuncDesc}鉛直座標に ETA 座標系を用いるときに、鉛直座標を定めるパラメータへの
アクセスを提供する。 
パラメータは4バイト単精度浮動小数点型の配列{\it a}, {\it b}, {\it c} で構成され、
{\it a}, {\it b}, は鉛直層数 {\it n\_levels} に対して、{\it n\_levels}+1 要素の配列、
{\it c} は1要素の配列(変数)を確保する必要がある。
{\it n\_levels} は nusdas\_subc\_inq\_nz で問い合わせることができる。
入出力指示が {\it GET} の場合においても、{\it n\_levels} は書込み対象変数として扱われる。
特に parameter 宣言された変数を {\it n\_levels} に指定してはならない。
\paragraph{\ResultCode}
\begin{quote}
\begin{description}
\item[{\bf 0}] 正常終了
\item[{\bf -2}] レコードが存在しない、またはレコードの書き込みがされていない。
\item[{\bf -3}] レコードサイズが不正
\item[{\bf -4}] ユーザーの鉛直層数がファイルの中の鉛直層数より小さい
\item[{\bf -5}] 入出力指示が不正。
\end{description}\end{quote}
\paragraph{ 履歴 }
この関数は NuSDaS1.0 から存在した。
NuSDaS1.1までは、レコードが書き込まれたかの情報を持ち合わせていなかった
ために無記録のレコードをファイルから読んで正常終了していた。 NuSDaS 1.3 からは
ファイルの初期化時にレコードを初期化し、未記録を判定できるようにした。
その場合のエラーは-2としている。
\paragraph{ 注意 }
SUBC ETA に使われている鉛直層数 {\it n\_levels} は実際のモデルの鉛直層数と
異なっている場合があるので、配列確保の際にはnusdas\_subc\_inq\_nzで問い
合わせた結果を用いること。
