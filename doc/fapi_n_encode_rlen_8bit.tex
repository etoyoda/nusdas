\subsection{N\_ENCODE\_RLEN\_8BIT: 4バイト整数を RLE 圧縮する}
\APILabel{n.encode.rlen.8bit}

\Prototype
\begin{quote}
CALL {\bf N\_ENCODE\_RLEN\_8BIT}({\it udata}, {\it compressed\_data}, {\it udata\_nelems}, {\it max\_compress\_nbytes}, {\it maxvalue}, {\it result})
\end{quote}

\begin{tabular}{l|rllp{16em}}
\hline
\ArgName & \ArgType & \ArrayDim & I/O & \ArgRole \\
\hline
{\it udata} & INTEGER(4) & \AnySize & IN &  元データ配列  \\
{\it compressed\_data} & CHARACTER & \AnySize & OUT &  結果格納配列  \\
{\it udata\_nelems} & INTEGER(4) &  & IN &  元データの要素数  \\
{\it max\_compress\_nbytes} & INTEGER(4) &  & IN &  結果配列のバイト数  \\
{\it maxvalue} & INTEGER(4) &  & OUT &  元データの最大値  \\
{\it result} & INTEGER(4) &  & OUT & \ResultCode \\
\hline
\end{tabular}
\paragraph{\FuncDesc}\paragraph{履歴}
この関数は NuSDaS 1.0 から存在するが、ドキュメントされていなかった。
NuSDaS 1.3 から Fortran API を伴う
サービスサブルーチンとして採録された。
