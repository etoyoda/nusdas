\subsection{NUSDAS\_INQ\_PARAMETER: オプション取得}
\APILabel{nusdas.inq.parameter}

\Prototype
\begin{quote}
CALL {\bf NUSDAS\_INQ\_PARAMETER}({\it param}, {\it value}, {\it result})
\end{quote}

\begin{tabular}{l|rllp{16em}}
\hline
\ArgName & \ArgType & \ArrayDim & I/O & \ArgRole \\
\hline
{\it param} & INTEGER(4) &  & IN &  設定項目コード  \\
{\it value} & \AnyType & \AnySize & OUT &  設定値  \\
{\it result} & INTEGER(4) &  & OUT & \ResultCode \\
\hline
\end{tabular}
\paragraph{\FuncDesc}
\APILink{nusdas.parameter.change}{nusdas\_parameter\_change} の項目 {\it param} で設定される
パラメータの値を {\it value} の指す領域 (型は以下を参照) に書き込む。
\begin{quote}\begin{description}
\item[{\bf N\_PC\_MISSING\_UI1}] 1バイト整数の欠損値
\item[{\bf N\_PC\_MISSING\_SI2}] 2バイト整数の欠損値
\item[{\bf N\_PC\_MISSING\_SI4}] 4バイト整数の欠損値
\item[{\bf N\_PC\_MISSING\_R4}] 4バイト実数の欠損値
\item[{\bf N\_PC\_MISSING\_R8}] 8バイト実数の欠損値
\item[{\bf N\_PC\_SIZEX}] 4バイト整数に x 方向強制格子サイズを与える
\item[{\bf N\_PC\_SIZEY}] 4バイト整数に y 方向強制格子サイズを与える
\item[{\bf N\_PC\_MASK\_BIT}] 
マスクビット配列を返す。
この問合せは設定値が \APILink{nusdas.make.mask}{nusdas\_make\_mask} で作られた場合にしか機能しない。
\item[{\bf N\_PC\_PACKING}] 
4バイトの文字列に強制パック方式名を与える。
設定されていない場合は 4 文字のスペースが書き込まれる。
\item[{\bf N\_PC\_ID\_SET}] 
NRD 番号制約がかかっている場合その値、かかっていない場合 -1 を与える。
\item[{\bf N\_PC\_WBUFFER}] 
4バイト整数に書き込みバッファサイズ (実行時オプション FWBF) を与える。
\item[{\bf N\_PC\_RBUFFER}] 
4バイト整数に読み取りバッファサイズ (実行時オプション FRBF) を与える。
\end{description}\end{quote}

\paragraph{\ResultCode}
\begin{quote}
\begin{description}
\item[{\bf 0}] 正常終了
\item[{\bf -1}] サポートされていないパラメタである
\item[{\bf -2}] マスクビット配列は設定されていない
\item[{\bf -3}] マスクビット配列は設定されているが長さがわからない
\end{description}\end{quote}

\paragraph{履歴}
NuSDaS 1.3 で導入された。
