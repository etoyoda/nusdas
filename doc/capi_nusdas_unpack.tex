\subsection{nusdas\_unpack: 生DATAレコードの解読}
\APILabel{nusdas.unpack}

\Prototype
\begin{quote}
N\_SI4 {\bf nusdas\_unpack}(const void $\ast${\it pdata}, void $\ast${\it udata}, const char {\it utype}[2], N\_SI4 {\it usize});
\end{quote}

\begin{tabular}{l|rp{20em}}
\hline
\ArgName & \ArgType & \ArgRole \\
\hline
{\it pdata} & const void $\ast$ &  パックされたバイト列  \\
{\it udata} & void $\ast$ &  展開先配列  \\
{\it utype} & const char [2] &  展開する型  \\
{\it usize} & N\_SI4 &  展開先配列の要素数  \\
\hline
\end{tabular}
\paragraph{\FuncDesc}
\APILink{nusdas.inq.data}{nusdas\_inq\_data} の問い合わせ N\_DATA\_CONTENT で得られるバイト列を
解読して数値配列を得る。パッキング型2UPJでは利用できない(2UPPは利用可)。

\paragraph{\ResultCode}
\begin{quote}
\begin{description}
\item[{\bf 正}] 正常終了、値は要素数
\item[{\bf -4}] 展開先の大きさ {\it usize} がデータレコードの要素数より少ない
\item[{\bf -5}] パッキング型・欠損値型・展開型の組合せが不適
\item[{\bf -6}] 利用できないパッキング型が与えられた
\end{description}\end{quote}

\paragraph{履歴}
本関数は NuSDaS 1.3 で追加された。
エラーコード -6 は NuSDaS 1.4 で新設されたもので、
それ以前はエラーチェックがなされていなかった。
