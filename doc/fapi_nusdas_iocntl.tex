\subsection{NUSDAS\_IOCNTL: 入出力フラグ設定}
\APILabel{nusdas.iocntl}

\Prototype
\begin{quote}
CALL {\bf NUSDAS\_IOCNTL}({\it param}, {\it value}, {\it result})
\end{quote}

\begin{tabular}{l|rllp{16em}}
\hline
\ArgName & \ArgType & \ArrayDim & I/O & \ArgRole \\
\hline
{\it param} & INTEGER(4) &  & IN &  設定項目コード  \\
{\it value} & INTEGER(4) &  & IN &  設定値  \\
{\it result} & INTEGER(4) &  & OUT & \ResultCode \\
\hline
\end{tabular}
\paragraph{\FuncDesc}
入出力にかかわるフラグを設定する.
\begin{quote}\begin{description}
\item[{\bf N\_IO\_MARK\_END}] 既定値1.
零にすると \APILink{nusdas.write}{nusdas\_write} などの出力関数を呼び出すたびに
データファイルへの出力を完結させ END 記録を書き出すのをやめる.
\item[{\bf N\_IO\_W\_FCLOSE}] 既定値1.
零にすると \APILink{nusdas.write}{nusdas\_write} などの出力関数を呼び出すたびに
書き込み用に開いたファイルを閉じるのをやめる.
速度上有利だが、データファイルの操作が終了した後で
ファイルを閉じる関数 \APILink{nusdas.allfile.close}{nusdas\_allfile\_close} または
\APILink{nusdas.onefile.close}{nusdas\_onefile\_close} を適切に呼んでファイルを閉じないと
出力ファイルが不完全となり、後で読むことができない.
なお、このフラグを変更すると N\_IO\_MARK\_END も連動する.
\item[{\bf N\_IO\_R\_FCLOSE}] 既定値1.
零にすると \APILink{nusdas.read}{nusdas\_read} などの入力関数を呼び出すたびに
読み込み用に開いたファイルを閉じるのをやめる.
速度上有利だが、多数のファイルから入力するプログラムでは
ファイルハンドルが枯渇する懸念があるので
ファイルを明示的に閉じることが推奨される.
\item[{\bf N\_IO\_WARNING\_OUT}] 既定値1.
0 にするとエラーメッセージだけが出力される.
1 にすると、それに加えて警告メッセージも出力されるようになる.
2 にすると、それに加えてデバッグメッセージも出力されるようになる.
\item[{\bf N\_IO\_BADGRID}] 既定値0.
1 にすると投影法パラメタの検査で不適切な値が検出されても
データファイルが作成できるようになる。
\end{description}\end{quote}

\paragraph{\ResultCode}
\begin{quote}
\begin{description}
\item[{\bf 0}] 正常終了
\item[{\bf -1}] サポートされていないパラメタである
\end{description}\end{quote}

\paragraph{履歴}
この関数は NuSDaS 1.0 から存在した.
{\bf N\_IO\_WARNING\_OUT} の値 2 は NuSDaS 1.3 からの拡張である.
{\bf N\_IO\_BADGRID} も NuSDaS 1.3 からの拡張である.
