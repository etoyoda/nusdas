\subsection{NUSDAS\_ALLFILE\_CLOSE: 全てのデータファイルを閉じる}
\APILabel{nusdas.allfile.close}

\Prototype
\begin{quote}
CALL {\bf NUSDAS\_ALLFILE\_CLOSE}({\it param}, {\it result})
\end{quote}

\begin{tabular}{l|rllp{16em}}
\hline
\ArgName & \ArgType & \ArrayDim & I/O & \ArgRole \\
\hline
{\it param} & INTEGER(4) &  & IN &  閉じるファイルの種類  \\
{\it result} & INTEGER(4) &  & OUT & \ResultCode \\
\hline
\end{tabular}
\paragraph{\FuncDesc}
今までに NuSDaS インターフェイスで開いた全てのファイルを閉じる.
引数 param は次のいずれかを用いる:
\begin{quote}\begin{description}
\item[{\bf N\_FOPEN\_READ}] 読み込み用に開いたファイルだけを閉じる
\item[{\bf N\_FOPEN\_WRITE}] 書き込み可で開いたファイルだけを閉じる
\item[{\bf N\_FOPEN\_ALL}] すべてのファイルを閉じる
\end{description}\end{quote}

\paragraph{\ResultCode}
\begin{quote}
\begin{description}
\item[{\bf 正}] 正常に閉じられたファイルの数
\item[{\bf 0}] 閉じるべきファイルがなかった
\item[{\bf 負}] 閉じる際にエラーが起こったファイルの数
\end{description}\end{quote}

\paragraph{履歴}
この関数は NuSDaS 1.0 から存在した.
