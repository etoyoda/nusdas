\subsection{NUSDAS\_SUBC\_TDIF: SUBC TDIF へのアクセス}
\APILabel{nusdas.subc.tdif}

\Prototype
\begin{quote}
CALL {\bf NUSDAS\_SUBC\_TDIF}({\it type1}, {\it type2}, {\it type3}, {\it basetime}, {\it member}, {\it validtime}, {\it diff\_time}, {\it total\_sec}, {\it getput}, {\it result})
\end{quote}

\begin{tabular}{l|rllp{16em}}
\hline
\ArgName & \ArgType & \ArrayDim & I/O & \ArgRole \\
\hline
{\it type1} & CHARACTER(8) &  & IN &  種別1  \\
{\it type2} & CHARACTER(4) &  & IN &  種別2  \\
{\it type3} & CHARACTER(4) &  & IN &  種別3  \\
{\it basetime} & INTEGER(4) &  & IN &  基準時刻(通算分)  \\
{\it member} & CHARACTER(4) &  & IN &  メンバー名  \\
{\it validtime} & INTEGER(4) &  & IN &  対象時刻(通算分)  \\
{\it diff\_time} & INTEGER(4) &  & I/O &  対象時刻からのずれ(秒)  \\
{\it total\_sec} & INTEGER(4) &  & I/O &  総予報時間(秒)  \\
{\it getput} & CHARACTER(3) &  & IN &  入出力指示 ({\it "GET}" または {\it "PUT}")  \\
{\it result} & INTEGER(4) &  & OUT & \ResultCode \\
\hline
\end{tabular}
\paragraph{\FuncDesc}格納した値の時刻の対象時間とのずれ、積算時間を格納する補助管理部 TDIF 
へのアクセスを提供する。
\paragraph{\ResultCode}
\begin{quote}
\begin{description}
\item[{\bf 0}] 正常終了
\item[{\bf -2}] 要求されたレコードが存在しない、または書き込まれていない。
\item[{\bf -3}] レコードサイズが不正
\item[{\bf -5}] 入出力指示が不正
\end{description}\end{quote}

\paragraph{補足}
\begin{itemize}
\item  diff\_time = 時間範囲始点 - 対象時刻 [秒単位]
\item  total\_sec = 時間範囲終点 - 時間範囲始点 [秒単位]
\end{itemize}

\paragraph{履歴}
この関数は NuSDaS1.0 から存在した。
