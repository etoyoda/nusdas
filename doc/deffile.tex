\Chapter{定義ファイルリファレンス}
\label{chap:deffile}

\section{はじめに}
\subsection{定義ファイルの文法}

NuSDaS 定義ファイルは概ねフリーフォーマットのテキストである。
\begin{itemize}
\item 定義ファイルは改行文字で区切られる行からなる。
\item 各行は空白 (スペース、タブ、鉛直タブ、フォームフィードの
	いずれか) で区切られる語に分解される。
\item 行頭の語が `{\tt \#}' で始まるとき、その行は無視される。
	(NuSDaS 1.1 では公式にはコメント機能はなかったが、
	たまたま定義ファイルパーザの作りがよかったので動作していた)
\item 行頭の語が次のキーワード (大文字・小文字は区別されない)
	のひとつであるとき\footnote{
		NuSDaS 1.3 以降でもルーチンに見られるスペル誤り
		OTHER, MISSSING, SUBCTNL と OPTION
		を許容する。
	}、
	その名をもつ文の解析が始まる。
\begin{quote}
NUSDAS PATH FILENAME CREATOR TYPE1 TYPE2 TYPE3 MEMBER {\nobreak MEMBERLIST}
BASETIME VALIDTIME VALIDTIME1 VALIDTIME2 PLANE PLANE1 PLANE2 \\*
ELEMENT ELEMENTMAP SIZE BASEPOINT DISTANCE STANDARD OTHERS VALUE PACKING
MISSING INFORMATION SUBCNTL FORCEDRLEN OPTIONS
\end{quote}
\item 複数の語からなる文は次の行にまたがって書いてもよい。
	ただし、行頭の語が上のキーワードになってはならない。
\item 複数の文をひとつの行に書くべきではない。
	NuSDaS 1.1 で読めなくなるからである。
\item 一部の文は決まった順序で書かなければならない。
\item ELEMENTMAP 文と INFORMATION 文をのぞき、
	同種の文が複数現れることはない。
\end{itemize}

\subsection{凡例}

\begin{itemize}
\item タイプライタ体 {\tt typewriter font} の文字は文字どおり用いられる
	ことを意味する。
\item 斜体 {\it italic} の文字は適宜置き換えるべき語を意味する。
\item 「書式」の項の各語は定義ファイルの1語に対応する。
\item 「書式」の項で ``...'' とあるのは語数が不定であることを示す。
	不定といっても先行する語や文などで数は定まるのであり、
	本文を参照されたい。
\item 経緯度を設定するとき
	$\lambda_*${\tt E}, $\varphi_*${\tt N} のように
	書いてあるところにそれぞれ
	$\lambda_*${\tt W}, $\varphi_*${\tt S} のように
	書けば、それぞれ西経・南緯を設定することができる。
\item 経緯度を設定する文の経緯度
	(書式の項に $\lambda_*${\tt E} \ $\varphi_*${\tt N} と書いてある)
	は逆順に書いてもよい。
\end{itemize}

\section{BASEPOINT: 参照点の設定}
\label{sec:def:BASEPOINT}
\paragraph{書式}
\begin{quote}
{\tt BASEPOINT} $x_0$ $y_0$ $\lambda_0${\tt E} $\varphi_0${\tt N}
\end{quote}
\paragraph{機能}
BASEPOINT 文はデータセットの格子番号 ($x_0$, $y_0$) が
地球上の経緯度 ($\lambda_0$, $\varphi_0$) を持つ地点であることを設定する。
\paragraph{省略時}
あたかも
{\tt BASEPOINT 0 0 0 0}
が書かれたかのように扱われる。
この設定値が適切か否かは2次元座標系しだいであり、
一概に誤りと断定することはできないが、
格子番号は1から数え始めることを考えるとほとんどの場合有用ではない。

\section{BASETIME: 基準時刻の設定}
\label{sec:def:BASETIME}
\paragraph{書式}
\begin{quote}
{\tt BASETIME} {\it yyyymmddhhnn}
\end{quote}
\paragraph{機能}
BASETIME 文は定数データセットが持つ唯一つの基準時刻を設定する。
ここに設定した以外の基準時刻によるデータの読み書きは失敗するように
コーディングされるべきである、が、
NuSDaS 1.1 のコードを見ても引数 {\it yyyymmddhhnn} を
使っているようには見えない。
\paragraph{省略時}
データセットには任意の基準時刻のデータファイルを作成できるようになる。
\paragraph{未実装注意}
この機能は使われていないので NuSDaS 1.4 でもまだ実装されていない。

\section{CREATOR: データ作成者の設定}
\label{sec:def:CREATOR}
\paragraph{書式}
\begin{quote}
{\tt CREATOR} {\it creator}
\end{quote}
\paragraph{機能}
CREATOR 文はデータファイルの NUSD レコード
作成者フィールド
(\TabRef{table.fmt.nusd} 項番 5)
に書き込む文字列を設定する。
\paragraph{省略時}
コンパイル時に設定した値が用いられる。
デフォルトは
\begin{quote}
{\tt Japan Meteorological Agency, http://www.jma.go.jp/}
\end{quote}
である。
\paragraph{バグ}
空白を含んだ任意文字列を設定する方法がない。

\section{DISTANCE: 格子間隔の設定}
\label{sec:def:DISTANCE}
\paragraph{書式}
以下のいずれか:
\begin{quote}
{\tt DISTANCE} $D_i$ $D_j$ \\
{\tt DISTANCE} $D_r$ $D_\theta$ \\
{\tt DISTANCE} $D_X$ $D_Y$
\end{quote}
\paragraph{機能}
DISTANCE 文は格子間隔を設定する。
\begin{itemize}
\item
 2次元座標系が {\tt LL}, {\tt GS}, {\tt RG} のとき、
 {\tt DISTANCE} $D_i$ $D_j$ 
 は経緯度の度単位での格子間隔 $D_i$, $D_j$ を設定する。
\item
 2次元座標系が {\tt RT} のとき、
 {\tt DISTANCE} $D_r$ $D_\theta$ 
 はメートル単位での格子間隔 $D_X$ と
 度単位での方位角間隔 $D_\theta$ を設定する。
\item
 2次元座標系が
 地図投影 ({\tt MR}, {\tt PS}, {\tt LM}, {\tt OL}) のとき、
 {\tt DISTANCE} $D_X$ $D_Y$ 
 はメートル単位での格子間隔 $D_X$, $D_Y$ を設定する。
\end{itemize}
\paragraph{省略時}
あたかも {\tt DISTANCE 0 0} が書かれたかのように動作する。
2次元座標系が {\tt ST} など、格子間隔の概念をもたず、
当然グリッド情報チェックでも値ゼロが許容される場合に使える。

\section{ELEMENT: 要素数の設定}
\label{sec:def:ELEMENT}
\paragraph{書式}
\begin{quote}
{\tt ELEMENT} $N_E$ \end{quote}
\paragraph{順序制約}
\DefRef{ELEMENTMAP} よりも先に記述しなければならない。
\paragraph{機能}
ELEMENT 文は要素の数を設定する。
ELEMENTMAP 文は $N_E$ 個設定する必要がある。
\paragraph{必須性}
ELEMENT 文を省略してはならない。

\section{ELEMENTMAP: 要素名と書込禁止制約}
\subsection{概要}
\label{sec:def:ELEMENTMAP}
\paragraph{書式}
以下のいずれか:
\begin{quote}
{\tt ELEMENTMAP} {\it elemname} {\tt 0} \\
{\tt ELEMENTMAP} {\it elemname} {\tt 1} {\it bitmap} ... \\
{\tt ELEMENTMAP} {\it elemname} {\tt 2} [$n_V$ {\it bitmap} ...] ...
\end{quote}
\paragraph{順序制約}
\DefRef{MEMBER} より前に記述してはならない。
\DefRef{VALIDTIME} よりも後に記述しなければならない。
\DefRef{PLANE} よりも後に記述しなければならない。
\DefRef{ELEMENT} よりも後に記述しなければならない。
ELEMENTMAP 文の $N_E$ 個記述しなければならないし、それを越えてもいけない。
\paragraph{機能}
定義ファイル中に現れた $i_E$ 番目の
ELEMENTMAP 文は要素名 {\it elemname} とともに
その要素に関する書込禁止制約のビットマップ (elementmap と呼ばれる) を
データセットの $i_E$ 番目の要素として登録する。
\paragraph{必須性}
ELEMENTMAP 文を省略してはならない。

\subsection{第0種ELEMENTMAP文}
第3語が {\tt 0} の ELEMENTMAP 文は、
要素 {\it elemname} に書込禁止制約を設定しないことを意味する。
たとえば 
\begin{screen}
{\tt ELEMENTMAP T  0}
\end{screen}
ならば、{\tt T} という要素はすべてのメンバー、対象時刻、
面の組み合わせについて書込可能である。

\subsection{第1種ELEMENTMAP文}
第3語が {\tt 1} の ELEMENTMAP 文は、
すべてのメンバー・対象時刻に共通のビットマップで
要素 {\it elemname} に書込禁止制約を設定する。
第4語以下 $N_Z$ [\DefRef{PLANE}] 個のビットが並び、
各ビットは {\tt 0} が書込禁止、{\tt 1} が書込許可を意味する。

たとえば 
\begin{screen}
\begin{verbatim}
PLANE         5
PLANE1                SURF 1000 700 500 300
ELEMENTMAP U       1  0    1    1   1   1
\end{verbatim}
\end{screen}
ならば、{\tt U} という要素は {\tt SURF} 以外の面について書込可能である。

なお、ビット数が多過ぎるときは行末まで読み飛ばされ、
少な過ぎるときは不足分に {\tt 0} が補われるが、
将来の版でエラーに変更されるかもしれない。

\subsection{第2種ELEMENTMAP文}
第3語が {\tt 2} の ELEMENTMAP 文は、
すべてのメンバーについて共通だけれど対象時刻によって異なるビットマップで
要素 {\it elemname} に書込禁止制約を設定する。
第4語以下はビットマップのくり返しである。
各ビットマップはくり返し数 $n_V$ を前置した $N_Z$ 個のビットで、
くり返し数の合計が $N_V$ [\DefRef{VALIDTIME}] と
なったところで文が終わる。

たとえば
\begin{screen}
\begin{verbatim}
VALIDTIME     12 HOUR in
VALIDTIME1 ARITHMETIC 0 1
PLANE         5
PLANE1                   SURF 1000 700 500 300
ELEMENTMAP TKE     2   1 0    0    0   0   0
                      10 0    1    1   0   0
                       1 0    0    0   0   0
\end{verbatim}
\end{screen}
ならば、要素 {\tt TKE} は
予報時間 1 から 10 の間だけ、
面 {\tt 1000} と {\tt 700} についてだけ書込できる。

なお、くり返し数が過大のときは行末まで読み飛ばされ、
過小であるときは {\tt 0} が補われる、
つまり該当する対象時刻は書込禁止となるが、
将来の版でエラーに変更されるかもしれない。

\subsection{第3種ELEMENTMAP文 (参考)}

第3語が {\tt 3} の ELEMENTMAP 文は
メンバーによっても対象時刻によっても異なる ELEMENTMAP を設定できる。
第4語以下は
「メンバーくり返し数を前置した第1種または第2種 ELEMENTMAP 文の第3語目以降」
のくり返しであり、
メンバーくり返し数の合計が $N_M$ [\DefRef{MEMBER}] と
なったところで文が終わる。
\paragraph{バグ}
第3種 ELEMENTMAP 文の実装はいいかげんであり、よくテストされていない。

\section{FILENAME: データファイル名の設定}
\label{sec:def:FILENAME}
\paragraph{書式}
\begin{quote}
{\tt FILENAME} {\it filename}
\end{quote}
\paragraph{機能}
FILENAME 文は \DefRef{PATH} と組み合わせてデータファイルの
名前を設定する。詳細は PATH 文の項を参照せよ。
\paragraph{注意}
いわゆる NWP 系のパス指定を用いると FILENAME 文の設定は無視される。

\section{FORCEDRLEN: 強制レコード長の設定}
\label{sec:def:FORCEDRLEN}
\paragraph{書式}
\begin{quote}
{\tt FORCEDRLEN} {\it nbytes}
\end{quote}
\paragraph{機能}
FORCEDRLEN 文はデータファイルのレコード長を {\it nbytes} バイトに設定する。
ここでレコード長とはレコード全体、
つまり \TabRef{table.fmt.general} の項番 1--7 の長さである。
\paragraph{省略時}
データファイルのレコード長は内容に応じて可変となる。
ただし、ファイル形式 13 以降では、レコード長は8の倍数となるように
調整される。

\section{INFORMATION: INFOレコードの定義}
\label{sec:def:INFORMATION}
\paragraph{書式}
\begin{quote}
{\tt INFORMATION} {\it group} {\it nbytes} {\it filename}
\end{quote}
\paragraph{位置制約}
INFORMATION 文は好きな数だけ記述してよい。
\paragraph{機能}
INFORMATION 文は群名 {\it group} の INFO レコードを定義する。
データファイル作成時には長さ {\it nbytes} のファイル {\it filename} を
読み込んでレコード内容が作られる。
したがって {\it nbytes} はレコードのペイロード長すなわち
\TabRef{table.fmt.general} の項番 5 の長さである。
\paragraph{省略時}
\DefRef{SUBCNTL} と同様に INFO レコードを作ることができなくなる、
と言いたいところであるが、
残念ながら INFO レコードは後から書き足すことができる。

\section{MEMBER: メンバー数の設定}
\label{sec:def:MEMBER}
\paragraph{書式}
つぎのいずれか:
\begin{quote}
{\tt MEMBER} $N_M$ {\tt IN}\\
{\tt MEMBER} $N_M$ {\tt OUT}\\
\end{quote}
\paragraph{位置制約}
\DefRef{MEMBERLIST},
\DefRef{ELEMENTMAP} より先に記述しなければならない。
\paragraph{機能}
MEMBER 文はメンバー数を設定する。
第3語の {\tt IN} は異なるメンバーのデータがひとつのデータファイルに
書かれることを意味し、
第3語の {\tt OUT} は異なるメンバーのデータは別のデータファイルに
書かれることを意味する。
\paragraph{省略時}
メンバーはただひとつスペース4文字 {\tt \SPC\SPC\SPC\SPC} のものが作られる。
\paragraph{バグ}
第3語の IN/OUT と \DefRef{PATH} や
\DefRef{FILENAME} との矛盾が検査されない。

\section{MEMBERLIST: メンバー名の設定}
\label{sec:def:MEMBERLIST}
\paragraph{書式}
\begin{quote}
{\tt MEMBERLIST} {\it name} ...
\end{quote}
\paragraph{位置制約}
\DefRef{MEMBER} より後に記述しなければならない。
MEMBER 文を書いたならば省略してはならない。
\paragraph{機能}
MEMBERLIST 文は $N_M$ 個のメンバー名 {\it name} を設定する。
メンバー名は 4文字以下の英数字または下線である。

\section{MISSING: 欠損値表現法の設定}
\label{sec:def:MISSING}
\paragraph{書式}
次のいずれか:
\begin{quote}
{\tt MISSING NONE}\\
{\tt MISSING UDFV}\\
{\tt MISSING MASK}
\end{quote}
\paragraph{機能}
MISSING 文は欠損値の表現方法を設定する。
それぞれの機能は\TabRef{tab:missing}を参照。

\paragraph{省略時}
{\tt MISSING NONE} が仮定される。
\paragraph{注意}
UDFVで欠損値を設定して書き込みを行う場合、PACKING が I1, I2, I4, R4, R8 の際には
以下の値は必ず欠損値として扱われる。これは
\APIRef{nusdas.parameter.change}{nusdas\_parameter\_change}で
欠損値として扱う値を変更した場合も変わらない。
\begin{itemize}
\item I1, I2, I4 では全bitが1となる -128, -32768, -2147483648
\item R4, R8 ではINFを除いた最大値に相当する 3.40282347e+38, 1.7976931348623157d+308
\end{itemize}
なお、 PACKING が 2UPC など I1, I2, I4, R4, R8 以外の場合は本件に該当しない。

\section{NUSDAS: データファイルの版番号}
\label{sec:def:NUSDAS}
\paragraph{書式}
\begin{quote}
{\tt NUSDAS} {\it version}
\end{quote}
\paragraph{機能}
この定義ファイルを持つデータセットで新規に作成される
データファイルの形式を {\it version} に設定する。
有効な {\it version} の値は次の通り:
\begin{description}
\item[10] ファイル形式 10 となる。
\item[11] ファイル形式 11 となる。
\item[13] ファイル形式 13 となる。
\item[14] ファイル形式 14 となる。
\end{description}
\paragraph{省略時}
configureオプションによる。特に設定せずにconfigureした場合は{\tt NUSDAS 14} が仮定される。
{\tt --enable-dfver}を設定した場合は{\tt NUSDAS 11}が、
{\tt --enable-dfver=13}を設定した場合は{\tt NUSDAS 13}が仮定される。
NAPS9, NAPS10 およびこれらの関連機器では{\tt NUSDAS 11}が仮定されるようになっている。
アプリケーションプログラマは将来デフォルトが
{\tt NUSDAS 14} などに変更される可能性を考慮すべきである。

\section{OPTIONS: データセットの実行時オプションの設定}
\label{sec:def:OPTIONS}
\paragraph{書式}
\begin{quote}
{\tt options} \ {\it optstring}
\end{quote}
\paragraph{機能}
OPTIONS 文は実行時オプションを設定する。
設定ファイル ``{\tt nusdas.ini}'' または環境変数 ``{\tt NUSDAS\_OPTS}''
と異なり、設定の効力はデータセットに限定される。
一方、システム全体の設定 (項目名が `G' ではじまるもの) は設定できない。
項目の詳細については \SectionRef{sec:opts:runtime} 参照。

\section{OTHERS: 斜軸ランベルトのパラメタ設定}
\label{sec:def:OTHERS}
\paragraph{書式}
\begin{quote}
{\tt others} \ $\hat\lambda_P${\tt E} \ $\hat\varphi_P${\tt N}
	\ $\lambda_E${\tt E} \ $\varphi_E${\tt N}
\end{quote}
\paragraph{機能}
OTHERS 文は地図投影パラメタが 5 個以上あるときの設定に用いる。
実際には斜軸ランベルト図法でしか使わないので、
\SectionRef{sec:proj:OL} を参照されたい。
\paragraph{省略時}
{\tt OTHERS 0E 0N 0E 0N} が仮定される。
斜軸ランベルト以外では OTHERS 文を省略する。

\section{PACKING: パッキング方式設定}
\label{sec:def:PACKING}
\paragraph{書式}
\begin{quote}
{\tt PACKING} {\it packing}
\end{quote}
\paragraph{機能}
PACKING 文はデータレコードのパック方式を {\it packing} に設定する。
許される値については\TabRef{tab:packing}を、
またパック方式については\ChapRef{chap:packing}参照。

\paragraph{省略時}
{\tt PACKING 2UPC} が仮定される。

\section{PATH: ディレクトリ構造の設定}
\label{sec:def:PATH}
\paragraph{書式}
次のいずれか:
\begin{quote}
{\tt PATH NWP\_PATH\_S} \\
{\tt PATH NWP\_PATH\_BS} \\
{\tt PATH NWP\_PATH\_M} \\
{\tt PATH NWP\_PATH\_VM} \\
{\tt PATH RELATIVE\_PATH} {\it relpath} \\
{\tt PATH NWP\_ESF}
\end{quote}
\paragraph{機能}
PATH 文は \DefRef{FILENAME} とともに
データファイルの位置を設定する。
PATH 文の第一引数に対応して、次のようにデータファイルが決められる。

\begin{tabular}{lp{35zw}}
{\tt RELATIVE\_PATH} & 
 引数 {\it relpath} はスラッシュを含みうる文字列である。
 データファイルの名前は \newline
 {\tt NUSDAS}{\it nn}{\tt /}{\it relpath}{\tt /}{\it filename}
 あるいは FILENAME 文を省略すると \newline
 {\tt NUSDAS}{\it nn}{\tt /}{\it relpath}
 となる。
 ここで {\it nn} は NRD 番号であり、
 {\it relpath} をスラッシュで区切ったものと {\it filename}
 のそれぞれについてパス名置換(後述)が行われる。
 MEMBER 文、VALIDTIME 文の設定はパス名に影響しない。
\\
 PATH 文省略時 &
 前項に準じて、
 定義ファイルにあたかも \newline
 {\tt PATH RELATIVE\_PATH}
 {\tt \_model/\_attribute/\_space/\_time/\_name/\_basetime} \newline
 と書かれているかのようにふるまう。
 ただし、MEMBER 文に OUT が設定されている場合は
 このあとに {\tt /\_member} が追加され、
 VALIDTIME 文に OUT が設定されている場合は
 さらにそのあとに {\tt /\_validtime} が追加される。
 FILENAME 文が存在すれば、さらにこの後に {\tt /}{\it filename} が追加される。
\\
 {\tt NWP\_PATH\_S} &
 データファイルの名前は
 {\tt NUSDAS}{\it nn}{\tt /}{\it 3dname}{\tt /}{\it validtime}
 となる。 
 ここで、{\it 3dname} はパス名置換の {\tt \_3d} と {\tt \_name} を
 連結したもの、
 {\it validtime} はパス名置換の {\tt \_validtime} と同じである。
 FILENAME 文、MEMBER 文、VALIDTIME 文の設定はパス名に影響しない。
\\
 {\tt NWP\_PATH\_BS} &
 データファイルの名前は
 {\tt NUSDAS}{\it nn}{\tt /}{\it 3dname}{\tt /}{\it basetime}
 となる。 
 ここで {\it basetime} はパス名置換の {\tt \_basetime} と同じである。
 FILENAME 文、MEMBER 文、VALIDTIME 文の設定はパス名に影響しない。
\\
 {\tt NWP\_PATH\_M} &
 データファイルの名前は
 {\tt NUSDAS}{\it nn}{\tt /}{\it 3dname}{\tt /}{\it member}%
 {\tt /}{\it validtime}
 となる。 
 ここで {\it member} はパス名置換の {\tt \_member} と同じである。
 FILENAME 文、MEMBER 文、VALIDTIME 文の設定はパス名に影響しない。
\\
 {\tt NWP\_PATH\_VM} &
 データファイルの名前は
 {\tt NUSDAS}{\it nn}{\tt /}{\it 3dname}{\tt /}{\it member}
 となる。 
 FILENAME 文、MEMBER 文、VALIDTIME 文の設定はパス名に影響しない。
\\
 {\tt NWP\_ESF} &
 データファイルの名前は
 {\tt NUSD}{\it type1type2type3}
 となる。
 通常は日立 CSES のために実行時オプション {\tt IESF}
 \SectionRef{sec:opts:runtime}
 とともに使う。
\\
 上記に該当しない時 &
 PATH 文省略時と同じ挙動になる。
\end{tabular}

\paragraph{パス名置換}
それぞれのパス要素が次のようなものであるとき、
次元値に応じた値に置換される。

\begin{tabular}{lp{35zw}}
{\tt \_basetime}
	& 基準時刻を12文字の数字列で表わしたもの。
	たとえば 200703312330 は 2007 年 3 月 31 日 12 時 30 分である。 \\
{\tt \_validtime} & 対象時刻を12文字の数字列で表わしたもの。 \\
{\tt \_member} & メンバ名。 \\
{\tt \_model} & モデル名 (種別1の前半4文字)。 \\
{\tt \_space} & 空間種別名 (種別1の後半4文字)。 \\
{\tt \_2d} & 2次元座標名 (空間種別名の前半2文字)。 \\
{\tt \_3d} & 3次元座標名 (空間種別名の後半2文字)。 \\
{\tt \_attribute} & 属性名 (種別2の前半2文字)。 \\
{\tt \_time} & 時間種別名 (種別2の後半2文字)。 \\
{\tt \_name} & 種別3。 \\
\end{tabular}

空白を含む名前が用いられたときは、ファイル名を生成するまえに
空白を下線 (`{\tt \_}') に置換する。

\paragraph{互換性}
{\tt PATH NWP\_ESF} 
は NAPS8, NAPS9 の数値予報ルーチンでは使えません
[NuSDaS 1.1 と NuSDaS 1.3 の r4374 (2014-12-11)
以降及び NuSDaS 1.4 でだけサポートされています。]

\section{PLANE: 面数の設定}
\label{sec:def:PLANE}
\paragraph{書式}
\begin{quote}
{\tt PLANE} $N_Z$
\end{quote}
\paragraph{位置制約}
\DefRef{ELEMENTMAP},
\DefRef{PLANE1} や
\DefRef{PLANE2} より先に書かねばならない。
\paragraph{機能}
PLANE 文は面の数を設定する。
\paragraph{必須性}
PLANE 文を省略してはならない。

\section{PLANE1: 面1の名前を設定}
\label{sec:def:PLANE1}
\paragraph{書式}
\begin{quote}
{\tt PLANE1} {\it name} ...
\end{quote}
\paragraph{位置制約}
\DefRef{PLANE} より後に書かねばならない。
\paragraph{機能}
PLANE1 文は $N_Z$ 個の面1の名前 {\it name} のリストを設定する。
名前は 6 文字以下の英数字または下線で、表 \ref{tab:plane} に従う。
ライブラリは名前を検査しないが、
その他の任意の名前は開発用途の一時的使用に限定される。
\paragraph{必須性}
PLANE1 文を省略してはならない。

\section{PLANE2: 面2の名前を設定}
\label{sec:def:PLANE2}
\paragraph{書式}
\begin{quote}
{\tt PLANE2} {\it name} ...
\end{quote}
\paragraph{機能}
PLANE2 文は $N_Z$ 個の面2の名前 {\it name} のリストを設定する。
その他の事項は PLANE1 文と共通である。
面を範囲で設定するデータセットでだけ用いられるため、
結局のところ実用例は皆無である。
\paragraph{省略時}
面2のリストには面1と同じ名前が登録される。

\section{SIZE: 格子数を設定}
\label{sec:def:SIZE}
\paragraph{書式}
\begin{quote}
{\tt SIZE} $N_X$ \ $N_Y$
\end{quote}
\paragraph{機能}
SIZE 文はデータレコード1つに含まれる格子の数
$N_X$, $N_Y$ を設定する。
\paragraph{必須性}
SIZE 文を省略してはならない。

\section{STANDARD: 地図投影法パラメタ設定}
\label{sec:def:STANDARD}
\paragraph{書式}
\begin{quote}
{\tt STANDARD}	\ $\lambda_1${\tt E} \ $\varphi_1${\tt N}
	\ $\lambda_1${\tt E} \ $\varphi_2${\tt N}
\end{quote}
\paragraph{機能}
STANDARD 文は地図投影法パラメタを設定する。
概ね $\varphi_1$ と $\varphi_2$ が標準緯度、
概ね $\lambda_1$ が標準経度と呼ばれるが、
その具体的な意味については付録 \ref{chap:proj} を参照されたい。
\paragraph{省略時}
{\tt STANDARD 0E 0N 0E 0N}
が仮定される。
それが適切か否かは2次元座標系しだいである。

\section{SUBCNTL: SUBC レコードの登録}
\label{sec:def:SUBCNTL}
\paragraph{書式}
\begin{quote}
{\tt SUBCNTL} $N_s$ [{\it name} {\it nbytes}] ...
\end{quote}
\paragraph{機能}
SUBCNTL 文はデータファイルが持つべき SUBC レコードについて設定する。
\DefRef{INFORMATION} と異なり、
1つの定義ファイルには SUBCNTL 文は1つだけ記述し、
そこに $N_s$ 個の SUBC レコードすべてを記述する。

名前 {\it name} は4文字以下の名前で、
使える名前は
{\tt ETA\SPC}, {\tt SIGM}, {\tt ZHYB}, {\tt RGAU},
{\tt RADR}, {\tt RADS}, {\tt DPRD}, {\tt ISPC}, {\tt DELT},
{\tt TDIF}, {\tt LOCA}
のいずれかである。

長さ {\it nbytes} は SUBC のペイロードつまり
\TabRef{table.fmt.subc} の項番 6 の長さであるが、
{\tt RADR}, {\tt RADS}, {\tt DPRD}, {\tt ISPC},
{\tt TDIF}, {\tt LOCA}
(いいかえると preset できないもの)
についてはメンバ数 $N_M$ (もしあれば) と対象時刻数 $N_V$ だけ繰り返されるので
本当のレコード長は次式で与えられる:
\[
	\hbox{(項番6のバイト数)} = N_M \times N_V \times {\it nbytes}.
\]

\paragraph{省略時}
登録されていない SUBC レコードは書き出せないので、
データファイルに
SUBC レコードを作ることはできなくなる。

\section{TYPE1: 種別1の設定}
\label{sec:def:TYPE1}
\paragraph{書式}
\begin{quote}
{\tt TYPE1} {\it model 2d 3d}
\end{quote}
\paragraph{機能}
TYPE1 文は種別1を
{\it model}, {\it 2d}, {\it 3d} を連結した文字列に
設定する。
引数 {\it model}, {\it 2d}, {\it 3d} は
それぞれモデル名 [\TabRef{tab:model}],
2次元座標名 [\TabRef{tab:2dname}],
3次元座標名 [\TabRef{tab:3dname}],
である。
\paragraph{必須性}
TYPE1 文を省略してはならない。
そのような定義ファイルが存在しても参照する方法がない。

\section{TYPE2: 種別2の設定}
\label{sec:def:TYPE2}
\paragraph{書式}
\begin{quote}
{\tt TYPE2} {\it attribute time}
\end{quote}
\paragraph{機能}
TYPE2 文は種別2を
{\it attribute} と {\it time} を連結した文字列に設定する。
引数 {\it attribute} と {\it time} は
それぞれ属性名 [\TabRef{tab:attribute}] と
時間種別名 [\TabRef{tab:time}]
である。
\paragraph{必須性}
TYPE2 文を省略してはならない。
そのような定義ファイルが存在しても参照する方法がない。

\section{TYPE3: 種別3の設定}
\label{sec:def:TYPE3}
\paragraph{書式}
\begin{quote}
{\tt TYPE3} {\it name}
\end{quote}
\paragraph{機能}
TYPE3 文は種別3を {\it name} に設定する。
種別3は4文字の英数字または下線である。
末尾には空白があってもよい、つまり3文字の種別3を設定することができるが、
混乱の元なので推奨しない。
\paragraph{必須性}
TYPE3 文を省略してはならない。
そのような定義ファイルが存在しても参照する方法がない。

\section{VALIDTIME: 対象時刻の数を設定}
\label{sec:def:VALIDTIME}
\paragraph{書式}
次のいずれか:
\begin{quote}
{\tt VALIDTIME} $N_V$ {\it tunits} {\tt IN}\\
{\tt VALIDTIME} $N_V$ {\it tunits} {\tt OUT}\\
{\tt VALIDTIME} $N_V$ {\tt IN} {\it tunits}\\
{\tt VALIDTIME} $N_V$ {\tt OUT} {\it tunits}\\
\end{quote}
\paragraph{位置制約}
\DefRef{ELEMENTMAP},
\DefRef{VALIDTIME1},
\DefRef{VALIDTIME2} (あれば)
より先に記述しなければならない。
\paragraph{機能}
VALIDTIME 文は対象時刻の数を設定する。
引数 {\tt IN} は異なる対象時刻のデータがひとつのデータファイルに
書かれることを意味し、
引数 {\tt OUT} は異なる対象時刻のデータは別のデータファイルに
書かれることを意味する。
時間の単位 {\it tunits} は予報時間の単位で、次のいずれかである;
{\tt MIN}: 分,
{\tt HOUR}: 時,
{\tt DAY}: 日,
{\tt WEEK}: 週,
{\tt PENT}: 暦日半旬,
{\tt JUN}: 旬,
{\tt MONT}: 月.
\paragraph{必須性}
VALIDTIME 文を省略してはならない。

\section{VALIDTIME1: 予報時間リストを設定}
\label{sec:def:VALIDTIME1}
\paragraph{書式}
次のいずれか:
\begin{quote}
{\tt VALIDTIME1 ARITHMETIC} {\it first} {\it step} \\
{\tt VALIDTIME1 ALL\_LIST} {\it ft} ...
\end{quote}
\paragraph{位置制約}
\DefRef{VALIDTIME} より後に記述しなければならない。
\paragraph{機能}
VALIDTIME1 文は予報時間のリストを設定する。
キーワード {\tt ARITHMETIC} が用いられた場合は
初項 {\it first}, 交差 {\it step} の等差数列となる。
キーワード {\tt ALL\_LIST} が用いられた場合は
$N_V$ 個の予報時間のリストを記述する。
予報時間は整数でなければならないが、負値でもよい。
\paragraph{必須性}
VALIDTIME1 文を省略してはならない。

\section{VALIDTIME2: 範囲の予報時間を設定}
\label{sec:def:VALIDTIME2}
\paragraph{書式}
次のいずれか
\begin{quote}
{\tt VALIDTIME2} {\tt -}{\it ft} \\
{\tt VALIDTIME2} {\it ft} ...
\end{quote}
\paragraph{機能}
VALIDTIME2 文は対象時刻を範囲で設定するデータセットのための機能で、
実例が皆無である。
最初の数値が負値 ${}-{}${\it ft} であれば、
$N_V$ 個の予報時間2 はすべて {\it ft} (符号反転値) である。
そうでなければ
あたかも VALIDTIME1 文の ALL\_LIST 設定時のように
$N_V$ 個の予報時間2 の一覧を設定する。
\paragraph{省略時}
{\tt VALIDTIME2 -1} が設定されたかのように
すべて $1$ が仮定される。

\section{VALUE: 格子の空間代表性を設定}
\label{sec:def:VALUE}
\paragraph{書式}
次のいずれか:
\begin{quote}
{\tt VALUE PVAL}\\
{\tt VALUE MEAN}\\
{\tt VALUE REPR}
\end{quote}
\paragraph{機能}
VALUE 文はデータが格子点近傍の場をどのように代表しているかを設定する。
許される値については\TabRef{tab:value}を参照。

\paragraph{省略時}
{\tt VALUE PVAL} が仮定される。
