\subsection{nusdas\_subc\_zhyb\_preset1: SUBC ZHYB のデフォルト値を設定}
\APILabel{nusdas.subc.zhyb.preset1}

\Prototype
\begin{quote}
N\_SI4 {\bf nusdas\_subc\_zhyb\_preset1}(const char {\it type1}[8], const char {\it type2}[4], const char {\it type3}[4], const N\_SI4 $\ast${\it nz}, const float $\ast${\it ptrf}, const float $\ast${\it presrf}, const float {\it zrp}[\,], const float {\it zrw}[\,], const float {\it vctrans\_p}[\,], const float {\it vctrans\_w}[\,], const float {\it dvtrans\_p}[\,], const float {\it dvtrans\_w}[\,]);
\end{quote}

\begin{tabular}{l|rp{20em}}
\hline
\ArgName & \ArgType & \ArgRole \\
\hline
{\it type1} & const char [8] &  種別1  \\
{\it type2} & const char [4] &  種別2  \\
{\it type3} & const char [4] &  種別3  \\
{\it nz} & const N\_SI4 $\ast$ &  鉛直層数  \\
{\it ptrf} & const float $\ast$ &  温位の参照値  \\
{\it presrf} & const float $\ast$ &  気圧の参照値  \\
{\it zrp} & const float [\,] &  モデル面高度 (フルレベル)  \\
{\it zrw} & const float [\,] &  モデル面高度 (ハーフレベル)  \\
{\it vctrans\_p} & const float [\,] &  座標変換関数 (フルレベル)  \\
{\it vctrans\_w} & const float [\,] &  座標変換関数 (ハーフレベル)  \\
{\it dvtrans\_p} & const float [\,] &  座標変換関数の鉛直微分 (フルレベル)  \\
{\it dvtrans\_w} & const float [\,] &  座標変換関数の鉛直微分 (ハーフレベル)  \\
\hline
\end{tabular}
\paragraph{\FuncDesc}ファイルが新たに生成される際にZHYBレコードに書き込む値を設定する。
ZHYB レコードや引数についてはnusdas\_subc\_zhyb を参照。
\paragraph{\ResultCode}
\begin{quote}
\begin{description}
\item[{\bf 0}] 正常終了
\item[{\bf -1}] 定義ファイルに "ZHYB" が登録されていない
\item[{\bf -2}] メモリの確保に失敗した
\end{description}\end{quote}
\paragraph{ 互換性 }
NuSDaS1.1 では、一つのNuSDaSデータセットに設定できる補助管理部の数は最大
10 に制限されており、それを超えると-2が返された。一方、 NuSDaS 1.3 からは
メモリが確保できる限り数に制限はなく、-2 をメモリ確保失敗のエラーコードに
読み替えている。
