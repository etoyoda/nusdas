\subsection{nusdas\_write: データ記録の書出}
\APILabel{nusdas.write}

\Prototype
\begin{quote}
N\_SI4 {\bf nusdas\_write}(const char {\it utype1}[8], const char {\it utype2}[4], const char {\it utype3}[4], const N\_SI4 $\ast${\it basetime}, const char {\it member}[4], const N\_SI4 $\ast${\it validtime}, const char {\it plane}[6], const char {\it element}[6], const void $\ast${\it data}, const char {\it fmt}[2], const N\_SI4 $\ast${\it nelems});
\end{quote}

\begin{tabular}{l|rp{20em}}
\hline
\ArgName & \ArgType & \ArgRole \\
\hline
{\it utype1} & const char [8] &  種別1  \\
{\it utype2} & const char [4] &  種別2  \\
{\it utype3} & const char [4] &  種別3  \\
{\it basetime} & const N\_SI4 $\ast$ &  基準時刻(通算分)  \\
{\it member} & const char [4] &  メンバー名  \\
{\it validtime} & const N\_SI4 $\ast$ &  対象時刻(通算分)  \\
{\it plane} & const char [6] &  面の名前  \\
{\it element} & const char [6] &  要素名  \\
{\it data} & const void $\ast$ &  データ配列  \\
{\it fmt} & const char [2] &  データ配列の型  \\
{\it nelems} & const N\_SI4 $\ast$ &  データ配列の要素数  \\
\hline
\end{tabular}
\paragraph{\FuncDesc}
データレコードを指定された場所に書き出す。

fmtは\ref{tab:typename}のものを指定するが、これらの値はN\_を接頭辞につけた定数で参照できる。
例えば'R4'であれば定数N\_R4で参照できる。

\paragraph{\ResultCode}
\begin{quote}
\begin{description}
\item[{\bf 正}] 実際に書き出された要素数
\item[{\bf -2}] メンバー名、面名、要素名が間違っている
\item[{\bf -2}] このレコードは ELEMENTMAP によって書き出しが禁止されている
\item[{\bf -3}] 与えられたデータ要素数 {\it nelems} が必要より小さい
\item[{\bf -4}] 指定データセットにはデータ配列の型 {\it fmt} は書き出せない
\item[{\bf -5}] データレコード長が固定レコード長を超える
\item[{\bf -6}] データセットの欠損値指定方式と RLEN 圧縮は併用できない
\item[{\bf -7}] マスクビットの設定がされていない
\item[{\bf -8}] エンコード過程でのエラー (数値が過大または RLEN 圧縮エラー)
\end{description}\end{quote}

\paragraph{注意}
\begin{itemize}
\item データセットの指定と異なる大きさのレコードを書き出すにはあらかじめ
\APILink{nusdas.parameter.change}{nusdas\_parameter\_change} を使って設定を変えておく。
\item 格子数 (データセットの指定または \APILink{nusdas.parameter.change}{nusdas\_parameter\_change} 設定)
より大きい要素数 {\it nelems} を指定するとエラーにはならず、
余った要素が書き出されない結果となるので注意されたい。
\end{itemize}

\paragraph{履歴}
この関数は NuSDaS 1.0 から存在した。
