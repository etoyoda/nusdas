\subsection{nusdas\_parameter\_change: オプション設定}
\APILabel{nusdas.parameter.change}

\Prototype
\begin{quote}
N\_SI4 {\bf nusdas\_parameter\_change}(N\_SI4 {\it param}, const void $\ast${\it value});
\end{quote}

\begin{tabular}{l|rp{20em}}
\hline
\ArgName & \ArgType & \ArgRole \\
\hline
{\it param} & N\_SI4 &  設定項目コード  \\
{\it value} & const void $\ast$ &  設定値  \\
\hline
\end{tabular}
\paragraph{\FuncDesc}
{\it param} で指定されるパラメータに値 {\it value} を設定する。
整数値の項目については、
互換性のため値ゼロのかわりに名前 N\_OFF を用いることができる。
\begin{quote}\begin{description}
\item[{\bf N\_PC\_MISSING\_UI1}] 1バイト整数の欠損値 (既定値: N\_MV\_UI1)
\item[{\bf N\_PC\_MISSING\_SI2}] 2バイト整数の欠損値 (既定値: N\_MV\_SI2)
\item[{\bf N\_PC\_MISSING\_SI4}] 4バイト整数の欠損値 (既定値: N\_MV\_SI4)
\item[{\bf N\_PC\_MISSING\_R4}] 4バイト実数の欠損値 (既定値: N\_MV\_R4)
\item[{\bf N\_PC\_MISSING\_R8}] 8バイト実数の欠損値 (既定値: N\_MV\_R8)
\item[{\bf N\_PC\_MASK\_BIT}] マスクビット配列へのポインタ
(既定値は NULL ポインタだが Fortran では直接設定できないので
\APILink{nusdas.parameter.reset}{nusdas\_parameter\_reset} を用いられたい)
\item[{\bf N\_PC\_SIZEX}] 
非零値を設定すると強制的にデータレコードの {\it x} 方向
格子数を設定する (0)
\item[{\bf N\_PC\_SIZEY}] 
強制格子サイズ:
既定値 (0) 以外を設定するとデータレコードの {\it y} 方向
格子数を設定する
\item[{\bf N\_PC\_PACKING}] 
4文字のパッキング名を設定すると、定義ファイルの指定にかかわらず
\APILink{nusdas.write}{nusdas\_write} 等データ記録書き込みの際に用いられる
パッキング方式が変更される。
既定値に戻す (定義ファイルどおりに書かせる) には
4 バイト整数値 0 を設定する。
\item[{\bf N\_PC\_ID\_SET}] 
NRD 番号制約:
既定値 (-1) 以外を設定すると、その番号の NRD だけを
入出力に用いるようになる
\item[{\bf N\_PC\_WBUFFER}] 
書き込みバッファサイズ (既定値: 0)
実行時オプション FWBF に同じ。
\item[{\bf N\_PC\_RBUFFER}] 
読み取りバッファサイズ (既定値: 17)
実行時オプション FRBF に同じ。
\item[{\bf N\_PC\_KEEP\_CFILE}] 
ファイルを閉じたあと CNTL/INDX などのヘッダ情報をキャッシュしておく
数。負にするとキャッシュを開放しなくなる(既定値: -1)。
実行時オプション GKCF に同じ。
\item[{\bf N\_PC\_OPTIONS}] 
設定のみでリセットはできない。
ヌル終端した文字列を与えると実行時オプションとして設定する。
Fortran インターフェイスでもヌル終端しなければならないことに注意。
\end{description}\end{quote}

\paragraph{\ResultCode}
\begin{quote}
\begin{description}
\item[{\bf 0}] 正常終了
\item[{\bf -1}] サポートされていないパラメタである
\end{description}\end{quote}

\paragraph{履歴}
NuSDaS 1.0 から存在する。

NuSDaS 1.1 ではデータセット探索のキャッシュ論理に問題があり、
N\_PC\_ID\_SET で NRD 番号制約をかけて入出力を行った後で
NRD 番号制約を解除して同じ種別にアクセスしても探索が行われない
(あらかじめ NRD 制約をかけずに入出力操作をしていれば探索される)。
この問題は NuSDaS 1.3 以降では解決されている。
