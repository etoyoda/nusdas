\subsection{nusdas\_inq\_subcinfo: SUBC/INFO の問合せ}
\APILabel{nusdas.inq.subcinfo}

\Prototype
\begin{quote}
N\_SI4 {\bf nusdas\_inq\_subcinfo}(const char {\it type1}[8], const char {\it type2}[4], const char {\it type3}[4], const N\_SI4 $\ast${\it basetime}, const char {\it member}[4], const N\_SI4 $\ast${\it validtime}, N\_SI4 {\it query}, const char {\it group}[4], void $\ast${\it buf}, const N\_SI4 {\it bufnelems});
\end{quote}

\begin{tabular}{l|rp{20em}}
\hline
\ArgName & \ArgType & \ArgRole \\
\hline
{\it type1} & const char [8] &  種別1  \\
{\it type2} & const char [4] &  種別2  \\
{\it type3} & const char [4] &  種別3  \\
{\it basetime} & const N\_SI4 $\ast$ &  基準時刻  \\
{\it member} & const char [4] &  メンバー  \\
{\it validtime} & const N\_SI4 $\ast$ &  対象時刻  \\
{\it query} & N\_SI4 &  問合せ項目  \\
{\it group} & const char [4] &  群名  \\
{\it buf} & void $\ast$ &  結果格納配列  \\
{\it bufnelems} & const N\_SI4 &  結果格納配列の要素数  \\
\hline
\end{tabular}
\paragraph{\FuncDesc}
引数 {\it type1} から {\it validtime} で指定されるデータファイルに書かれた
SUBC または INFO 記録について、
引数 {\it query} で指定される問合せを行う。
\begin{quote}\begin{description}
\item[{\bf N\_SUBC\_NUM}] 
SUBC 記録の個数が4バイト整数型変数 {\it buf} に書かれる。
引数 {\it group} は無視される。
\item[{\bf N\_SUBC\_LIST}] 
データファイルに定義された SUBC 記録の群名が配列 {\it buf} に書かれる。
配列 {\it buf} は長さ 4 文字の文字型で
{\it N\_SUBC\_NUM} 要素存在しなければならない。
引数 {\it group} は無視される。
\item[{\bf N\_SUBC\_NBYTES}] 
群名 {\it group} の SUBC 記録のバイト数が4バイト整数型変数 {\it buf} に書かれる。
\item[{\bf N\_SUBC\_CONTENT}] 
群名 {\it group} の SUBC 記録が配列 {\it buf} に書かれる。
上述のバイト数だけの長さを確保しておかねばならない。
\item[{\bf N\_INFO\_NUM}] 
INFO 記録の個数が4バイト整数型変数 {\it buf} に書かれる。
引数 {\it group} は無視される。
\item[{\bf N\_INFO\_LIST}] 
データファイルに定義された INFO 記録の群名が配列 {\it buf} に書かれる。
配列 {\it buf} は長さ 4 文字の文字型で
{\it N\_INFO\_NUM} 要素存在しなければならない。
引数 {\it group} は無視される。
\item[{\bf N\_INFO\_NBYTES}] 
群名 {\it group} の INFO 記録のバイト数が4バイト整数型変数 {\it buf} に書かれる。
\end{description}\end{quote}

\paragraph{\ResultCode}
\begin{quote}
\begin{description}
\item[{\bf 正}] 格納要素数
\end{description}\end{quote}
\paragraph{履歴}
この関数は NuSDaS 1.3 で新設された。
