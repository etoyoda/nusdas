\subsection{nusdas\_subc\_delt: SUBC DELT へのアクセス }
\APILabel{nusdas.subc.delt}

\Prototype
\begin{quote}
N\_SI4 {\bf nusdas\_subc\_delt}(const char {\it type1}[8], const char {\it type2}[4], const char {\it type3}[4], const N\_SI4 $\ast${\it basetime}, const char {\it member}[4], const N\_SI4 $\ast${\it validtime}, float $\ast${\it delt}, const char {\it getput}[3]);
\end{quote}

\begin{tabular}{l|rp{20em}}
\hline
\ArgName & \ArgType & \ArgRole \\
\hline
{\it type1} & const char [8] &  種別1  \\
{\it type2} & const char [4] &  種別2  \\
{\it type3} & const char [4] &  種別3  \\
{\it basetime} & const N\_SI4 $\ast$ &  基準時刻(通算分)  \\
{\it member} & const char [4] &  メンバー名  \\
{\it validtime} & const N\_SI4 $\ast$ &  対象時刻(通算分)  \\
{\it delt} & float $\ast$ &  DELT 数値へのポインタ  \\
{\it getput} & const char [3] &  入出力指示 ({\it "GET}" または {\it "PUT}")  \\
\hline
\end{tabular}
\paragraph{\FuncDesc}モデルの時間積分間隔を補助管理情報に記録しておくものである。
\paragraph{\ResultCode}
\begin{quote}
\begin{description}
\item[{\bf 0}] 正常終了
\item[{\bf -2}] レコードが存在しない、または書き込まれていない。
\item[{\bf -3}] レコードサイズが不正
\item[{\bf -5}] 入出力指示が不正
\end{description}\end{quote}
\paragraph{ 履歴 }
この関数は NuSDaS1.2で導入された。
