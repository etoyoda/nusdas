\subsection{nusdas\_uncpsd: 2UPPを2UPCに展開する}
\APILabel{nusdas.uncpsd}

\Prototype
\begin{quote}
N\_SI4 {\bf nusdas\_uncpsd}(const void $\ast${\it pdata}, void $\ast${\it cdata}, N\_SI4 {\it csize});
\end{quote}

\begin{tabular}{l|rp{20em}}
\hline
\ArgName & \ArgType & \ArgRole \\
\hline
{\it pdata} & const void $\ast$ &  入力する2UPPデータ  \\
{\it udata} & void $\ast$ &  展開先配列  \\
{\it usize} & N\_SI4 &  展開先配列のバイト数  \\
\hline
\end{tabular}
\paragraph{\FuncDesc}
\APILink{nusdas.inq.data}{nusdas\_inq\_data} の問い合わせ N\_DATA\_CONTENT で得られる 2UPP のバイト列を
展開して 2UPC のバイト列として得ます。結果として返るバイト列は 2UPC 形式のデータに対して
\APILink{nusdas.inq.data}{nusdas\_inq\_data} の問い合わせ N\_DATA\_CONTENT で得られるデータと同じ形式です。

\paragraph{\ResultCode}
\begin{quote}
\begin{description}
\item[{\bf 正}] 正常終了、値は展開後のバイト数
\item[{\bf -4}] 展開先配列の大きさ {\it csize} が必要バイト数より少ない
\item[{\bf -5}] 入力データが2UPPではない
\item[{\bf -6}] 2UPPから2UPCへの展開時にエラーが発生
\end{description}\end{quote}

\paragraph{履歴}
本関数は NuSDaS 1.4 で追加された。
