\subsection{NUSDAS\_CUT\_RAW: 領域限定の DATA 記録直接読取}
\APILabel{nusdas.cut.raw}

\Prototype
\begin{quote}
CALL {\bf NUSDAS\_CUT\_RAW}({\it type1}, {\it type2}, {\it type3}, {\it basetime}, {\it member}, {\it validtime}, {\it plane}, {\it element}, {\it udata}, {\it usize}, {\it ixstart}, {\it ixfinal}, {\it iystart}, {\it iyfinal}, {\it result})
\end{quote}

\begin{tabular}{l|rllp{16em}}
\hline
\ArgName & \ArgType & \ArrayDim & I/O & \ArgRole \\
\hline
{\it type1} & CHARACTER(8) &  & IN &  種別1  \\
{\it type2} & CHARACTER(4) &  & IN &  種別2  \\
{\it type3} & CHARACTER(4) &  & IN &  種別3  \\
{\it basetime} & INTEGER(4) &  & IN &  基準時刻(通算分)  \\
{\it member} & CHARACTER(4) &  & IN &  メンバー名  \\
{\it validtime} & INTEGER(4) &  & IN &  対象時刻(通算分)  \\
{\it plane} & CHARACTER(6) &  & IN &  面  \\
{\it element} & CHARACTER(6) &  & IN &  要素名  \\
{\it udata} & \AnyType & \AnySize & OUT &  データ格納先配列  \\
{\it usize} & INTEGER(4) &  & IN &  データ格納先配列のバイト数  \\
{\it ixstart} & INTEGER(4) &  & IN &  $x$ 方向格子番号下限  \\
{\it ixfinal} & INTEGER(4) &  & IN &  $x$ 方向格子番号上限  \\
{\it iystart} & INTEGER(4) &  & IN &  $y$ 方向格子番号下限  \\
{\it iyfinal} & INTEGER(4) &  & IN &  $y$ 方向格子番号上限  \\
{\it result} & INTEGER(4) &  & OUT & \ResultCode \\
\hline
\end{tabular}
\paragraph{\FuncDesc}
\APILink{nusdas.read2.raw}{nusdas\_read2\_raw} と類似だが、データレコードのうち格子点
({\it ixstart} , {\it iystart} )--({\it ixfinal} , {\it iyfinal} )
に対応する部分だけが {\it udata} に格納される。
ただしパッキングが 2UPP, 2UPJ, RLEN の場合、または欠損値が MASK の場合は全体が格納される。

\paragraph{\ResultCode}
\begin{quote}
\begin{description}
\item[{\bf 正}] 読み出して格納したバイト数
\item[{\bf 0}] 指定したデータは未記録(定義ファイルの elementmap によって書き込まれることは許容されているが、まだデータが書き込まれていない)
\item[{\bf -2}] 指定したデータは記録することが許容されていない(elementmap によって禁止されている場合と指定した面名、要素名が登録されていない場合の両方を含む)。
\item[{\bf -4}] 格納配列が不足
\end{description}\end{quote}

\paragraph{履歴}
この関数は NuSDaS 1.1 で導入された。
エラーコード -4 は NuSDaS 1.3 で新設されたもので、
それ以前はエラーチェックがなされていなかった。
