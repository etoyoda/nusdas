\subsection{srf\_amd\_slct: アメダスデータを指定の地点番号順に並べる}
\APILabel{srf.amd.slct}

\Prototype
\begin{quote}
int {\bf srf\_amd\_slct}(const N\_SI4 $\ast${\it st\_r}, int {\it n\_st\_r}, void $\ast${\it d\_r}, const char $\ast${\it t\_r}, N\_SI4 $\ast${\it st\_n}, int {\it n\_st\_n}, void $\ast${\it d\_n}, const char $\ast${\it t\_n}, int {\it sort\_f});
\end{quote}

\begin{tabular}{l|rp{20em}}
\hline
\ArgName & \ArgType & \ArgRole \\
\hline
{\it st\_r} & const N\_SI4 $\ast$ &  結果の順を指示する地点番号表  \\
{\it n\_st\_r} & int &  結果配列長  \\
{\it d\_r} & void $\ast$ &  結果配列  \\
{\it t\_r} & const char $\ast$ &  結果配列の型  \\
{\it st\_n} & N\_SI4 $\ast$ &  元データ地点番号配列  \\
{\it n\_st\_n} & int &  元データ配列長  \\
{\it d\_n} & void $\ast$ &  元データ配列  \\
{\it t\_n} & const char $\ast$ &  元データ配列の型  \\
{\it sort\_f} & int &  未ソートフラグ  \\
\hline
\end{tabular}
\paragraph{\FuncDesc}
長さ {\it n\_st\_n} の地点番号配列 {\it st\_n} と
対応する順に並んだ {\it t\_n} 型の配列 {\it d\_n} から、
別の地点番号配列 {\it st\_r} (要素数 {\it n\_st\_r} 個) の順に並んだ
{\it t\_r} 型の配列 {\it d\_r} (要素数 {\it n\_st\_r} 個) を作る。

配列 {\it st\_n} と {\it d\_n} があらかじめソートされている場合 {\it sort\_f} に
N\_OFF (nusdas.h で定義される) を指定する。
そうでない場合 {\it sort\_f} に N\_ON を指定するとソートされる。

\paragraph{\ResultCode}
\begin{quote}
\begin{description}
\item[{\bf 0}] 配列 {\it st\_r} の全地点が見付かった
\item[{\bf 正}] みつからなかった地点数
\end{description}\end{quote}

\begin{itemize}
\item 型は nusdas.h で定義される N\_R4, N\_I4, N\_I2 のいずれかで指定する。
\item 配列 {\it st\_r} に含まれる地点番号が {\it st\_n} で見付からない場合は
nusdas.h で定義される欠損値 N\_MV\_R4, N\_MV\_SI4, N\_MV\_SI2 が入る。
\end{itemize}
\paragraph{履歴}
この関数は NAPS7 時代から存在した。
