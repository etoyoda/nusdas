\subsection{bfwrite\_native: バイトオーダー変換付きファイル出力}
\APILabel{bfwrite.native}

\Prototype
\begin{quote}
unsigned long {\bf bfwrite\_native}(void $\ast${\it ptr}, unsigned long {\it size}, unsigned long {\it nmemb}, N\_BIGFILE $\ast${\it bf});
\end{quote}

\begin{tabular}{l|rp{20em}}
\hline
\ArgName & \ArgType & \ArgRole \\
\hline
{\it ptr} & void $\ast$ &  データ  \\
{\it size} & unsigned long &  オブジェクト長  \\
{\it nmemb} & unsigned long &  オブジェクト数  \\
{\it bf} & N\_BIGFILE $\ast$ &  ファイル  \\
\hline
\end{tabular}
\paragraph{\FuncDesc}
あらかじめ \APILink{bfopen}{bfopen} で開かれた
ファイル {\it bf} に幅 @size バイトのオブジェクトを {\it nmemb} 個書き込む。
書き込むデータは機械に自然なバイトオーダーで {\it ptr} から読み込まれ、
ファイルにはビッグエンディアンで書かれる。
\paragraph{\ResultCode}
\begin{quote}
\begin{description}
\item[{\bf 正}] 書き出されたオブジェクト数 (エラー時に nmemb より少ないことがある)
\item[{\bf 0}] エラー
\end{description}\end{quote}
\paragraph{参考}
\begin{itemize}
\item 引数 {\it size} にふさわしい値は sizeof 演算子によって得られる。
\end{itemize}
\paragraph{履歴}
この関数は NuSDaS 1.3 で追加された。
