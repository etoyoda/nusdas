\subsection{nusdas\_subc\_tdif: SUBC TDIF へのアクセス}
\APILabel{nusdas.subc.tdif}

\Prototype
\begin{quote}
N\_SI4 {\bf nusdas\_subc\_tdif}(const char {\it type1}[8], const char {\it type2}[4], const char {\it type3}[4], const N\_SI4 $\ast${\it basetime}, const char {\it member}[4], const N\_SI4 $\ast${\it validtime}, N\_SI4 $\ast${\it diff\_time}, N\_SI4 $\ast${\it total\_sec}, const char {\it getput}[3]);
\end{quote}

\begin{tabular}{l|rp{20em}}
\hline
\ArgName & \ArgType & \ArgRole \\
\hline
{\it type1} & const char [8] &  種別1  \\
{\it type2} & const char [4] &  種別2  \\
{\it type3} & const char [4] &  種別3  \\
{\it basetime} & const N\_SI4 $\ast$ &  基準時刻(通算分)  \\
{\it member} & const char [4] &  メンバー名  \\
{\it validtime} & const N\_SI4 $\ast$ &  対象時刻(通算分)  \\
{\it diff\_time} & N\_SI4 $\ast$ &  対象時刻からのずれ(秒)  \\
{\it total\_sec} & N\_SI4 $\ast$ &  総予報時間(秒)  \\
{\it getput} & const char [3] &  入出力指示 ({\it "GET}" または {\it "PUT}")  \\
\hline
\end{tabular}
\paragraph{\FuncDesc}格納した値の時刻の対象時間とのずれ、積算時間を格納する補助管理部 TDIF 
へのアクセスを提供する。
\paragraph{\ResultCode}
\begin{quote}
\begin{description}
\item[{\bf 0}] 正常終了
\item[{\bf -2}] 要求されたレコードが存在しない、または書き込まれていない。
\item[{\bf -3}] レコードサイズが不正
\item[{\bf -5}] 入出力指示が不正
\end{description}\end{quote}

\paragraph{補足}
\begin{itemize}
\item  diff\_time = 時間範囲始点 - 対象時刻 [秒単位]
\item  total\_sec = 時間範囲終点 - 時間範囲始点 [秒単位]
\end{itemize}

\paragraph{履歴}
この関数は NuSDaS1.0 から存在した。
