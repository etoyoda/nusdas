\subsection{NUSDAS\_CUT2: 領域限定のデータ読取 }
\APILabel{nusdas.cut2}

\Prototype
\begin{quote}
CALL {\bf NUSDAS\_CUT2}({\it type1}, {\it type2}, {\it type3}, {\it basetime}, {\it member}, {\it validtime1}, {\it validtime2}, {\it plane1}, {\it plane2}, {\it element}, {\it udata}, {\it utype}, {\it usize}, {\it ixstart}, {\it ixfinal}, {\it iystart}, {\it iyfinal}, {\it result})
\end{quote}

\begin{tabular}{l|rllp{16em}}
\hline
\ArgName & \ArgType & \ArrayDim & I/O & \ArgRole \\
\hline
{\it type1} & CHARACTER(8) &  & IN &  種別1  \\
{\it type2} & CHARACTER(4) &  & IN &  種別2  \\
{\it type3} & CHARACTER(4) &  & IN &  種別3  \\
{\it basetime} & INTEGER(4) &  & IN &  基準時刻(通算分)  \\
{\it member} & CHARACTER(4) &  & IN &  メンバー名  \\
{\it validtime1} & INTEGER(4) &  & IN &  対象時刻1  \\
{\it validtime2} & INTEGER(4) &  & IN &  対象時刻2  \\
{\it plane1} & CHARACTER(6) &  & IN &  面1  \\
{\it plane2} & CHARACTER(6) &  & IN &  面2  \\
{\it element} & CHARACTER(6) &  & IN &  要素名  \\
{\it udata} & \AnyType & \AnySize & OUT &  データ格納配列  \\
{\it utype} & CHARACTER(2) &  & IN &  データ格納配列の型  \\
{\it usize} & INTEGER(4) &  & IN &  データ格納配列の要素数  \\
{\it ixstart} & INTEGER(4) &  & IN &  $x$ 方向格子番号下限  \\
{\it ixfinal} & INTEGER(4) &  & IN &  $x$ 方向格子番号上限  \\
{\it iystart} & INTEGER(4) &  & IN &  $y$ 方向格子番号下限  \\
{\it iyfinal} & INTEGER(4) &  & IN &  $y$ 方向格子番号上限  \\
{\it result} & INTEGER(4) &  & OUT & \ResultCode \\
\hline
\end{tabular}
