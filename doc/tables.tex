\Chapter{数値・名前の表}

\section{共通エラーコードの表}

NuSDaS インターフェイスが返すエラーコードのうち、
$-10$ 以下の値は関数によらず次のような共通の意味をもつ。

\begin{description}
 \item[-10]
	    メモリが足りない。
 \item[-13]
 	    種別1の後半4文字をスペースにした場合の補間機能が働かなかった。
 \item[-18]
	    データファイルに書かれたバージョン番号指定が不正。
 \item[-21]
	    指定された種別に対応するNuSDaS Root Directory が見つからなかっ
	    た。
 \item[-33] 
	    定義ファイルの行の順序が不正である。
 \item[-34]
	    定義ファイルの elementmap の指定が不正である。
 \item[-35] 
	    定義ファイルの解読中に予期せぬEOFがあった。
 \item[-36]
	    定義ファイルの解読中にエラーが発生した(その他の定義ファイル
	    関連のエラーに属しないもの)。
 \item[-40] 
	    定義ファイルで TYPE1 が未定義。
 \item[-41] 
	    定義ファイルで TYPE2 が未定義。
 \item[-42] 
	    定義ファイルで TYPE3 が未定義。
 \item[-43] 
	    定義ファイルで対象時刻の数が未定義。
 \item[-44] 
	    定義ファイルで対象時刻のリストが未定義。
 \item[-45] 
	    定義ファイルで層の数が未定義。
 \item[-46] 
	    定義ファイルで層のリストが未定義。
 \item[-47] 
	    定義ファイルで物理量の数が未定義。
 \item[-48] 
	    定義ファイルで物理量のリストが未定義。
 \item[-49] 
	    定義ファイルで格子の数が未定義。
 \item[-50]
	    要求された対象時刻が、定義ファイルで指定した validtime に含
	    まれていない。または定義ファイルにmemberlistが未定義。
 \item[-51] 
	    type123, 基準時刻、メンバー、対象時刻から決まるNuSDaS データ
	    ファイルが存在しない(open に失敗した)。
 \item[-53]
	    データファイルの書き込み時、新規作成に失敗。
 \item[-54] 
	    既存データファイルの識別部(NUSD)が不正または読み込み
	    に失敗。
 \item[-55] 
	    既存データファイルの管理部(CNTL)が不正または読み込み
	    に失敗。
 \item[-56] 
	    既存データファイルのアドレス部(INDX)が不正または読み込み
	    に失敗。
 \item[-57] 
	    既存データファイルの終了部(END)が不正または読み込み
	    に失敗。
 \item[-58]
	    基準時刻に -1 を指定したときに、指定された対象時刻を持つデー
	    タが見つからなかった。
 \item[-59] 
	    既存データファイルのデータ部(DATA)が不正または読み込み
	    に失敗。
 \item[-63] 
	    レコード長がレコード強制長(定義ファイルにおいて forced rlen
	    で指定した値)を超えた。
 \item[-64]
	    定義ファイルに指定したINFOレコードの内容を記述したファイルの
	    読み込みに失敗した。
 \item[-65]
	    ファイルのclose処理において、識別部(NUSD)の書き込みに失敗し
	    た。
 \item[-66]
	    ファイルのclose処理において、管理部(CNTL)またはアドレス部
	    (INDX)の書き込みに失敗した。
 \item[-67]
	    ファイルのclose処理において、終了部(END)の書き込みに失敗した。
 \item[-68]
	    読み込み限定でopenしているファイルに書き込みをしようとした。
 \item[-69]
	    ID が 50 以上の NuSDaS Root Directoryに、書き込みを行おうと
	    した。
 \item[-80]
	    4GiB 以上の書き込みをサポートしていないNuSDaS1.1形式データに
	    4GiB を超える書き込みをしようとした。
 \item[-81]
 	    不正な投影法パラメタを持つデータファイルを作成しようとした、
	    または、nusdas\_grid で不正な投影法パラメタを設定しようとした。
 \item[-83]
	    ファイルのclose処理において、書き込まれたファイルサイズが
	    ライブラリが書き込んだサイズと異なっていた。
 \item[-98]
	    gz 圧縮に関する操作を行おうとしたが、Zlib がリンクされていな
	    いため、できない。
 \item[-99]
	    ファイルI/Oエラーが発生。pandora サーバーとの通信におけるエ
	    ラーも含む。
\end{description}

\noindent
{\bf 互換性}
\begin{description}
 \item[-18]
	   NuSDaS1.3 で新設。
 \item[-20] 
	   NuSDaS1.1 では「定義ファイルの同時オープン数がNuSDaSの最大値
	   を超えた」としているが、NuSDaS1.3 ではメモリが許す限りこの
	   上限がないので廃止する。ちなみに、NuSDaS1.1 にお
	   けるこの最大値は99。
 \item[-31]
	    NuSDaS1.1 のドキュメントには「定義ファイルのある項目の数の指
	    定がNuSDaS最大値を超えた」となっているが、
	    NuSDaS1.3 ではメモリが許す限り上限がないので廃止する。
	    ちなみに、NuSDaS 1.1 における上限値はメンバー数、層数、
	    要素数がそれぞれ256、対象時刻の数が36500であった。
 \item[-32]
	    NuSDaS1.1 では定義ファイルを読み込むときのメモリ不足に用いて
	    いたが、NuSDaS1.3 では $-10$ に統合して廃止する。
 \item[-35, -36]
	    NuSDaS1.3 で新設。
 \item[-52]
	   NuSDaS1.1 では「データファイルの同時オープン数がNuSDaS最大値
	   を超えた」としているが、NuSDaS1.3 ではメモリおよびファイルハ
	   ンドルの数が許される限りは上限がないので廃止する。
	   ちなみに、NuSDaS1.1におけるこの最大値は99。
 \item[-58] 
	    NuSDaS1.1 のドキュメントでは説明されていなかったが、同意味。
 \item[-59]
	    NuSDaS1.3 で新設。
 \item[-60, -61, -62]
	    NuSDaS1.1 ではファイルの初期化(新規作成)におけるエラーとして
	    これらのエラーが定義されていたが、
	    NuSDaS1.1 と NuSDaS1.3 ではファイルの初期作成の方法が異なる
	    ため、NuSDaS1.3 ではこのエラーに対応するエラーは発生しない
	    (発生する可能性があるのは$-53$である)。よってNuSDaS1.3 では廃
	    止する。
 \item[-63]
	    NuSDaS1.1 では初期化時に限定したエラーであったが、NuSDaS1.3
	    では初期化時に限らず、レコード長がレコード強制長を超えた場合
	    に用いる。
 \item[-66]
	    NuSDaS1.1 ではアドレス部(INDX)の書き込みに限定しているが、
	    NuSDaS1.1 ではファイルのclose処理においては、管理部(CNTL)は
	    書き込みをしてないためである。NuSDaS1.3 においては書き込み処
	    理をしているため、ここに管理部の書き込みエラーを含めることに
	    する。
 \item[-70〜-79]
	    NuSDaS1.3 では ES 利用をサポートしないので、廃止。
 \item[-80]
	    ドキュメントされていなかったが、NuSDaS1.1 で設定された。
 \item[-81, -98]
	    NuSDaS1.3 で新設された。
 \item[-83]
	    NuSDaS1.4 で新設された。
\end{description}
\newpage
\section{種別名}
\label{sec:nustype}

現在はオンライン登録制となっている。最新登録状況は
{\tt http://nusdas.npd.naps.kishou.go.jp/} 参照のこと(庁内限定)。

\begin{table}[htp]
\begin{center}
\begin{tabular}{l|l}
\hline
名前 & 意味 \\
\hline
{\tt \_AVN} & 米国全球モデル \\
{\tt \_CTM} & オゾン \\
{\tt \_CWM} & 波浪モデル \\
{\tt \_DCD} & 観測デコード \\
{\tt \_FDV} & 4次元変分同化用定数 \\
{\tt \_GSM} & 全球モデル \\
{\tt \_LLM} & 低解像度モデル \\
{\tt \_LRM} & 低解像度モデル \\
{\tt \_MSG} & エエロゾル (MASINGER) \\
{\tt \_MSM} & メソモデル \\
{\tt \_\_QA} & 毎時解析 \\
{\tt \_QMA} & 毎時解析 \\
{\tt \_RSM} & 領域モデル \\
{\tt \_SF1} & 1ヶ月予報 \\
{\tt \_SF4} & 4ヶ月予報 \\
{\tt \_SGM} & 高潮 \\
{\tt \_SRF} & 降水短時間予報 \\
{\tt \_SST} & SST解析 \\
{\tt \_SWM} & 浅海波浪モデル \\
{\tt \_TEM} & 台風アンサンブルモデル \\
{\tt \_TYM} & 台風モデル \\
{\tt \_VRF} & 検証 \\
{\tt \_WFM} & 週間予報 \\
{\tt \_XXX} & 不明 \\
\hline
\end{tabular}
\caption{種別に用いるモデル名}
\label{tab:model}
\end{center}
\end{table}

\begin{table}[htp]
\begin{center}
\begin{tabular}{ll|l}
\hline
2字略号 & 3字略号 & 意味 \\
\hline
{\tt FG} & {\tt FG\SPC\SPC} & 自由格子 \\
{\tt GS} & {\tt GS\SPC\SPC} & 矩形ガウス格子( SUBC RGAUを設定した場合
を除く ) \\
{\tt LL} & {\tt LL\SPC\SPC} & 経緯度座標 \\
{\tt LM} & {\tt LMN\SPC} & ランベルト正角円錐図法 \\
{\tt LM} & {\tt LMS\SPC} & 同 (南半球) \\
{\tt MR} & {\tt MER\SPC} & メルカトル図法 \\
{\tt OL} & {\tt OL\SPC\SPC} & 斜軸ランベルト \\
{\tt PS} & {\tt PSN\SPC} & ポーラーステレオ図法 \\
{\tt PS} & {\tt PSS\SPC} & 同 (南半球) \\
{\tt RD} & {\tt RD\SPC\SPC} & レーダー単サイト \\
{\tt RG} & {\tt RG\SPC\SPC} & 適合ガウス格子( SUBC RGAUを設定した矩形
ガウス格子を含む ) \\
{\tt RT} & {\tt RT\SPC\SPC} & 極座標レーダー \\
{\tt SB} & {\tt SB\SPC\SPC} & 細分 \\
{\tt ST} & {\tt ST\SPC\SPC} & 地点観測 \\
{\tt XX} & {\tt XX\SPC\SPC} & 不明 \\
{\tt YP} & {\tt YP\SPC\SPC} & 子午面断面 \\
\hline
\end{tabular}
\caption{2次元座標の名称。種別1には2字略号を用い、
	その他の場合、つまり CNTL レコードや NUSDAS\_GRID で
	受け渡す値には3字略号を用いる。}
\label{tab:2dname}
\end{center}
\end{table}

\begin{table}[htp]
\begin{center}
\begin{tabular}{l|l}
\hline
名前 & 意味 \\
\hline
{\tt ET} & モデル面 ($\eta$ 座標) \\
{\tt FL} & フライトレベル ({\tt F010} ... {\tt F450}) \\
{\tt LO} & 経度 (用例は {\tt ZONAL} のみ) \\
{\tt LY} & 高度面、エコートップ、大気全層など \\
{\tt PP} & 気圧座標 \\
{\tt SF} & 地表面 \\
{\tt SG} & モデル面 ($\sigma$ 座標) \\
{\tt TO} & 閾値 \\
{\tt ZS} & モデル面 ($z^*$ 座標) \\
{\tt XX} & 不明 \\
\hline
\end{tabular}
\caption{種別に用いる3次元座標名}
\label{tab:3dname}
\end{center}
\end{table}

種別1の先頭4文字はモデル名と呼ばれ、
表 \ref{tab:model} に従って
データを作成した処理の名称をつける。
種別1の続く2文字は2次元座標名と呼ばれ、
表 \ref{tab:2dname} に従って
データ記録の2次元配列がとられた座標系の名称をつける。
種別1の続く2文字は3次元座標名と呼ばれ、
表 \ref{tab:3dname} に従って
面名がつけられた座標系の名称をつける。

\begin{table}[htp]
\begin{center}
\begin{tabular}{l|l}
\hline
名前 & 意味 \\
\hline
{\tt AA} & 同化解析 \\
{\tt AF} & 解析 (変分同化等) の中の予報 \\
{\tt EF}* & アンサンブル予報 \\
{\tt CC} & 定数 \\
{\tt EA} & 速報解析 \\
{\tt FC} & 予報値 \\
{\tt GS} & 解析の推定値 (guess) \\
{\tt IN} & 初期値 \\
{\tt OB} & 観測値 \\
{\tt PT} & アンサンブルの摂動 \\
{\tt RA}* & 再解析 \\
{\tt VR} & 検証値 \\
{\tt XX} & 不明 \\
\hline
\end{tabular}
\caption{種別に用いる属性名.}
\label{tab:attribute}
\end{center}
\end{table}

\begin{table}[htp]
\begin{center}
\begin{tabular}{l|l}
\hline
名前 & 意味 \\
\hline
{\tt AN} & 積算値の気候値 \\
{\tt AV} & 積算値 \\
{\tt DN} & 標準偏差の気候値 \\
{\tt DV} & 標準偏差 \\
{\tt MA} & 期間平均の気候値からの偏差 \\
{\tt MN} & 期間平均の気候値 \\
{\tt MV} & (気候に比べ短期間の) 平均値 \\
{\tt PV} & 確率値 \\
{\tt SN} & 瞬間値の気候値 \\
{\tt SV} & 瞬間値 \\
{\tt XX} & 不明 \\
\hline
\end{tabular}
\caption{種別に用いる時間種類名}
\label{tab:time}
\end{center}
\end{table}

種別2の最初の2文字は属性名と呼ばれ、
表 \ref{tab:attribute} に従って
予報値、解析値、観測値などの区別を表わす。
種別2の末尾2文字は時間種類名と呼ばれ、
表 \ref{tab:time} に従って
瞬間値、時間平均値などの区別を表わす。
ただし実運用においては時間平均値や積算値を瞬間値とまとめて
{\tt SV} のデータセットに格納することがある。

種別3は任意に付けられるデータセット名である。
名前 {\tt STD1} はデータ作成処理にとって
最も標準的な出力 (標準ファイル) に付けられる名前である。
\newpage
\section{面名の表}

\begin{table}[htp]
\begin{center}
\begin{tabular}{l|l}
\hline
名前 & 意味 \\
\hline
{\tt ATMTOP} & 大気上端 \\
{\tt CBTOP} & 積乱雲の雲頂 \\
{\tt COLUMN} & 大気全層 \\
{\tt ECTOP} & レーダーのエコー頂 \\
{\tt MXWIND} & 風速最大の高度 \\
{\tt SURF} & 地表面 (ただし転用多し) \\
{\tt TOTAL} & 土壌全層 \\
{\tt TROPO} & 対流圏界面 \\
{\tt ZONAL} & 大気全体 ({\tt YP} 座標系での東西平均) \\
\hline
\end{tabular}
\caption{面名}
\label{tab:plane}
\end{center}
\end{table}

面名はこの表に示したものの他に、種別1の3次元座標名(表 \ref{tab:3dname})に応じて
気圧座標(PP)であれば 850, 500 といった気圧の値、
フライトレベル(FL)であれば F450 といった値を利用可能。

\section{その他の表}

\begin{table}[htp]
\begin{center}
\begin{tabular}{l|l}
\hline
名前 & 意味 \\
\hline
{\tt I1} & 1バイト符号なし整数型 \\
{\tt I2} & 2バイト符号付き整数型 \\
{\tt I4} & 4バイト符号付き整数型 \\
{\tt R4} & 4バイト単精度浮動小数点型 \\
{\tt R8} & 8バイト倍精度浮動小数点型 \\
{\tt NC} & パッキングをしたままのバイト列 \\
{\tt ND} & 生のバイナリデータ(1.4-1以降) \\ 
\hline
\end{tabular}
\caption{ユーザ配列の型の名称}
\label{tab:typename}
\end{center}
\end{table}

NC は パッキング方式 \ref{tab:packing} が 2UPC,2UPP,2UPP で base, amp を展開しないデータを扱う。
2UPP,2UPJの場合はそれぞれ複合差分圧縮,JPEG2000圧縮が展開された状態のデータとなる。
また、エンディアンも変換された状態のデータを扱う。
扱うデータ範囲は \ref{table.fmt.data} の項番13のデータ。
サイズ指定ではbaseやampを考慮する必要はなく、純粋に格子数$(x*y)$で指定する。
なお、欠損値表現方式 \ref{tab:missing} が MASK のデータでは利用できない。

ND はファイルに書き込んだものと同じ生のバイナリデータを扱う。エンディアン変換も行われない。
任意のパッキング方式で利用できるが、\APIRef{nusdas.write}{nusdas\_write} 以外のサブルーチンでは利用できない。
扱うデータ範囲は \ref{table.fmt.data} の項番10から13のデータで、
\APIRef{nusdas.inq.data}{nusdas\_inq\_data} に N\_DATA\_CONTENT を指定して取得したデータをそのまま利用できる。
サイズは格子数ではなくバイナリデータのサイズを指定する必要があり、
\APIRef{nusdas.inq.data}{nusdas\_inq\_data} に N\_DATA\_NBYTES を指定して取得した値を利用すべきである。

\begin{table}[htp]
\begin{center}
\begin{tabular}{l|l}
\hline
名前 & 意味 \\
\hline
{\tt 1PAC} & 符号なし1バイト整数によるバイトパック \\
{\tt 2PAC} & 符号付き2バイト整数によるバイトパック \\
{\tt 2UPC} & 符号なし2バイト整数によるバイトパック\\
{\tt 2UPJ} & 符号なし2バイト整数によるバイトパック、 JPEG2000圧縮\\
{\tt 2UPP} & 符号なし2バイト整数によるバイトパック、 複合差分圧縮\\
{\tt 4PAC} & 符号つき4バイト整数によるバイトパック \\
{\tt N1I2} & 10倍値の2バイト符号付き整数による格納 \\
{\tt RLEN} & 8ビットランレングス圧縮 \\
{\tt I1  } & 1バイト符号なし整数による格納 \\
{\tt I2  } & 2バイト符号つき整数による格納 \\
{\tt I4  } & 4バイト符号つき整数による格納 \\
{\tt R4  } & 単精度浮動小数点型による格納 \\
{\tt R8  } & 倍精度浮動小数点型による格納 \\
\hline
\end{tabular}
\caption{パッキング方式名称。
それぞれの形式については\ChapRef{chap:packing}参照}
\label{tab:packing}
\end{center}
\end{table}

\begin{table}[htp]
\begin{center}
\begin{tabular}{l|l}
\hline
名前 & 意味 \\
\hline
{\tt NONE} & 欠損値なし。あらゆる数値が意味を持つデータである(デフォルト) \\
{\tt UDFV} & データレコード毎に定めるある値が欠損値である。 \\
 & 欠損値は\APIRef{nusdas.parameter.change}{nusdas\_parameter\_change}で設定する \\
{\tt MASK} & マスクビットによってデータの存在格子を指定する。 \\
 & \APIRef{nusdas.make.mask}{nusdas\_make\_mask}でマスクビットを生成し、\\
 & \APIRef{nusdas.parameter.change}{nusdas\_parameter\_change} または \\
 & \APIRef{nusdas.set.mask}{nusdas\_set\_mask} で欠損格子を設定する。\\
\hline
\end{tabular}
\caption{欠損値表現方式}
\label{tab:missing}
\end{center}
\end{table}

\begin{table}[htp]
\begin{center}
\begin{tabular}{l|l}
\hline
名前 & 意味 \\
\hline
{\tt PVAL} & データは格子点における値である(デフォルト) \\
{\tt MEAN} & データは格子点近傍の場の平均値である\\
{\tt REPR} & データは格子点近傍の場を何らかの意味で代表する値である\\
\hline
\end{tabular}
\caption{格子点の空間代表性}
\label{tab:value}
\end{center}
\end{table}

\clearpage
\section{要素名の表}

SUBC TDIF レコードとともに期間内の平均、積算、最大、最小値
を格納する場合、要素名の先頭に、
期間平均値の場合「 \_ 」、期間積算値の場合「 . 」、
期間最大値の場合「 A\_ 」、期間最小値の場合「 I\_ 」を付加する。

現在はオンライン登録制となっている。最新登録状況は
{\tt http://nusdas.npd.naps.kishou.go.jp/} 参照のこと(庁内限定)。

\begin{longtable}{l|rrrp{20zw}}
\hline
要素名 & GRIB1番号 & 旧形式 & 単位 & 物理量 \\
\hline
{\tt Pres} & 1 &  & ${\rm\,Pa}$ & 気圧 \\
{\tt P} & 1 & GVD & ${\rm\,hPa}$ & 気圧 \\
{\tt PAI} & 1 & GVD & ${\rm\,hPa}$ & 気圧 \\
{\tt Pmsl} & 2 &  & ${\rm\,Pa}$ & 海面更正気圧 \\
{\tt PSEA} & 2 & GVD & ${\rm\,hPa}$ & 海面更正気圧 \\
{\tt SLP} & 2 & GVD & ${\rm\,hPa}$ & 海面更正気圧 \\
{\tt Ptend} & 3 &  & ${\rm\,Pa/s}$ & 気圧変化の傾向 \\
{\tt pVOR} & 4 &  & ${\rm\,K\,\cdot\,m^2/\,(kg\,\cdot\,s)}$ & ポテンシャル渦度 \\
{\tt sarH} & 5 &  & ${\rm\,m}$ & ICAO標準大気参照高度 \\
{\tt G} & 6 &  & ${\rm\,m^2/s^2}$ & ジオポテンシャル \\
{\tt PHI} & 6 & GVD & ${\rm\,m^2/s^2}$ & ジオポテンシャル \\
{\tt gpH} & 7 &  & ${\rm\,gpm}$ & ジオポテンシャル高度 \\
{\tt Z} & 7 & GVD & ${\rm\,m}$ & ジオポテンシャル高度 \\
{\tt gmH} & 8 &  & ${\rm\,m}$ & 幾何学的高度 \\
{\tt sdH} & 9 &  & ${\rm\,m}$ & 高度の標準偏差 \\
{\tt tOZON} & 10 &  & ${\rm\,Dobson}$ & オゾン全量 \\
{\tt T} & 11 & GVD & ${\rm\,K}$ & 気温 \\
{\tt vT} & 12 &  & ${\rm\,K}$ & 仮温度 \\
{\tt pT} & 13 &  & ${\rm\,K}$ & 温位 \\
{\tt papT} & 14 &  & ${\rm\,K}$ & 偽断熱温位 \\
{\tt maxT} & 15 &  & ${\rm\,K}$ & 最高気温 \\
{\tt minT} & 16 &  & ${\rm\,K}$ & 最低気温 \\
{\tt dT} & 17 &  & ${\rm\,K}$ & 露点温度 \\
{\tt TTD} & 18 & GVD & ${\rm\,K}$ & 湿数 \\
{\tt TRate} & 19 &  & ${\rm\,K/m}$ & 気温減率 \\
{\tt VIS} & 20 &  & ${\rm\,m}$ & 視程 \\
{\tt Radr1} & 21 &  &  & レーダースペクトル(a) \\
{\tt Radr2} & 22 &  &  & レーダースペクトル(b) \\
{\tt Radr3} & 23 &  &  & レーダースペクトル(c) \\
{\tt PLI50} & 24 &  & ${\rm\,K}$ & 500hPa面への気塊持ち上げ指数 \\
{\tt Tano} & 25 &  & ${\rm\,K}$ & 気温偏差 \\
{\tt Pano} & 26 &  & ${\rm\,hPa}$ & 気圧偏差 \\
{\tt gpHan} & 27 &  & ${\rm\,gpm}$ & ジオポテンシャル高度偏差 \\
{\tt Zan} & 27 &  & ${\rm\,m}$ & ジオポテンシャル高度偏差 \\
{\tt Wave1} & 28 &  &  & 波のスペクトル(a) \\
{\tt Wave2} & 29 &  &  & 波のスペクトル(b) \\
{\tt Wave3} & 30 &  &  & 波のスペクトル(c) \\
{\tt WindD} & 31 &  & ${\rm\,degree\,true}$ & 風向 \\
{\tt WindS} & 32 &  & ${\rm\,m/s}$ & 風速 \\
{\tt U} & 33 & GVD & ${\rm\,m/s}$ & 風のx軸成分 \\
{\tt WindX} & 33 &  & ${\rm\,m/s}$ & 風のx軸成分 \\
{\tt UU} & 33 & GVD & ${\rm\,m/s}$ & 風の東西成分 \\
{\tt V} & 34 & GVD & ${\rm\,m/s}$ & 風のy軸成分 \\
{\tt WindY} & 34 &  & ${\rm\,m/s}$ & 風のy軸成分 \\
{\tt VV} & 34 & GVD & ${\rm\,m/s}$ & 風の南北成分 \\
{\tt PSI} & 35 &  & ${\rm\,m^2/s}$ & 流線関数 \\
{\tt CHI} & 36 & GVD & ${\rm\,m^2/s}$ & 速度ポテンシャル \\
{\tt mPSI} & 37 &  & ${\rm\,m^2/s^2}$ & モンゴメリーの流線関数 \\
{\tt sVV} & 38 &  & ${\rm\,1/s}$ & シグマ座標における鉛直速度 \\
{\tt VVPa} & 39 &  & ${\rm\,Pa/s}$ & 鉛直速度(気圧座標) \\
{\tt OMG} & 39 & GVD & ${\rm\,hPa/hour}$ & 鉛直速度(気圧座標) \\
{\tt VVm} & 40 &  & ${\rm\,m/s}$ & 鉛直速度(m単位) \\
{\tt W} & 40 &  & ${\rm\,m/s}$ & 鉛直速度 \\
{\tt aVOR} & 41 &  & ${\rm\,1/s}$ & 絶対渦度 \\
{\tt aDIV} & 42 &  & ${\rm\,1/s}$ & 絶対発散 \\
{\tt rVOR} & 43 &  & ${\rm\,1/s}$ & 相対渦度 \\
{\tt VOR} & 43 & GVD & ${\rm\,10^{-6}/s}$ & 相対渦度 \\
{\tt rDIV} & 44 &  & ${\rm\,1/s}$ & 相対発散 \\
{\tt DIV} & 44 & GVD & ${\rm\,10^{-6}/s}$ & 相対発散 \\
{\tt vUS} & 45 &  & ${\rm\,1/s}$ & 鉛直シヤーのx成分 \\
{\tt vVS} & 46 &  & ${\rm\,1/s}$ & 鉛直シヤーのy成分 \\
{\tt VWS} & 未定義 &  & ${\rm\,kt/1000ft}$ & 鉛直速度シアー \\
{\tt CrntD} & 47 &  & ${\rm\,degree\,true}$ & 流れの方向 \\
{\tt CrntS} & 48 &  & ${\rm\,m/s}$ & 流れの速さ \\
{\tt CrntU} & 49 &  & ${\rm\,m/s}$ & 流れのx成分 \\
{\tt CrntV} & 50 &  & ${\rm\,m/s}$ & 流れのy成分 \\
{\tt Q} & 51 &  & ${\rm\,kg/kg}$ & 比湿 \\
{\tt RH} & 52 &  & ${\rm\,Percent}$ & 相対湿度 \\
{\tt HMR} & 53 &  & ${\rm\,kg/kg}$ & 混合比 \\
{\tt TPW} & 54 &  & ${\rm\,kg/m^2}$ & 可降水量 \\
{\tt VP} & 55 &  & ${\rm\,Pa}$ & 蒸気圧 \\
{\tt VPVPD} & 56 &  & ${\rm\,Pa}$ & 飽差 \\
{\tt Evap} & 57 &  & ${\rm\,kg/m^2}$ & 蒸発量 \\
{\tt CIC} & 58 &  & ${\rm\,kg/m^2}$ & 雲氷 \\
{\tt RRate} & 59 &  & ${\rm\,kg/(m^2\,\cdot\,s)}$ & 降水率 \\
{\tt ThndP} & 60 &  & ${\rm\,Percent}$ & 雷電の発生確率 \\
{\tt RAIN} & 61 & GVD & ${\rm\,kg/m^2}$ & 総降水量 \\
{\tt RR10} & 61 &  & ${\rm\,mm}$ & 前10分間降水量 \\
{\tt RR60} & 61 &  & ${\rm\,mm}$ & 前60分間降水量 \\
{\tt RR1H} & 61 &  & ${\rm\,mm}$ & 前1時間降水量 \\
{\tt RR3H} & 61 &  & ${\rm\,mm}$ & 前3時間降水量 \\
{\tt RR6H} & 61 &  & ${\rm\,mm}$ & 前6時間降水量 \\
{\tt RR1D} & 61 &  & ${\rm\,mm}$ & 前1日間降水量 \\
{\tt RR1M} & 61 &  & ${\rm\,mm}$ & 前1月間降水量 \\
{\tt RRfr0} & 61 &  & ${\rm\,mm}$ & 正時からの降水量 \\
{\tt RRL} & 62 & GVD & ${\rm\,kg/m^2}$ & 層状性降水量 \\
{\tt RRLpD} & 62 &  & ${\rm\,mm}$ & 層状性降水量(1日当たり) \\
{\tt RRC} & 63 & GVD & ${\rm\,kg/m^2}$ & 対流性降水量 \\
{\tt RRCpD} & 63 &  & ${\rm\,mm}$ & 対流性降水量(1日当たり) \\
{\tt SnRWe} & 64 &  & ${\rm\,kg/(m^2\,\cdot\,s)}$ & 降雪率の水当量 \\
{\tt SnWe} & 65 &  & ${\rm\,m}$ & 積雪の水当量 \\
{\tt SnowD} & 66 &  & ${\rm\,m}$ & 積雪の深さ \\
{\tt MLD} & 67 &  & ${\rm\,m}$ & 混合層の厚さ \\
{\tt tTcD} & 68 &  & ${\rm\,m}$ & 非定常水温躍層の深さ \\
{\tt mTcD} & 69 &  & ${\rm\,m}$ & 主水温躍層の深さ \\
{\tt mTcan} & 70 &  & ${\rm\,m}$ & 主水温躍層の偏差 \\
{\tt CLA} & 71 & GVD &  & 全雲量 \\
{\tt CLC} & 72 &  &  & 対流雲の雲量 \\
{\tt CLL} & 73 & GVD &  & 下層雲量 \\
{\tt CLM} & 74 & GVD &  & 中層雲量 \\
{\tt CLH} & 75 & GVD &  & 上層雲量 \\
{\tt CWC} &  &  & ${\rm\,kg/kg}$ & 雲水量(氷相を含む) \\
{\tt TCWC} & 76 &  & ${\rm\,kg/m^2}$ & 雲水量 \\
{\tt BLI50} & 77 &  & ${\rm\,K}$ & 500hPa面への最適持ち上げ指数 \\
{\tt SnC} & 78 &  & ${\rm\,kg/m^2}$ & 対流性の降雪量 \\
{\tt SnL} & 79 &  & ${\rm\,kg/m^2}$ & ラージスケールの降雪量 \\
{\tt WatrT} & 80 &  & ${\rm\,K}$ & 水温 \\
{\tt SST} & 80 & GVD & ${\rm\,K}$ & 水温 \\
{\tt Land} & 81 &  & ${\rm\,Proportion}$ & 陸域 \\
{\tt Sldev} & 82 &  & ${\rm\,m}$ & 海面水位の平均値からの偏差 \\
{\tt Z0} & 83 & GVD & ${\rm\,m}$ & 地表面粗度 \\
{\tt Albed} & 84 &  &  & アルベド \\
{\tt SoilT} & 85 &  & ${\rm\,K}$ & 土壌温度 \\
{\tt SoilW} & 86 &  &  & 積算土壌水分量 \\
{\tt Veget} & 87 &  &  & 植生被覆率 \\
{\tt Sali} & 88 &  & ${\rm\,kg/kg}$ & 塩分 \\
{\tt Dens} & 89 &  & ${\rm\,kg/m^3}$ & 密度 \\
{\tt RunOf} & 90 &  & ${\rm\,kg/m^2}$ & 流出量 \\
{\tt ROF} & 90 & GVD & ${\rm\,mm/day}$ & 流出量 \\
{\tt ROFS} & 90 & GVD & ${\rm\,mm/day}$ & 表面排水による流出量 \\
{\tt ROFD} & 90 & GVD & ${\rm\,mm/day}$ & 重力排水による流出量 \\
{\tt IceC} & 91 &  & ${\rm\,Proportion}$ & 氷域 \\
{\tt ICE} & 91 & GVD & ${\rm\,Proportion}$ & 氷域 \\
{\tt IceD} & 92 &  & ${\rm\,m}$ & 氷の厚さ \\
{\tt IceMD} & 93 &  & ${\rm\,degree\,true}$ & 氷の移動方向 \\
{\tt IceMS} & 94 &  & ${\rm\,m/s}$ & 氷の移動速度 \\
{\tt IceMU} & 95 &  & ${\rm\,m/s}$ & 氷の移動速度のx成分 \\
{\tt IceMV} & 96 &  & ${\rm\,m/s}$ & 氷の移動速度のy成分 \\
{\tt IceGR} & 97 &  & ${\rm\,m/s}$ & 氷の成長率 \\
{\tt IceDV} & 98 &  & ${\rm\,1/s}$ & 氷の発散 \\
{\tt SNMlt} & 99 &  & ${\rm\,kg/m^2}$ & 融雪量 \\
{\tt CWSSH} & 100 &  & ${\rm\,m}$ & 風浪とうねりの合成有義波高 \\
{\tt WWvD} & 101 &  & ${\rm\,degree\,true}$ & 風浪の向き \\
{\tt WWvSH} & 102 &  & ${\rm\,m}$ & 風浪の有義波高 \\
{\tt WWvMP} & 103 &  & ${\rm\,s}$ & 風浪の平均周期 \\
{\tt SWvD} & 104 &  & ${\rm\,degree\,true}$ & うねりの向き \\
{\tt SWvSH} & 105 &  & ${\rm\,m}$ & うねりの有義波高 \\
{\tt SWvMP} & 106 &  & ${\rm\,s}$ & うねりの平均周期 \\
{\tt PWvD} & 107 &  & ${\rm\,degree\,true}$ & 第1波の方向 \\
{\tt PWvMP} & 108 &  & ${\rm\,s}$ & 第1波の平均周期 \\
{\tt 2WvD} & 109 &  & ${\rm\,degree\,true}$ & 第2波の方向 \\
{\tt 2WvMP} & 110 &  & ${\rm\,s}$ & 第2波の平均周期 \\
{\tt RSNB} & 111 & GVD & ${\rm\,W/m^2}$ & 正味短波放射フラックス(地表面) \\
{\tt RLNB} & 112 & GVD & ${\rm\,W/m^2}$ & 正味長波放射フラックス(地表面) \\
{\tt RSNT} & 113 & GVD & ${\rm\,W/m^2}$ & 正味短波放射フラックス(大気上端) \\
{\tt RLNT} & 114 & GVD & ${\rm\,W/m^2}$ & 正味長波放射フラックス(大気上端) \\
{\tt RL} & 115 &  & ${\rm\,W/m^2}$ & 長波放射フラックス \\
{\tt RLUB} & 115 & GVD & ${\rm\,W/m^2}$ & 長波放射フラックス(上向き、地表面) \\
{\tt RLUBc} & 115 &  & ${\rm\,W/m^2}$ & 長波放射フラックス(上向き、地表面、晴天) \\
{\tt RLDB} & 115 & GVD & ${\rm\,W/m^2}$ & 長波放射フラックス(下向き、地表面) \\
{\tt RLUT} & 115 & GVD & ${\rm\,W/m^2}$ & 長波放射フラックス(上向き、大気上端) \\
{\tt RLDBc} & 115 & GVD & ${\rm\,W/m^2}$ & 長波放射フラックス(下向き、地表面、晴天) \\
{\tt RLUTc} & 115 & GVD & ${\rm\,W/m^2}$ & 長波放射フラックス(上向き、大気上端、晴天) \\
{\tt RS} & 116 &  & ${\rm\,W/m^2}$ & 短波放射フラックス \\
{\tt RSUB} & 116 & GVD & ${\rm\,W/m^2}$ & 短波放射フラックス(上向き、地表面) \\
{\tt RSDB} & 116 & GVD & ${\rm\,W/m^2}$ & 短波放射フラックス(下向き、地表面) \\
{\tt RSUT} & 116 & GVD & ${\rm\,W/m^2}$ & 短波放射フラックス(上向き、大気上端) \\
{\tt RSDT} & 116 & GVD & ${\rm\,W/m^2}$ & 短波放射フラックス(下向き、大気上端) \\
{\tt RSDBc} & 116 & GVD & ${\rm\,W/m^2}$ & 短波放射フラックス(下向き、地表面、晴天) \\
{\tt RSUBc} & 116 & GVD & ${\rm\,W/m^2}$ & 短波放射フラックス(上向き、地表面、晴天) \\
{\tt RSUTc} & 116 & GVD & ${\rm\,W/m^2}$ & 短波放射フラックス(上向き、大気上端、晴天) \\
{\tt RSDSn} & 116 & GVD & ${\rm\,W/m^2}$ & 短波放射フラックス(下向き、地表面、積雪内) \\
{\tt GlRad} & 117 &  & ${\rm\,W/m^2}$ & 全天日射フラックス \\
{\tt BrT} & 118 &  & ${\rm\,K}$ & 輝度温度 \\
{\tt WNRad} & 119 &  & ${\rm\,W/(m\,\cdot\,sr)}$ & 放射(波数に関して) \\
{\tt WLRad} & 120 &  & ${\rm\,W/(m^3\,\cdot\,sr)}$ & 放射(波長に関して) \\
{\tt FLLH} & 121 & GVD & ${\rm\,W/m^2}$ & 潜熱フラックス \\
{\tt FLSH} & 122 & GVD & ${\rm\,W/m^2}$ & 顕熱フラックス \\
{\tt BLDsp} & 123 &  & ${\rm\,W/m^2}$ & 境界層における散逸 \\
{\tt FLMU} & 124 & GVD & ${\rm\,N/m^2}$ & 運動量フラックス(地表面風応力)x成分 \\
{\tt FLMV} & 125 & GVD & ${\rm\,N/m^2}$ & 運動量フラックス(地表面風応力)y成分 \\
{\tt WMixE} & 126 &  & ${\rm\,J}$ & 風の混合エネルギー \\
{\tt Image} & 127 &  &  & 画像資料 \\
{\tt WatrT} & 128 &  & ${\rm\,K}$ & 水温 \\
{\tt CLC2} & 129 &  & ${\rm\,Percent}$ & 雲量 \\
{\tt AvTBB} & 130 &  & ${\rm\,K}$ & TBBの平均値 \\
{\tt MnTBB} & 131 &  & ${\rm\,K}$ & TBBの最小値 \\
{\tt SDTBB} & 132 &  & ${\rm\,K}$ & TBBの標準偏差 \\
{\tt SNCov} & 133 &  & ${\rm\,Percent}$ & 雪氷域 \\
{\tt Tsun} & 134 &  & ${\rm\,J/m^2}$ & 全天日射量 \\
{\tt HZanP} & 140 &  &  & ジオポテンシャル高度の高偏差確率 \\
{\tt PSprd} & 141 &  &  & 気圧のスプレッド \\
{\tt ZSprd} & 142 &  &  & ジオポテンシャル高度のスプレッド \\
{\tt TSprd} & 143 &  &  & 気温のスプレッド \\
{\tt EAvSLP} & 200 &  & ${\rm\,hPa}$ & 海面更正気圧(アンサンブルメンバーの平均) \\
{\tt EAvZ} & 201 &  & ${\rm\,gpm}$ & ジオポテンシャル高度(アンサンブルメンバーの平均) \\
{\tt EAvT} & 202 &  & ${\rm\,K}$ & 気温(アンサンブルメンバーの平均) \\
{\tt EAvU} & 203 &  & ${\rm\,m/s}$ & 風のx軸方向成分(アンサンブルメンバーの平均) \\
{\tt EAvV} & 204 &  & ${\rm\,m/s}$ & 風のy軸方向成分(アンサンブルメンバーの平均) \\
{\tt ESDSLP} & 210 &  & ${\rm\,hPa}$ & 海面更正気圧(アンサンブルメンバーの標準偏差) \\
{\tt ESDZ} & 211 &  & ${\rm\,gpm}$ & ジオポテンシャル高度(アンサンブルメンバーの標準偏差) \\
{\tt ESDT} & 212 &  & ${\rm\,K}$ & 気温(アンサンブルメンバーの標準偏差) \\
{\tt ESDU} & 213 &  & ${\rm\,m/s}$ & 風のx軸方向成分(アンサンブルメンバーの標準偏差) \\
{\tt ESDV} & 214 &  & ${\rm\,m/s}$ & 風のy軸方向成分(アンサンブルメンバーの標準偏差) \\
{\tt RAM0} & 未定義 &  &  & 雨量換算係数 \\
{\tt RAM1} & 未定義 &  &  & 雨量換算係数 \\
{\tt RAM2} & 未定義 &  &  & 雨量換算係数 \\
{\tt RAM3} & 未定義 &  &  & 雨量換算係数 \\
{\tt TANKLV} & 未定義 &  &  & 土壌雨量タンクレベル値 \\
{\tt TKRANK} & 未定義 &  &  & 土壌雨量履歴順位レベル値 \\
{\tt cUVI} & 未定義 &  &  & 紫外線指数(晴天時) \\
{\tt wUVI} & 未定義 &  &  & 紫外線指数 \\
{\tt TDSCS} & 未定義 &  & ${\rm\,g/m^3}$ & ダスト下層濃度 \\
{\tt TDSCI} & 未定義 &  & ${\rm\,kg/m^2}$ & ダスト気柱積算量 \\
{\tt CWMR} & 未定義 &  & ${\rm\,kg/kg}$ & 雲水混合比 \\
{\tt CIMR} & 未定義 &  & ${\rm\,kg/kg}$ & 雲氷混合比 \\
{\tt SkinT} & 未定義 &  & ${\rm\,K}$ & 地面からの長波放射量(温度換算値) \\
{\tt WindDM} & 未定義 &  &  & レーダーVVP風ベクトル 最大風速差 \\
{\tt VVPSD} & 未定義 &  &  & レーダーVVP風ベクトル 標準偏差 \\
{\tt VVPSN} & 未定義 &  &  & レーダーVVP風ベクトル 標本数 \\
{\tt Altit} & 未定義 &  & ${\rm\,m}$ & モデル地面の高度 \\
{\tt FGSU} & 未定義 & GVD &  & 重力波抵抗短波運動量フラックスx成分 \\
{\tt FGSV} & 未定義 & GVD &  & 重力波抵抗短波運動量フラックスy成分 \\
{\tt FGLU} & 未定義 & GVD &  & 重力波抵抗長波運動量フラックスx成分 \\
{\tt FGLV} & 未定義 & GVD &  & 重力波抵抗長波運動量フラックスy成分 \\
{\tt LTRS} & 未定義 & GVD &  & 蒸散 \\
{\tt LINT} & 未定義 & GVD &  & キャノピー面にたまった水からの潜熱フラックス \\
{\tt MSC} & 未定義 & GVD &  & キャノピーの水分量 \\
{\tt MSG} & 未定義 & GVD &  & 地面・下草の水分量 \\
{\tt TSC} & 未定義 & GVD &  & キャノピーの温度 \\
{\tt TSG} & 未定義 &  &  & 地面・下草の温度 \\
{\tt ISC} & 未定義 &  &  & キャノピーの氷 \\
{\tt ISG} & 未定義 &  &  & 下草の氷 \\
{\tt SoilI} & 未定義 &  &  & 積算土壌氷分量 \\
{\tt SoilQ} & 未定義 &  &  & 土壌伝導熱 \\
{\tt TSN} & 未定義 &  &  & 積雪表面温度 \\
{\tt SnTmp} & 未定義 &  &  & 積雪温度 \\
{\tt SnQ} & 未定義 &  &  & 積雪伝導熱 \\
{\tt SnW} & 未定義 &  &  & 積雪の含水量 \\
{\tt SnDen} & 未定義 &  &  & 積雪密度 \\
{\tt SnFr} & 未定義 &  & ${\rm\,Proportion}$ & 積雪被覆率(部分積雪の面積比率) \\
{\tt KIND} & 未定義 & GVD &  & 地表面状態 \\
{\tt U1} & 未定義 &  &  & モデル最下層の風のx成分 \\
{\tt V1} & 未定義 &  &  & モデル最下層の風のy成分 \\
{\tt T1} & 未定義 &  &  & モデル最下層の気温 \\
{\tt Q1} & 未定義 &  &  & モデル最下層の比湿 \\
{\tt WET} & 未定義 & GVD &  & 土壌水分飽和度 \\
{\tt UWV} & 未定義 & GVD &  & 水蒸気フラックスx成分 \\
{\tt VWV} & 未定義 & GVD &  & 水蒸気フラックスy成分 \\
{\tt RCST} & 未定義 & GVD &  & 放射強制力(短波・天頂) \\
{\tt RCSB} & 未定義 & GVD &  & 放射強制力(短波・地表面) \\
{\tt RCLT} & 未定義 & GVD &  & 放射強制力(長波・天頂) \\
{\tt RCLB} & 未定義 & GVD &  & 放射強制力(長波・地表面) \\
{\tt PBLH} & 未定義 &  &  & 境界層の高さ \\
{\tt HRRS} & 未定義 &  & ${\rm\,K/day}$ & 短波放射による気温の変化率 ( 加熱率 ) \\
{\tt HRRL} & 未定義 &  & ${\rm\,K/day}$ & 長波放射による気温の変化率 ( 加熱率 ) \\
{\tt HRCV} & 未定義 &  & ${\rm\,K/day}$ & 積雲対流による気温の変化率 ( 加熱率 ) \\
{\tt HRLC} & 未定義 &  & ${\rm\,K/day}$ & 層状性降水による気温の変化率 ( 加熱率 ) \\
{\tt HRVD} & 未定義 &  & ${\rm\,K/day}$ & 鉛直拡散による気温の変化率 ( 加熱率 ) \\
{\tt HRAD} & 未定義 &  & ${\rm\,K/day}$ & 断熱過程による気温の変化率 ( 加熱率 ) \\
{\tt HR} & 未定義 &  & ${\rm\,K/day}$ & 気温の変化率 ( 加熱率 ) \\
{\tt QRCV} & 未定義 &  & ${\rm\,kg/(kg\,\cdot\,day)}$ & 積雲対流による比湿の変化率 \\
{\tt QRLC} & 未定義 &  & ${\rm\,kg/(kg\,\cdot\,day)}$ & 層状性降水による比湿の変化率 \\
{\tt QRVD} & 未定義 &  & ${\rm\,kg/(kg\,\cdot\,day)}$ & 鉛直拡散による比湿の変化率 \\
{\tt QRAD} & 未定義 &  & ${\rm\,kg/(kg\,\cdot\,day)}$ & 断熱過程による比湿の変化率 \\
{\tt URGW} & 未定義 &  & ${\rm\,m/(s\,\cdot\,day)}$ & 重力波抵抗による u の変化率 \\
{\tt URCV} & 未定義 &  & ${\rm\,m/(s\,\cdot\,day)}$ & 積雲対流による u の変化率 \\
{\tt URVD} & 未定義 &  & ${\rm\,m/(s\,\cdot\,day)}$ & 鉛直拡散による u の変化率 \\
{\tt URAD} & 未定義 &  & ${\rm\,m/(s\,\cdot\,day)}$ & 断熱過程による u の変化率 \\
{\tt VRGW} & 未定義 &  & ${\rm\,m/(s\,\cdot\,day)}$ & 重力波抵抗による v の変化率 \\
{\tt VRCV} & 未定義 &  & ${\rm\,m/(s\,\cdot\,day)}$ & 積雲対流による v の変化率 \\
{\tt VRVD} & 未定義 &  & ${\rm\,m/(s\,\cdot\,day)}$ & 鉛直拡散による v の変化率 \\
{\tt VRAD} & 未定義 &  & ${\rm\,m/(s\,\cdot\,day)}$ & 断熱過程による v の変化率 \\
{\tt CVR} & 未定義 &  &  & 雲量 \\
{\tt UMF} & 未定義 &  &  & 上向きマスフラックス \\
{\tt UMB} & 未定義 &  &  & 雲底での上向きマスフラックス \\
{\tt CWF} & 未定義 &  &  & 雲仕事関数 \\
{\tt MXWIN} & 未定義 &  &  & 最大風速面 \\
{\tt TROP1} & 未定義 &  &  & 第1圏界面 \\
{\tt TROP2} & 未定義 &  &  & 第2圏界面 \\
{\tt CBTOP} & 未定義 &  &  & 積乱雲頂 \\
{\tt NUM} & 未定義 &  &  & 地点番号 \\
{\tt MLAT} & 未定義 &  & ${\rm\,deg}$ & 緯度 \\
{\tt MLON} & 未定義 &  & ${\rm\,deg}$ & 経度 \\
{\tt LAT} & 未定義 &  & ${\rm\,deg}$ & 緯度 \\
{\tt LON} & 未定義 &  & ${\rm\,deg}$ & 経度 \\
{\tt HIGH} & 未定義 &  & ${\rm\,m}$ & 高度 \\
{\tt AQC} & 未定義 &  &  & アメダスAQC情報 \\
{\tt Sunsh} & 未定義 &  &  & 日照時間のある割合 \\
{\tt SSfr0} & 未定義 &  & ${\rm\,min}$ & 日照時間 \\
{\tt SEC} & 未定義 &  & ${\rm\,s}$ & 時間 \\
{\tt CSEC} & 未定義 &  & ${\rm\,s}$ & 時間 \\
{\tt TDDKK} & 未定義 &  &  & 雷多重度+放電種別 \\
{\tt TEC} & 未定義 &  & ${\rm\,kA}$ & 雷推定電流 \\
{\tt SM} & 未定義 &  &  & マップファクター \\
{\tt PI10LV} & 未定義 &  &  & レーダー降水強度10分レベル値 \\
{\tt RR60LV} & 未定義 &  &  & レーダー積算降水量60分レベル値 \\
{\tt RR10LV} & 未定義 &  &  & レーダー積算降水量10分レベル値 \\
{\tt HIGHLV} & 未定義 &  &  & レーダーエコー頂高度レベル値 \\
{\tt SB\_NUM} & 未定義 &  &  & 二次細分区番号(注警報に使用しているもの) \\
{\tt GRDFLG} & 未定義 &  &  & 二次細分区における検証格子存在チェック \\
{\tt FO} & 未定義 &  &  & 検証四分割表における予報あり、観測あり \\
{\tt NFO} & 未定義 &  &  & 検証四分割表における予報なし、観測あり \\
{\tt FNO} & 未定義 &  &  & 検証四分割表における予報あり、観測なし \\
{\tt NFNO} & 未定義 &  &  & 検証四分割表における予報なし、観測なし \\
{\tt FCST} & 未定義 &  &  & 検証四分割表における予報ありの数 \\
{\tt OBS} & 未定義 &  &  & 検証四分割表における観測ありの数 \\
{\tt PS} & 1 &  &  & 地表面気圧 \\
{\tt P1} & 1 &  & ${\rm\,hPa}$ & モデル最下層気圧面 \\
{\tt US} & 33 &  & ${\rm\,m/s}$ & 高度 10m におけるx軸成分 \\
{\tt VS} & 34 &  & ${\rm\,m/s}$ & 高度 10m におけるy軸成分 \\
{\tt TS} & 11 &  & ${\rm\,K}$ & 高度 2m における気温 \\
{\tt QS} & 51 &  & ${\rm\,kg/kg}$ & 高度 2m における比湿 \\
{\tt maxWS} & 未定義 &  & ${\rm\,m/s}$ & 前出力からの高度10mにおける最大風速 \\
{\tt CLAR} & 71 &  &  & 放射スキームの全雲量 \\
{\tt CW} & 76 &  & ${\rm\,kg/m^2}$ & 鉛直積算雲液水量 \\
{\tt TCWCR} & 76 &  & ${\rm\,kg/m^2}$ & 放射スキームの鉛直積算雲水量 \\
{\tt XL0} & 未定義 &  & ${\rm\,m}$ & 混合距離 \\
{\tt Sunsc} & 未定義 &  & ${\rm\,Proportion}$ & 晴天時の日照時間のある割合 \\
{\tt Sn} & 未定義 &  & ${\rm\,mm/day}$ & 降雪強度 ( 雨換算 ) \\
{\tt CTOP} & 未定義 &  & ${\rm\,hPa}$ & 雲頂 \\
{\tt CBASE} & 未定義 &  & ${\rm\,hPa}$ & 雲底 \\
{\tt CVTOP} & 未定義 &  & ${\rm\,hPa}$ & 対流スキームにおける雲頂 \\
{\tt FrCV} & 未定義 &  &  & 深い積雲対流の発生率 \\
{\tt FrCVs} & 未定義 &  &  & 浅い積雲対流の発生率 \\
{\tt UMBdf} & 未定義 &  & ${\rm\,kg/(m^2\,\cdot\,s)}$ & 雲底での対流性下降流によるマスフラックス \\
{\tt UMBmc} & 未定義 &  & ${\rm\,kg/(m^2\,\cdot\,s)}$ & 雲底での中層対流によるマスフラックス \\
{\tt DCAPE} & 未定義 &  &  & 力学過程によるCAPE生成率 \\
{\tt FrSc} & 未定義 &  & ${\rm\,Proportion}$ & 層積雲スキームの働く割合 \\
{\tt CWCI} & 未定義 &  & ${\rm\,kg/kg}$ & 雲氷量 \\
{\tt CWCW} & 未定義 &  & ${\rm\,kg/kg}$ & 雲液水量 \\
{\tt ROFB} & 90 &  & ${\rm\,mm/day}$ & 底面排水による流出量 \\
{\tt SMC} & 未定義 &  &  & 土壌水分量 \\
{\tt VEG1} & 未定義 &  & ${\rm\,Proportion}$ & キャノピー被覆率 \\
{\tt VEG2} & 未定義 &  & ${\rm\,Proportion}$ & 下草被覆率 \\
{\tt SMCFC} & 未定義 &  & ${\rm\,kg/m^3}$ & 圃場容水量(Field Capacity) \\
{\tt WETFC} & 未定義 &  &  & 圃場容水量(Field Capacity)における土壌含水量 \\
{\tt SMCWP} & 未定義 &  & ${\rm\,kg/m^3}$ & しおれ点における土壌水分量 \\
{\tt WETWP} & 未定義 &  &  & しおれ点における単位体積あたりの土壌水分量 \\
{\tt SMC02} & 未定義 &  &  & 土壌 0 〜 20cm の土壌水分 \\
{\tt LSBL} & 未定義 &  & ${\rm\,W/m^2}$ & 昇華潜熱 \\
{\tt FLG0} & 未定義 &  & ${\rm\,W/m^2}$ & 土壌への下向き熱フラックス \\
{\tt FLS0} & 未定義 &  & ${\rm\,W/m^2}$ & 雪への下向き熱フラックス \\
{\tt ALBSV} & 未定義 &  &  & 可視に対する雪アルベド \\
{\tt ALBSI} & 未定義 &  &  & 近赤外に対する雪アルベド \\
{\tt ST02} & 未定義 &  & ${\rm\,K}$ & 土壌 0 〜 20cm の土壌温度 \\
{\tt TSCn} & 未定義 &  & ${\rm\,K}$ & キャノピーの温度(雪なし域) \\
{\tt TSCs} & 未定義 &  & ${\rm\,K}$ & キャノピーの温度(雪あり域) \\
{\tt TSGn} & 未定義 &  & ${\rm\,K}$ & 地面・下草の温度(雪なし域) \\
{\tt TSGs} & 未定義 &  & ${\rm\,K}$ & 地面・下草の温度(雪あり域) \\
{\tt SoilTn} & 未定義 &  & ${\rm\,K}$ & 土壌温度(雪なし域) \\
{\tt SoilTs} & 未定義 &  & ${\rm\,K}$ & 土壌温度(雪あり域) \\
{\tt SoilWn} & 未定義 &  &  & 土壌液水率(雪なし域) \\
{\tt SoilWs} & 未定義 &  &  & 土壌液水率(雪あり域) \\
{\tt SoilIn} & 未定義 &  &  & 土壌氷率(雪なし域) \\
{\tt SoilIs} & 未定義 &  &  & 土壌氷率(雪あり域) \\
{\tt SnI} & 未定義 &  & ${\rm\,kg/m^2}$ & 積雪の含氷量 \\
{\tt R1Dy} & 未定義 &  & ${\rm\,mm/day}$ & 日平均降水量 \\
{\tt R1Dano} & 未定義 &  & ${\rm\,mm/day}$ & 日平均降水量偏差 \\
{\tt Uano} & 未定義 &  & ${\rm\,m/s}$ & 風のx軸成分偏差 \\
{\tt Vano} & 未定義 &  & ${\rm\,m/s}$ & 風のy軸成分偏差 \\
{\tt RHano} & 未定義 &  & ${\rm\,Percent}$ & 相対湿度偏差 \\
{\tt Qano} & 未定義 &  & ${\rm\,kg/kg}$ & 比湿偏差 \\
{\tt SSTano} & 未定義 &  & ${\rm\,K}$ & 海面水温偏差 \\
{\tt SMQR} & 未定義 &  & ${\rm\,kg/m^2}$ & 予報開始からの積算降水量(雨) \\
{\tt SMQS} & 未定義 &  & ${\rm\,kg/m^2}$ & 予報開始からの積算降水量(雪) \\
{\tt SMQG} & 未定義 &  & ${\rm\,kg/m^2}$ & 予報開始からの積算降水量(あられ) \\
{\tt SMQH} & 未定義 &  & ${\rm\,kg/m^2}$ & 予報開始からの積算降水量(ひょう) \\
{\tt RU} & 未定義 &  &  & 単位体積あたりのx方向の運動量と座標変換係数との積 \\
{\tt RV} & 未定義 &  &  & 単位体積あたりのy方向の運動量と座標変換係数の積 \\
{\tt PAIRF} & 未定義 &  &  & 無次元気圧 \\
{\tt AsTide} & 未定義 &  & ${\rm\,m}$ & 天文潮位 \\
{\tt SeaLev} & 未定義 &  & ${\rm\,m}$ & 潮位(=天文潮位+潮位偏差) \\
{\tt EOWvH} & 未定義 &  & ${\rm\,m}$ & 換算沖波波高 \\
\hline
\end{longtable}

\label{tab:element}
