\subsection{nusdas\_subc\_srf: 降短系 SUBC へのアクセス}
\APILabel{nusdas.subc.srf}

\Prototype
\begin{quote}
N\_SI4 {\bf nusdas\_subc\_srf}(const char {\it type1}[8], const char {\it type2}[4], const char {\it type3}[4], const N\_SI4 $\ast${\it basetime}, const char {\it member}[4], const N\_SI4 $\ast${\it validtime}, const char {\it plane}[6], const char {\it element}[6], const char {\it group}[4], N\_SI4 $\ast${\it data}, const char {\it getput}[3]);
\end{quote}

\begin{tabular}{l|rp{20em}}
\hline
\ArgName & \ArgType & \ArgRole \\
\hline
{\it type1} & const char [8] &  種別1  \\
{\it type2} & const char [4] &  種別2  \\
{\it type3} & const char [4] &  種別3  \\
{\it basetime} & const N\_SI4 $\ast$ &  基準時刻(通算分)  \\
{\it member} & const char [4] &  メンバー名  \\
{\it validtime} & const N\_SI4 $\ast$ &  対象時刻(通算分)  \\
{\it plane} & const char [6] &  面  \\
{\it element} & const char [6] &  要素名  \\
{\it group} & const char [4] &  群名  \\
{\it data} & N\_SI4 $\ast$ &  データ配列  \\
{\it getput} & const char [3] &  入出力指示 ({\it "GET}" または {\it "PUT}")  \\
\hline
\end{tabular}
\paragraph{\FuncDesc}降水短時間予報系のデータの補助管理部へのアクセスを提供する。
群名には次のもののいずれかを指定する。
\begin{quote}\begin{description}
\item[{\bf ISPC}] 
レーダーや雨量計の運用情報、レベル値変換テーブルが格納される。
data には 128要素の4バイト整数型配列を用意する。内部のフォーマットは
4バイト整数型であることは関係ないが、バイトオーダーの変換はされるので
注意が必要。
\item[{\bf THUN}] 
詳細未詳。
data には 4バイト整数型変数を用意する。
\item[{\bf RADR}] 
レーダー観測に関する情報。data には 4バイト整数型変数を用意する。
\item[{\bf RADS}] 
レーダー観測に関する情報。data には 6要素の4バイト整数型配列を用意する。
\item[{\bf DPRD}] 
ドップラーレーダー観測に関する情報。
data には 8要素の4バイト整数型配列を用意する。
\end{description}\end{quote}
\paragraph{\ResultCode}
\begin{quote}
\begin{description}
\item[{\bf 0}] 正常終了
\item[{\bf -2}] 要求されたレコードが存在しない、または書かれていない。
\item[{\bf -3}] レコードサイズが不正
\item[{\bf -4}] 群名が不正
\item[{\bf -5}] 入出力指示が不正
\end{description}\end{quote}
\paragraph{ 履歴 }
この関数は NuSDaS1.0 から存在した。
