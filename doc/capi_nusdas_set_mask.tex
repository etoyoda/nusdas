\subsection{nusdas\_set\_mask: 改善型マスクビット設定関数}
\APILabel{nusdas.set.mask}

\Prototype
\begin{quote}
N\_SI4 {\bf nusdas\_set\_mask}(const char {\it type1}[8], const char {\it type2}[4], const char {\it type3}[4], const void $\ast${\it udata}, const char {\it utype}[2], N\_SI4 {\it usize});
\end{quote}

\begin{tabular}{l|rp{20em}}
\hline
\ArgName & \ArgType & \ArgRole \\
\hline
{\it type1} & const char [8] &  種別1  \\
{\it type2} & const char [4] &  種別2  \\
{\it type3} & const char [4] &  種別3  \\
{\it udata} & const void $\ast$ &  データ配列  \\
{\it utype} & const char [2] &  データ配列の型  \\
{\it usize} & N\_SI4 &  配列の要素数  \\
\hline
\end{tabular}
\paragraph{\FuncDesc}
配列 {\it udata} の内容に従って \APILink{nusdas.make.mask}{nusdas\_make\_mask} と同様に
マスクビット列を作成し
指定した種別のデータセットに対して設定する。

\paragraph{\ResultCode}
\begin{quote}
\begin{description}
\item[{\bf 0}] 正常終了
\item[{\bf -5}] 未知の型名 {\it utype} が与えられた
\end{description}\end{quote}

\paragraph{注意}
本関数によるマスクビットの設定は \APILink{nusdas.parameter.change}{nusdas\_parameter\_change} に
優先するが、他のデータセットには効果をもたない。

\paragraph{履歴}
本関数は NuSDaS 1.3 で新設された。
