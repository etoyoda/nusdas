\subsection{bfseek: ファイル位置設定}
\APILabel{bfseek}

\Prototype
\begin{quote}
int {\bf bfseek}(N\_BIGFILE $\ast${\it bf}, long {\it offset}, int {\it whence});
\end{quote}

\begin{tabular}{l|rp{20em}}
\hline
\ArgName & \ArgType & \ArgRole \\
\hline
{\it bf} & N\_BIGFILE $\ast$ &  ファイル  \\
{\it offset} & long &  相対位置  \\
{\it whence} & int &  起点  \\
\hline
\end{tabular}
\paragraph{\FuncDesc}
あらかじめ \APILink{bfopen}{bfopen} で開かれた
ファイル {\it bf} の現在位置を設定する。
{\it offset} は本来 off\_t 型が望ましいのだろうが、
long型で実装されているので注意。

ファイル位置の起算原点は {\it whence} によって異なる。
\begin{quote}\begin{description}
\item[{\bf SEEK\_SET}] ファイル先頭から {\it offset} バイト (非負) 進んだ位置
\item[{\bf SEEK\_CUR}] 現在位置から {\it offset} バイト (負でもよい) 進んだ位置
\item[{\bf SEEK\_END}] ファイル末尾から {\it offset} バイト進んだ位置
\end{description}\end{quote}
\paragraph{\ResultCode}
\begin{quote}
\begin{description}
\item[{\bf 0}] 正常終了
\item[{\bf -1}] エラー
\end{description}\end{quote}
\paragraph{注意}
\begin{itemize}
\item long が 32 ビット幅の場合、2 ギガバイトを超えるファイルでは
指定できない場所がある。
\item {\it whence} に SEEK\_END を指定して正の {\it offset} を指定した場合の
挙動については OS の lseek(2) 等のマニュアルを参照されたい。
\end{itemize}
\paragraph{履歴}
この関数は NuSDaS 1.3 で追加された。
