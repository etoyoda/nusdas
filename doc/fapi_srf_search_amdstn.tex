\subsection{SRF\_SEARCH\_AMDSTN: 地点番号の辞書内通番を探す}
\APILabel{srf.search.amdstn}

\Prototype
\begin{quote}
CALL {\bf SRF\_SEARCH\_AMDSTN}({\it amd}, {\it ac}, {\it stn}, {\it amd\_type}, {\it result})
\end{quote}

\begin{tabular}{l|rllp{16em}}
\hline
\ArgName & \ArgType & \ArrayDim & I/O & \ArgRole \\
\hline
{\it amd} & type(SRF\_AMD\_SINFO) & \AnySize & IN &  地点辞書配列  \\
{\it ac} & INTEGER(4) &  & IN &  地点辞書配列の長さ  \\
{\it stn} & INTEGER(4) &  & IN &  地点番号  \\
{\it amd\_type} & INTEGER(4) &  & IN &  地点種別  \\
{\it result} & INTEGER(4) &  & OUT & \ResultCode \\
\hline
\end{tabular}
\paragraph{\FuncDesc}
SRF\_AMD\_SINFO 構造型の配列 {\it amd} (地点数 {\it ac} 個) から
地点番号 {\it stn} の地点情報を収めた添字 (1始まり) を返す。

\paragraph{\ResultCode}
\begin{quote}
\begin{description}
\item[{\bf 正}] 地点の辞書内格納順位 (1始まり)
\item[{\bf -1}] 地点がみつからない
\end{description}\end{quote}

\paragraph{注意}
\begin{itemize}
\item 地点種別 {\it amd\_type} は無視される。
\item 配列が地点番号順にソートされていることを前提に二分探索を使っている。
\end{itemize}
\paragraph{履歴}
この関数は NAPS7 時代から存在したようであるが
ドキュメントされていなかった。
NuSDaS 1.3 リリースに際してドキュメントされるようになった。
