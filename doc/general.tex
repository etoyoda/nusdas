\Chapter{はじめに}
\section{NuSDaS の概要}\label{機能}

\subsection{NuSDaS とは}

NuSDaS は NWP Standard Dataset System の略で、
数値予報格子点データ (GPV; grid point value)
を格納するために作られたデータ形式である。
C および Fortran で NuSDaS データを読み書きするための
サブルーチン集を NuSDaS インターフェイス (または NuSDaS ライブラリ) という。

気象庁の数値予報ルーチンにおいては
NuSDaS 形式および NuSDaS インターフェイスの利用が必須とされている%
\footnote{ジョブグループ内で閉じたデータのやりとりを除く}。
これはディレクトリも含めたデータ形式や入出力手段の標準化によって、
次のような目標を達成するためである。
\begin{itemize}
\item データの読み書きの方法について調整する手間を簡便化する
\item データ作成者によるメタデータ
	(データを利用するために必要となる付随的情報)
	の提供し忘れを防ぐ
\item デコード・可視化・エンコードなどのアプリケーションの共通化を図る
\end{itemize}

\subsection{NuSDaS の歴史}

数値解析予報システムは第 6 世代 (NAPS6; 1996年3月--2001年2月)
GPV の格納には GVD1 (直接編成ファイル) および
GVS1 (順番編成ファイル) という形式が用いられていた。
GVD1 や GVS1 という言葉がファイル形式およびその読み書きライブラリ
を指す点を含め、NuSDaS の先駆者といえるが、
この頃のオペレーティングシステム (OS) は UNIX とはまったく異なった
\href{http://www.google.com/search?q=VOS3}{VOS3}
というもので、たとえば階層的ファイルシステムが存在しないとか、
OS のレベルで直接編成ファイルと順番編成ファイルが区別されるなど、
現在とはまったく異なる世界であった。

2001年3月から運用された NAPS7 では UNIX (HI-UX/MPP) が用いられることとなり、
可変データの取り回しなどの数値予報ルーチン運用上の諸規則
(数値予報ルーチンルール) はすべて再構築しなければならなくなった。
データ形式も例外ではなく、
% 藤川プログラマを中心として
ディレクトリを用いたデータセット管理のため NuSDaS が開発され、
NAPS7 行用期間中の数値予報ルーチンで使われた。
このときの NuSDaS が NuSDaS 1.0 と呼ばれる\footnote{
	開発管理サーバのリポジトリ trunk/pnusdas/honke 以下が
	NAPS7 ルーチン版に対応する。
}。

NAPS7 はほとんどビッグエンディアン機で構成されていたため、
i386-linux などのリトルエンディアン環境での NuSDaS の利用が課題となった。
また JRA-25 (電力中央研究所の富士通機) や共生プロジェクト (地球シミュレータ)
など気象庁モデルを用いた共同研究が盛んになったこともあり、
プログラムの移植性が問われるようになった。
このため NuSDaS 1.0 をベースとして
バイトオーダー対応、configure システムなどを備えた portable NuSDaS
(pnusdas) が開発され、開発環境で利用された。
後に pnusdas にはパンドラ手順によって
ネットワークを通じて遠隔ホスト上のデータセットにアクセスする
機能が追加され、レーダー情報作成装置などで活躍することとなった。

NAPS7 における開発・運用を通じて NuSDaS の設計には
さまざまな問題点が指摘されるようになった。
\begin{itemize}
 \item データファイルの形式が Fortran 順番探査形式に似ているが違うので
 直接 Fortran で開くことができない%
 \footnote{国土地理院向け配信 GPV は特別に Fortran 形式に変換されていた。
 一方研究所向け配信ではこのような措置が行われず、
 NuSDaS の普及を妨げる一因となった}
 \item
 データファイルの最大長が 2GB (NuSDaS 1.1 以降は 4GB)
 に制限されている
 \item 複数の基準時刻のデータを1ファイルに納めることができないので
 小さなデータセットのファイル数が過大となる%
 \footnote{NAPS7 では季節予報モデルがリスタートするために
 ルーチンルール上は基準時刻毎にファイルを別ける必要があり、
 ファイル数が過大となるという問題であった}
\end{itemize}
データモデルの再設計を含んだライブラリの全面的刷新は
2003 年度後半 (NuSDaS 2.0) と
NAPS8 移行期 (NuSDaS 3.0) との 2 回試みられたが、
いずれも最終的には断念されることとなり、
NAPS8 更新当初は pnusdas をベースとしてレコード形式を
Fortran の順番編成形式に改め、
またデータ出力の効率化を図った NuSDaS 1.1\footnote{
	数値予報課 CVS サーバの pandora/pnusdas
	(特に honke ではなく src サブディレクトリ) の
	タグ NAPS8\_0603 を付した版が対応していたが、
	該当する版は現在運用されているの開発管理サーバのリポジトリに存在しない
}が用いられることとなった。

しかしながらこの措置は一時的なものでしかなく、
2007 年 5 月からメソ予報が 33 時間に延長され、
気圧面予報値ファイルが 4.8 GB に達するため
ついに NuSDaS ファイル形式の変更とこれに伴う
ライブラリの大規模改修が不可避となった。

この機にライブラリの構造を全面刷新し見通しのよいプログラムにするとともに、
NuSDaS 2.0/3.0 の反省をふまえてデータモデルや
API レベルでの互換性を極力重視しつつ、
データサイズ制限の撤廃、新しいメタデータのための補助管理部
(2007 年秋に更新予定の ${\rm T}_L959$ 全球モデルの対応を含む)、
仕様の明確化、さらなる入出力の効率化など
(詳細は \SectionRef{changes13} 参照)
などを行ったのが NuSDaS 1.3 \footnote{
	 開発管理サーバのリポジトリ trunk/nusdas1.3 以下
}である。
NuSDaS 1.3 はメソ 33 時間予報から導入されることになる。

なお新しい補助管理部関連 API の実験として
先行的に NuSDaS 1.1 に当該機能を追加したもの\footnote{
	 開発管理サーバのリポジトリ trunk/pnusdas 以下
}が NuSDaS 1.2 と呼ばれている。

NAPS8 においては NuSDaS 1.1 と NuSDaS 1.3 が用いられた。
移行当初に NuSDaS 1.1 とリンクして導入されたロードモジュールは
特に支障がないかぎりそのまま用いられ、
ルーチン変更にあたっても従来どおり NuSDaS 1.1 (libnusdas.pbf, libnwp.pbf) を
リンクする申請を出すことができた。

2012年6月から運用されている NAPS9 においては NuSDaS 1.1 が廃され、
NuSDaS 1.3 だけが用いられている。

さらに NAPS9 での機能拡張とバグフィックス(詳細は \SectionRef{changes14} 参照)を
導入したものが NuSDaS 1.4 \footnote{
	開発管理サーバのリポジトリ trunk/nusdas1.4 以下
}である。
2018 年 6 月に運用開始された NAPS10 では NuSDaS 1.4 のみ利用可能であり、
今後の NuSDaS ライブラリのメンテナンスは 1.4 版についてだけ行われる。
