\subsection{NUSDAS\_SCAN\_DS: データセットの一覧}
\APILabel{nusdas.scan.ds}

\Prototype
\begin{quote}
CALL {\bf NUSDAS\_SCAN\_DS}({\it type1}, {\it type2}, {\it type3}, {\it nrd}, {\it result})
\end{quote}

\begin{tabular}{l|rllp{16em}}
\hline
\ArgName & \ArgType & \ArrayDim & I/O & \ArgRole \\
\hline
{\it type1} & CHARACTER(8) &  & OUT &  種別1  \\
{\it type2} & CHARACTER(4) &  & OUT &  種別2  \\
{\it type3} & CHARACTER(4) &  & OUT &  種別3  \\
{\it nrd} & INTEGER(4) &  & OUT &  NRD番号 \\
{\it result} & INTEGER(4) &  & OUT & \ResultCode \\
\hline
\end{tabular}
\paragraph{\FuncDesc}
返却値が負になるまで呼出しを繰り返すと、ライブラリが認識している
データセットの一覧が得られる。

\paragraph{\ResultCode}
\begin{quote}
\begin{description}
\item[{\bf 0}] 引数の配列にデータセットの情報が格納された。
\item[{\bf -1}] もうこれ以上データセットは認識されていない。
\end{description}\end{quote}
\paragraph{履歴}
この関数は NuSDaS 1.3 で追加された。
pnusdas には非公開の nusdas\_list\_type という関数があり類似の機能を持つ。
