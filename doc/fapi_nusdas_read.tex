\subsection{NUSDAS\_READ: データ記録の読取}
\APILabel{nusdas.read}

\Prototype
\begin{quote}
CALL {\bf NUSDAS\_READ}({\it utype1}, {\it utype2}, {\it utype3}, {\it basetime}, {\it member}, {\it validtime}, {\it plane}, {\it element}, {\it data}, {\it fmt}, {\it size}, {\it result})
\end{quote}

\begin{tabular}{l|rllp{16em}}
\hline
\ArgName & \ArgType & \ArrayDim & I/O & \ArgRole \\
\hline
{\it utype1} & CHARACTER(8) &  & IN &  種別1  \\
{\it utype2} & CHARACTER(4) &  & IN &  種別2  \\
{\it utype3} & CHARACTER(4) &  & IN &  種別3  \\
{\it basetime} & INTEGER(4) &  & IN &  基準時刻(通算分)  \\
{\it member} & CHARACTER(4) &  & IN &  メンバー  \\
{\it validtime} & INTEGER(4) &  & IN &  対象時刻(通算分)  \\
{\it plane} & CHARACTER(6) &  & IN &  面の名前  \\
{\it element} & CHARACTER(6) &  & IN &  要素名  \\
{\it data} & \AnyType & \AnySize & OUT &  結果格納配列  \\
{\it fmt} & CHARACTER(2) &  & IN &  結果格納配列の型  \\
{\it size} & INTEGER(4) &  & IN &  結果格納配列の要素数  \\
{\it result} & INTEGER(4) &  & OUT & \ResultCode \\
\hline
\end{tabular}
\paragraph{\FuncDesc}引数で指定したTYPE, 基準時刻、メンバー、対象時刻、面、要素のデータを
読み出す。 

fmtは\ref{tab:typename}のものを指定するが、これらの値はN\_を接頭辞につけた定数で参照できる。
例えば'R4'であれば定数N\_R4で参照できる。

\paragraph{\ResultCode}
\begin{quote}
\begin{description}
\item[{\bf 正}] 読み出して格納した格子数
\item[{\bf 0}] 指定したデータは未記録(定義ファイルの elementmap によって書き込まれることは許容されているが、まだデータが書き込まれていない)
\item[{\bf -2}] 指定したデータは記録することが許容されていない(elementmap によって禁止されている場合と指定した面名、要素名が登録されていない場合の両方を含む)。
\item[{\bf -4}] 格納配列が不足
\item[{\bf -5}] 格納配列の型とレコードの記録形式が不整合
\end{description}\end{quote}

\paragraph{ 注意 }
nusdas\_read では、返却値 0 はエラーであることに注意が必要。
nusdas\_read のエラーチェックは返却値が求めている格子数と一致していること
を確認するのが望ましい。
\paragraph{ 互換性 }
NuSDaS1.1 では「ランレングス圧縮で、データが指定最大値を超えている」
(返却値-6)が定義されていたが、はデータの最初だけを
見ているだけで意味がないと思われるので、 NuSDaS 1.3 からはこのエラーは
返さない。また、「ユーザーオープンファイルの管理部又はアドレス部が不正
である」(返却値-7)は、共通部分の-54〜-57に対応するので、このエラーは返さない
