\subsection{srf\_amd\_rdic: アメダス地点辞書の読み込み}
\APILabel{srf.amd.rdic}

\Prototype
\begin{quote}
int {\bf srf\_amd\_rdic}(SRF\_AMD\_SINFO $\ast${\it amd}, int {\it amdnum}, int {\it btime}, int {\it amd\_type});
\end{quote}

\begin{tabular}{l|rp{20em}}
\hline
\ArgName & \ArgType & \ArgRole \\
\hline
{\it amd} & SRF\_AMD\_SINFO $\ast$ &  地点辞書格納配列  \\
{\it amdnum} & int &  地点辞書配列長  \\
{\it btime} & int &  探索日時 (通算分)  \\
{\it amd\_type} & int &  地点種別  \\
\hline
\end{tabular}
\paragraph{\FuncDesc}
環境変数 NWP\_AMDDCD\_STDICT が指す地点辞書ファイル
(環境変数未定義時は amddic.txt となる) から
SRF\_AMD\_SINFO 構造型の配列 {\it amd} にアメダス地点情報を読出す。
読出される地点は時刻 {\it btime} に存在するものが選ばれ、
さらに引数 {\it amd\_type} によって次のように限定される。
配列長 {\it amdnum} を越えて書き出すことはない。

\begin{quote}\begin{description}
\item[{\bf SRF\_KANS}] 官署
\item[{\bf SRF\_ELM4}] 4要素を観測している地点
\item[{\bf SRF\_AMEL}] ロボット雨量計
\item[{\bf SRF\_AIRP}] 航空官署
\item[{\bf SRF\_YUKI}] 積雪観測地点
\item[{\bf SRF\_ALL}] 全地点
\end{description}\end{quote}

\paragraph{\ResultCode}
\begin{quote}
\begin{description}
\item[{\bf 非負}] 地点数
\item[{\bf -1}] 地点種別が不正
\item[{\bf -2}] 結果格納配列の長さ不足
\item[{\bf -3}] 地点辞書ファイルが開けない
\end{description}\end{quote}

\paragraph{参考}
NAPS8 においては地点辞書は
/grpK/nwp/Open/Const/Pre/Dcd/amddic.txt に置かれている。
\paragraph{履歴}
この関数は NAPS7 時代から存在した。
