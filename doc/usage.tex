\Chapter{利用の手引}

\section{動作環境}

\begin{itemize}
\item 現在のところ、トランク更新のための動作確認は
	x86\_64 Linux + gcc で行われている
\item 些細でない変更は x86\_64 Linux + Intel C および XTC OS + 富士通 C でテストされる
% \item 頻繁ではないが x86\_64 Windows 7 + Cygwin, SPARC Linux + 富士通 C
% 	への移植上の問題が報告された場合には対応するようにしている
\item HPPA2 HP-UX や SPARC Solaris は
	2007 年頃から動作確認されていない (実機がない)。
	レコード長が4の倍数でないかマスクビットを使うと
	SIGBUS を起こすことがある
% \item Sun SPARC でもconfigure時に\verb|"--disable-mmap"|を指定することで
%  コンパイルはできるが、SIGBUS を起こすことがある\\
%	これを回避するには、CCをccにし(gcc不可)、
%	\verb|CFLAGS|に\verb|-xmemalign=1i|を設定するとよい
%	(あわせて、\verb|--disable-inline|が必要)
\item gcc4.0以降では、コンパイラの最適化に問題がありfloat型が正常に扱えない場合がある\\
	gcc4.0以降を利用する場合は、\verb|CFLAGS="-g -O1"|として最適化のレベルを下げること
	によりこの問題を回避できる。
      ただし、本件は少なくとも gcc4.8.5 では解決しているようである。
\end{itemize}

\section{利用手順}

\subsection{利用手順概説}

\begin{enumerate}
\item 使っている計算機に NuSDaS ライブラリがインストールされていなければ
	インストールする
    \begin{itemize}
    \item NuSDaS ライブラリのソースコードを取得する
    \item NuSDaS ライブラリをコンパイル・インストールする
    (詳細は INSTALL ファイルをあわせて参照されたい。)
    \end{itemize}
\item アプリケーションを書く
    \begin{itemize}
    \item Fortran ならば
    	\ChapRef{chap:fapi} のサブルーチンを呼ぶサブルーチンの先頭で
    	\verb|use "nusdas_fort.h"| する
    \item C ならば
    	\ChapRef{chap:capi} の関数を呼ぶソースコードの先頭で
    	\verb|#include <nusdas.h>| と、通算分の計算で必要ならば
    	\verb|#include <nwpl_capi.h>| を使う
    \item その他のヘッダ
	(数値予報標準ライブラリと降水短時間ライブラリを除く)
        を使っている C プログラムは
    	NuSDaS のバージョンに依存するため NuSDaS 1.3 以降では使えない
    \end{itemize}
\item アプリケーションをコンパイル・リンクする
    \begin{itemize}
    \item 数値予報ルーチンでは PBF を書くのでコンパイラに渡す
    オプションの詳細に付いて心配する必要はない
    \end{itemize}
\item 定義ファイル・データセットを作成する
    \begin{itemize}
    \item 定義ファイルの構文に付いては \ChapRef{chap:deffile} を参照されたい。
    \item データ設計については \SectionRef{sec:writers} を参考にされたい。
    \end{itemize}
\item アプリケーションを実行する
\end{enumerate}


\subsection{典型的なコマンドライン}

とりあえず最低限動かすためにシンプルな例を挙げる。

\subsubsection*{インストール}

\begin{screen}
\begin{verbatim}
 $ tar xvfz nusdas-1.4.tar.gz
 $ cd nusdas1.4
 $ F90=f90 sh configure --prefix=/opt/nusdas1.4 --without-zlib
 $ make
 $ sudo make install
\end{verbatim}
\end{screen}

ここで \verb|/opt/nusdas1.4| は任意でよいが、書込ができなければならない。
一般ユーザ権限しかないならば \verb|/opt/nusdas1.4| とあるところを
\verb|${HOME}| と読みかえるとよいだろう。


\subsubsection*{C 言語のコンパイル}

\begin{screen}
\begin{verbatim}
 $ cc -I/opt/nusdas1.4/include -c app.c
\end{verbatim}
\end{screen}

\subsubsection*{C 言語のリンク}

\begin{screen}
\begin{verbatim}
 $ cc -o app app.o -L/opt/nusdas1.4/lib -lnusdas -lnwp -lm
\end{verbatim}
\end{screen}

\subsubsection*{Fortran のコンパイル}

\begin{screen}
\begin{verbatim}
 $ f90 -I/opt/nusdas1.4/include -c app.f90
\end{verbatim}
\end{screen}

\subsubsection*{Fortran のリンク}

\begin{screen}
\begin{verbatim}
 $ f90 -o app app.o -L/opt/nusdas1.4/lib -lnusdas -lnwp -lm
\end{verbatim}
\end{screen}

\subsubsection*{データセットの作成}

\begin{screen}
\begin{verbatim}
$ mkdir fcst_p.nus
$ mkdir fcst_p.nus/nusdas_def
$ vi fcst_p.nus/nusdas_def/fcst_p.def
$ ln -s fcst_p.nus NUSDAS10
\end{verbatim}
\end{screen}

このようにして始めてプログラムが実行できる。

\subsection{ライブラリ間の依存関係}

\paragraph{気象庁のライブラリ間の依存性}

降水短時間ライブラリ (libsrf) は NuSDaS に依存する。
そして NuSDaS は数値予報標準ライブラリ (nwplib) に依存する。
したがって、アプリケーションのリンク時には
\verb|-lsrf -lnusdas -lnwp|
の順で記述すべきである。

\paragraph{ZLib 依存性}

ネットワーク NuSDaS で gzip 圧縮を可能に設定していると
ZLib が必要になる。

アプリケーションのリンク時に deflate, Inflate, crc32 などの
シンボルが未定義だといってエラーになるようならば、
リンクのコマンドラインに \verb|-lz| オプションを追加\footnote{
	コマンドラインの最後がよい。
}されたい。
ZLib が OS 標準ではないディレクトリにインストールされている場合は
\verb|libz.a| を探して \verb|-L| オプションを \verb|-lz| の前に挿入する。
たとえば \verb|/usr/local/lib/libz.a| があるならば、
\verb|-L/usr/local/lib -lz| を cc なり f90 のコマンドラインの
末尾に追加するとよい。

\paragraph{数学ライブラリ依存性}

NuSDaS 1.1 では使っている関数によっては数学ライブラリをまったく
使わないでアプリケーションをビルド出来る場合があったが、
NuSDaS 1.3 からはほとんど全てのアプリケーションで数学ライブラリが必要となった。

アプリケーションのリンク時に lrint, sin, cos などのシンボルが未定義だといって
エラーになるならば、コマンドラインの最後に \verb|-lm| を追加せよ。

\subsection{設定情報提供スクリプト nusdas-conf}

上述のように NuSDaS ライブラリがどのような設定になっているかによって
必要なライブラリが変わって来るため、
NuSDaS 1.3 からは必要なコンパイルオプションに関する情報を提供する
スクリプト nusdas-conf を提供している。

このスクリプトは「必要なコンパイルオプション」
「必要なライブラリを示すリンカオプション」などの
文字列を標準出力に印字するので、たとえば次のように使うことができる。

\subsubsection*{コンパイル}
\begin{screen}
\begin{verbatim}
 $ `/opt/nusdas1.4/bin/nusdas-config --cc --cflags -cppflags` \
   -c app.c
\end{verbatim}
\end{screen}

\subsubsection*{リンク}
\begin{screen}
\begin{verbatim}
 $ `/opt/nusdas1.4/bin/nusdas-config --cc --cflags` \
   -o app app.o `/opt/nusdas1.4/bin/nusdas-config --ldflags --libs`
\end{verbatim}
\end{screen}

