\subsection{NUSDAS\_SUBC\_SIGM2: SUBC SIGM へのアクセス }
\APILabel{nusdas.subc.sigm2}

\Prototype
\begin{quote}
CALL {\bf NUSDAS\_SUBC\_SIGM2}({\it type1}, {\it type2}, {\it type3}, {\it basetime}, {\it member}, {\it validtime1}, {\it validtime2}, {\it n\_levels}, {\it a}, {\it b}, {\it c}, {\it getput}, {\it result})
\end{quote}

\begin{tabular}{l|rllp{16em}}
\hline
\ArgName & \ArgType & \ArrayDim & I/O & \ArgRole \\
\hline
{\it type1} & CHARACTER(8) &  & IN &  種別1  \\
{\it type2} & CHARACTER(4) &  & IN &  種別2  \\
{\it type3} & CHARACTER(4) &  & IN &  種別3  \\
{\it basetime} & INTEGER(4) &  & IN &  基準時刻(通算分)  \\
{\it member} & CHARACTER(4) &  & IN &  メンバー名  \\
{\it validtime1} & INTEGER(4) &  & IN &  対象時刻1(通算分)  \\
{\it validtime2} & INTEGER(4) &  & IN &  対象時刻2(通算分)  \\
{\it n\_levels} & INTEGER(4) &  & I/O &  鉛直層数  \\
{\it a} & REAL(4) & \AnySize & I/O &  係数 a  \\
{\it b} & REAL(4) & \AnySize & I/O &  係数 b  \\
{\it c} & REAL(4) &  & I/O &  係数 c  \\
{\it getput} & CHARACTER(3) &  & IN &  入出力指示 ({\it "GET}" または {\it "PUT}")  \\
{\it result} & INTEGER(4) &  & OUT & \ResultCode \\
\hline
\end{tabular}
