\section{データ構造}

\subsection{データモデル}

NuSDaS で扱うデータは整数または浮動小数点型の 2 次元配列の集まりである。
この 2 次元配列はデータ記録と呼ばれる。
ほとんどの場合、データ記録の 2 つの次元は水平方向にとられる。
データ記録は次の 6 項目の情報で特定される。
%
\begin{description}
\item[種別] {\em type}
	データセットを識別する 16 文字の名前。
	先頭 8 文字の{\bf 種別1} (type1), 続く 4 文字の{\bf 種別2},
	続く 4 文字の{\bf 種別3}からなる。
	たとえば {\tt \_GSMLLPP}\,{\tt FCSV}\,{\tt STD1} は
	全球気圧面予報値を表わす、といったように
	データの種類によって大部分その名前が決まる。
	詳細は \SectionRef{sec:nustype} 参照。
\item[基準時刻] {\em basetime}
	データセットを作成するジョブが基準として参照する時刻。
	予報・解析データについては初期時刻が用いられ、
	観測データについては観測時刻の近傍の「きりのよい」
	(日単位、時単位など) 時刻が用いられる。
	計算機上では 4 バイト (32 ビット) 符号付き整数で
	グレゴリオ暦 1801 年 1 月 1 日 0 時 GMT
	(寛政12年11月17日巳刻) を 0 として分単位で表現される。
	この表現を{\em 通算分}という。
\item[メンバ名] {\em member}
	アンサンブルモデルのメンバー、
	またはレーダーサイトをあらわす 4 文字の名前。
	メンバが1つしか存在しないデータセットにあっては
	スペース 4 つ ``{\tt \SPC\SPC\SPC\SPC}'' が用いられる。
\item[対象時刻] {\em validtime}
	データ記録の表現する時刻。
	基準時刻と同様、通算分で表現される。
	対象時刻から基準時刻を引いた差を{\em 予報時間 forecast time}
	といって対象時刻と区別する%
	\footnote{にもかかわらずプログラムで用いる各種の名前および
	旧ドキュメントには多少の混乱がみられるので注意されたい。}。
\item[面名] {\em plane}
	ほとんどの場合高度を表わす 6 文字の名前。
	きちんというと、
	データ記録の 2 つの次元と直交する空間方向の位置。
	気圧面データの ``{\tt 500\SPC\SPC\SPC}'' は 500 hPa 面、
	地表面は ``{\tt SURF\SPC\SPC}'' など。
	表 \ref{tab:plane} 参照。
\item[要素名] {\em element}
	物理量を識別するための 6 文字の名前。
	表 \ref{tab:element} 参照。
\end{description}
%
上記の 6 項目は {\em 次元 dimensions} とも呼ばれる。
関係データベースに親しい読者は、これらをキーと考えてもよいだろう。
名前すなわち種別名、メンバ名、面名、要素名には
英字 (大文字と小文字は区別される)、下線、数字が用いられる。

なお、対象時刻および面は範囲指定もできるようになっていたが
実際には使われていない。詳細に付いては \SectionRef{api2} を参照。

\subsection{データの物理構造}

NuSDaS データは大から小へ次のような階層構造をもつ。

\begin{description}
\item[NuSDaS ルートディレクトリ]
	NuSDaS インターフェイスが取り扱うデータはすべて
	環境変数 {\tt NUSDAS}{\it nn} の指すディレクトリまたは
	{\tt ./NUSDAS}{\it nn} に存在しなければならない。
	これらのディレクトリを NuSDaS ルートディレクトリ (NRD) という。
	ここで {\it nn} は $01$ から $99$ までの整数で、
	ファイルの書き込み許可属性にかかわらず
	$01$ から $49$ までの NRD は書き込み可、
	$50$ から $99$ までの NRD は読み込みのみ許可される。
	数値予報ルーチンにおいては NRD を
	{\tt fcst\_p.nus} などといった (サフィックス {\tt .nus} を持つ) 名称
	で作成し、{\tt NUSDAS}{\it nn} はそれへのシンボリックリンクとして
	作成する\footnote{これらの処理は new\_nusdas.sh で行われる}。
\item[データセット]
	NRD の下には {\tt nusdas\_def} というディレクトリがあり、
	その中の {\tt .def} または {\tt .DEF} で終わる名前のファイル
	[{\bf 定義ファイル}, \ChapRef{chap:deffile}]
	が NRD 内のデータファイル (後述)
	の配置およびデータの構造を記述している。
	ひとつの定義ファイルが記述するデータの総体をデータセットという。
	ひとつの NRD には複数のデータセットがあってよいが、
	たいていの NRD はひとつだけのデータセットを含む。
	データセットに対する操作を行う NuSDaS インターフェイス関数では
	種別を使ってデータセットを指示する。
\item[データファイル]
	データファイルは Fortran 順番探査ファイルで、
	その構造は \SectionRef{sec:datafile} で詳述する。
	種別や基準時刻の異なるデータは必ず別のデータファイルに格納される。
	また、面や要素だけが違う2つのデータは必ず同じデータファイルに
	格納される。
	メンバや対象時刻がデータファイルを別にするかどうかは
	定義ファイルの設定による。
	データセットに対する操作を行う NuSDaS インターフェイス関数では
	種別、基準時刻、メンバ名、対象時刻を使ってデータセットを指示する。
\item[データ記録]
	データファイル中に格納されている2次元配列。
	データ記録に対する操作を行う NuSDaS インターフェイス関数では
	種別から要素名までのすべての次元を使ってデータ記録を指示する。
\end{description}

\subsection{定義ファイルとデータファイルの関係}
\label{sec:dataflow}

定義ファイルに書かれている情報は3種類に大別される。
\begin{itemize}
\item 種別 [\DefRef{TYPE1}, TYPE2 文, TYPE3 文]
\item ディレクトリ構造 [\DefRef{PATH}, \DefRef{FILENAME}]
\item メタデータ (その他)
\end{itemize}
メタデータはすべてデータファイル (主に CNTL レコード) に格納される。
そのため両者の関係に注意が必要である。

データセットには多数の基準時刻のデータファイルを置くことができる。
たとえば数年間分のデータファイルが蓄積されて行くような場合、
運用の都合上で要素数・要素名・格子系などが微妙に変更されることがある。
このような時は \APIRef{nusdas.inq.def}{nusdas\_inq\_def} と
\APIRef{nusdas.inq.cntl}{nusdas\_inq\_cntl} の結果が違うことになる。
