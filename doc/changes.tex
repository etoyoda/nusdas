\Chapter{仕様の変更点}

\section{pnusdas}

リトルエンディアン機で動作するようになりました。

\section{NuSDaS 1.1}
\subsection{ファイルにおける「レコード長」の変更}
ファイルの各レコードの先頭、末尾各4バイトに書き込まれているレコード長を
そのレコード長そのものの8バイトを除いた長さにするようにしました
(NuSDaS10では、その8バイトを含めた長さになっていました)。これによっ
て、ファイルがFortran 順編成ファイルになりました。
\subsection{ファイル長の上限が2GBから4GBへ}
従来は、{\tt INDX}レコードを符号付き4byte整数としていたため、ファイルの大きさ
が符号付き4byte整数の表現範囲の上限である約 2GB に制限されていましたが、これ
を符号なし4byte整数として解釈するように変更して、約 4GB までに制限が緩和され
ました。

\subsection{同一TYPEのNRDが複数ある時のNRD探索}
同一のTYPE1,2,3 の NuSDaS Root Directory(NRD) が複数ある場合、
従来は最初に見つかったNRDからデータ取得ができなければ、失敗として
エラーを返していましたが、データ取得が成功するまでNRDの探索を継続
するようにしました。
\subsection{ランレングス圧縮において1byte整数のユーザーデータをサポート}
従来はランレングス圧縮のユーザーデータの型は4byte整数だけをサポー
トしていましたが、1byte整数もサポートするようにしました。
\subsection{高速化}
ファイル書き込みの際にバッファリングを行うことにより、ファイル書き
込みの高速化を図りました。

%%%% 確かに多数の関数が追加されたけど、広く公開していないので、
%%%% 言及しなくてもよいかも  tabito
%多数の関数が追加されました。

\section{NuSDaS 1.2}

SUBC RGAU, SUBC ZHYB, SUBC DELT 記録のアクセス関数が追加されました。

\section{NuSDaS 1.3}
\label{changes13}

\subsection{Fortran での定数 NULL の廃止}

あまりに実装が難しいため、
Fortran インターフェイスでの定数 {\bf NULL} が廃止されました。
\\* nusdas\_parameter\_change()
で設定したパラメタを既定値に戻すには、明示的に既定値を設定するか、新設の
\\* nusdas\_parameter\_reset()
を用いてください。

\subsection{ファイル名生成}

定義ファイルの filename 文にスラッシュと
{\tt \_basetime} などの置換指定を使うと互換性がなくなります。
ディレクトリは path 文に記述するようにしてください。

\subsection{CNTL 記録の大きさ}

NuSDaS 1.3 で作られる CNTL 記録に書かれる対象時刻のリストは、
定義ファイルの予報時間リスト (VALIDTIME1 文) と
基準時刻から導かれる対象時刻すべてとなります。

従来は定義ファイルの設定により、対象時刻のリストが1対象時刻だけとなる
場合がありました。

\subsection{コーデック}\label{chgs-codec}

nusdas\_read や nusdas\_write などで与えるユーザ配列の型と
パッキング方式の組合せはなんでもよいわけではないのですが、
許される組合せが増えました。

特に、{\tt RLEN} パッキングで不適切なユーザ配列型
({\bf N\_I2}, {\bf N\_R4}, {\bf N\_R8})
を与えた場合に不定動作となるバグに対処しました。

一方、{\bf N\_NC} 機能は 2012年9月19日時点では {\tt 2UPC} パッキング
および {\tt 2UPJ} パッキングだけについて実装されています。
(NuSDaS 1.4 からは {\tt 2UPP} も追加)

\subsection{SUBC 記録}

SUBC 記録を書き込んでいない状況で
nusdas\_subc\_tdif(),
nusdas\_subc\_srf(),
nusdas\_subc\_delt() (NuSDaS 1.2 のみ)
で読出しを行った場合、
従来は異常が通知されずすべてゼロの値が返されますが、
エラーコード $-2$ で異常終了するようにしました。

データファイルが存在しない状況で
SUBC 記録を書き出すと従来はエラーコード $-51$ で異常終了していましたが、
正常に書き出せるようになりました。

性能上の理由から
NUSD--SUBC, INFO, END 記録の書き出しはファイルを閉じるときまで遅延されます。
そこで SUBC 記録を出力したプログラムがファイルを閉じることなく
異常終了すると当該 SUBC 記録の内容が読出せなくなります
(従来版では読出せることがある)。

\subsection{INFO 記録}

リトルエンディアン機で定義ファイルによって INFO 記録を作成すると
群名が反転 (例: ${\tt VSRF} \to {\tt FRSV}$) するバグが対処されています。

INFO 記録を読み出すときバッファ長が不足していると
従来はバッファ長だけ読出して正常終了していましたが
エラーコード $-3$ で異常終了するようにしました。

データファイルが存在しない状況で
INFO 記録を書き出すと従来はエラーコード $-51$ で異常終了していましたが、
正常に書き出せるようになりました。

\section{NuSDaS 1.4}
\label{changes14}

NAPS9 運用開始後 2020年1月までの主な変更は次のとおりです。
チケット番号 (\#717) 、リビジョン番号 (r4369) などは開発管理サーバ Redmine の番号です。

\paragraph{\#717, r4369}
入出力モジュール aio, mmap, ibmshmat が廃止されました。

\paragraph{r4376}
実行時オプション GSVB が新設されました。

\paragraph{\#720, r4380}
ES 対応が入りました。

\paragraph{\#745, r4386}
OpenMP に対応しました。

\paragraph{\#752, r4407}
リトルエンディアン機でのバグ対応。
負値を I2 または I4 型のパッキングで保存して R4 に読み出すと、
あたかも符号なし整数であったかのような大きな値が読み出されていました。

\paragraph{\#765, r4413}
SUBCを上書きするとENDレコードが壊れる問題に対処しました。

\paragraph{\#771, r4420}
複合差分圧縮 packing = 2UPP を導入しました。

\paragraph{\#770, r4432}
NAPS インストールツールを再び動作するようにしました。

\paragraph{\#848}
複合差分圧縮でOpenMPに対応しました。

\paragraph{\#934}
計算誤差によって欠損値が設定していない格子に欠損値が格納される不具合を修正しました。

\paragraph{\#1037}
2UPP展開サブルーチンを追加しました。

\paragraph{\#1162}
intelコンパイラでjpeg2000利用時に配列侵害を起こす不具合を修正しました。

\paragraph{\#1242}
SUBC INFOレコードで初期化せず利用する変数を初期化するよう変更しました。

\paragraph{\#1248}
矩形ガウス格子の出力でリトルエンディアン機に対応しました。

\paragraph{\#1643}
2UPP,2UPJ 書き込み時のメモリリークが起きないよう対応しました。

\paragraph{\#1678}
[122]PAC, 2UP[CPJ] で 負の0による不一致がでないよう修正しました。

\paragraph{\#1684}
ISS(GPFS)のテープアクセス判定を追加しました。

\paragraph{\#1764}
NuSDaS\_inq\_data で対象ファイルが存在しない場合に N\_DATA\_EXIST の戻り値が不定とならないよう修正しました。

\paragraph{\#1788}
close時にfsyncを実行する処理を追加しました。

\paragraph{\#1803}
デバッグ時にタイムスタンプを出力しないオプションNTMSを追加しました。

\paragraph{\#1856}
動的ライブラリとして利用しやすいよう、リンク切れ修正しました。

\paragraph{\#1876}
N\_ND を新規実装しました。

\paragraph{\#1881}
NUSDレコードとENDレコードのレコード数が不正な値になる不具合を修正しました。

\paragraph{\#2015}
lustre flush を利用してLustreの末尾欠落問題に対応しました。

\paragraph{\#2045}
末尾欠落に該当したレコード読み込みの戻り値が正常に読み込みの値となるバグを修正しました。

\paragraph{\#2080}
NTMSオプションの名称をGNTSに修正しました。



