\subsection{NUSDAS\_READ\_3D: 高次元読み込み}
\APILabel{nusdas.read.3d}

\Prototype
\begin{quote}
CALL {\bf NUSDAS\_READ\_3D}({\it type1}, {\it type2}, {\it type3}, {\it basetime}, {\it member}, {\it validtime}, {\it plane}, {\it element}, {\it nrecs}, {\it udata}, {\it utype}, {\it usize}, {\it result})
\end{quote}

\begin{tabular}{l|rllp{16em}}
\hline
\ArgName & \ArgType & \ArrayDim & I/O & \ArgRole \\
\hline
{\it type1} & CHARACTER(8) &  & IN &  種別1  \\
{\it type2} & CHARACTER(4) &  & IN &  種別2  \\
{\it type3} & CHARACTER(4) &  & IN &  種別3  \\
{\it basetime} & INTEGER(4) &  & IN &  基準時刻  \\
{\it member} & CHARACTER(4) & \AnySize & IN &  メンバ名の配列  \\
{\it validtime} & INTEGER(4) & \AnySize & IN &  対象時刻の配列  \\
{\it plane} & CHARACTER(6) & \AnySize & IN &  面名の配列  \\
{\it element} & CHARACTER(6) & \AnySize & IN &  要素名の配列  \\
{\it nrecs} & INTEGER(4) &  & IN &  レコード数  \\
{\it udata} & \AnyType & \AnySize & OUT &  結果格納配列  \\
{\it utype} & CHARACTER(2) &  & IN &  結果格納配列の型  \\
{\it usize} & INTEGER(4) &  & IN &  レコードあたり要素数  \\
{\it result} & INTEGER(4) &  & OUT & \ResultCode \\
\hline
\end{tabular}

\paragraph{\FuncDesc}
{\it member}, {\it validtime}, {\it plane}, {\it element} には {\it nrecs} 個の大きさを持つ配列を指定する。
これらの配列の各要素を指定して \APILink{nusdas.read}{nusdas\_read} を順次呼びだし {\it udata} に格納する。
{\it udata} の要素数は {\it nrecs} * {\it usize} 個以上でなければならない。
nusdas\_read の返却値(終了コード)が {\it usize} と一致しなかった場合は読み込みを終了する。

utypeは\ref{tab:typename}のものを指定するが、これらの値はN\_を接頭辞につけた定数で参照できる。
例えば'R4'であれば定数N\_R4で参照できる。

\paragraph{\ResultCode}
\begin{quote}
\begin{description}
\item[{\bf 正}] 全てのデータを読むことができた場合は {\it nrecs} * {\it usize} を返す。
\item[{\bf 負}] 最後に呼び出した \APILink{nusdas.read}{nusdas\_read} の終了コードを返す。
\end{description}\end{quote}

\paragraph{注意}
読み込みを途中で終了した場合、どこまで処理したかを知る方法は無い。
