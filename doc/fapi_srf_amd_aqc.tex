\subsection{SRF\_AMD\_AQC: AQCのパックを展開}
\APILabel{srf.amd.aqc}

\Prototype
\begin{quote}
CALL {\bf SRF\_AMD\_AQC}({\it aqc\_in}, {\it num}, {\it aqc\_out}, {\it param})
\end{quote}

\begin{tabular}{l|rllp{16em}}
\hline
\ArgName & \ArgType & \ArrayDim & I/O & \ArgRole \\
\hline
{\it aqc\_in} & INTEGER(2) & \AnySize & IN &  AQC 配列  \\
{\it num} & INTEGER(4) &  & IN &  配列要素数  \\
{\it aqc\_out} & INTEGER(2) & \AnySize & I/O &  結果配列  \\
{\it param} & CHARACTER($\ast$) & \AnySize & IN &  要素名  \\
\hline
\end{tabular}
\paragraph{\FuncDesc}
アメダス デコード データセットに含まれる AQC から
要素名 {\it param} で指定される各ビットフィールドを取り出す。
\begin{quote}\begin{description}
\item[{\bf UNYOU△}] 入電・休止・運用情報 (-1:休止, 0:入電無し, 正:入電回数)
\item[{\bf RRfr0△}] 降水量の情報
(0:入電無し, 1:ハードエラー・欠測・休止, 2:AQC該当値, 3:正常値; 以下同じ)
\item[{\bf SSfr0△}] 日照時間の情報
\item[{\bf T△△△△△}] 気温の情報
\item[{\bf WindD△}] 風向の情報
\item[{\bf WindS△}] 風速の情報
\item[{\bf SnowD△}] 積雪深の情報
\end{description}\end{quote}

\paragraph{注意}
要素名が不正な場合は警告後なにもせず終了する。
(NuSDaS 1.3 より前は不定動作)
\paragraph{履歴}
この関数は NAPS7 時代から存在した。
