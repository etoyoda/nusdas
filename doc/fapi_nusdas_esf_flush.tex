\subsection{NUSDAS\_ESF\_FLUSH: NAPS7型ESファイルの出力完了}
\APILabel{nusdas.esf.flush}

\Prototype
\begin{quote}
CALL {\bf NUSDAS\_ESF\_FLUSH}({\it type1}, {\it type2}, {\it type3}, {\it basetime}, {\it member}, {\it validtime}, {\it result})
\end{quote}

\begin{tabular}{l|rllp{16em}}
\hline
\ArgName & \ArgType & \ArrayDim & I/O & \ArgRole \\
\hline
{\it type1} & CHARACTER(8) &  & IN &  種別1  \\
{\it type2} & CHARACTER(4) &  & IN &  種別2  \\
{\it type3} & CHARACTER(4) &  & IN &  種別3  \\
{\it basetime} & INTEGER(4) &  & IN &  基準時刻  \\
{\it member} & CHARACTER(4) &  & IN &  メンバー名  \\
{\it validtime} & INTEGER(4) &  & IN &  対象時刻  \\
{\it result} & INTEGER(4) &  & OUT & \ResultCode \\
\hline
\end{tabular}
\paragraph{\FuncDesc}
\paragraph{履歴} \APILink{nusdas.esf.flush}{nusdas\_esf\_flush} は NuSDaS 1.0 から存在する。
\paragraph{\Bug} NuSDaS 1.3 からは ES をサポートしていないため、
この関数はダミーである。
