\subsection{NUSDAS\_READ2\_RAW: DATA記録内容の直接読取}
\APILabel{nusdas.read2.raw}

\Prototype
\begin{quote}
CALL {\bf NUSDAS\_READ2\_RAW}({\it type1}, {\it type2}, {\it type3}, {\it basetime}, {\it member}, {\it validtime1}, {\it validtime2}, {\it plane1}, {\it plane2}, {\it element}, {\it buf}, {\it buf\_nbytes}, {\it result})
\end{quote}

\begin{tabular}{l|rllp{16em}}
\hline
\ArgName & \ArgType & \ArrayDim & I/O & \ArgRole \\
\hline
{\it type1} & CHARACTER(8) &  & IN &  種別1  \\
{\it type2} & CHARACTER(4) &  & IN &  種別2  \\
{\it type3} & CHARACTER(4) &  & IN &  種別3  \\
{\it basetime} & INTEGER(4) &  & IN &  基準時刻(通算分)  \\
{\it member} & CHARACTER(4) &  & IN &  メンバー名  \\
{\it validtime1} & INTEGER(4) &  & IN &  対象時刻1  \\
{\it validtime2} & INTEGER(4) &  & IN &  対象時刻2  \\
{\it plane1} & CHARACTER(6) &  & IN &  面1  \\
{\it plane2} & CHARACTER(6) &  & IN &  面2  \\
{\it element} & CHARACTER(6) &  & IN &  要素名  \\
{\it buf} & \AnyType & \AnySize & OUT &  データ格納配列  \\
{\it buf\_nbytes} & INTEGER(4) &  & IN &  データ格納配列のバイト数  \\
{\it result} & INTEGER(4) &  & OUT & \ResultCode \\
\hline
\end{tabular}
\paragraph{\FuncDesc}引数で指定したTYPE, 基準時刻、メンバー、対象時刻、面、要素のデータを
ファイルに格納されたままの形式で読み出す。
データは、DATA レコードのフォーマット表\ref{table.fmt.data}の項番10〜13までのデータが
格納される。
\paragraph{\ResultCode}
\begin{quote}
\begin{description}
\item[{\bf 正}] 読み出して格納したバイト数。
\item[{\bf 0}] 指定したデータは未記録(定義ファイルの elementmap によって書き込まれることは許容されているが、まだデータが書き込まれていない)
\item[{\bf -2}] 指定したデータは記録することが許容されていない(elementmap によって禁止されている場合と指定した面名、要素名が登録されていない場合の両方を含む)。
\item[{\bf -4}] 格納配列が不足
\end{description}\end{quote}
\paragraph{ 履歴 }
この関数は NuSDaS1.1 で導入された。
