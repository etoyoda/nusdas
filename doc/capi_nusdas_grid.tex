\subsection{nusdas\_grid: 格子情報へのアクセス}
\APILabel{nusdas.grid}

\Prototype
\begin{quote}
N\_SI4 {\bf nusdas\_grid}(const char {\it type1}[8], const char {\it type2}[4], const char {\it type3}[4], const N\_SI4 $\ast${\it basetime}, const char {\it member}[4], const N\_SI4 $\ast${\it validtime}, char {\it proj}[4], N\_SI4 {\it gridsize}[2], float {\it gridinfo}[14], char {\it value}[4], const char {\it getput}[3]);
\end{quote}

\begin{tabular}{l|rp{20em}}
\hline
\ArgName & \ArgType & \ArgRole \\
\hline
{\it type1} & const char [8] &  種別1  \\
{\it type2} & const char [4] &  種別2  \\
{\it type3} & const char [4] &  種別3  \\
{\it basetime} & const N\_SI4 $\ast$ &  基準時刻(通算分)  \\
{\it member} & const char [4] &  メンバー名  \\
{\it validtime} & const N\_SI4 $\ast$ &  対象時刻(通算分)  \\
{\it proj} & char [4] &  投影法3字略号  \\
{\it gridsize} & N\_SI4 [2] &  格子数  \\
{\it gridinfo} & float [14] &  投影法緒元  \\
{\it value} & char [4] &  格子点値が周囲の場を代表する方法  \\
{\it getput} & const char [3] &  入出力指示 ({\it "GET}" または {\it "PUT}")  \\
\hline
\end{tabular}
\paragraph{\FuncDesc}このAPIは、CNTLレコードに格納された格子情報(つまり定義ファイルに書かれた
格子情報)を返す。nusdas\_parameter\_change を使って、定義ファイルに書いた
格子数から変更した場合には正しい情報が得られない。このような場合は 
nusdas\_inq\_data を使う。

gridinfo には4バイト単精度浮動小数点型の配列で14要素存在するものを指定する。

これはCNTLレコードの項番 15 〜 21に対応する。
順に基準点X座標、基準点Y座標、基準点緯度、基準点経度、
X方向格子間隔、Y方向格子間隔、標準緯度、標準経度、第2標準緯度、第2標準経度、
緯度1、経度1、緯度2、経度2となる。

value の値については\TabRef{tab:value}を参照。

\paragraph{\ResultCode}
\begin{quote}
\begin{description}
\item[{\bf 0}] 正常
\item[{\bf -5}] 入出力指示が不正
\end{description}\end{quote}
\paragraph{ 履歴 }
この関数は NuSDaS 1.0 から実装されていた。
