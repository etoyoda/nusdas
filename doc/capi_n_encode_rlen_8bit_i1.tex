\subsection{n\_encode\_rlen\_8bit\_I1: 1バイト整数を RLE 圧縮する}
\APILabel{n.encode.rlen.8bit.I1}

\Prototype
\begin{quote}
N\_SI4 {\bf n\_encode\_rlen\_8bit\_I1}(const unsigned char {\it udata}[\,], unsigned char {\it compressed\_data}[\,], N\_SI4 {\it udata\_nelems}, N\_SI4 {\it max\_compress\_nbytes}, N\_SI4 $\ast${\it maxvalue});
\end{quote}

\begin{tabular}{l|rp{20em}}
\hline
\ArgName & \ArgType & \ArgRole \\
\hline
{\it udata} & const unsigned char [\,] &  元データ配列  \\
{\it compressed\_data} & unsigned char [\,] &  結果格納配列  \\
{\it udata\_nelems} & N\_SI4 &  元データの要素数  \\
{\it max\_compress\_nbytes} & N\_SI4 &  結果配列のバイト数  \\
{\it maxvalue} & N\_SI4 $\ast$ &  元データの最大値  \\
\hline
\end{tabular}
\paragraph{\FuncDesc}\paragraph{履歴}
この関数は NuSDaS 1.0 から存在するが、ドキュメントされていなかった。
NuSDaS 1.3 から Fortran API を伴う
サービスサブルーチンとして採録された。
