\Chapter{データファイル形式}
\label{sec:datafile}

NuSDaS データファイル形式はライブラリ改修とともに若干変遷をとげている。
これらはファイル形式番号という整数値で区別される。
今まで存在したのは 1 (10 と同一形式), 10, 11, 13, 14 である。
ここでは最新ファイル形式 14 を基本としつつ、旧形式で異なる点は
その都度別に説明する。

\section{レコード形式の一般形}
\label{sec:fmt.general}

NuSDaS データファイル形式は、
ほとんどの Fortran コンパイラが順番探査ファイルとして採用している
形式を持つ (ファイル形式 10 は微妙に違っていた)。
データファイルはレコード (\TabRef{table.fmt.general}) の集まりであり、
レコード内容は最大約 4 ギガバイト
\((4 \times 1024^3 - 24) = {\rm 4\,294\,967\,272 byte}\)
の長さを持つことができる。

レコードの機能は先頭部に書かれている 4 文字のレコード名で決まる。
レコード名には
{\tt NUSD},
{\tt CNTL},
{\tt INDY},
{\tt INDX},
{\tt SUBC},
{\tt DATA},
{\tt INFO}, および
{\tt END\SPC}
\footnote{以下 {\tt \SPC} はスペース文字である。}
があり、配列には以下の規則がある。

\begin{itemize}
\item{} NUSD レコードはファイルの先頭に 1 つだけ存在する。
\item{} CNTL レコードは NUSD レコードの次に 1 つだけ存在する。
\item{} INDY レコードが CNTL レコードの次に 1 つだけ存在する。
	ファイル形式 11 以前ではこの場所に INDX レコードがある
	\footnote{NAPS7 の ES ファイルにはどちらも存在しない。
	しかし、今となってはそれを言っても仕方がないようにも思う}%
	。
\item{} SUBC レコードは INDY レコードの後に、
	定義ファイルで指定した数だけ存在する。
  \begin{itemize}
  \item 定義ファイルで指定しない SUBC レコードを作成することはできない。
  \item 定義ファイルで指定した SUBC レコードを実際に書き込まなくても
	ファイルを閉じる際にエラーになる (初期値設定が可能な場合) か
	不定内容のレコードが作成される。
  \end{itemize}
\item{} END レコードはファイルの末尾に 1つだけ存在する。 
\item{} DATA レコードと INFO レコードは SUBC%
	\footnote{NuSDaS 1.1 系では SUBC より INFO が後と保証できるか?}%
	と END の間に任意個存在する。
  \begin{itemize}
  \item これらの順序はデータ作成プログラムが
	\APIRef{nusdas.write}{nusdas\_write} および
	\APIRef{nusdas.info}{nusdas\_info} を呼び出した順序に依存するため、
	INFO レコードが DATA レコードの後に来ることは保証されない。
  \item 本来 INFO レコードは定義ファイルに記述すべきものであり、
  	正しく記述すれば DATA レコードの前に配置される。
  \end{itemize}
\end{itemize}

\newpage

\begin{table}[htp]
 \begin{center}
 \begin{tabular}{r|llrl}
 \hline
 項番 & フィールド & 型 & 長さ & 内容 \\
      &            &    & オクテット &  \\
 \hline
 1 & レコード長 & 非負整数 & 4 & 項番 2〜6 の占めるオクテット数 (注1) \\
 2 & レコード名 & 文字 & 4 & {\tt NUSD}, {\tt CNTL}, .... \\
 3 & レコード有効長 & 非負整数 & 4 & 項番 2〜5 の占めるオクテット数 \\
 4 & 更新時刻 & 整数 & 4 & データ作成機上での time\_t のシステム時間 \\
 \hline
 5 & レコード内容 & & 可変 & 後続の表を見よ \\
 \hline
 6 & すき間 & & 可変 & レコード長調整のため (注2) \\
 7 & レコード長 & 非負整数 & 4 & 項番1と同じ \\
 \hline
 \end{tabular}
 \end{center}
 \caption[NuSDaS データファイルのレコード形式の一般型]{%
 NuSDaS データファイルのレコード形式の一般型。
 注1: ファイルバージョン 10 においては項番 1〜7 の占めるオクテット数だった。
 注2: 定義ファイルの FIXEDRLEN 機能を用いた場合は、
 項番 1〜7 の長さが指定のレコード長に一致するように幅が決まる。
 そうでない場合、ファイル形式 13 以降においてはレコード長が 8 の倍数となるように
 0〜7 オクテットのすき間が挿入される。
 }
 \label{table.fmt.general}
\end{table}


\newpage
\section{NUSDレコード}
\label{sec:fmt.nusd}

NUSD レコードの形式を \TabRef{table.fmt.nusd} に示す。

\begin{table}[htp]
 \begin{center}
 \begin{tabular}{rr|llrl}
 \hline
 項番 & 位置 & フィールド & 型 & 長さ & 内容 \\
      &      &            & \multicolumn{2}{r}{オクテット} &  \\
 \hline
 1 & 0 & レコード長 & 非負整数 & 4 & \\
 2 & 4 & レコード名 & 文字 & 4 & {\tt NUSD} \\
 3 & 8 & レコード有効長 & 非負整数 & 4 & \\
 4 & 12 & 更新時刻 & 整数 & 4 & \\
 \hline
 5 & 16 & 作成元 & 文字 & 72 & 定義ファイルの CREATOR 文 (注1) \\
 6 & 88 & ファイル長 & 整数 & 8 & ファイルのオクテット数 \\
 7 & 96 & ファイル版番号 & 整数 & 4 & 14 \\
 8 & 100 & ファイル長 & 非負整数 & 4 & (注2) \\
 9 & 104 & レコード数 & 非負整数 & 4 & NUSD から END まで \\
 10 & 108 & INFO レコード数 & 非負整数 & 4 & $N_i$ \\
 11 & 112 & SUBC レコード数 & 非負整数 & 4 & $N_s$ \\
 \hline
 12 & 116 & すき間 & & 可変 &  \\
 13 &  & レコード長 & 非負整数 & 4 &  \\
 \hline
 \end{tabular}
 \end{center}
 \caption[NUSD レコードの形式]{%
  NUSD レコードの形式。
  注1: ファイルバージョン 11 以前では 80 オクテット長で項番 6 は存在しない。
  注2: ファイル長が 0xFFFF\,FFFF オクテットを超えるときは 0xFFFF\,FFFF.
 }
 \label{table.fmt.nusd}
\end{table}

\newpage
\section{CNTLレコード}
\label{sec:fmt.cntl}

CNTL レコードの形式を \TabRef{table.fmt.cntl} に示す。

\begin{table}[htp]
 \begin{center}
 \begin{tabular}{rr|llrl}
 \hline
 項番 & 位置 & フィールド & 型 & 長さ & 内容 \\
      &      &            & \multicolumn{2}{r}{オクテット} &  \\
 \hline
 1 & 0 & レコード長 & 非負整数 & 4 & \\
 2 & 4 & レコード名 & 文字 & 4 & {\tt CNTL} \\
 3 & 8 & レコード有効長 & 非負整数 & 4 & \\
 4 & 12 & 更新時刻 & 整数 & 4 & \\
 \hline
 5 & 16 & 種別		& 文字 & 16 & 定義ファイルの TYPE1〜TYPE3 \\
 6 & 32 & 基準時刻	& 文字 & 12 & {\tt yyyymmddhhnn} 形式 \\
 7 & 44 & 基準時刻	& 整数 & 4 & 通算分 \\
 8 & 48 & 予報時間単位	& 文字 & 4 & 定義ファイルの VALIDTIME 文 \\
 9 & 52 & メンバー数	& 非負整数 & 4 & \(N_M\) \\
 10 & 56 & 予報時間数	& 非負整数 & 4 & \(N_V\) \\
 11 & 60 & 面数		& 非負整数 & 4 & \(N_Z\) \\
 12 & 64 & 要素数	& 非負整数 & 4 & \(N_E\) \\
 13 & 68 & 投影法	& 文字 & 4 & 定義ファイルの PROJECTION 文 \\
 14 & 72 & \(x, y\) 格子数 & 非負整数 &
 	\(4 \times 2\) & 定義ファイルの SIZE 文 \\
 15 & 80 & 基準点座標	& 単精度 & \(4 \times 2\) &
 	定義ファイルの BASEPOINT 文 \\
 16 & 88 & 基準点緯経度 & 単精度 & \(4 \times 2\) & 同上 \\
 17 & 96 & 格子間隔	& 単精度 & \(4 \times 2\) &
 	定義ファイルの DISTANCE 文 \\
 18 & 104 & 標準緯経度	& 単精度 & \(4 \times 2\) &
 	定義ファイルの STANDARD 文 \\
 19 & 112 & 第2標準緯経度 & 単精度 & \(4 \times 2\) & 同上 \\
 20 & 120 & 緯経度1 & 単精度 & \(4 \times 2\) &
 	定義ファイルの OTHERS 文 \\
 21 & 128 & 緯経度2 & 単精度 & \(4 \times 2\) & 同上 \\
 22 & 136 & 格子点の空間代表性 & 文字 & 4 & 定義ファイルの VALUE 文 \\
 23 & 140 & 予約 (投影) & 単精度 & \(4 \times 2\) & \\
 24 & 148 & 予約	& 未定義 & \(4 \times 6\) & \\
 25 & 172 & メンバー名	& 文字 & \(4 \times N_M\) & \\
 26 & $+4N_M$ & 対象時刻1 & 整数 & \(4 \times N_V\) & \\
 27 & $+4N_V$ & 対象時刻2 & 整数 & \(4 \times N_V\) & \\
 28 & $+4N_V$ & 面名1 & 文字 & \(6 \times N_Z\) & \\
 29 & $+6N_Z$ & 面名2 & 文字 & \(6 \times N_Z\) & \\
 30 & $+6N_Z$ & 要素名 & 文字 & \(6 \times N_E\) & \\
 \hline
 31 & $+6N_E$ & すき間 & & 可変 & \\
 32 &  & レコード長 & 非負整数 & 4 & \\
 \hline
 \end{tabular}
 \end{center}
 \caption[CNTL レコードの形式]{%
  CNTL レコードの形式。
  繰り返し数 $N_M$, $N_V$, $N_Z$ および $N_E$ はデータファイル作成時の
  定義ファイルのリスト長である。
  ただし、NuSDaS 1.2 以前は定義ファイルの設定により $N_V$ がこれより
  短くなる場合があった。
 }
 \label{table.fmt.cntl}
\end{table}

\newpage
\section{INDX/INDYレコード}
\label{sec:fmt.indx}

INDY レコードの形式を \TabRef{table.fmt.indy} に、
同じ役割でバージョン 1.1 まで用いられた
INDX レコードの形式を \TabRef{table.fmt.indx} に示す。

\begin{table}[htp]
 \begin{center}
 \begin{tabular}{rr|llrl}
 \hline
 項番 & 位置 & フィールド & 型 & 長さ & 内容 \\
      &      &            & \multicolumn{2}{r}{オクテット} &  \\
 \hline
 1 & 0 & レコード長 & 非負整数 & 4 & \\
 2 & 4 & レコード名 & 文字 & 4 & {\tt INDY} \\
 3 & 8 & レコード有効長 & 非負整数 & 4 & \\
 4 & 12 & 更新時刻 & 整数 & 4 & \\
 \hline
 5 & 16 & DATAレコード位置 & 整数 &
 	\(8\times N_M\times N_V\times N_Z\times N_E\) & (注1,2) \\
 6 &    & DATAレコード長 & 非負整数 &
 	\(4\times N_M\times N_V\times N_Z\times N_E\) & (注1,3) \\
 7 &    & DATAレコード要素数 & 非負整数 &
 	\(4\times N_M\times N_V\times N_Z\times N_E\) & (注1) \\
 \hline
 8 &  & すき間 & & 可変 &  \\
 9 &  & レコード長 & 非負整数 & 4 &  \\
 \hline
 \end{tabular}
 \end{center}
 \caption[INDY レコードの形式]{%
  INDY レコードの形式。
  注1:
  次元順を間違えないようにくどく書くと、この構造は
  「『「『整数値を \(N_E\) 個並べた1次元配列』が
  \(N_Z\) 個並んだ2次元配列」が
  \(N_V\) 個並んだ3次元配列』が
  \(N_M\) 個並んだ4次元配列」である。
  各次元の中での順序は定義ファイルの記述順に同じ。
  ただし、定義ファイルで
  {\tt MEMBER} \(N_M\) {\tt OUT} または
  {\tt VALIDTIME} \(N_V\) {\tt OUT}
  を指定した場合は、NuSDaS 1.1 および NAPS9 以降ではそれぞれ
  \(N_M\) または \(N_V\) に代えて次元長が1となる(当該データファイルで
  使われているメンバーまたは対象時刻だけが記述される)。 
  注2:
  定義ファイルの ELEMENTMAP により出力禁止されているレコードは $-1$,
  出力許可だが未だ書かれていないレコードは $0$ が書かれる。
  注3:
  ここで言うレコード長とはTable\ref{table.fmt.general}で示している
  項番2〜6 の占めるオクテット数ではなく、レコード全体の長さである
  項番1~7の占めるオクテット数である。
 }
 \label{table.fmt.indy}
\end{table}

\begin{table}[htp]
 \begin{center}
 \begin{tabular}{rr|llrl}
 \hline
 項番 & 位置 & フィールド & 型 & 長さ & 内容 \\
      &      &            & \multicolumn{2}{r}{オクテット} &  \\
 \hline
 1 & 0 & レコード長 & 非負整数 & 4 & \\
 2 & 4 & レコード名 & 文字 & 4 & {\tt INDX} \\
 3 & 8 & レコード有効長 & 非負整数 & 4 & \\
 4 & 12 & 更新時刻 & 整数 & 4 & \\
 \hline
 5 & 16 & DATAレコード位置 & 非負整数 &
 	\(4\times N_M\times N_V\times N_Z\times N_E\) & (注1,2) \\
 \hline
 6 &  & すき間 & & 可変 &  \\
 7 &  & レコード長 & 非負整数 & 4 &  \\
 \hline
 \end{tabular}
 \end{center}
 \caption[ファイルバージョン 11 までの INDX レコードの形式]{%
  ファイルバージョン 11 までの INDX レコードの形式。
  注1: 次元順は INDY レコードに同じ。
  注2:
  定義ファイルの ELEMENTMAP により出力禁止されているレコードは
  0xFFFF\,FFFF,
  出力許可だが未だ書かれていないレコードは $0$ が書かれる。
 }
 \label{table.fmt.indx}
\end{table}

\newpage
\section{SUBC レコード}
\label{sec:fmt.subc}

SUBC レコードの一般形式を \TabRef{table.fmt.subc} に示す。

\begin{table}[htp]
 \begin{center}
 \begin{tabular}{rr|llrl}
 \hline
 項番 & 位置 & フィールド & 型 & 長さ & 内容 \\
      &      &            & \multicolumn{2}{r}{オクテット} &  \\
 \hline
 1 & 0 & レコード長 & 非負整数 & 4 & \\
 2 & 4 & レコード名 & 文字 & 4 & {\tt SUBC} \\
 3 & 8 & レコード有効長 & 非負整数 & 4 & \\
 4 & 12 & 更新時刻 & 整数 & 4 & \\
 \hline
 5 & 16 & 群名 & 文字 & 4 & {\tt ETA\SPC}, {\tt ZHYB}, {\tt TDIF}, .... \\
 \hline
 6 & 20 & レコード内容 & 本節下表参照 & 可変 & 群によって異なる \\
 \hline
 7 &  & すき間 & & 可変 &  \\
 8 &  & レコード長 & 非負整数 & 4 &  \\
 \hline
 \end{tabular}
 \end{center}
 \caption[SUBC レコードの形式 (一般型)]{%
  SUBC レコードの形式 (一般型)。
 }
 \label{table.fmt.subc}
\end{table}

\begin{table}[htp]
 \begin{center}
  \begin{tabular}{rr|llrl}
 \hline
 項番 & 位置 & フィールド & 型 & 長さ & 内容 \\
      &      &            & \multicolumn{2}{r}{オクテット} &  \\
 \hline
   6-1 & 20             & 面の数      & 非負整数      & 4 & $N$ \\
   6-2 & 24             & 係数 $A$         & 単精度   & $4\times (N+1)$ & \\ 
   6-3 & $24+4\times (N+1)$ & 係数 $B$         & 単精度   & $4\times (N+1)$ & \\ 
   6-4 & $24+8\times (N+1)$ & 係数 $C$         & 単精度   & 4 & \\ \hline 
  \end{tabular}
 \end{center}
 \caption{SUBC ETA のTable\ref{table.fmt.subc}の項番6の形式}
 \label{table.fmt.subc.eta}
\end{table}

\begin{table}[htp]
 \begin{center}
  \begin{tabular}{rr|llrl}
 \hline
 項番 & 位置 & フィールド & 型 & 長さ & 内容 \\
      &      &            & \multicolumn{2}{r}{オクテット} &  \\
 \hline
   6-1 & 20  & 対象時刻からのずれ & 整数   & $4\times N_M  \times N_V$ & \\
   6-2 & $24+4\times N_M  \times N_V$ 
               & 積算秒         & 整数   & $4\times N_M  \times N_V$ & \\ \hline 
  \end{tabular}
 \end{center}
 \caption{SUBC TDIF のTable\ref{table.fmt.subc}の項番6の形式。
  内容については \SectionRef{sec:timeaxis} 参照。
 }
 \label{table.fmt.subc.tdif}
\end{table}

\begin{table}[htp]
 \begin{center}
  \begin{tabular}{rr|llrl}
 \hline
 項番 & 位置 & フィールド & 型 & 長さ & 内容 \\
      &      &            & \multicolumn{2}{r}{オクテット} &  \\
 \hline
   6-1 & 20  & 時間積分間隔 & 整数   & 4 & \\
  \hline
  \end{tabular}
 \end{center}
 \caption{SUBC DELT のTable\ref{table.fmt.subc}の項番6の形式。
  内容については \SectionRef{sec:timeaxis} 参照。
 }
 \label{table.fmt.subc.delt}
\end{table}

\begin{table}[htp]
 \begin{center}
  \begin{tabular}{rr|llrl}
 \hline
 項番 & 位置 & フィールド & 型 & 長さ & 内容 \\
      &      &            & \multicolumn{2}{r}{オクテット} &  \\
 \hline
   6-1 & 20 & 鉛直層数 & 整数   &     4  & {\tt nz} \\
   6-2 & 24 & 温位の基準値 & 単精度 & 4  & {\tt ptrf} \\
   6-3 & 28 & 気圧の基準値 & 単精度 & 4  & {\tt presrf} \\
   6-4 & 32 & モデル面高度(フルレベル) & 単精度 & $4\times {\tt nz}$ &
   {\tt zrp(nz)}\\
   6-5 & $32 + 4\times{\tt nz}$ & 
    モデル面高度(ハーフレベル) & 単精度 & $4\times {\tt nz}$ & {\tt zrw(nz)}\\
   6-6 & $32 + 8\times{\tt nz}$ & 
    座標変換関数(フルレベル) & 単精度 & $4\times {\tt nz}$ & {\tt vctrans\_p(nz)}\\
   6-7 & $32 + 12\times{\tt nz}$ & 
    座標変換関数(ハーフレベル) & 単精度 & $4\times {\tt nz}$ & {\tt vctrans\_w(nz)}\\
   6-8 & $32 + 16\times{\tt nz}$ & 
    座標変換関数の微分(フルレベル) & 単精度 & $4\times {\tt nz}$ & {\tt dvtrans\_p(nz)}\\
   6-9 & $32 + 20\times{\tt nz}$ & 
    座標変換関数の微分(ハーフレベル) & 単精度 & $4\times {\tt nz}$ &
   {\tt dvtrans\_w(nz)}\\ \hline
  \end{tabular}
 \end{center}
 \caption{SUBC ZHYB のTable\ref{table.fmt.subc}の項番6の形式}
 \label{table.fmt.subc.zhyb}
\end{table}

\begin{table}[htp]
 \begin{center}
  \begin{tabular}{rr|llrl}
 \hline
 項番 & 位置 & フィールド & 型 & 長さ & 内容 \\
      &      &            & \multicolumn{2}{r}{オクテット} &  \\
 \hline
   6-1 & 20 & 全球の南北分割数 & 整数   &     4  & {\tt j} \\
   6-2 & 24 & 格納されている最北の緯度の格子番号 & 整数 & 4  & {\tt j\_start} \\
   6-3 & 28 & データの南北格子数 & 整数 & 4  & {\tt j\_n} \\
   6-4 & 32 & 全球の東西分割数 & 整数 & $4\times {\tt j\_n}$ &
   {\tt i(j\_n)}\\
   6-5 & $32+4\times{\tt j\_n}$ & データの最西格子の番号 & 整数 
   & $4\times {\tt j\_n}$ & {\tt i\_start(j\_n)}\\
   6-6 & $32+8\times{\tt j\_n}$ & データの東西格子数 & 整数
   & $4\times {\tt j\_n}$ & {\tt i\_n(j\_n)}\\
   6-7 & $32+12\times{\tt j\_n}$ & 緯度 & 単精度
   & $4\times {\tt j\_n}$ & {\tt lat(j\_n)}\\
   \hline
  \end{tabular}
 \end{center}
 \caption{SUBC RGAU のTable\ref{table.fmt.subc}の項番6の形式}
 \label{table.fmt.subc.rgau}
\end{table}

\begin{table}[htp]
 \begin{center}
  \begin{tabular}{rr|llrl}
 \hline
 項番 & 位置 & フィールド & 型 & 長さ & 内容 \\
      &      &            & \multicolumn{2}{r}{オクテット} &  \\
 \hline
   6-1 & 20 &レーダー運用情報 & 整数 
   & $4\times N_M \times N_V \times N_Z \times N_E$ & 
   0:No Data, 1: Echo, \\
       &  &  &  &  & 2: No Echo, 3: No Ope \\
   \hline
  \end{tabular}
 \end{center}
 \caption{SUBC RADR のTable\ref{table.fmt.subc}の項番6の形式}
 \label{table.fmt.subc.radr}
\end{table}

\begin{table}[htp]
 \begin{center}
  \begin{tabular}{rr|llrl}
 \hline
 項番 & 位置 & フィールド & 型 & 長さ & 内容 \\
      &      &            & \multicolumn{2}{r}{オクテット} &  \\
 \hline
   6-1 & 20 &合成情報 & 整数 
   & $4 \times 128 \times \times N_M \times N_V \times N_Z \times N_E$ 
   & 表?? 参照\\
   \hline
  \end{tabular}
 \end{center}
 \caption{SUBC ISPC のTable\ref{table.fmt.subc}の項番6の形式}
 \label{table.fmt.subc.ispc}
\end{table}

\begin{table}[htp]
 \begin{center}
  \begin{tabular}{rr|llrl}
 \hline
 項番 & 位置 & フィールド & 型 & 長さ & 内容 \\
      &      &            & \multicolumn{2}{r}{オクテット} &  \\
 \hline
   6-1 & 20 &観測モード & 整数 
   & $4\times N_M \times N_V \times N_Z \times N_E$ & 
   1: MODE1, 2: MODE2, \\
       &  &  &  &  & 4: MODE3, 7: off \\
   6-2 & 24 &エコーフラグ & 整数 
   & $4\times N_M \times N_V \times N_Z \times N_E$ & 
   0:Echo あり, 1: No-Echo, \\
       &  &  &  &  & 2: No Ope \\
   6-3 & 28 & $N_0$値の10倍 & 整数 
      & $4\times N_M \times N_V \times N_Z \times N_E$ &  \\
   6-4 & 32 & パラメータ $B$ &  
      & $4\times N_M \times N_V \times N_Z \times N_E$ &  \\
   6-5 & 36 & パラメータ $\beta$ &  
      & $4\times N_M \times N_V \times N_Z \times N_E$ &  \\
   \hline
  \end{tabular}
 \end{center}
 \caption{SUBC RADS のTable\ref{table.fmt.subc}の項番6の形式。
 雨量強度$R$[mm/hr] は、レーダーごとに異なる定数$N_0$、パラメータ$B$,
 $\beta$、距離補正済みの受信電力 $N$ を用いると、
 $R=\left(\frac{200}{B}\right)^{1/\beta}
 \times 10 ^{\frac{80}{256}\times \frac{N-N_0}{10\beta}}$}
 で算出される。
 \label{table.fmt.subc.rads}
\end{table}

\begin{table}[htp]
 \begin{center}
  \begin{tabular}{rr|llrl}
 \hline
 項番 & 位置 & フィールド & 型 & 長さ & 内容 \\
      &      &            & \multicolumn{2}{r}{オクテット} &  \\
 \hline
   6-1 & 20 & 観測時刻 & 整数  & 4 & \\
   6-2 & 24 & 周波数 & 整数 & 4 & \\
   6-3 & 28 & レンジ解像度 & 整数 & 4 & \\
   6-4 & 32 & 方位角解像度 & 整数 & 4 & \\
   6-5 & 36 & データ解像度 & 整数 & 4 & \\
   6-6 & 40 & MTIフィルタ & 整数 & 4 & \\
   6-7 & 44 & 有効データ数 & 整数 & 4 &  \\
   6-8 & 48 & 仰角 & 整数 & 4 &  \\
   \hline
  \end{tabular}
 \end{center}
 \caption{SUBC DPRD のTable\ref{table.fmt.subc}の項番6の形式。}
 \label{table.fmt.subc.dprd}
\end{table}
%\begin{table}[htp]
% \begin{center}
%  \begin{tabular}{rr|llrl}
% \hline
% 項番 & 位置 & フィールド & 型 & 長さ & 内容 \\
%      &      &            & \multicolumn{2}{r}{オクテット} &  \\
% \hline
%   1 & 0 &データ種別       & 文字  & 4 & {\tt PRR}, {\tt PMF} \\
%   2 & 4 &対象時刻(通算分) & 整数  & 4 &  \\
%   3 & 8 &データ使用フラグ1 &      & 8 &  \\
%   4 & 16 & 初期時刻(通算分) & 整数  & 4 &  \\
%   5 & 20 & 処理時刻(通算分) & 整数  & 4 &  \\
%   6 & 24 & データ使用フラグ2 &     & 8 &  \\
%   7 & 32 & コメント・予備等 &     & 16 &  \\
%   8 & 48 & データ使用フラグ3 &     & 8 &  \\
%   9 & 56 & データ使用フラグ3 &     & 8 &  \\
%   \hline
%  \end{tabular}
% \end{center}
% \caption{SUBC ISPC のTable\ref{table.fmt.subc.ispc}の項番6-1の形式}
% \label{table.fmt.subc.ispc.content}
%\end{table}


\newpage
\section{INFO レコード}
\label{sec:fmt.subc}

INFO レコードの形式を \TabRef{table.fmt.info} に示す。

\begin{table}[htp]
 \begin{center}
 \begin{tabular}{rr|llrl}
 \hline
 項番 & 位置 & フィールド & 型 & 長さ & 内容 \\
      &      &            & \multicolumn{2}{r}{オクテット} &  \\
 \hline
 1 & 0 & レコード長 & 非負整数 & 4 & \\
 2 & 4 & レコード名 & 文字 & 4 & {\tt INFO} \\
 3 & 8 & レコード有効長 & 非負整数 & 4 & \\
 4 & 12 & 更新時刻 & 整数 & 4 & \\
 \hline
 5 & 16 & 群名 & 文字 & 4 & 利用者定義 \\
 6 & 20 & レコード内容 & 文字 & 可変 & 利用者定義 \\
 \hline
 7 &  & すき間 & & 可変 &  \\
 8 &  & レコード長 & 非負整数 & 4 &  \\
 \hline
 \end{tabular}
 \end{center}
 \caption[INFO レコードの形式]{%
  INFO レコードの形式。
  レコード内容はバイナリであってもよいが、
  NuSDaS インターフェイスはバイトオーダーの相違について関知しない。
 }
 \label{table.fmt.info}
\end{table}

\newpage
\section{DATA レコード}
\label{sec:fmt.subc}

DATA レコードの一般形式を \TabRef{table.fmt.data} に示す。

\begin{table}[htp]
 \begin{center}
 \begin{tabular}{rr|llrl}
 \hline
 項番 & 位置 & フィールド & 型 & 長さ & 内容 \\
      &      &            & \multicolumn{2}{r}{オクテット} &  \\
 \hline
 1 & 0 & レコード長 & 非負整数 & 4 & \\
 2 & 4 & レコード名 & 文字 & 4 & {\tt DATA} \\
 3 & 8 & レコード有効長 & 非負整数 & 4 & \\
 4 & 12 & 更新時刻 & 整数 & 4 & \\
 \hline
 5 & 16 & メンバー名	& 文字 & 4 & \\
 6 & 20 & 対象時刻	& 整数 & \(4\times 2\) & \\
 7 & 28 & 面名		& 文字 & \(6\times 2\) & \\
 8 & 40 & 要素名	& 文字 & 6 & \\
 9 & 46 & 予約		& 未定義 & 2 & \\
 10 & 48 & \(x, y\) 格子数 & 非負整数 & \(4\times 2\) & \\
 11 & 56 & パッキング方式 & 文字 & 4 & 定義ファイルの PACKING 文 \\
 12 & 60 & 欠損値表現方式 & 文字 & 4 & 定義ファイルの MISSING 文 \\
 \hline
 13 & 64 & パックデータ & 本節下表参照 & 可変 & \\
 \hline
 14 &  & すき間 & & 可変 &  \\
 15 &  & レコード長 & 非負整数 & 4 &  \\
 \hline
 \end{tabular}
 \end{center}
 \caption[DATA レコードの形式]{%
  DATA レコードの形式。
 }
 \label{table.fmt.data}
\end{table}

\begin{table}[htp]
 \begin{center}
 \begin{tabular}{ll|llrl}
 \hline
 項番 & 位置 & フィールド & 型 & 長さ & 内容 \\
      &      &            & \multicolumn{2}{r}{オクテット} &  \\
 \hline
 13-DATA &  64  & データ &   & 可変 & \\ \hline
 \end{tabular}
 \end{center}
 \caption[DATA レコードの項番13の形式]{%
  欠損値表現方式が ``NONE'' の場合の
 Table \ref{table.fmt.data}の項番13の形式。
 }
 \label{table.fmt.data.none}
\end{table}

\begin{table}[htp]
 \begin{center}
 \begin{tabular}{ll|llrl}
 \hline
 項番 & 位置 & フィールド & 型 & 長さ & 内容 \\
      &      &            & \multicolumn{2}{r}{オクテット} &  \\
 \hline
 13-1 &  64  & 欠損値の値 & PACK に依存 & n\_ud = 1$\sim$8 & \\
 13-DATA & 64 + n\_ud & データ &   & 可変 & \\ \hline
 \end{tabular}
 \end{center}
 \caption{%
  欠損値表現方式が ``UDFV'' の場合の Table \ref{table.fmt.data}の項番13の形式。
 }
 \label{table.fmt.data.udfv}
\end{table}

\begin{table}[htp]
 \begin{center}
 \begin{tabular}{ll|llrl}
 \hline
 項番 & 位置 & フィールド & 型 & 長さ & 内容 \\
      &      &            & \multicolumn{2}{r}{オクテット} &  \\
 \hline
 13-1 &  64  & マスクビット & ビット列
      & {\tt n\_ms} = ($x$格子数 $\times$ $y$格子数 - 1) / 8 + 1\\
 13-DATA & 64 + n\_ms & データ     &  & 可変 \\ \hline
 \end{tabular}
 \end{center}
 \caption{%
  欠損値表現方式が ``MASK'' の場合の
 Table \ref{table.fmt.data}の項番13の形式。
 }
 \label{table.fmt.data.mask}
\end{table}

\begin{table}[htp]
 \begin{center}
 \begin{tabular}{ll|llrl}
 \hline
 項番 & 位置 & フィールド & 型 & 長さ & 内容 \\
      &      &            & \multicolumn{2}{r}{オクテット} &  \\
 \hline
   1  &  0   & base($b$) & 単精度 & 4 &  \\
   2  &  4   & amp($a$)  & 単精度 & 4 &  \\
   3  &  8   & パックされたデータ & 非負整数(1PAC) & 1 $\times$ 格子数 &
  \\ 
      &      &                    & 整数(2PAC) & 2 $\times$ 格子数 & \\ 
      &      &                    & 非負整数(2UPC) & 2 $\times$ 格子数 &
  \\ 
  \hline
 \end{tabular}
 \end{center}
 \caption{%
  パッキング方式が ``1PAC'', ``2PAC'', ``2UPC'' の場合の
 Table \ref{table.fmt.data.none},
 Table \ref{table.fmt.data.udfv},
 Table \ref{table.fmt.data.mask}, の項番13-DATAの形式。
 }
 \label{table.fmt.data.2pac}
\end{table}

\begin{table}[htp]
 \begin{center}
 \begin{tabular}{ll|llrl}
 \hline
 項番 & 位置 & フィールド & 型 & 長さ & 内容 \\
      &      &            & \multicolumn{2}{r}{オクテット} &  \\
 \hline
   1  &  0   & base($b$) & 単精度 & 4 &  \\
   2  &  4   & amp($a$)  & 単精度 & 4 &  \\
   3  &  8   & $N$ & 非負整数 & 4 & 資料数 \\ 
   4  &  12  & $R_i$ & 非負整数 & 2 $\times$ 群数 $N_G$ &
   各群の参照値 ($i = 0 \cdots N_G - 1$) \\ 
   5  &  $12 + 2N_G$ & $W_i$ & 4bit & $N_H$ &
   各群の幅 ($i = 0 \cdots N_G - 1$) \\ 
   6  &  $12 + 2N_G + N_H $ & 圧縮データ & $(W_i+1)$ bit & 各群 4 $(W_i+1)$ &
   32 $(W_i+1)$bit = 4 $(W_i+1)$ byte
   \\ 
  \hline
 \end{tabular}
 \end{center}
 \caption{%
  パッキング方式が ``2UPP'' の場合の
 Table \ref{table.fmt.data.none},
 Table \ref{table.fmt.data.udfv},
 Table \ref{table.fmt.data.mask}, の項番13-DATAの形式。
 \(N_G = ((N-1)/32)+1\).
 \(N_H = ((N_G-1)/2)+1\).
 圧縮データは各群について $(W_i + 1)$ ビット幅の整数32個 (末端に欠有り得)
 の2階差分値 (オフセットあり)。
 }
 \label{table.fmt.data.2upp}
\end{table}

\begin{table}[htp]
 \begin{center}
 \begin{tabular}{ll|llrl}
 \hline
 項番 & 位置 & フィールド & 型 & 長さ & 内容 \\
      &      &            & \multicolumn{2}{r}{オクテット} &  \\
 \hline
   1  &  0   & base($b$) & 倍精度 & 8 &  \\
   2  &  8   & amp($a$)  & 倍精度 & 8 &  \\
   3  &  16   & パックされたデータ & 整数(4PAC) & 4 $\times$ 格子数 &
  \\ 
  \hline
 \end{tabular}
 \end{center}
 \caption{%
  パッキング方式が ``4PAC'' の場合の
 Table \ref{table.fmt.data.none},
 Table \ref{table.fmt.data.udfv},
 Table \ref{table.fmt.data.mask}, の項番13-DATAの形式。
 }
 \label{table.fmt.data.4pac}
\end{table}

\begin{table}[htp]
 \begin{center}
 \begin{tabular}{ll|llrl}
 \hline
 項番 & 位置 & フィールド & 型 & 長さ & 内容 \\
      &      &            & \multicolumn{2}{r}{オクテット} &  \\
 \hline
   1  &  0   & nbit & 非負整数 & 4 &  ランレングスビット数\\
   2  &  4   & maxv  & 非負整数 & 4 &  データ最大値\\
   3  &  8   & num & 非負整数 & 4 & データ列の数 \\
   3  &  12  & データ列    &    &  (nbit $\times$ num - 1)/ 8  + 1 & 
  \\ 
  \hline
 \end{tabular}
 \end{center}
 \caption{%
  パッキング方式が ``RLEN'' の場合の
 Table \ref{table.fmt.data.none},
 Table \ref{table.fmt.data.udfv},
 Table \ref{table.fmt.data.mask}, の項番13-DATAの形式。
 }
 \label{table.fmt.data.rlen}
\end{table}

\begin{table}[htp]
 \begin{center}
 \begin{tabular}{ll|llrl}
 \hline
 項番 & 位置 & フィールド & 型 & 長さ & 内容 \\
      &      &            & \multicolumn{2}{r}{オクテット} &  \\
 \hline
   1  &  0  & データ    & 非負整数(I1)  & 1 $\times$ 格子数 & \\  
      &     &           & 整数(I2)      & 2 $\times$ 格子数 & \\  
      &     &           & 整数(N1I2)    & 2 $\times$ 格子数 & \\  
      &     &           & 整数(I4)      & 4 $\times$ 格子数 & \\  
      &     &           & 単精度(R4)      & 4 $\times$ 格子数 & \\  
      &     &           & 倍精度(R8)      & 8 $\times$ 格子数 & \\  
  \hline
 \end{tabular}
 \end{center}
 \caption{%
  パッキング方式が ``I1  '', ``I2  '', ``N1I2'', ``I4  '', ``R4  '', 
 ``R8  ''の場合の
 Table \ref{table.fmt.data.none},
 Table \ref{table.fmt.data.udfv},
 Table \ref{table.fmt.data.mask}, の項番13-DATAの形式。
 }
 \label{table.fmt.data.r4}
\end{table}

\newpage
\section{END レコード}
\label{sec:fmt.subc}

END レコードの形式を \TabRef{table.fmt.end} に示す。
NuSDaS 1.3 にはバグがあって、既に補助レコードが存在する
データファイルを開いて書き込むと、補助レコード表 (項番7) に不正値が入っていた。
このバグはNuSDaS 1.4 で修正されている。

\begin{table}[htp]
 \begin{center}
 \begin{tabular}{ll|llrl}
 \hline
 項番 & 位置 & フィールド & 型 & 長さ & 内容 \\
      &      &            & \multicolumn{2}{r}{オクテット} &  \\
 \hline
 1 & 0 & レコード長 & 非負整数 & 4 & \\
 2 & 4 & レコード名 & 文字 & 4 & {\tt END\SPC} \\
 3 & 8 & レコード有効長 & 非負整数 & 4 & \\
 4 & 12 & 更新時刻 & 整数 & 4 & \\
 \hline
 5 & 16 & ファイル長 & 非負整数 & 4 & NUSD 項番8に同じ \\
 6 & 20 & レコード数 & 非負整数 & 4 & NUSD 項番9に同じ \\
 \hline
 7 & 24 & 補助レコード表 & 
 	& \(24 \times (N_s+N_i)\) & 7.1〜7.5 を繰り返し \\
 7.1 & \quad+0 & レコード名 & 文字 & 4 & {\tt SUBC} または {\tt INFO} \\
 7.2 & \quad+4 & 群名 & 文字 & 4 & \\
 7.3 & \quad+8 & レコード位置 & 整数 & 8 & \\
 7.4 & \quad+16 & レコード長 & 非負整数 & 4 & (注1) \\
 7.5 & \quad+20 & 予約 & 未定義 & 4 & 0 \\
 \hline
 8 &  & すき間 & & 可変 &  \\
 9 &  & レコード長 & 非負整数 & 4 &  \\
 \hline
 \end{tabular}
 \end{center}
 \caption[END レコードの形式]{%
  END レコードの形式。
  ファイル形式 11 以前は項番 7 を欠く。
  注1:
  ここで言うレコード長とはTable\ref{table.fmt.general}で示している
  項番2〜6 の占めるオクテット数ではなく、レコード全体の長さである
  項番1~7の占めるオクテット数である。
 }
 \label{table.fmt.end}
\end{table}

