\subsection{NUSDAS\_SUBC\_PRESET1: SUBC ETA/SIGM のデフォルト値設定 }
\APILabel{nusdas.subc.preset1}

\Prototype
\begin{quote}
CALL {\bf NUSDAS\_SUBC\_PRESET1}({\it type1}, {\it type2}, {\it type3}, {\it group}, {\it n\_levels}, {\it a}, {\it b}, {\it c}, {\it result})
\end{quote}

\begin{tabular}{l|rllp{16em}}
\hline
\ArgName & \ArgType & \ArrayDim & I/O & \ArgRole \\
\hline
{\it type1} & CHARACTER(8) &  & IN &  種別1  \\
{\it type2} & CHARACTER(4) &  & IN &  種別2  \\
{\it type3} & CHARACTER(4) &  & IN &  種別3  \\
{\it group} & CHARACTER(4) &  & IN &  群名  \\
{\it n\_levels} & INTEGER(4) &  & IN &  鉛直層数  \\
{\it a} & REAL(4) & \AnySize & I/O &  係数 a  \\
{\it b} & REAL(4) & \AnySize & I/O &  係数 b  \\
{\it c} & REAL(4) &  & I/O &  係数 c  \\
{\it result} & INTEGER(4) &  & OUT & \ResultCode \\
\hline
\end{tabular}
\paragraph{\FuncDesc}ファイルが新たに生成される際にETA, SIGMに書き込む値を設定する。
SIGM や引数については nusdas\_subc\_eta を参照。
引数の「群名」には、"ETA " または "SIGM" を指定する。
\paragraph{\ResultCode}
\begin{quote}
\begin{description}
\item[{\bf 0}] 正常終了
\item[{\bf -1}] 定義ファイルに指定した群名が登録されていない
\item[{\bf -2}] メモリの確保に失敗した
\item[{\bf -3}] レコードのサイズが不正
\end{description}\end{quote}

\paragraph{ 互換性 }
NuSDaS1.1 では、一つのNuSDaSデータセットに設定できる補助管理部の数は最大
10 に制限されており、それを超えると-2が返された。一方、 NuSDaS 1.3 からは
メモリが確保できる限り数に制限はなく、-2 をメモリ確保失敗のエラーコードに
読み替えている。
