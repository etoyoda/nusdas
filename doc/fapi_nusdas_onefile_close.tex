\subsection{NUSDAS\_ONEFILE\_CLOSE: 指定データファイルを閉じる}
\APILabel{nusdas.onefile.close}

\Prototype
\begin{quote}
CALL {\bf NUSDAS\_ONEFILE\_CLOSE}({\it type1}, {\it type2}, {\it type3}, {\it basetime}, {\it member}, {\it validtime}, {\it result})
\end{quote}

\begin{tabular}{l|rllp{16em}}
\hline
\ArgName & \ArgType & \ArrayDim & I/O & \ArgRole \\
\hline
{\it type1} & CHARACTER(8) &  & IN &  種別1  \\
{\it type2} & CHARACTER(4) &  & IN &  種別2  \\
{\it type3} & CHARACTER(4) &  & IN &  種別3  \\
{\it basetime} & INTEGER(4) &  & IN &  基準時刻(通算分)  \\
{\it member} & CHARACTER(4) &  & IN &  メンバー名  \\
{\it validtime} & INTEGER(4) &  & IN &  対象時刻  \\
{\it result} & INTEGER(4) &  & OUT & \ResultCode \\
\hline
\end{tabular}

\paragraph{\ResultCode}
\begin{quote}
\begin{description}
\item[{\bf 0}] 正常終了
\item[{\bf 1}] ファイルクローズ前の書き込み時に定義ファイルを読み込めなかった
\item[{\bf -1}] ファイルクローズ前の書き込み時にIOエラーが発生
\end{description}\end{quote}

\paragraph{\FuncDesc}\paragraph{履歴}
この関数は NuSDaS 1.0 から存在した.
